\documentclass[a4paper, 11pt]{report}
\usepackage[czech]{babel}
\usepackage[utf8]{inputenc}
\usepackage{multirow}
\usepackage{amsmath}
\usepackage{amsfonts}
\usepackage{enumerate}
\usepackage{verbatim}
\usepackage{tikz-qtree}

\usepackage{amsthm}
\newtheorem{mydef}{Definice}
\newtheorem{veta}{Věta}
\newtheorem{lemma}{Lemma}

\usepackage{geometry}
\usepackage{layout}

\geometry{
  includeheadfoot,
  hmargin=2.0cm,
  vmargin={0cm, 1.0cm}
}

\usepackage{color}
\usepackage[unicode,colorlinks,hyperindex,plainpages=false,pdftex]{hyperref}

\usepackage{listings}  
\definecolor{mygreen}{rgb}{0,0.6,0}
\lstset{language=VHDL,commentstyle=\color{mygreen},tabsize=4}

\usepackage{fancyhdr}
\pagestyle{fancyplain}
\fancyhf{}
\renewcommand{\headrulewidth}{0pt}

\cfoot{\hfill © Jakuje \hfill \thepage }


\begin{document}

\ref{cha:1}
\ref{cha:2}
\ref{cha:3}
\ref{cha:4}
\ref{cha:5}
\ref{cha:6}
\ref{cha:7}
\ref{cha:8}
\ref{cha:9}
\ref{cha:10}
\ref{cha:11}
\ref{cha:12}
\ref{cha:13}
\ref{cha:14}
\ref{cha:15}
\ref{cha:16}
\ref{cha:17}
\ref{cha:18}
\ref{cha:19}
\ref{cha:20}

\ref{cha:21}
\ref{cha:22}
\ref{cha:23}
\ref{cha:24}
\ref{cha:25}
\ref{cha:26}
\ref{cha:27}
\ref{cha:28}
\ref{cha:29}
\ref{cha:30}
\ref{cha:31}
\ref{cha:32}
\ref{cha:33}
\ref{cha:34}
\ref{cha:35}
\ref{cha:36}
\ref{cha:37}
\ref{cha:38}
\ref{cha:39}
\ref{cha:40}

\ref{cha:41}
\ref{cha:42}
\ref{cha:43}
\ref{cha:44}
\ref{cha:45}
\ref{cha:46}
\ref{cha:47}
\ref{cha:48}
\ref{cha:49}
\ref{cha:50}
\ref{cha:51}
\ref{cha:52}
\ref{cha:53}
\ref{cha:54}
\ref{cha:55}
\ref{cha:56}
\ref{cha:57}
\ref{cha:58}
\ref{cha:59}
\ref{cha:60}

\ref{cha:61}
\ref{cha:62}
\ref{cha:63}
\newpage

\tableofcontents

%%%%%%%%%%%%%%%%%%%%%%%%%%%%%%%%%%%%%%%%%%%%%%%%%%%%%%%%%%%%%%%%%%%%%%%%%%%%%%%%
%%%%%%%%%%%%%%%%%%%%%%%%%%%%%%%%%%%%%%%%%%%%%%%%%%%%%%%%%%%%%%%%%%%%%%%%%%%%%%%%
\chapter{Metodika návrhu HW/SW codesign, platformy, programovatelné obvody.} \label{cha:1}
X. semestr, XXX, ??
\chapter{Výpočetní modely} \label{cha:2}
X. semestr, XXX, ??

(StateCharts, Kahnova síť procesů, synchronní dataflow)
\chapter{Specifikace (chování, struktura), syntéza (alokace, přidělení, plánování) a integrace systémů (rozhraní, synchronizace, komunikace).} \label{cha:3}
X. semestr, XXX, ??
\chapter{Syntéza HW z vyšších programovacích jazyků (reprezentace, alokace, plánování, přiřazení) a jazyk Catapult C.} \label{cha:4}
X. semestr, XXX, ??
\chapter{Odhady (přesnost, věrnost, metriky, metody) a optimalizace vlastností systému (příkon, energie).} \label{cha:5}

%%%%%%%%%%%%%%%%%%%%%%%%%%%%%%%%%%%%%%%%%%%%%%%%%%%%%%%%%%%%%%%%%%%%%%%%%%%%%%%%
%%%%%%%%%%%%%%%%%%%%%%%%%%%%%%%%%%%%%%%%%%%%%%%%%%%%%%%%%%%%%%%%%%%%%%%%%%%%%%%%
\chapter{Jazyk a sémantika predikátové logiky} \label{cha:6}
1. semestr, MAT, \texttt{logikaaktual3.pdf}, 3., 4. kapitola

(termy, formule, realizace jazyka, pravdivost formulí)

\section{Jazyk predikátové logiky}

\begin{itemize}
	\item Logické symboly
	\begin{itemize}
		\item proměnné: $x, y, z, \dots, x_1, x_2, \dots$
		\item logické spojky: $ \lnot, \land, \lor, \to, \leftrightarrow$
		\item kvantifikátory: $\exists, \forall$
		\item závorky, čárka: $(,)$
		\item predikátový symbol rovnosti $=$
	\end{itemize}
	\item Speciální symboly
	\begin{itemize}
		\item funkční symboly $f, g, h, \dots, f_1, f_2, f_3$, nezáporné celé číslo -- četnost
		\item predikátové symboly $p, q, r, \dots, p_1, p_2, \dots$, kladné celé číslo -- jeho četnost.
	\end{itemize}
\end{itemize}

Obsahuje-li jazyk symbol $=$ pro rovnost, mluvíme o \emph{jazyku s rovností}.
Specifiku jazyka určují jeho funkční a predikátové symboly (určující oblast kterou jazyk popisuje)

\subsection{Termy}
\begin{enumerate}[(i)]
	\item Každá proměnná je term
	\item Je-li $f$ funkční symbol s četností $n$ a jsou-li $t_1, \dots, t_n$ termy, pak také $f(t_1, \dots, t_n)$ je term
	\item Každý term vznikne konečným počtem užití (i), (ii)
\end{enumerate}
(ze (ii) plyne, že každá konstanta je term)

\subsection{Atomické formule}
Je-li $p$ predikátový symbol s četností $n$ a jsou-li $t_1, \dots t_n$ termy, pak $p(t_1, \dots, t_n)$ je \emph{atomická formule}.

Speciální, máme-li jazyk s rovností a jsou-li $t_1, t_2$ termy, pak $(t_1 = t_2)$ je atomická formule.
Píšeme $(t_1 = t_2)$ místo $=(t_1, t_2)$. Podobný zápis používáme i pro jiné binární predikátové operátory, např. místo $< (t_1, t_2)$ píšeme $(t_1 < t_2)$.

\subsection{Formule}
\begin{enumerate}[(i)]
	\item Každá atomická formule je formule
	\item Jsou-li $\varphi, \psi$ formule, pak také $(\lnot \varphi), (\varphi \land \psi), (\varphi \lor \psi), (\varphi \to \psi), (\varphi \leftrightarrow \psi)$ jsou formule.
	\item Je-li $x$ proměnná a $\varphi$ formule, pak také $(\forall x \varphi)$, $(\exists x \varphi)$ jsou formule.
	\item Každá formule vznikne konečným počtem užití (i), (ii), (iii)
\end{enumerate}

\tikzset{every tree node/.style={align=center,anchor=north}}
\Tree[.{Formule}
	[.{Atomická formule}
		{Predikátový symbol\\ + Term\\ $p(t_1, \dots, t_n)$}
		[.{Term}
			{Proměnná\\ $x$}
			{Funkční symbol\\ + Term\\ $f(t_1, \dots, t_n)$}
			]
		]
	{Logické spojky\\ + Formule\\ $\varphi \land \psi$}
	{Kvantifikátory\\ + Proměnné\\ + Formule\\ $\forall x \varphi$}
]

\begin{description}
	\item[Vázaný výskyt proměnné] nachází-li se v nějaké podformuli tvaru $\forall x \varphi$ nebo $\exists x \varphi$.
	\item[Obor kvantifikátoru] $\varphi$
	\item[Volný výskyt proměnné] není vázaný
	\item[Volná (Vázaná) proměnná] existuje-li volný (vázaný) výskyt proměnné v této formuli
	\item[Uzavřená formule (Výrok)] Formule neobsahující žádnou volnou proměnnou
	\item[Otevřená formule (Výrok)] Formule neobsahující žádnou vázanou proměnnou
	\item[Formule s čistými proměnnými] Otevřené a uzavřené formule
\end{description}

\section{Sémantika predikátové logiky}

\begin{mydef}
Nechť $L$ je jazyk 1. řádu. \emph{Realizací jazyka} $L$ rozumíme algebraickou strukturu $\mathcal{M}$, která se skládá z
\begin{enumerate}[(i)]
	\item neprázdné množiny $M$, kterou nazveme \emph{univerzum}
	\item pro každý funkční symbol $f$ četností $n$ je dáno zobrazení $f_\mathcal{M} : M^n \to M$
	\item pro každý predikátový symbol $p$ četnosti $n$, kromě rovnosti, je dána relace $p_\mathcal{M} \subset M^n$
\end{enumerate}
Poznamenejme, že pro nulární funkční symbol, tj. pro konstantu, je $M^0 = \{0\}$ a příslušné zobrazení $M^0 \to M$ lze chápat jako vyznačení určitého prvku z $M$ odpovídajícího daného konstantě.
\end{mydef}

\paragraph{Ohodnocení proměnných}: Libovolné zobrazení $e$ množiny všech proměnných do univerza $M$ dané realizace $\mathcal{M}$ jazyka $L$. Pokud proměnné $x$ přiřazuje prvek $m$, budeme značit $e(x/m)$.

\begin{mydef}
\emph{Hodnota termu} $t$ v realizaci $\mathcal{M}$ jazyka $L$ při daném ohodnocení $e$ proměnných, označovaná $t[e]$, se definuje indukcí následovně:
\begin{enumerate}[(i)]
	\item Je-li $t$ proměnná $x$, potom $t[e]$ je $e(x)$
	\item je-li $t$ term tvaru $f(t_1, \dots, t_n)$, kde $f$ je funkční symbol četnosti $n$ a $t_1, \dots, t_n$ jsou termy, potom $t[e]$ je $f_\mathcal{M}(t_1[e], \dots, t_n[e])$
\end{enumerate}

\end{mydef}
\begin{mydef}
Nechť $\mathcal{M}$ je realizace jazyka $L$, nechť $e$ je ohodnocení proměnných a nechť $\varphi$ je formule jazyka $L$. Indukcí podle složitosti formule $\varphi$ definujeme, co znamená, že \emph{formule $\varphi$ je pravdivá v $\mathcal{M}$ při ohodnocení $e$}. Tuto skutečnost budeme značit $\mathcal{M} \models \varphi[e]$
\begin{enumerate}[(i)]
	\item Je-li $\varphi$ atomická formule tvaru $p(t_1, \dots, t_n)$, kde $p$ je predikátový symbol četnosti $n$ a $t_1, \dots, t_n$ jsou termy, pak $\mathcal{M} \models \varphi[e]$ právě když $(t_1[e], \dots, t_n[e]) \in p_\mathcal{M}$.
	\item Je-li $\varphi$ atomická formule tvaru $t_1 = t_2$, kde $t_1, t_2$ jsou termy, pak $\mathcal{M} \models \varphi[e]$ právě když $t_1[e]$ je tentýž prvek jako $t_2[e]$ v $M$
	\item Je-li $\varphi$ tvaru $\lnot \psi$, kde je $\psi$ je formule jazyka $L$, pak $\mathcal{M} \models \varphi[e]$ je právě když $\mathcal{M} \not\models \psi[e]$.
	\item Je-li $\varphi$ některého z tvarů $(\eta \land \psi), (\eta \lor \psi), (\eta \to \psi), (\eta \leftrightarrow \psi)$, kde $\eta$, $\psi$ jsou formule, klademe: \\
	$\mathcal{M} \models (\eta \land \psi)[e]$ právě když současně $\mathcal{M} \models \eta[e]$ a $\mathcal{M} \models \psi[e]$. \\
	$\mathcal{M} \models (\eta \lor \psi)[e]$ právě platí alespoň jedno z $\mathcal{M} \models \eta[e]$ a $\mathcal{M} \models \psi[e]$ a podobně další logické spojky.
	\item Je-li $\varphi$ tvaru $(\forall x \psi)$, kde $\psi$ je formule jazyka $L$, pak $\mathcal{M} \models \varphi[e]$ právě když pro každý prvek $m \in M$ je $\mathcal{M} \models \psi[e(x/m)]$.
	\item Je-li $\varphi$ tvaru $(\exists x \psi)$, kde $\psi$ je formule jazyka $L$, pak $\mathcal{M} \models \varphi[e]$ právě když existuje $m \in M$ taková, že $\mathcal{M} \models \psi[e(x/m)]$.
\end{enumerate}
\end{mydef}

\begin{comment}
\section{Ostatní}

Formalizovaná axiomatická teorie je dána
\begin{itemize}
	\item symboly -- tvoří abecedu
	\item formulemi -- určitá slova této abecedy, která tvoří jazyk této teorie
	\item axiomy -- výchozí tvrzení této teorie zapsaná pomocí abecedy jako jisté formule
	\item odvozovací pravidla --  pravidla pro manipulaci s formulemi, pomocí kterých odvozujeme z axiomů důsledky.
\end{itemize}

\begin{description}
	\item[Proměnné] Označení libovolného prvku z daného oboru $(x, y, z, \dots, x_1, x_2, \dots)$
	\item[Konstanty] Význačné objekty $(0, 1, \dots)$
	\item[Funkční symboly] Operace $(f, g, h, \dots, f_1, f_2, f_3)$
	\item[Četnost funkčního symbolu] počet argumentů dané operace
	\item[Predikáty] Vlastnosti a vztahy mezi objekty
	\item[Predikátové symboly] Vyjádření predikátů? $(p, q, r, \dots, p_1, p_2, \dots)$.
	\item[Četnost predikátového symbolu] počet argumentů predikátu
	\item[Atomické formule] Nejjednodušší tvrzení, složená z \emph{Proměnných}, \emph{konstant}, \emph{funkčních symbolů}, \emph{predikátových symbolů}.
	\item[Složitější formule] Atomické formule + logické spojky + kvantifikace proměnných ($\forall, \exists$)
	\item[Abeceda predikátové logiky 1. řádu] výše uvedené symboly s logickými spojkami a pomocnými symboly (závorky, čárka).
\end{description}


\section{Predikátová logika 1. řádu}

Matematické teorie pracují s celými soubory objektů (čísla, body v prostoru, prvky algebraických struktur).
Pro označení lib. prvků z daného oboru používáme \emph{proměnné} $(x, y, z, \dots, x_1, x_2, \dots)$

Mezi prvky z daného oboru mohou být některé význačné objekty (0, neutrální prvek grupy, \dots), pro než užíváme zvláštní symboly -- \emph{konstanty} (např. 0, 1, \dots).

S objekty daného oboru lze provádět různé operace (sčítání a násobení čísel, násobení v grupách, \dots).
K označení operace užíváme \emph{funkční symboly} $(f, g, h, \dots, f_1, f_2, f_3)$.
Ke každému funkčnímu symbolu je přiřazeno přirozené číslo, které vyjadřuje jeho \emph{četnost}, tj. počet argumentů dané operace.
Je-li četnost symbolu rovna $n$, říkáme, že symbol je $n$-ární.
Je přirozené chápat konstanty jako nulární funkční symboly.

Matematika zkoumá vlastnosti objektů a vztahy mezi objekty.
Vlastnosti  a vztahy mezi objekty daného oboru, tzv. \emph{predikáty} (\uv{být záporným číslem} (vlastnost), \uv{být menší než}, \uv{být prvkem} (vztahy)) vyjadřujeme pomocí \emph{predikátových symbolů} $(p, q, r, \dots, p_1, p_2, \dots)$.
Predikát znamená vztah mezi užitým počtem objektů.
Tím je každému predikátovému symbolu přiřazeno přirozené číslo, jeho četnost udávající počet jeho argumentů.
Je-li četnost rovna $n$, říkáme, že symbol je \emph{n-ární}.
V mnoha případech používáme zvláštní označení = pro binární predikátový symbol označující rovnost, tj. totožnost objektů z daného oboru.

Z proměnných, konstant, funkčních symbolů a predikátových symbolů sestavujeme jistým způsobem nejjednodušší tvrzení, vyjádřená tzv. \emph{atomickými formulemi}.
Z nich vytváříme složitější formule pomocí \emph{logických spojek} (stejných jako ve výrokové logice) a pomocí \emph{kvantifikace proměnných}.\\
\emph{Univerzální (obecný) kvantifikátor} $\forall$ vyjadřuje platnost pro všechny objekty z daného oboru.\\
\emph{Existenční kvantifikátor} $\exists$ vyjadřuje existenci požadovaného objektu v daném oboru.

Uvedené symboly spolu s logickými spojkami a \emph{pomocnými symboly} (závorka, čárka) tvoří abecedu jazyka \emph{predikátové logiky 1. řádu}.
Proměnné jazyka prvního řádu jsou obecná jména pro objekty daného oboru, tj. pro individua (např. čísla).
Jazyk neobsahuje proměnné pro množiny individuí (např. množiny čísel, relací, \dots), vyšších řádů, které dovolují kvantifikovat např. množiny, relace.

%%%%%%%%%%%%%%%%%%%%%%%%%%%%%%%%%%
\section{Jazyk predikátové logiky}

\begin{itemize}
	\item Logické symboly
	\begin{itemize}
		\item proměnné: $x, y, z, \dots, x_1, x_2, \dots$
		\item logické spojky: $ \lnot, \land, \lor, \to, \leftrightarrow$
		\item kvantifikátory: $\exists, \forall$
		\item závorky, čárka: $(,)$
		\item predikátový symbol rovnosti $=$
	\end{itemize}
	\item Speciální symboly
	\begin{itemize}
		\item funkční symboly $f, g, h, \dots, f_1, f_2, f_3$, nezáporné celé číslo -- četnost
		\item predikátové symboly $p, q, r, \dots, p_1, p_2, \dots$, kladné celé číslo -- jeho četnost.
	\end{itemize}
\end{itemize}

Obsahuje-li jazyk symbol $=$ pro rovnost, mluvíme o \emph{jazyku s rovností}.
Specifiku jazyka určují jeho funkční a predikátové symboly (určující oblast kterou jazyk popisuje)

\subsection{Termy}
\begin{enumerate}[(i)]
	\item Každá proměnná je term
	\item Je-li $f$ funkční symbol s četností $n$ a jsou-li $t_1, \dots, t_n$ termy, pak také $f(t_1, \dots, t_n)$ je term
	\item Každý term vznikne konečným počtem užití (i), (ii)
\end{enumerate}
(ze (ii) plyne, že každá konstanta je term)

\subsection{Atomické formule}
Je-li $p$ predikátový symbol s četností $n$ a jsou-li $t_1, \dots t_n$ termy, pak $p(t_1, \dots, t_n)$ je \emph{atomická formule}.

Speciální, máme-li jazyk s rovností a jsou-li $t_1, t_2$ termy, pak $(t_1 = t_2)$ je atomická formule.
Píšeme $(t_1 = t_2)$ místo $=(t_1, t_2)$. Podobný zápis používáme i pro jiné binární predikátové operátory, např. místo $< (t_1, t_2)$ píšeme $(t_1 < t_2)$.

\subsection{Formule}
\begin{enumerate}[(i)]
	\item Každá atomická formule je formule
	\item Jsou-li $\varphi, \psi$ formule, pak také $(\lnot \varphi), (\varphi \land \psi), (\varphi \lor \psi), (\varphi \to \psi), (\varphi \leftrightarrow \psi)$ jsou formule.
	\item Je-li $x$ proměnná a $\varphi$ formule, pak také $(\forall x \varphi)$, $(\exists x \varphi)$ jsou formule.
	\item Každá formule vznikne konečným počtem užití (i), (ii), (iii)
\end{enumerate}

Poznamenejme, že píšeme $x \not= y$ místo $\lnot (x, y)$ a také, pokud to nemůže narušit srozumitelnost, vynecháváme některé dvojice závorek.

Při tvorbě formule $\varphi$ podle předchozí definice vytváříme určitou posloupnost formulí, která začíná atomickými formulemi a končí formulí $\varphi$ a každá formule v této posloupnosti vzniká z některých předcházejících pomocí logických spojek a kvantifikátorů. Každá z těchto formulí se nazývá podformule $\varphi$.

Každá formule je konečnou posloupností symbolů. Každý symbol, zejména každá proměnná, se může ve formuli vyskytovat na jednom nebo více místech.
Řekněme, že daný \emph{výskyt} proměnné $x$ ve formuli $\varphi$ je \emph{vázaný}, nenachází-li se v nějaké podformuli tvaru $\forall x \psi$ nebo $\exists x \psi$. V tomto případě se proměnná $x$ vyskytuje v kvantifikátoru samém nebo ve formuli $\psi$ (podformule $\psi$ se nazývá \emph{obor kvantifikátoru} $\forall x$ nebo $\exists x$.
V opačném případě (výskyt není vázaný) řekneme, že daný výskyt proměnné $x$ ve formuli $\phi$ je \emph{volný}.
Proměnná $x$ se nazývá \emph{volnou (vázanou) proměnnou} ve formuli $\phi$, existuje-li její volný (vázáný) výskyt v této formuli.
Proměnná tedy může být ve formuli volná i vázaná. Formule neobsahující žádnou volnou proměnnou se nazývá \emph{uzavřená formule} nebo též \emph{výrok}.
Naopak, formule neobsahující žádnou vázanou proměnnou se nazývá \emph{otevřenou formulí}. Uzavření a otevřené formule nazýváme \emph{formulemi s čistými proměnnými}.

%%%%%%%%%%%%%%%%%%%%%%%%%%%%%%%%%%%%%%
\section{Sémantika predikátové logiky}

Chceme dát interpretaci symbolům jazyka predikátové logiky 1. řádu.
Nejprve vymezíme obor, který budeme určovat možné hodnoty proměnných, bude to určitý soubor $M$ uvažovaných objektů.
Funkčním symbolům budou odpovídat operace na $M$ příslušných četností.
Predikátovým symbolům budou odpovídat vztahy mezi objekty z $M$, které lze popsat jako relace na $M$ příslušných četností. Máme-li jazyk s rovností, interpretujeme symbol $=$ jako rovnost objektů z $M$.

\begin{mydef}
Nechť $L$ je jazyk 1. řádu. \emph{Realizací jazyka} $L$ rozumíme algebraickou strukturu $\mathcal{M}$, která se skládá z
\begin{enumerate}[(i)]
	\item neprázdné množiny $M$, kterou nazveme \emph{univerzum}
	\item pro každý funkční symbol $f$ četností $n$ je dáno zobrazení $f_\mathcal{M} : M^n \to M$
	\item pro každý predikátový symbol $p$ četnosti $n$, kromě rovnosti, je dána relace $p_\mathcal{M} \subset M^n$
\end{enumerate}
\end{mydef}

Poznamenejme, že pro nulární funkční symbol, tj. pro konstantu, je $M^0 = \{0\}$ a příslušné zobrazení $M^0 \to M$ lze chápat jako vyznačení určitého prvku z $M$ odpovídajícího daného konstantě.

Chceme-li zkoumat pravdivost formulí jazyka $L$ v nějaké jeho realizaci $\mathcal{M}$ musíme volným proměnným přiřadit hodnoty, jimiž budou nějaké prvky množiny $M$.

\begin{mydef}
Libovolné zobrazení $e$ množiny všech proměnných do univerza $M$ dané realizace $\mathcal{M}$ jazyka $L$ budeme nazývat \emph{ohodnocení proměnných}.

Je-li $x$ proměnná a $e$ ohodnocení proměnných a $m \in M$, potom ohodnocení proměnných, které proměnné $x$ přiřazuje prvek $m$ a pro všechny ostatní proměnné splývá s ohodnocením $e$, budeme značit $e(x/m)$.
\end{mydef}

\begin{mydef}
\emph{Hodnota termu} $t$ v realizaci $\mathcal{M}$ jazyka $L$ při daném ohodnocení $e$ proměnných, označovaná $t[e]$, se definuje indukcí následovně:
\begin{enumerate}[(i)]
	\item Je-li $t$ proměnná $x$, potom $t[e]$ je $e(x)$
	\item je-li $t$ term tvaru $f(t_1, \dots, t_n)$, kde $f$ je funkční symbol četnosti $n$ a $t_1, \dots, t_n$ jsou termy, potom $t[e]$ je $f_\mathcal{M}(t_1[e], \dots, t_n[e])$
\end{enumerate}
\end{mydef}

\begin{mydef}
Nechť $\mathcal{M}$ je realizace jazyka $L$, nechť $e$ je ohodnocení proměnných a nechť $\varphi$ je formule jazyka $L$. Indukcí podle složitosti formule $\varphi$ definujeme, co znamená, že \emph{formule $\varphi$ je pravdivá v $\mathcal{M}$ při ohodnocení $e$}. Tuto skutečnost budeme značit $\mathcal{M} \models \varphi[e]$
\begin{enumerate}[(i)]
	\item Je-li $\varphi$ atomická formule tvaru $p(t_1, \dots, t_n)$, kde $p$ je predikátový symbol četnosti $n$ a $t_1, \dots, t_n$ jsou termy, pak $\mathcal{M} \models \varphi[e]$ právě když $(t_1[e], \dots, t_n[e]) \in p_\mathcal{M}$.
	\item Je-li $\varphi$ atomická formule tvaru $t_1 = t_2$, kde $t_1, t_2$ jsou termy, pak $\mathcal{M} \models \varphi[e]$ právě když $t_1[e]$ je tentýž prvek jako $t_2[e]$ v $M$
	\item Je-li $\varphi$ tvaru $\lnot \psi$, kde je $\psi$ je formule jazyka $L$, pak $\mathcal{M} \models \varphi[e]$ je právě když $\mathcal{M} \not\models \psi[e]$.
	\item Je-li $\varphi$ některého z tvarů $(\eta \land \psi), (\eta \lor \psi), (\eta \to \psi), (\eta \leftrightarrow \psi)$, kde $\eta$, $\psi$ jsou formule, klademe: \\
	$\mathcal{M} \models (\eta \land \psi)[e]$ právě když současně $\mathcal{M} \models \eta[e]$ a $\mathcal{M} \models \psi[e]$. \\
	$\mathcal{M} \models (\eta \lor \psi)[e]$ právě platí alespoň jedno z $\mathcal{M} \models \eta[e]$ a $\mathcal{M} \models \psi[e]$ a podobně další logické spojky.
	\item Je-li $\varphi$ tvaru $(\forall x \psi)$, kde $\psi$ je formule jazyka $L$, pak $\mathcal{M} \models \varphi[e]$ právě když pro každý prvek $m \in M$ je $\mathcal{M} \models \psi[e(x/m)]$.
	\item Je-li $\varphi$ tvaru $(\exists x \psi)$, kde $\psi$ je formule jazyka $L$, pak $\mathcal{M} \models \varphi[e]$ právě když existuje $m \in M$ taková, že $\mathcal{M} \models \psi[e(x/m)]$.
\end{enumerate}
\end{mydef}

\begin{mydef}
Řekneme, že formule $\varphi$ jazyka $L$ je \emph{logicky platná}, jestliže pro každou realizaci $\mathcal{M}$ jazyka $L$ je $\mathcal{M} \models \varphi$, píšeme $\models \varphi$.
\end{mydef}
\end{comment}
















%%%%%%%%%%%%%%%%%%%%%%%%%%%%%%%%%%%%%%%%%%%%%%%%%%%%%%%%%%%%%%%%%%%%%%%%%%%%%%%%
%%%%%%%%%%%%%%%%%%%%%%%%%%%%%%%%%%%%%%%%%%%%%%%%%%%%%%%%%%%%%%%%%%%%%%%%%%%%%%%%
\chapter{Formální systém predikátové logiky} \label{cha:7}

1. semestr, MAT, \texttt{logikaaktual3.pdf}, 5., 6., 7., 8. kapitola

(axiomy a odvozovací pravidla, dokazatelnost, model a důsledek teorie, věty o úplnosti a kompaktnosti, prenexní tvar formulí)

\section{Axiomy}

%Budujeme predikátovou logiku jako formální axiomatický systém. Jazyk $L$ predikátové logiky přebíráme z předchozího s tím, že z logických spojek bereme jako základní $\lnot$ a $\to$ (ostatní mohou být definovány jako ve výrokovém počtu). Z kvantifikátorů bereme jako základní $\forall$, kvantifikátor $\exists$ je možno zavést takto: Je-li $\varphi$ formule, pak $\exists x \varphi$ je zkratka pro $\lnot(\forall x (\lnot \varphi))$. Omezíme se tedy pouze na ty formule, které jsou vytvořeny z atomických formulí jen pomocí spojek $\lnot$, $\to$ a kvantifikátoru $\forall$.. Axiomy predikátové logiky lze rozdělit do čtyř skupin.

\subsection{Schémata výrokových axiomů}

Jsou-li $\varphi, \psi, \eta$ formule jazyka $L$, pak
\begin{eqnarray*}
&\varphi \to (\psi \to \varphi) & \\
&(\varphi \to (\psi \to \eta)) \to ((\varphi \to \psi) \to (\varphi \to \eta)) & \\
&((\lnot \psi) \to (\not \varphi)) \to (\varphi \to \psi) & 
\end{eqnarray*}
jsou axiomy predikátové logiky.

\subsection{Schéma axiomu kvantifikátoru}
Jsou-li $\varphi, \psi$ formule a je-li $x$ proměnná, která nemá volný výskyt ve formuli $\varphi$, pak
$$ (\forall x (\varphi \to \psi)) \to (\varphi \to (\forall x \psi)) $$
je axiom predikátové logiky.

\subsection{Schéma axiomu substituce}
Je-li $\varphi$ formule, $x$ proměnná a $t$ term substituovatelný za $x$ do $\varphi$, pak
$$ (\forall x \varphi) \to \varphi_x[t] $$
je axiom predikátové logiky.

Jestliže $t = x$, pak schéma axiomu substituce má tvar
$$ (\forall x \varphi) \to \varphi $$


\subsection{Schémata axiomů rovnosti}
Je-li $x$ proměnná, pak $x = x$ je axiom. Jsou-li $x_1, \dots, x_n, y_1, \dots, y_n$ proměnné a je-li $f$ funkční symbol s četností $n$, pak
$$(x_1 = y_1 \to (x_2 = y_2 \to ( \dots (x_n = y_n \to f(x_1, \dots, x_n) = f(y_1, \dots, y_n)) \dots ))) $$
je axiom. Jsou-li $x_1, \dots, x_n, y_1, \dots, y_n$ proměnné, je-li $p$ predikátový symbol s četností $n$, pak
$$(x_1 = y_1 \to (x_2 = y_2 \to ( \dots (x_n = y_n \to p(x_1, \dots, x_n) = p(y_1, \dots, y_n)) \dots ))) $$
je axiom.

\section{Odvozovací pravidla predikátové logiky}

\subsection{Pravidlo odloučení (modus ponens}
Z formulí $\varphi, \varphi \to \psi$ se odvodí formule $\psi$.

\subsection{Pravidlo zobecnění (generalizace)}
Pro libovolnou proměnnou $x$ se z formule $\varphi$ odvodí formule $\forall x \varphi$.

%Spolu se schématy výrokových axiomů a pravidle odloučení přechází do predikátové logiky celá výroková logika.

\section{Dokazatelnost?}
\begin{veta}
(O korektnosti) Libovolná formule jazyka $L$ dokazatelná v predikátové logice 1. řádu je logicky platnou formulí, tj. je splněna v každé realizaci jazyka L.
\end{veta}

\begin{lemma}
(Pravidlo $\forall$) Je-li $\vdash \varphi \to \psi$ a proměnná $x$ nemá volný výskyt ve $\varphi$, pak $\vdash \varphi \to (\forall x \psi)$.
\end{lemma}

\begin{lemma}
(Pravidlo $\exists$) Je-li $\vdash \varphi \to \psi$ a proměnná $x$ nemá volný výskyt ve $\psi$, pak $\vdash (\exists x \varphi) \to \psi$.
\end{lemma}

\begin{lemma}
Je-li $\varphi$ formule, $x$ proměnná, $t$ term substituovatelný za $x$ do $\varphi$, pak $\vdash \varphi_x[t] \to (\exists x \varphi)$
\end{lemma}

\begin{lemma}
Nechť $\varphi'$ je instancí formule $\varphi$, tj. nechť $\varphi'$ je tvaru $\varphi_{x_1, \dots, x_n}[t_1, \dots t_n]$ pro nějaké termy $t_1, \dots, t_n$ substituovatelné za $x_1, \dots, x_n$ do $\varphi$. Jestliže $\vdash \varphi$, pak $\vdash \varphi'$.
\end{lemma}

\subsection{Uzávěr formule}
\begin{mydef}
Jsou-li $x_1, \dots, x_n$ všechny volné proměnné ve formuli $\varphi$ v nějakém pořadí, pak formuli $(\forall x_1 \dots \forall x_n \varphi$ nazveme uzávěrem formule $\varphi$.
\end{mydef}

\begin{veta}
(O uzávěru) Je-li $T$ množina formulí a $\varphi'$ uzávěr formule $\varphi$, pak $T \vdash \varphi$ právě když $T \vdash \varphi'$.
\end{veta}

\begin{lemma}
(Distribuce kvantifikátorů) Je-li $\vdash \varphi \to \psi$, potom $\vdash (\forall x \varphi ) \to (\forall x \psi), \vdash (\exists x \varphi) \to (\exists x \psi)$.
\end{lemma}

\begin{veta}
(O dedukci) Nechť $T$ je množina formulí jazyka L, nechť $\varphi$ je uzávřená formule, $\psi$ je libovolná formule jazyka $L$. Potom $T \vdash \varphi \to \psi$, právě když $T, \varphi \vdash \psi$.
\end{veta}

\begin{veta}
(O konstantách) Nechť $T$ je množina formulí jazyka $L$, nechť $\varphi$ je formule. Nechť $x_1, \dots, x_n$ jsou proměnné a nechť $c_1, \dots, c_n$ jsou nové konstanty, jejichž přidáním k $L$ vznikne jazyk $L'$. Potom $T \vdash \varphi_{x_1, \dots x_n}[c_1, \dots, c_n]$, právě když $T \vdash \varphi$.
\end{veta}

\begin{lemma}
Je-li $L$ jazyk s rovností, pak

$$ \vdash x = y \to y = x $$
$$ \vdash x = y \to (y = z \to x = z) $$
\end{lemma}

\begin{lemma}
Je-li $f$ funkční symbol četnosti $n$, je-li $p$ predikátová symbol četnosti $m$ a jsou-li $u$, $v$, $w$, $s_1, \dots, s_n$, $t_1, \dots, t_n$ termy jazyka $L$, pak
\begin{enumerate}[(i)]
	\item $\vdash u = u $
	\item $\vdash u = v \to v = u $
	\item $\vdash u = v \to (v = w \to u = w) $
	\item $\vdash s_1 = t_1 \to (s_2 = t_2 \to \dots (s_n = t_n \to f(s_1, \dots, s_n) = f(t_1, \dots, t_n)) \dots ) $
	\item $\vdash s_1 = t_1 \to (s_2 = t_2 \to \dots (s_n = t_n \to p(s_1, \dots, s_n) = p(t_1, \dots, t_n)) \dots ) $
\end{enumerate}
\end{lemma}

\section{Prenexní tvar formulí}
Základní tvar formulí.

\begin{comment}
\begin{lemma}
Buď $i_1, \dots, i_n$ libovolná permutace čísel $\{1, \dots, n\}$. Nechť $x_1, \dots, x_n$ jsou proměnné a $A$ formule predikátové logiky. Pak platí:
\begin{enumerate}
	\item $\vdash (\forall x_1) \dots (\forall x_n) A \leftrightarrow (\forall x_{i_1}) \dots (\forall x_{i_n}) A $
	\item $\vdash (\exists x_1) \dots (\exists x_n) A \leftrightarrow (\exists x_{i_1}) \dots (\exists x_{i_n}) A $
\end{enumerate}
\end{lemma}

\begin{veta}
Buď $A$ formule taková, že proměnné $x_1, \dots x_n$ jsou jediné proměnné s volným výskytem v $A$. Pak $\vdash A$, právě když $\vdash \forall x_1 \dots \forall x_n A$
\end{veta}

\begin{veta}
(O ekvivalenci) Nechť formule $A'$ vznikne s formule $A$ nahrazením některých výskytů podformulí $B_1 \dots B_n$ pro řadě formulemi $B'_1, \dots, B'_n$. Je-li $\vdash B_i \leftrightarrow B'_i$ pro všechna $i = 1, \dots, n$, pak platí $\vdash A \leftrightarrow A'$.
\end{veta}

\begin{veta}
Buďte $A, B$ formule a $x$ proměnná. Pak

$$ \vdash (\exists x) \lnot A \leftrightarrow \lnot (\forall x) A $$
$$ \vdash (\forall x) \lnot A \leftrightarrow \lnot (\exists x) A $$

Jestliže $x$ není volná ve formuli $A$ a $\circ$ značí některou z výrokových spojek $\land, \lor, \to$, pak platí

$$ \vdash \forall x (A \circ B) \leftrightarrow (A \circ \forall x B) $$
$$ \vdash \exists x (A \circ B) \leftrightarrow (A \circ \exists x B) $$

pro opačnou implikaci $B \to A$ platí:

$$ \vdash \forall x (B \to A) \leftrightarrow (\exists x B \to A)$$
$$ \vdash \exists x (B \to A) \leftrightarrow (\forall x B \to A)$$
\end{veta}
\end{comment}

\begin{mydef}
Nechť $A$ je formule predikátové logiky. Formule $A'$ je \emph{variantou} formule $A$, jestliže vznikne z $A$ postupným nahrazením podformulí tvaru $(Qx) B$ podformulemi $(Q y) B_x[y]$, kde $Q$ je obecný nebo existenční kvantifikátor a $y$ je proměnná, která není volná v $B$.
\end{mydef}

Důsledek: Je-li $A'$ variantou formule $A$, pak je dokazatelné, že obě formule jsou ekvivalentní: $\vdash A \leftrightarrow A'$

\begin{mydef}
Formule $A$ je v \emph{prenexním tvaru}, jestliže má tvar $Q_1 x_1 \dots Q_n x_n B$, kde
\begin{enumerate}[(i)]
	\item $n \geq 0$ a pro každé $i = 1, \dots n$ je $Q_i$ buď $\forall$ nebo $\exists$,
	\item $x_i, \dots, x_n$  jsou navzájem různé proměnné,
	\item $B$ je otevřená formule (neobsahuje kvantifikátory).
\end{enumerate}
\end{mydef}

\begin{veta}
Ke každé formuli $A$ lze sestrojit formuli $A'$ v prenexním tvaru tak, že $\vdash A \leftrightarrow A'$.
\end{veta}

\subsection{Převedení formule na prenexní tvar}
\begin{description}
	\item[Vyloučení zbytečných kvantifikátorů] vynecháme všechny kvantifikátory $\forall x$, resp. $\exists x$ v podformulích tvaru $\forall x B$ nebo $\exists x B$, pokud se proměnná $x$ nevyskytuje volně v $B$.

	\item[Přejmenování proměnných] Vyhledáme podformuli $Q x A$ nejvíce vlevo takovou, že proměnná $x$ se vyskytuje volně v $A$. Pokud $x$ má ještě další výskyt ve výchozí formuli, nahradíme podformuli $Q x A$ její variantou $Q x' A'$, kde $x'$ je proměnná různá od všech proměnných vyskytujících se v převáděné formuli. Tento proces opakujeme do té doby, až všechny kvantifikátory mají různé proměnné a žádná proměnná není v získané formule současně volná i vázaná (formule s čistými proměnnými).

	\item[Eliminace spojky $\leftrightarrow$] provede se podle následujícího schématu:
	$$ A \leftrightarrow B \dots (A \to B) \land (B \to A)$$

	\item[Přesun negace dovnitř] - provádíme postupně náhrady podformulí podle schémat
	\begin{eqnarray*}
	& \lnot (\forall x A)	\dots \exists x \lnot A & \\
	& \lnot (\exists x A)	\dots \forall x \lnot A & \\
	& \lnot (A \to B)		\dots A \land \lnot B & \\
	& \lnot (A \lor B)	\dots \lnot A \land \lnot B & \\
	& \lnot (A \land B)	\dots \lnot A \lor \lnot B & \\
	& \lnot (\lnot A )	\dots A &
	\end{eqnarray*}

	\item[Přesun kvantifikátoru doleva] pro podformuli $B$, ve které se nevyskytuje proměnná $x$, provádíme náhrady podle schémat
	\begin{eqnarray*}
	& (QxA) \lor B	\dots Qx(A \lor B) &\\
	& (QxA) \land B	\dots Qx(A \land B) &\\
	& (QxA) \to B		\dots \bar{Q}x(A \to B) &\\
	& B \to (QxA)		\dots Qx(B \to A) &\\
	& (\exists x A) \lor (\exists y B) 	\dots \exists x (A \lor B_y[x]) &\\
	& (\forall x A) \land (\forall y B) 	\dots \forall x (A \land B_y[x]) &
	\end{eqnarray*}
	kde $\bar{Q}$ je kvantifikátor "opačný" ke $Q$. N+kde lze snížit počet kvantifikátorů pomocí schémat
\end{description}

\section{Věta o úplnosti}

\begin{mydef}
Je-li $L$ jazyk 1. řádu a $T$ množina formulí jazyka $L$, říkáme, že $T$ je \emph{teorie 1. řádu} s jazykem L.
\end{mydef}

\begin{mydef}
Říkáme, že teorie je \emph{sporná}, jestliže pro každou formuli $\varphi$ jazyka $L$ platí $T \vdash \varphi$. V opačném případě je teorie \emph{bezesporná}.
\end{mydef}
(tedy platí $T \vdash \varphi$ a zároveň $T \vdash \lnot \varphi$)

Důsledek: Nechť $T$ je množina formulí a nechť $\varphi'$ je uzávěr formule $\varphi$. Potom $T \vdash \varphi$, právě když $T \cup \{\lnot \varphi'\}$ je sporná teorie.

\subsection{Model a důsledek teorie}

\begin{mydef}
Buď $T$ teorie s jazykem $L$ a nechť $\mathcal{M}$ je nějaká realizace jazyka $L$. Řekněme, že $\mathcal{M}$ je model teorie $T$, jestliže $\mathcal{M} \models \varphi$ pro každou formuli $\varphi \in T$. Pak píšeme $\mathcal{M} \models T$.
\end{mydef}

\begin{mydef}
Řekneme, že formule $\varphi$ je \emph{důsledkem teorie $T$}, jestliže pro každý model teorie $\mathcal{M}$ teorie $T$ je $\mathcal{M} \models \varphi$. Pak píšeme $T \models \varphi$.
\end{mydef}

\begin{veta}
(O korektnosti) Je-li teorie s jazykem $L$ a $\varphi$ formule taková, že $T \vdash \varphi$, pak $T \models \varphi$.
\end{veta}

Důsledek: Má-li teorie $T$ s jazykem $L$ nějaký model, potom je bezesporná.

\begin{veta}
(Gödelova věta o úplnosti) Je-li $T$ teorie s jazykem $L$ a je-li $\varphi$ libovolná formule jazyka $L$, pak $T \vdash \varphi$ právě když $T \models \varphi$.
\end{veta}

\begin{veta}
(Gödelova věta o úplnosti) Teorie $T$ je bezesporná, právě když má nějaký model.
\end{veta}

\begin{mydef}
Řekneme, že teorie $T$ s jazykem L je \emph{úplná}, jestliže $T$ je bezesporná a pro každou uzavřenou formuli $\varphi$ platí $T \vdash \varphi$ nebo $T \vdash \lnot \varphi$ (v důsledku bezespornosti nemůže platit $T \vdash \varphi$ i $T \vdash \lnot \varphi$ současně), V opačném případě říkáme, že $T$ je \emph{neúplná}.
\end{mydef}

\begin{mydef}
Řekneme, že teorie $T$ s jazykem $L$ je \emph{Henkinova}, jestliže pro libovolnou uzavřenou formuli tvaru $\exists x \psi$ jazyka $L$ existuje konstanta $c$ jazyka $L$ taková, že $T \vdash (\exists x \psi) \to \psi_x[x]$.
\end{mydef}

\begin{lemma}
Libovolná Henkinova teorie má model.
\end{lemma}

\begin{mydef}
Jazyk $L'$ je rozšířením jazyka $L$ jestliže každý speciální symbol jazyka $L$ je obsažen v jazyce $L'$. Teorie $T'$ jazyka $L'$ je rozšířením teorie $T$ jazyka $L$, jestliže pro libovolnou formuli $\varphi$ jazyka $L$ takovou, že $T \vdash \varphi$, je také $T' \vdash \varphi$. Teorie $T'$ je konzervativním rozšířením teorie $T$, jestliže  navíc pro každou formuli $\psi$ jazyka $L$ takovou, že $T' \vdash \psi$, je již $T \vdash \psi$.
\end{mydef}

\begin{lemma}
(Heinkin) K libovolné teorii lze sestrojit Heinkinovu teorii $T_H$, která je konzervativním rozšířením teorie $T$.
\end{lemma}

\begin{lemma}
Teorie $T$ je bezesporná, právě když každá její konečná podmnožina $Q \subseteq T$ je bezesporná.
\end{lemma}

\begin{veta}
(Lindenbaum) Je-li $T$ bezesporná teorie s jazykem $L$, pak existuje rozšíření $T'$ teorie $T$ se stejným jazykem $L$.
\end{veta}

\section{Věta o kompaktnosti a věta Herbrandova}

\begin{veta}
(O kompaktnosti) Nechť $T$ je množina formulí jazyka $L$. Pak teorie $T$ má nějaký model, právě když každá její konečná podmnožina $Q \subseteq T$ má model.
\end{veta}

\begin{veta}
(Löwenheim,Skolem) Má-li teorie $T$ s jazykem $L$ nekonečný model, pak má model libovolné mohutnosti $n \geq max\{\aleph_0, |L|\}$
\end{veta}

??? more?





%%%%%%%%%%%%%%%%%%%%%%%%%%%%%%%%%%%%%%%%%%%%%%%%%%%%%%%%%%%%%%%%%%%%%%%%%%%%%%%%
%%%%%%%%%%%%%%%%%%%%%%%%%%%%%%%%%%%%%%%%%%%%%%%%%%%%%%%%%%%%%%%%%%%%%%%%%%%%%%%%
\chapter{Algebraické struktury (grupy, okruhy, obory integrity a tělesa, svazy a Boolovy algebry, univerzální algebry).} \label{cha:8}
\chapter{Základní algebraické metody (podalgebry, homomorfismy, přímé součiny, kongruence a faktorové algebry, normální podgrupy a ideály okruhů).} \label{cha:9}
\chapter{Obory integrity a dělitelnost (okruhy polynomů, pravidla dělitelnosti, Gaussovy a Eukleidovy okruhy).} \label{cha:10}
\chapter{Teorie polí (minimální pole, rozšíření pole, konečná pole a jejich konstrukce).} \label{cha:11}
\chapter{Metrické prostory (příklady, konvergence posloupností, spojitá a izometrická zobrazení, úplnost, Banachova věta o pevném bodu).} \label{cha:12}
\chapter{Normované a unitární prostory (základní vlastnosti a příklady, normované prostory konečné dimenze, uzavřené ortonormální systémy a Fourierovy řady).} \label{cha:13}
\chapter{Obyčejné grafy (stupně uzlů, sledy, souvislost, izomorfismy, stromy, kostry, Kruskalův a Primův algoritmus pro hledání minimální kostry ohodnoceného grafu, eulerovské a hamiltonovské grafy, planarita a obarvitelnost).} \label{cha:14}
\chapter{Orientované grafy (orientované sledy, souvislost a silná souvislost, turnaje, eulerovské a hamiltonovské grafy, Dijkstrův a Floyd-Warshallův algoritmus pro hledání cesty minimální délky).} \label{cha:15}
\chapter{Klasifikace gramatik, formálních jazyků a automatů přijímajících jazyky.} \label{cha:16}
\chapter{Vlastnosti formálních jazyků (typické vlastnosti a jejich rozhodnutelnost).} \label{cha:17}
\chapter{Konečné automaty (jazyky přijímané jazyky KA, varianty KA, minimalizace KA).} \label{cha:18}
\chapter{Regulární množiny, regulární výrazy a rovnice nad regulárními výrazy.} \label{cha:19}
\chapter{Transformace a normální formy bezkontextových gramatik.} \label{cha:20}
\chapter{Zásobníkové automaty (jazyky přijímané ZA, varianty ZA).} \label{cha:21}
\chapter{Turingovy stroje (jazyky přijímané TS, varianty TS, lineárně omezené automaty, univerzální TS).} \label{cha:22}
\chapter{Nerozhodnutelnost (problém zastavení TS, princip diagonalizace a redukce, Postův korespondenční problém).} \label{cha:23}
\chapter{Parciální rekurzivní funkce.} \label{cha:24}
\chapter{Časová a paměťová složitost (třídy složitosti, úplnost, SAT problém).} \label{cha:25}
\chapter{Ukazatele a zákony paralelního zpracování. Funkce konst. účinnosti a škálovatelnost.} \label{cha:26}
\chapter{Paralelizace programů: vzory programových a datových struktur, podpůrné struktury.} \label{cha:27}
\chapter{Paralelní zpracování v OpenMP, SPMD, smyčky, sekce a tasky. Synchronizační prostředky.} \label{cha:28}
\chapter{Architektury se sdílenou pamětí, UMA i NUMA, zajištění koherence pamětí cache.} \label{cha:29}
\chapter{Architektury distribuovaných systémů se zasílání zpráv.} \label{cha:30}
\chapter{Blokující a neblokující párové (point-to-point) komunikace v MPI.} \label{cha:31}
\chapter{Kolektivní komunikace v MPI, paralelní vstup a výstup.} \label{cha:32}
\chapter{Propojovací sítě: Topologie a směrovací algoritmu, přepínání a řízení toku.} \label{cha:33}
\chapter{Klasifikace metod komprese dat (ztrátové, bezeztrátové, intuitivní, algoritmické, četnost výskytu, pravděpodobnost výskytu), princip kódování délek sledů, kódování „přesuň na začátek.‘‘} \label{cha:34}
\chapter{Kódy s proměnnou délkou - princip, zdůvodnění, Huffmanovy kódy - různé typy, kanonický Huffmanův kód, adaptivní Huffmanův kód, aritmetický kód.} \label{cha:35}
\chapter{Slovníkové metody (LZ77, LZ78, práce se slovníkem, pohyblivé okno, prodlužování položek).} \label{cha:36}
\chapter{Informace a entropie, Shannova věta o kódování.} \label{cha:37}
\chapter{Bezpečnostní kódy: lineární, Hammingovy, cyklické, konvoluční. Detekce a oprava chyb.} \label{cha:38}
\chapter{Základní architektury přepínačů, algoritmy pro plánování, řešení blokování, vícestupňové přepínací sítě.} \label{cha:39}
\chapter{Základní funkce směrovače, zpracování paketů ve směrovači, typy architektur.} \label{cha:40}
\chapter{Sítě Peer-to-Peer (P2P), Milgramův problém malého světa, model sítě P2P, směrování v P2P sítích, strukturované a nestrukturované sítě.} \label{cha:41}
\chapter{Základní principy softwarově definovaných sítí SDN, architektura, technologie OpenFlow.} \label{cha:42}
\chapter{Formální metody v počítačových sítích.} \label{cha:43}
\chapter{Konflikty a závislosti při řetězovém zpracování instrukcí a jejich HW/SW ošetření.} \label{cha:44}
\chapter{Architektura superskalárních procesorů a algoritmy OOO zpracování instrukcí.} \label{cha:45}
\chapter{Procesory VLIW a používané optimalizační techniky s HW podporou.} \label{cha:46}
\chapter{Multivláknové procesory, hrubý, jemný a simultánní MT.} \label{cha:47}
\chapter{Datový paralelismus SIMD a SIMT, HW implementace a SW podpora.} \label{cha:48}
\chapter{Architektura grafických procesorů, odlišnosti od superskalárních procesorů.} \label{cha:49}
\chapter{Programovací jazyk CUDA, model vláken a paměťový model.} \label{cha:50}
\chapter{Základní rysy nízkopříkonových procesorů (požadavky, architektura, výkonnost).} \label{cha:51}
\chapter{Jazyky pro popis obvodů (VHDL, syntetizovatelné konstrukce).} \label{cha:52}
\chapter{Logická syntéza obvodů (návrh pro technologie FPGA a ASIC, fáze syntézy, optimalizace, mapování, techniky zřetězení a vyvážení).} \label{cha:53}
\chapter{Moderní přístupy k syntéze číslicových obvodů (reprezentace obvodu pomocí AIG, techniky odstraňování funkční redundance v AIG, tradiční mapování AIG do LUT).} \label{cha:54}
\chapter{Aplikace omezujících podmínek (časová a fyzická omezení).} \label{cha:55}
\chapter{Verifikace číslicových obvodů (metodologie OVM).} \label{cha:56}
\chapter{Technologie programovatelného hardware (architektura FPGA, struktura konfigurovatelných bloků a vestavěných bloků, propojovací sít, způsoby konfigurace, srovnání s technologií ASIC).} \label{cha:57}
\chapter{Vestavěný počítačový systém (shody a odlišnosti s běžným univerzálním počítačovým systémem).} \label{cha:58}
\chapter{Implementace funkcí vestavěného systému SW a HW prostředky (výhody a nevýhody - dopady SW a HW implementace konkrétní funkce na vlastnosti systému, příklad).} \label{cha:59}
\chapter{Číslicové vstupy a výstupy vestavěných systémů (problémy a jejich řešení, přizpůsobení napěťových úrovní, snímání stavu mechanického kontaktu, ovládání zátěže, posílení výstupu, H-můstek).} \label{cha:60}
\chapter{Architektura SW pro vestavěné systémy (hlavní smyčka, implementace stavového automatu, obsluha přerušení).} \label{cha:61}
\chapter{Konstrukce adaptéru systémové sběrnice: návrh adresového dekodérů, obsluha čtecí a zápisové transakce.} \label{cha:62}
\chapter{Architektura sběrnice PCI-Express a USB: typy transakcí, způsob komunikace a směrování transakcí, detekce chyb a způsob zotavení.} \label{cha:63}

% 2. semestr, IPR, zdroje % XXX


\end{document}

