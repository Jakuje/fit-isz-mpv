\documentclass[a4paper, 11pt]{report}
\usepackage[czech]{babel}
\usepackage[utf8]{inputenc}
\usepackage{multirow}
\usepackage{amsmath}
\usepackage{amsfonts}
\usepackage{enumerate}
\usepackage{verbatim}
\usepackage{tikz-qtree}
\usepackage{tikz}
\usetikzlibrary{automata,positioning,fit}
\usepackage{mathtools}

\usepackage{amsthm}
\newtheorem{mydef}{Definice}[chapter]
\newtheorem{veta}{Věta}[chapter]
\newtheorem{lemma}{Lemma}[chapter]
\newtheorem{alg}{Algoritmus}[chapter]

\usepackage{geometry}
\usepackage{layout}

\geometry{
  includeheadfoot,
  hmargin=2.0cm,
  vmargin={0cm, 1.0cm}
}

\usepackage{color}
\usepackage[unicode,colorlinks,hyperindex,plainpages=false,pdftex]{hyperref}

\usepackage{listings}  
\definecolor{mygreen}{rgb}{0,0.6,0}
\lstset{language=VHDL,commentstyle=\color{mygreen},tabsize=4}

\usepackage{fancyhdr}
\pagestyle{fancyplain}
\fancyhf{}
\renewcommand{\headrulewidth}{0pt}

\cfoot{\hfill © Jakuje \hfill \thepage }


\begin{document}

\ref{cha:1}
\ref{cha:2}
\ref{cha:3}
\ref{cha:4}
\ref{cha:5}
\ref{cha:6}
\ref{cha:7}
\ref{cha:8}
\ref{cha:9}
\ref{cha:10}
\ref{cha:11}
\ref{cha:12}
\ref{cha:13}
\ref{cha:14}
\ref{cha:15}
\ref{cha:16}
\ref{cha:17}
\ref{cha:18}
\ref{cha:19}
\ref{cha:20}

\ref{cha:21}
\ref{cha:22}
\ref{cha:23}
\ref{cha:24}
\ref{cha:25}
\ref{cha:26}
\ref{cha:27}
\ref{cha:28}
\ref{cha:29}
\ref{cha:30}
\ref{cha:31}
\ref{cha:32}
\ref{cha:33}
\ref{cha:34}
\ref{cha:35}
\ref{cha:36}
\ref{cha:37}
\ref{cha:38}
\ref{cha:39}
\ref{cha:40}

\ref{cha:41}
\ref{cha:42}
\ref{cha:43}
\ref{cha:44}
\ref{cha:45}
\ref{cha:46}
\ref{cha:47}
\ref{cha:48}
\ref{cha:49}
\ref{cha:50}
\ref{cha:51}
\ref{cha:52}
\ref{cha:53}
\ref{cha:54}
\ref{cha:55}
\ref{cha:56}
\ref{cha:57}
\ref{cha:58}
\ref{cha:59}
\ref{cha:60}

\ref{cha:61}
\ref{cha:62}
\ref{cha:63}
\newpage

\tableofcontents

%%%%%%%%%%%%%%%%%%%%%%%%%%%%%%%%%%%%%%%%%%%%%%%%%%%%%%%%%%%%%%%%%%%%%%%%%%%%%%%%
%%%%%%%%%%%%%%%%%%%%%%%%%%%%%%%%%%%%%%%%%%%%%%%%%%%%%%%%%%%%%%%%%%%%%%%%%%%%%%%%
\chapter{Metodika návrhu HW/SW codesign, platformy, programovatelné obvody.} \label{cha:1}
X. semestr, XXX, ??
\chapter{Výpočetní modely} \label{cha:2}
X. semestr, XXX, ??

(StateCharts, Kahnova síť procesů, synchronní dataflow)
\chapter{Specifikace (chování, struktura), syntéza (alokace, přidělení, plánování) a integrace systémů (rozhraní, synchronizace, komunikace).} \label{cha:3}
X. semestr, XXX, ??
\chapter{Syntéza HW z vyšších programovacích jazyků (reprezentace, alokace, plánování, přiřazení) a jazyk Catapult C.} \label{cha:4}
X. semestr, XXX, ??
\chapter{Odhady (přesnost, věrnost, metriky, metody) a optimalizace vlastností systému (příkon, energie).} \label{cha:5}

%%%%%%%%%%%%%%%%%%%%%%%%%%%%%%%%%%%%%%%%%%%%%%%%%%%%%%%%%%%%%%%%%%%%%%%%%%%%%%%%
%%%%%%%%%%%%%%%%%%%%%%%%%%%%%%%%%%%%%%%%%%%%%%%%%%%%%%%%%%%%%%%%%%%%%%%%%%%%%%%%
\chapter{Jazyk a sémantika predikátové logiky} \label{cha:6}
1. semestr, MAT, \texttt{logikaaktual3.pdf}, 3., 4. kapitola

(termy, formule, realizace jazyka, pravdivost formulí)

\section{Jazyk predikátové logiky}

\begin{itemize}
	\item Logické symboly
	\begin{itemize}
		\item proměnné: $x, y, z, \dots, x_1, x_2, \dots$
		\item logické spojky: $ \lnot, \land, \lor, \to, \leftrightarrow$
		\item kvantifikátory: $\exists, \forall$
		\item závorky, čárka: $(,)$
		\item predikátový symbol rovnosti $=$
	\end{itemize}
	\item Speciální symboly
	\begin{itemize}
		\item funkční symboly $f, g, h, \dots, f_1, f_2, f_3$, nezáporné celé číslo -- četnost
		\item predikátové symboly $p, q, r, \dots, p_1, p_2, \dots$, kladné celé číslo -- jeho četnost.
	\end{itemize}
\end{itemize}

Obsahuje-li jazyk symbol $=$ pro rovnost, mluvíme o \emph{jazyku s rovností}.
Specifiku jazyka určují jeho funkční a predikátové symboly (určující oblast kterou jazyk popisuje)

\subsection{Termy}
\begin{enumerate}[(i)]
	\item Každá proměnná je term
	\item Je-li $f$ funkční symbol s četností $n$ a jsou-li $t_1, \dots, t_n$ termy, pak také $f(t_1, \dots, t_n)$ je term
	\item Každý term vznikne konečným počtem užití (i), (ii)
\end{enumerate}
(ze (ii) plyne, že každá konstanta je term)

\subsection{Atomické formule}
Je-li $p$ predikátový symbol s četností $n$ a jsou-li $t_1, \dots t_n$ termy, pak $p(t_1, \dots, t_n)$ je \emph{atomická formule}.

Speciální, máme-li jazyk s rovností a jsou-li $t_1, t_2$ termy, pak $(t_1 = t_2)$ je atomická formule.
Píšeme $(t_1 = t_2)$ místo $=(t_1, t_2)$. Podobný zápis používáme i pro jiné binární predikátové operátory, např. místo $< (t_1, t_2)$ píšeme $(t_1 < t_2)$.

\subsection{Formule}
\begin{enumerate}[(i)]
	\item Každá atomická formule je formule
	\item Jsou-li $\varphi, \psi$ formule, pak také $(\lnot \varphi), (\varphi \land \psi), (\varphi \lor \psi), (\varphi \to \psi), (\varphi \leftrightarrow \psi)$ jsou formule.
	\item Je-li $x$ proměnná a $\varphi$ formule, pak také $(\forall x \varphi)$, $(\exists x \varphi)$ jsou formule.
	\item Každá formule vznikne konečným počtem užití (i), (ii), (iii)
\end{enumerate}

\tikzset{every tree node/.style={align=center,anchor=north}}
\Tree[.{Formule}
	[.{Atomická formule}
		{Predikátový symbol\\ + Term\\ $p(t_1, \dots, t_n)$}
		[.{Term}
			{Proměnná\\ $x$}
			{Funkční symbol\\ + Term\\ $f(t_1, \dots, t_n)$}
			]
		]
	{Logické spojky\\ + Formule\\ $\varphi \land \psi$}
	{Kvantifikátory\\ + Proměnné\\ + Formule\\ $\forall x \varphi$}
]

\begin{description}
	\item[Vázaný výskyt proměnné] nachází-li se v nějaké podformuli tvaru $\forall x \varphi$ nebo $\exists x \varphi$.
	\item[Obor kvantifikátoru] $\varphi$
	\item[Volný výskyt proměnné] není vázaný
	\item[Volná (Vázaná) proměnná] existuje-li volný (vázaný) výskyt proměnné v této formuli
	\item[Uzavřená formule (Výrok)] Formule neobsahující žádnou volnou proměnnou
	\item[Otevřená formule (Výrok)] Formule neobsahující žádnou vázanou proměnnou
	\item[Formule s čistými proměnnými] Otevřené a uzavřené formule
\end{description}

\section{Sémantika predikátové logiky}

\begin{mydef}
Nechť $L$ je jazyk 1. řádu. \emph{Realizací jazyka} $L$ rozumíme algebraickou strukturu $\mathcal{M}$, která se skládá z
\begin{enumerate}[(i)]
	\item neprázdné množiny $M$, kterou nazveme \emph{univerzum}
	\item pro každý funkční symbol $f$ četností $n$ je dáno zobrazení $f_\mathcal{M} : M^n \to M$
	\item pro každý predikátový symbol $p$ četnosti $n$, kromě rovnosti, je dána relace $p_\mathcal{M} \subset M^n$
\end{enumerate}
Poznamenejme, že pro nulární funkční symbol, tj. pro konstantu, je $M^0 = \{0\}$ a příslušné zobrazení $M^0 \to M$ lze chápat jako vyznačení určitého prvku z $M$ odpovídajícího daného konstantě.
\end{mydef}

\paragraph{Ohodnocení proměnných}: Libovolné zobrazení $e$ množiny všech proměnných do univerza $M$ dané realizace $\mathcal{M}$ jazyka $L$. Pokud proměnné $x$ přiřazuje prvek $m$, budeme značit $e(x/m)$.

\begin{mydef}
\emph{Hodnota termu} $t$ v realizaci $\mathcal{M}$ jazyka $L$ při daném ohodnocení $e$ proměnných, označovaná $t[e]$, se definuje indukcí následovně:
\begin{enumerate}[(i)]
	\item Je-li $t$ proměnná $x$, potom $t[e]$ je $e(x)$
	\item je-li $t$ term tvaru $f(t_1, \dots, t_n)$, kde $f$ je funkční symbol četnosti $n$ a $t_1, \dots, t_n$ jsou termy, potom $t[e]$ je $f_\mathcal{M}(t_1[e], \dots, t_n[e])$
\end{enumerate}

\end{mydef}
\begin{mydef}
Nechť $\mathcal{M}$ je realizace jazyka $L$, nechť $e$ je ohodnocení proměnných a nechť $\varphi$ je formule jazyka $L$. Indukcí podle složitosti formule $\varphi$ definujeme, co znamená, že \emph{formule $\varphi$ je pravdivá v $\mathcal{M}$ při ohodnocení $e$}. Tuto skutečnost budeme značit $\mathcal{M} \models \varphi[e]$
\begin{enumerate}[(i)]
	\item Je-li $\varphi$ atomická formule tvaru $p(t_1, \dots, t_n)$, kde $p$ je predikátový symbol četnosti $n$ a $t_1, \dots, t_n$ jsou termy, pak $\mathcal{M} \models \varphi[e]$ právě když $(t_1[e], \dots, t_n[e]) \in p_\mathcal{M}$.
	\item Je-li $\varphi$ atomická formule tvaru $t_1 = t_2$, kde $t_1, t_2$ jsou termy, pak $\mathcal{M} \models \varphi[e]$ právě když $t_1[e]$ je tentýž prvek jako $t_2[e]$ v $M$
	\item Je-li $\varphi$ tvaru $\lnot \psi$, kde je $\psi$ je formule jazyka $L$, pak $\mathcal{M} \models \varphi[e]$ je právě když $\mathcal{M} \not\models \psi[e]$.
	\item Je-li $\varphi$ některého z tvarů $(\eta \land \psi), (\eta \lor \psi), (\eta \to \psi), (\eta \leftrightarrow \psi)$, kde $\eta$, $\psi$ jsou formule, klademe: \\
	$\mathcal{M} \models (\eta \land \psi)[e]$ právě když současně $\mathcal{M} \models \eta[e]$ a $\mathcal{M} \models \psi[e]$. \\
	$\mathcal{M} \models (\eta \lor \psi)[e]$ právě platí alespoň jedno z $\mathcal{M} \models \eta[e]$ a $\mathcal{M} \models \psi[e]$ a podobně další logické spojky.
	\item Je-li $\varphi$ tvaru $(\forall x \psi)$, kde $\psi$ je formule jazyka $L$, pak $\mathcal{M} \models \varphi[e]$ právě když pro každý prvek $m \in M$ je $\mathcal{M} \models \psi[e(x/m)]$.
	\item Je-li $\varphi$ tvaru $(\exists x \psi)$, kde $\psi$ je formule jazyka $L$, pak $\mathcal{M} \models \varphi[e]$ právě když existuje $m \in M$ taková, že $\mathcal{M} \models \psi[e(x/m)]$.
\end{enumerate}
\end{mydef}

\begin{comment}
\section{Ostatní}

Formalizovaná axiomatická teorie je dána
\begin{itemize}
	\item symboly -- tvoří abecedu
	\item formulemi -- určitá slova této abecedy, která tvoří jazyk této teorie
	\item axiomy -- výchozí tvrzení této teorie zapsaná pomocí abecedy jako jisté formule
	\item odvozovací pravidla --  pravidla pro manipulaci s formulemi, pomocí kterých odvozujeme z axiomů důsledky.
\end{itemize}

\begin{description}
	\item[Proměnné] Označení libovolného prvku z daného oboru $(x, y, z, \dots, x_1, x_2, \dots)$
	\item[Konstanty] Význačné objekty $(0, 1, \dots)$
	\item[Funkční symboly] Operace $(f, g, h, \dots, f_1, f_2, f_3)$
	\item[Četnost funkčního symbolu] počet argumentů dané operace
	\item[Predikáty] Vlastnosti a vztahy mezi objekty
	\item[Predikátové symboly] Vyjádření predikátů? $(p, q, r, \dots, p_1, p_2, \dots)$.
	\item[Četnost predikátového symbolu] počet argumentů predikátu
	\item[Atomické formule] Nejjednodušší tvrzení, složená z \emph{Proměnných}, \emph{konstant}, \emph{funkčních symbolů}, \emph{predikátových symbolů}.
	\item[Složitější formule] Atomické formule + logické spojky + kvantifikace proměnných ($\forall, \exists$)
	\item[Abeceda predikátové logiky 1. řádu] výše uvedené symboly s logickými spojkami a pomocnými symboly (závorky, čárka).
\end{description}


\section{Predikátová logika 1. řádu}

Matematické teorie pracují s celými soubory objektů (čísla, body v prostoru, prvky algebraických struktur).
Pro označení lib. prvků z daného oboru používáme \emph{proměnné} $(x, y, z, \dots, x_1, x_2, \dots)$

Mezi prvky z daného oboru mohou být některé význačné objekty (0, neutrální prvek grupy, \dots), pro než užíváme zvláštní symboly -- \emph{konstanty} (např. 0, 1, \dots).

S objekty daného oboru lze provádět různé operace (sčítání a násobení čísel, násobení v grupách, \dots).
K označení operace užíváme \emph{funkční symboly} $(f, g, h, \dots, f_1, f_2, f_3)$.
Ke každému funkčnímu symbolu je přiřazeno přirozené číslo, které vyjadřuje jeho \emph{četnost}, tj. počet argumentů dané operace.
Je-li četnost symbolu rovna $n$, říkáme, že symbol je $n$-ární.
Je přirozené chápat konstanty jako nulární funkční symboly.

Matematika zkoumá vlastnosti objektů a vztahy mezi objekty.
Vlastnosti  a vztahy mezi objekty daného oboru, tzv. \emph{predikáty} (\uv{být záporným číslem} (vlastnost), \uv{být menší než}, \uv{být prvkem} (vztahy)) vyjadřujeme pomocí \emph{predikátových symbolů} $(p, q, r, \dots, p_1, p_2, \dots)$.
Predikát znamená vztah mezi užitým počtem objektů.
Tím je každému predikátovému symbolu přiřazeno přirozené číslo, jeho četnost udávající počet jeho argumentů.
Je-li četnost rovna $n$, říkáme, že symbol je \emph{n-ární}.
V mnoha případech používáme zvláštní označení = pro binární predikátový symbol označující rovnost, tj. totožnost objektů z daného oboru.

Z proměnných, konstant, funkčních symbolů a predikátových symbolů sestavujeme jistým způsobem nejjednodušší tvrzení, vyjádřená tzv. \emph{atomickými formulemi}.
Z nich vytváříme složitější formule pomocí \emph{logických spojek} (stejných jako ve výrokové logice) a pomocí \emph{kvantifikace proměnných}.\\
\emph{Univerzální (obecný) kvantifikátor} $\forall$ vyjadřuje platnost pro všechny objekty z daného oboru.\\
\emph{Existenční kvantifikátor} $\exists$ vyjadřuje existenci požadovaného objektu v daném oboru.

Uvedené symboly spolu s logickými spojkami a \emph{pomocnými symboly} (závorka, čárka) tvoří abecedu jazyka \emph{predikátové logiky 1. řádu}.
Proměnné jazyka prvního řádu jsou obecná jména pro objekty daného oboru, tj. pro individua (např. čísla).
Jazyk neobsahuje proměnné pro množiny individuí (např. množiny čísel, relací, \dots), vyšších řádů, které dovolují kvantifikovat např. množiny, relace.

%%%%%%%%%%%%%%%%%%%%%%%%%%%%%%%%%%
\section{Jazyk predikátové logiky}

\begin{itemize}
	\item Logické symboly
	\begin{itemize}
		\item proměnné: $x, y, z, \dots, x_1, x_2, \dots$
		\item logické spojky: $ \lnot, \land, \lor, \to, \leftrightarrow$
		\item kvantifikátory: $\exists, \forall$
		\item závorky, čárka: $(,)$
		\item predikátový symbol rovnosti $=$
	\end{itemize}
	\item Speciální symboly
	\begin{itemize}
		\item funkční symboly $f, g, h, \dots, f_1, f_2, f_3$, nezáporné celé číslo -- četnost
		\item predikátové symboly $p, q, r, \dots, p_1, p_2, \dots$, kladné celé číslo -- jeho četnost.
	\end{itemize}
\end{itemize}

Obsahuje-li jazyk symbol $=$ pro rovnost, mluvíme o \emph{jazyku s rovností}.
Specifiku jazyka určují jeho funkční a predikátové symboly (určující oblast kterou jazyk popisuje)

\subsection{Termy}
\begin{enumerate}[(i)]
	\item Každá proměnná je term
	\item Je-li $f$ funkční symbol s četností $n$ a jsou-li $t_1, \dots, t_n$ termy, pak také $f(t_1, \dots, t_n)$ je term
	\item Každý term vznikne konečným počtem užití (i), (ii)
\end{enumerate}
(ze (ii) plyne, že každá konstanta je term)

\subsection{Atomické formule}
Je-li $p$ predikátový symbol s četností $n$ a jsou-li $t_1, \dots t_n$ termy, pak $p(t_1, \dots, t_n)$ je \emph{atomická formule}.

Speciální, máme-li jazyk s rovností a jsou-li $t_1, t_2$ termy, pak $(t_1 = t_2)$ je atomická formule.
Píšeme $(t_1 = t_2)$ místo $=(t_1, t_2)$. Podobný zápis používáme i pro jiné binární predikátové operátory, např. místo $< (t_1, t_2)$ píšeme $(t_1 < t_2)$.

\subsection{Formule}
\begin{enumerate}[(i)]
	\item Každá atomická formule je formule
	\item Jsou-li $\varphi, \psi$ formule, pak také $(\lnot \varphi), (\varphi \land \psi), (\varphi \lor \psi), (\varphi \to \psi), (\varphi \leftrightarrow \psi)$ jsou formule.
	\item Je-li $x$ proměnná a $\varphi$ formule, pak také $(\forall x \varphi)$, $(\exists x \varphi)$ jsou formule.
	\item Každá formule vznikne konečným počtem užití (i), (ii), (iii)
\end{enumerate}

Poznamenejme, že píšeme $x \not= y$ místo $\lnot (x, y)$ a také, pokud to nemůže narušit srozumitelnost, vynecháváme některé dvojice závorek.

Při tvorbě formule $\varphi$ podle předchozí definice vytváříme určitou posloupnost formulí, která začíná atomickými formulemi a končí formulí $\varphi$ a každá formule v této posloupnosti vzniká z některých předcházejících pomocí logických spojek a kvantifikátorů. Každá z těchto formulí se nazývá podformule $\varphi$.

Každá formule je konečnou posloupností symbolů. Každý symbol, zejména každá proměnná, se může ve formuli vyskytovat na jednom nebo více místech.
Řekněme, že daný \emph{výskyt} proměnné $x$ ve formuli $\varphi$ je \emph{vázaný}, nenachází-li se v nějaké podformuli tvaru $\forall x \psi$ nebo $\exists x \psi$. V tomto případě se proměnná $x$ vyskytuje v kvantifikátoru samém nebo ve formuli $\psi$ (podformule $\psi$ se nazývá \emph{obor kvantifikátoru} $\forall x$ nebo $\exists x$.
V opačném případě (výskyt není vázaný) řekneme, že daný výskyt proměnné $x$ ve formuli $\phi$ je \emph{volný}.
Proměnná $x$ se nazývá \emph{volnou (vázanou) proměnnou} ve formuli $\phi$, existuje-li její volný (vázáný) výskyt v této formuli.
Proměnná tedy může být ve formuli volná i vázaná. Formule neobsahující žádnou volnou proměnnou se nazývá \emph{uzavřená formule} nebo též \emph{výrok}.
Naopak, formule neobsahující žádnou vázanou proměnnou se nazývá \emph{otevřenou formulí}. Uzavření a otevřené formule nazýváme \emph{formulemi s čistými proměnnými}.

%%%%%%%%%%%%%%%%%%%%%%%%%%%%%%%%%%%%%%
\section{Sémantika predikátové logiky}

Chceme dát interpretaci symbolům jazyka predikátové logiky 1. řádu.
Nejprve vymezíme obor, který budeme určovat možné hodnoty proměnných, bude to určitý soubor $M$ uvažovaných objektů.
Funkčním symbolům budou odpovídat operace na $M$ příslušných četností.
Predikátovým symbolům budou odpovídat vztahy mezi objekty z $M$, které lze popsat jako relace na $M$ příslušných četností. Máme-li jazyk s rovností, interpretujeme symbol $=$ jako rovnost objektů z $M$.

\begin{mydef}
Nechť $L$ je jazyk 1. řádu. \emph{Realizací jazyka} $L$ rozumíme algebraickou strukturu $\mathcal{M}$, která se skládá z
\begin{enumerate}[(i)]
	\item neprázdné množiny $M$, kterou nazveme \emph{univerzum}
	\item pro každý funkční symbol $f$ četností $n$ je dáno zobrazení $f_\mathcal{M} : M^n \to M$
	\item pro každý predikátový symbol $p$ četnosti $n$, kromě rovnosti, je dána relace $p_\mathcal{M} \subset M^n$
\end{enumerate}
\end{mydef}

Poznamenejme, že pro nulární funkční symbol, tj. pro konstantu, je $M^0 = \{0\}$ a příslušné zobrazení $M^0 \to M$ lze chápat jako vyznačení určitého prvku z $M$ odpovídajícího daného konstantě.

Chceme-li zkoumat pravdivost formulí jazyka $L$ v nějaké jeho realizaci $\mathcal{M}$ musíme volným proměnným přiřadit hodnoty, jimiž budou nějaké prvky množiny $M$.

\begin{mydef}
Libovolné zobrazení $e$ množiny všech proměnných do univerza $M$ dané realizace $\mathcal{M}$ jazyka $L$ budeme nazývat \emph{ohodnocení proměnných}.

Je-li $x$ proměnná a $e$ ohodnocení proměnných a $m \in M$, potom ohodnocení proměnných, které proměnné $x$ přiřazuje prvek $m$ a pro všechny ostatní proměnné splývá s ohodnocením $e$, budeme značit $e(x/m)$.
\end{mydef}

\begin{mydef}
\emph{Hodnota termu} $t$ v realizaci $\mathcal{M}$ jazyka $L$ při daném ohodnocení $e$ proměnných, označovaná $t[e]$, se definuje indukcí následovně:
\begin{enumerate}[(i)]
	\item Je-li $t$ proměnná $x$, potom $t[e]$ je $e(x)$
	\item je-li $t$ term tvaru $f(t_1, \dots, t_n)$, kde $f$ je funkční symbol četnosti $n$ a $t_1, \dots, t_n$ jsou termy, potom $t[e]$ je $f_\mathcal{M}(t_1[e], \dots, t_n[e])$
\end{enumerate}
\end{mydef}

\begin{mydef}
Nechť $\mathcal{M}$ je realizace jazyka $L$, nechť $e$ je ohodnocení proměnných a nechť $\varphi$ je formule jazyka $L$. Indukcí podle složitosti formule $\varphi$ definujeme, co znamená, že \emph{formule $\varphi$ je pravdivá v $\mathcal{M}$ při ohodnocení $e$}. Tuto skutečnost budeme značit $\mathcal{M} \models \varphi[e]$
\begin{enumerate}[(i)]
	\item Je-li $\varphi$ atomická formule tvaru $p(t_1, \dots, t_n)$, kde $p$ je predikátový symbol četnosti $n$ a $t_1, \dots, t_n$ jsou termy, pak $\mathcal{M} \models \varphi[e]$ právě když $(t_1[e], \dots, t_n[e]) \in p_\mathcal{M}$.
	\item Je-li $\varphi$ atomická formule tvaru $t_1 = t_2$, kde $t_1, t_2$ jsou termy, pak $\mathcal{M} \models \varphi[e]$ právě když $t_1[e]$ je tentýž prvek jako $t_2[e]$ v $M$
	\item Je-li $\varphi$ tvaru $\lnot \psi$, kde je $\psi$ je formule jazyka $L$, pak $\mathcal{M} \models \varphi[e]$ je právě když $\mathcal{M} \not\models \psi[e]$.
	\item Je-li $\varphi$ některého z tvarů $(\eta \land \psi), (\eta \lor \psi), (\eta \to \psi), (\eta \leftrightarrow \psi)$, kde $\eta$, $\psi$ jsou formule, klademe: \\
	$\mathcal{M} \models (\eta \land \psi)[e]$ právě když současně $\mathcal{M} \models \eta[e]$ a $\mathcal{M} \models \psi[e]$. \\
	$\mathcal{M} \models (\eta \lor \psi)[e]$ právě platí alespoň jedno z $\mathcal{M} \models \eta[e]$ a $\mathcal{M} \models \psi[e]$ a podobně další logické spojky.
	\item Je-li $\varphi$ tvaru $(\forall x \psi)$, kde $\psi$ je formule jazyka $L$, pak $\mathcal{M} \models \varphi[e]$ právě když pro každý prvek $m \in M$ je $\mathcal{M} \models \psi[e(x/m)]$.
	\item Je-li $\varphi$ tvaru $(\exists x \psi)$, kde $\psi$ je formule jazyka $L$, pak $\mathcal{M} \models \varphi[e]$ právě když existuje $m \in M$ taková, že $\mathcal{M} \models \psi[e(x/m)]$.
\end{enumerate}
\end{mydef}

\begin{mydef}
Řekneme, že formule $\varphi$ jazyka $L$ je \emph{logicky platná}, jestliže pro každou realizaci $\mathcal{M}$ jazyka $L$ je $\mathcal{M} \models \varphi$, píšeme $\models \varphi$.
\end{mydef}
\end{comment}
















%%%%%%%%%%%%%%%%%%%%%%%%%%%%%%%%%%%%%%%%%%%%%%%%%%%%%%%%%%%%%%%%%%%%%%%%%%%%%%%%
%%%%%%%%%%%%%%%%%%%%%%%%%%%%%%%%%%%%%%%%%%%%%%%%%%%%%%%%%%%%%%%%%%%%%%%%%%%%%%%%
\chapter{Formální systém predikátové logiky} \label{cha:7}

1. semestr, MAT, \texttt{logikaaktual3.pdf}, 5., 6., 7., 8. kapitola

(axiomy a odvozovací pravidla, dokazatelnost, model a důsledek teorie, věty o úplnosti a kompaktnosti, prenexní tvar formulí)

\section{Axiomy}

%Budujeme predikátovou logiku jako formální axiomatický systém. Jazyk $L$ predikátové logiky přebíráme z předchozího s tím, že z logických spojek bereme jako základní $\lnot$ a $\to$ (ostatní mohou být definovány jako ve výrokovém počtu). Z kvantifikátorů bereme jako základní $\forall$, kvantifikátor $\exists$ je možno zavést takto: Je-li $\varphi$ formule, pak $\exists x \varphi$ je zkratka pro $\lnot(\forall x (\lnot \varphi))$. Omezíme se tedy pouze na ty formule, které jsou vytvořeny z atomických formulí jen pomocí spojek $\lnot$, $\to$ a kvantifikátoru $\forall$.. Axiomy predikátové logiky lze rozdělit do čtyř skupin.

\subsection{Schémata výrokových axiomů}

Jsou-li $\varphi, \psi, \eta$ formule jazyka $L$, pak
\begin{eqnarray*}
&\varphi \to (\psi \to \varphi) & \\
&(\varphi \to (\psi \to \eta)) \to ((\varphi \to \psi) \to (\varphi \to \eta)) & \\
&((\lnot \psi) \to (\not \varphi)) \to (\varphi \to \psi) & 
\end{eqnarray*}
jsou axiomy predikátové logiky.

\subsection{Schéma axiomu kvantifikátoru}
Jsou-li $\varphi, \psi$ formule a je-li $x$ proměnná, která nemá volný výskyt ve formuli $\varphi$, pak
$$ (\forall x (\varphi \to \psi)) \to (\varphi \to (\forall x \psi)) $$
je axiom predikátové logiky.

\subsection{Schéma axiomu substituce}
Je-li $\varphi$ formule, $x$ proměnná a $t$ term substituovatelný za $x$ do $\varphi$, pak
$$ (\forall x \varphi) \to \varphi_x[t] $$
je axiom predikátové logiky.

Jestliže $t = x$, pak schéma axiomu substituce má tvar
$$ (\forall x \varphi) \to \varphi $$


\subsection{Schémata axiomů rovnosti}
Je-li $x$ proměnná, pak $x = x$ je axiom. Jsou-li $x_1, \dots, x_n, y_1, \dots, y_n$ proměnné a je-li $f$ funkční symbol s četností $n$, pak
$$(x_1 = y_1 \to (x_2 = y_2 \to ( \dots (x_n = y_n \to f(x_1, \dots, x_n) = f(y_1, \dots, y_n)) \dots ))) $$
je axiom. Jsou-li $x_1, \dots, x_n, y_1, \dots, y_n$ proměnné, je-li $p$ predikátový symbol s četností $n$, pak
$$(x_1 = y_1 \to (x_2 = y_2 \to ( \dots (x_n = y_n \to p(x_1, \dots, x_n) = p(y_1, \dots, y_n)) \dots ))) $$
je axiom.

\section{Odvozovací pravidla predikátové logiky}

\subsection{Pravidlo odloučení (modus ponens}
Z formulí $\varphi, \varphi \to \psi$ se odvodí formule $\psi$.

\subsection{Pravidlo zobecnění (generalizace)}
Pro libovolnou proměnnou $x$ se z formule $\varphi$ odvodí formule $\forall x \varphi$.

%Spolu se schématy výrokových axiomů a pravidle odloučení přechází do predikátové logiky celá výroková logika.

\section{Dokazatelnost?}
\begin{veta}
(O korektnosti) Libovolná formule jazyka $L$ dokazatelná v predikátové logice 1. řádu je logicky platnou formulí, tj. je splněna v každé realizaci jazyka L.
\end{veta}

\begin{lemma}
(Pravidlo $\forall$) Je-li $\vdash \varphi \to \psi$ a proměnná $x$ nemá volný výskyt ve $\varphi$, pak $\vdash \varphi \to (\forall x \psi)$.
\end{lemma}

\begin{lemma}
(Pravidlo $\exists$) Je-li $\vdash \varphi \to \psi$ a proměnná $x$ nemá volný výskyt ve $\psi$, pak $\vdash (\exists x \varphi) \to \psi$.
\end{lemma}

\begin{lemma}
Je-li $\varphi$ formule, $x$ proměnná, $t$ term substituovatelný za $x$ do $\varphi$, pak $\vdash \varphi_x[t] \to (\exists x \varphi)$
\end{lemma}

\begin{lemma}
Nechť $\varphi'$ je instancí formule $\varphi$, tj. nechť $\varphi'$ je tvaru $\varphi_{x_1, \dots, x_n}[t_1, \dots t_n]$ pro nějaké termy $t_1, \dots, t_n$ substituovatelné za $x_1, \dots, x_n$ do $\varphi$. Jestliže $\vdash \varphi$, pak $\vdash \varphi'$.
\end{lemma}

\subsection{Uzávěr formule}
\begin{mydef}
Jsou-li $x_1, \dots, x_n$ všechny volné proměnné ve formuli $\varphi$ v nějakém pořadí, pak formuli $(\forall x_1 \dots \forall x_n \varphi$ nazveme uzávěrem formule $\varphi$.
\end{mydef}

\begin{veta}
(O uzávěru) Je-li $T$ množina formulí a $\varphi'$ uzávěr formule $\varphi$, pak $T \vdash \varphi$ právě když $T \vdash \varphi'$.
\end{veta}

\begin{lemma}
(Distribuce kvantifikátorů) Je-li $\vdash \varphi \to \psi$, potom $\vdash (\forall x \varphi ) \to (\forall x \psi), \vdash (\exists x \varphi) \to (\exists x \psi)$.
\end{lemma}

\begin{veta}
(O dedukci) Nechť $T$ je množina formulí jazyka L, nechť $\varphi$ je uzávřená formule, $\psi$ je libovolná formule jazyka $L$. Potom $T \vdash \varphi \to \psi$, právě když $T, \varphi \vdash \psi$.
\end{veta}

\begin{veta}
(O konstantách) Nechť $T$ je množina formulí jazyka $L$, nechť $\varphi$ je formule. Nechť $x_1, \dots, x_n$ jsou proměnné a nechť $c_1, \dots, c_n$ jsou nové konstanty, jejichž přidáním k $L$ vznikne jazyk $L'$. Potom $T \vdash \varphi_{x_1, \dots x_n}[c_1, \dots, c_n]$, právě když $T \vdash \varphi$.
\end{veta}

\begin{lemma}
Je-li $L$ jazyk s rovností, pak

$$ \vdash x = y \to y = x $$
$$ \vdash x = y \to (y = z \to x = z) $$
\end{lemma}

\begin{lemma}
Je-li $f$ funkční symbol četnosti $n$, je-li $p$ predikátová symbol četnosti $m$ a jsou-li $u$, $v$, $w$, $s_1, \dots, s_n$, $t_1, \dots, t_n$ termy jazyka $L$, pak
\begin{enumerate}[(i)]
	\item $\vdash u = u $
	\item $\vdash u = v \to v = u $
	\item $\vdash u = v \to (v = w \to u = w) $
	\item $\vdash s_1 = t_1 \to (s_2 = t_2 \to \dots (s_n = t_n \to f(s_1, \dots, s_n) = f(t_1, \dots, t_n)) \dots ) $
	\item $\vdash s_1 = t_1 \to (s_2 = t_2 \to \dots (s_n = t_n \to p(s_1, \dots, s_n) = p(t_1, \dots, t_n)) \dots ) $
\end{enumerate}
\end{lemma}

\section{Prenexní tvar formulí}
Základní tvar formulí.

\begin{comment}
\begin{lemma}
Buď $i_1, \dots, i_n$ libovolná permutace čísel $\{1, \dots, n\}$. Nechť $x_1, \dots, x_n$ jsou proměnné a $A$ formule predikátové logiky. Pak platí:
\begin{enumerate}
	\item $\vdash (\forall x_1) \dots (\forall x_n) A \leftrightarrow (\forall x_{i_1}) \dots (\forall x_{i_n}) A $
	\item $\vdash (\exists x_1) \dots (\exists x_n) A \leftrightarrow (\exists x_{i_1}) \dots (\exists x_{i_n}) A $
\end{enumerate}
\end{lemma}

\begin{veta}
Buď $A$ formule taková, že proměnné $x_1, \dots x_n$ jsou jediné proměnné s volným výskytem v $A$. Pak $\vdash A$, právě když $\vdash \forall x_1 \dots \forall x_n A$
\end{veta}

\begin{veta}
(O ekvivalenci) Nechť formule $A'$ vznikne s formule $A$ nahrazením některých výskytů podformulí $B_1 \dots B_n$ pro řadě formulemi $B'_1, \dots, B'_n$. Je-li $\vdash B_i \leftrightarrow B'_i$ pro všechna $i = 1, \dots, n$, pak platí $\vdash A \leftrightarrow A'$.
\end{veta}

\begin{veta}
Buďte $A, B$ formule a $x$ proměnná. Pak

$$ \vdash (\exists x) \lnot A \leftrightarrow \lnot (\forall x) A $$
$$ \vdash (\forall x) \lnot A \leftrightarrow \lnot (\exists x) A $$

Jestliže $x$ není volná ve formuli $A$ a $\circ$ značí některou z výrokových spojek $\land, \lor, \to$, pak platí

$$ \vdash \forall x (A \circ B) \leftrightarrow (A \circ \forall x B) $$
$$ \vdash \exists x (A \circ B) \leftrightarrow (A \circ \exists x B) $$

pro opačnou implikaci $B \to A$ platí:

$$ \vdash \forall x (B \to A) \leftrightarrow (\exists x B \to A)$$
$$ \vdash \exists x (B \to A) \leftrightarrow (\forall x B \to A)$$
\end{veta}
\end{comment}

\begin{mydef}
Nechť $A$ je formule predikátové logiky. Formule $A'$ je \emph{variantou} formule $A$, jestliže vznikne z $A$ postupným nahrazením podformulí tvaru $(Qx) B$ podformulemi $(Q y) B_x[y]$, kde $Q$ je obecný nebo existenční kvantifikátor a $y$ je proměnná, která není volná v $B$.
\end{mydef}

Důsledek: Je-li $A'$ variantou formule $A$, pak je dokazatelné, že obě formule jsou ekvivalentní: $\vdash A \leftrightarrow A'$

\begin{mydef}
Formule $A$ je v \emph{prenexním tvaru}, jestliže má tvar $Q_1 x_1 \dots Q_n x_n B$, kde
\begin{enumerate}[(i)]
	\item $n \geq 0$ a pro každé $i = 1, \dots n$ je $Q_i$ buď $\forall$ nebo $\exists$,
	\item $x_i, \dots, x_n$  jsou navzájem různé proměnné,
	\item $B$ je otevřená formule (neobsahuje kvantifikátory).
\end{enumerate}
\end{mydef}

\begin{veta}
Ke každé formuli $A$ lze sestrojit formuli $A'$ v prenexním tvaru tak, že $\vdash A \leftrightarrow A'$.
\end{veta}

\subsection{Převedení formule na prenexní tvar}
\begin{description}
	\item[Vyloučení zbytečných kvantifikátorů] vynecháme všechny kvantifikátory $\forall x$, resp. $\exists x$ v podformulích tvaru $\forall x B$ nebo $\exists x B$, pokud se proměnná $x$ nevyskytuje volně v $B$.

	\item[Přejmenování proměnných] Vyhledáme podformuli $Q x A$ nejvíce vlevo takovou, že proměnná $x$ se vyskytuje volně v $A$. Pokud $x$ má ještě další výskyt ve výchozí formuli, nahradíme podformuli $Q x A$ její variantou $Q x' A'$, kde $x'$ je proměnná různá od všech proměnných vyskytujících se v převáděné formuli. Tento proces opakujeme do té doby, až všechny kvantifikátory mají různé proměnné a žádná proměnná není v získané formule současně volná i vázaná (formule s čistými proměnnými).

	\item[Eliminace spojky $\leftrightarrow$] provede se podle následujícího schématu:
	$$ A \leftrightarrow B \dots (A \to B) \land (B \to A)$$

	\item[Přesun negace dovnitř] - provádíme postupně náhrady podformulí podle schémat
	\begin{eqnarray*}
	& \lnot (\forall x A)	\dots \exists x \lnot A & \\
	& \lnot (\exists x A)	\dots \forall x \lnot A & \\
	& \lnot (A \to B)		\dots A \land \lnot B & \\
	& \lnot (A \lor B)	\dots \lnot A \land \lnot B & \\
	& \lnot (A \land B)	\dots \lnot A \lor \lnot B & \\
	& \lnot (\lnot A )	\dots A &
	\end{eqnarray*}

	\item[Přesun kvantifikátoru doleva] pro podformuli $B$, ve které se nevyskytuje proměnná $x$, provádíme náhrady podle schémat
	\begin{eqnarray*}
	& (QxA) \lor B	\dots Qx(A \lor B) &\\
	& (QxA) \land B	\dots Qx(A \land B) &\\
	& (QxA) \to B		\dots \bar{Q}x(A \to B) &\\
	& B \to (QxA)		\dots Qx(B \to A) &\\
	& (\exists x A) \lor (\exists y B) 	\dots \exists x (A \lor B_y[x]) &\\
	& (\forall x A) \land (\forall y B) 	\dots \forall x (A \land B_y[x]) &
	\end{eqnarray*}
	kde $\bar{Q}$ je kvantifikátor "opačný" ke $Q$. N+kde lze snížit počet kvantifikátorů pomocí schémat
\end{description}

\section{Věta o úplnosti}

\begin{mydef}
Je-li $L$ jazyk 1. řádu a $T$ množina formulí jazyka $L$, říkáme, že $T$ je \emph{teorie 1. řádu} s jazykem L.
\end{mydef}

\begin{mydef}
Říkáme, že teorie je \emph{sporná}, jestliže pro každou formuli $\varphi$ jazyka $L$ platí $T \vdash \varphi$. V opačném případě je teorie \emph{bezesporná}.
\end{mydef}
(tedy platí $T \vdash \varphi$ a zároveň $T \vdash \lnot \varphi$)

Důsledek: Nechť $T$ je množina formulí a nechť $\varphi'$ je uzávěr formule $\varphi$. Potom $T \vdash \varphi$, právě když $T \cup \{\lnot \varphi'\}$ je sporná teorie.

\subsection{Model a důsledek teorie}

\begin{mydef}
Buď $T$ teorie s jazykem $L$ a nechť $\mathcal{M}$ je nějaká realizace jazyka $L$. Řekněme, že $\mathcal{M}$ je model teorie $T$, jestliže $\mathcal{M} \models \varphi$ pro každou formuli $\varphi \in T$. Pak píšeme $\mathcal{M} \models T$.
\end{mydef}

\begin{mydef}
Řekneme, že formule $\varphi$ je \emph{důsledkem teorie $T$}, jestliže pro každý model teorie $\mathcal{M}$ teorie $T$ je $\mathcal{M} \models \varphi$. Pak píšeme $T \models \varphi$.
\end{mydef}

\begin{veta}
(O korektnosti) Je-li teorie s jazykem $L$ a $\varphi$ formule taková, že $T \vdash \varphi$, pak $T \models \varphi$.
\end{veta}

Důsledek: Má-li teorie $T$ s jazykem $L$ nějaký model, potom je bezesporná.

\begin{veta}
(Gödelova věta o úplnosti) Je-li $T$ teorie s jazykem $L$ a je-li $\varphi$ libovolná formule jazyka $L$, pak $T \vdash \varphi$ právě když $T \models \varphi$.
\end{veta}

\begin{veta}
(Gödelova věta o úplnosti) Teorie $T$ je bezesporná, právě když má nějaký model.
\end{veta}

\begin{mydef}
Řekneme, že teorie $T$ s jazykem L je \emph{úplná}, jestliže $T$ je bezesporná a pro každou uzavřenou formuli $\varphi$ platí $T \vdash \varphi$ nebo $T \vdash \lnot \varphi$ (v důsledku bezespornosti nemůže platit $T \vdash \varphi$ i $T \vdash \lnot \varphi$ současně), V opačném případě říkáme, že $T$ je \emph{neúplná}.
\end{mydef}

\begin{mydef}
Řekneme, že teorie $T$ s jazykem $L$ je \emph{Henkinova}, jestliže pro libovolnou uzavřenou formuli tvaru $\exists x \psi$ jazyka $L$ existuje konstanta $c$ jazyka $L$ taková, že $T \vdash (\exists x \psi) \to \psi_x[x]$.
\end{mydef}

\begin{lemma}
Libovolná Henkinova teorie má model.
\end{lemma}

\begin{mydef}
Jazyk $L'$ je rozšířením jazyka $L$ jestliže každý speciální symbol jazyka $L$ je obsažen v jazyce $L'$. Teorie $T'$ jazyka $L'$ je rozšířením teorie $T$ jazyka $L$, jestliže pro libovolnou formuli $\varphi$ jazyka $L$ takovou, že $T \vdash \varphi$, je také $T' \vdash \varphi$. Teorie $T'$ je konzervativním rozšířením teorie $T$, jestliže  navíc pro každou formuli $\psi$ jazyka $L$ takovou, že $T' \vdash \psi$, je již $T \vdash \psi$.
\end{mydef}

\begin{lemma}
(Heinkin) K libovolné teorii lze sestrojit Heinkinovu teorii $T_H$, která je konzervativním rozšířením teorie $T$.
\end{lemma}

\begin{lemma}
Teorie $T$ je bezesporná, právě když každá její konečná podmnožina $Q \subseteq T$ je bezesporná.
\end{lemma}

\begin{veta}
(Lindenbaum) Je-li $T$ bezesporná teorie s jazykem $L$, pak existuje rozšíření $T'$ teorie $T$ se stejným jazykem $L$.
\end{veta}

\section{Věta o kompaktnosti a věta Herbrandova}

\begin{veta}
(O kompaktnosti) Nechť $T$ je množina formulí jazyka $L$. Pak teorie $T$ má nějaký model, právě když každá její konečná podmnožina $Q \subseteq T$ má model.
\end{veta}

\begin{veta}
(Löwenheim,Skolem) Má-li teorie $T$ s jazykem $L$ nekonečný model, pak má model libovolné mohutnosti $n \geq max\{\aleph_0, |L|\}$
\end{veta}

Is there more?





%%%%%%%%%%%%%%%%%%%%%%%%%%%%%%%%%%%%%%%%%%%%%%%%%%%%%%%%%%%%%%%%%%%%%%%%%%%%%%%%
%%%%%%%%%%%%%%%%%%%%%%%%%%%%%%%%%%%%%%%%%%%%%%%%%%%%%%%%%%%%%%%%%%%%%%%%%%%%%%%%
\chapter{Algebraické struktury} \label{cha:8}

1. semestr, MAT, \texttt{Zaklady\_obecne\_algebry.pdf}, 1. kapitola?

(grupy, okruhy, obory integrity a tělesa, svazy a Boolovy algebry, univerzální algebry)

\section{Operace a zákony}

\begin{mydef}
Buď $A$ množina, $n \in \mathbb{N}_0$. Potom zobrazení $\omega : A^n \to A$ se nazývá n-ární operace na $A$. Tedy pro $n \in \mathbb{N}$:

$$ \omega : A^n \to A $$
$$ \omega : (x_1, \dots, x_n) \to \omega x_1 \dots x_n$$
\end{mydef}

\begin{mydef}
Buď $A$ množina, $n \in \mathbb{N}_0, D \subseteq A^n$. Potom zobrazení $\omega : D \to A$ se nazývá n-ární parciální operace na $A$.
\end{mydef}

\begin{mydef}
Buď $A$ množina $I$ množina (indexů). Pro $i \in I$ buď $\omega_i$ n-ární operace na $A$, $n_i \in \mathbb{N}$. Potom $\mathcal{A} := (A, (\omega_i)_{i \in I}$ označujeme (univerzální) algebru s nosnou množinou $A$ a souborem operací $(\omega_i)_{i \in I} =: \Omega$.
\end{mydef}

\begin{mydef}
Buď $A$ množina, $\circ$ binární operace na $A$. Prvek $e \in A$ se nazývá
a) \emph{levý neutrální prvek} vzhledem k $\circ: \Leftrightarrow x \in A : e \circ x = x$,
b) \emph{pravý neutrální prvek} vzhledem k $\circ: \Leftrightarrow x \in A : x \circ e = x$,
c) \emph{neutrální prvek} vzhledem k $\circ: \Leftrightarrow x \in A : e \circ x = x \circ e = x$
\end{mydef}

\begin{veta}
Buď $\circ$ binární operace na $A$, $e_1$ levý neutrální prvek a $e_2$ pravý neutrální prvek. Potom platí: $e_1 = e_2$ a $e_1 (= e_2)$ je neutrální prvek.
\end{veta}
Existuje nanejvýše jeden neutrální prvek.

\begin{mydef}
Buď $A$ množina $\circ$ binární operace, $e$ neutrální prvek, $x \in A$. Potom nazýváme prvek $y \in A$
a) \emph{levým inverzním prvkem} k $x: \Leftrightarrow y \circ x = e$
b) \emph{pravým inverzním prvkem} k $x: \Leftrightarrow x \circ y = e$
c) \emph{inverzním prvkem} k $x: \Leftrightarrow x \circ y = y \circ x = e$
\end{mydef}

\begin{mydef}
Prvek $x$ se nazývá invertibilní: $\Leftrightarrow$ existuje inverzní prvek k $x$.
\end{mydef}

\begin{mydef}
Buď $A$ množina, $\circ$ binární operace na $A$. $\circ$ se nazývá \emph{asociativní}: $\Leftrightarrow \forall x,y,z \in A: (x \circ y) \circ = x \circ (y \circ z)$ (asociativní zákon). 
\end{mydef}

\begin{veta}
Buď $\circ$ \textbf{asociativní} binární operace na $A$, $x \in A$, $y_1$ levý inverzní prvek k $x$, $y_2$ pravý inverzní prvek k $x$. Potom platí $y_1 = y_2$.
\end{veta}

Je-li operace asociativní, existuje ke každému prvku nejvýše jeden inverzní prvek.

\begin{mydef}
Binární operace $\circ$ se nazývá operace s dělením na $A$: $\Leftrightarrow \forall (a,b) \in A^2 \exists (x,y) \in A^2: a \circ x = b$ (levý zákon o dělení) $\land y \circ a = b$ (pravý zákon o dělení).
\end{mydef}

\begin{veta}
Buď $A \not= \emptyset$ a $\circ$ asociativní binární operace na $A$. Potom jsou následující tvrzení ekvivalentní:
\begin{enumerate}[a)]
	\item $\circ$ je operace s dělením na $A$
	\item Existuje neutrální prvek $e$ (vzhledem k $\circ$) a každý prvek $x \in A$ je invertibilní, tzn. $\exists y \in A: x \circ y = y \circ x = e$.
\end{enumerate}
\end{veta}

\begin{mydef}
Binární operace $\circ$ na $A$ se nazývá operace s krácením: $\Leftrightarrow \forall a, x_1, x_2, y_1, y_2 \in A: (a \circ x_1 = a \circ x_2 \Rightarrow x_1 = x_2)$ (levý zákon o krácení) $\land (y_1 \circ a = y_2 \circ a \Rightarrow y_1 = y_2$ (pravý zákon o krácení).
\end{mydef}

\begin{mydef}
Binární operace $\circ$ na $A$ se nazývá komutativní: $\Leftrightarrow \forall x,y \in A: x \circ y = y \circ x$ (komutativní zákon).
\end{mydef}

\begin{mydef}
Pokud jsou $+, \cdot$ binární operace na $A$, potom se $\cdot$ nazývá distributivní nad $+$: $\Leftrightarrow \forall x,y,z \in A: x \cdot (y + z) = x \cdot y + x \cdot z$ (levý komutativní zákon) $\land (y + z) \cdot x = y \cdot x + z \cdot x$ (pravý komutativní zákon).
\end{mydef}

\section{Důležité typy algeber}
\begin{mydef}
Algebra $(A, \cdot)$ typu (2) se nazývá \emph{grupoid}.
\end{mydef}

\begin{mydef}
Grupoid $(H, \cdot)$ se nazývá \emph{pologrupa}: $\Leftrightarrow \cdot$ je asociativní.
\end{mydef}

\begin{mydef}
\begin{enumerate}[a)]
	\item Pologrupa $(H, \cdot)$ se nazývá \emph{monoid} typu (2): $\Leftrightarrow$ existuje neutrální prvek $e$.
	\item Algebra $(H, \cdot, e)$ typu (2,0) se nazývá \emph{monoid} typu (2, 0): $\Leftrightarrow$ platí následující zákony pro všechna $x,y,z \in H$:
	\begin{enumerate}[1)]
		\item $x(yz) = (xy)z$
		\item $ex = x$, $xe = x$
	\end{enumerate}
\end{enumerate}
\end{mydef}

\begin{mydef}
\begin{enumerate}[a)]
	\item Monoid $(G, \circ)$ s neutrálním prvkem $e$ se nazývá \emph{grupa} typu (2): $\Leftrightarrow$ každý prvek $x \in G$ je invertibilní, tj., $\forall x \in G \exists x^{-1} \in G: x x^{-1} = x^{-1} x = e$.
	\item Algebra $(G, \cdot, e, ^{-1}$ typu (2, 0, 1) se nazývá \emph{grupa} typu (2, 0, 1): $\Leftrightarrow$ platí následující zákony pro všechna $x, y, z \in G$:
	\begin{enumerate}[1)]
		\item $x(yz) = (xy)z$
		\item $ex = x$, $xe = x$
		\item $x x^{-1} = e$, $x^{-1} x = e$
	\end{enumerate}
	\item Grupa $(G, \cdot)$, resp. $(G, \cdot, e, ^{-1})$ se nazývá \emph{komutativní nebo abelovská}ů $\Leftrightarrow \forall x,y \in G: xy = yx$.
\end{enumerate}
\end{mydef}

\begin{mydef}
\begin{enumerate}[a)]
	\item Algebra $(R, +, \cdot)$ typu (2, 2) se nazývá \emph{okruh} typu (2, 2): $\Leftrightarrow$
	\begin{enumerate}[1)]
		\item $(R, +)$ je abelovská grupa
		\item $(R, \cdot)$ je pologrupa
		\item $\cdot$ je distributivní nad $+$
	\end{enumerate}
	\item Algebra $(R, +, 0, -, \cdot)$ typu (2, 0, 1, 2) se nazývá \emph{okruh} typu (2, 0, 1, 2): $\Leftrightarrow$
		\begin{enumerate}[1)]
		\item $(R, +, 0, -)$ je abelovská grupa
		\item $(R, \cdot)$ je pologrupa
		\item $\cdot$ je distributivní nad $+$
	\end{enumerate}
	Prvek $0$ se nazývá "nulový prvek" okruhu. Budeme psát $x - y:= x + (-y)$
\end{enumerate}
\end{mydef}

\begin{mydef}
\begin{enumerate}[a)]
	\item Algebra $(R, +, 0, -, \cdot, 1)$ typu (2, 0, 1, 2, 0) se nazývá \emph{okruh s jednotkovým prvkem}: $\Leftrightarrow$
	\begin{enumerate}[1)]
		\item $(R, +, 0, -, \cdot)$ je okruh
		\item $1$ je neutrální prvek vzhledem k $\cdot$, tj. $\forall x \in R: 1 \cdot x = x \cdot 1 = x$ (1 se nazývá \emph{jednotkový prvek} okruhu).
	\end{enumerate}
	\item Okruh $(R, +, 0, -, \cdot)$ se nazývá \emph{komutativní}: $\Leftrightarrow \forall x ,y \in R: xy = yx$
	\item Algebra $(R, +, 0, -, \cdot, 1)$ se nazývá \emph{komutativní okruh s jednotkovým prvkem}: $\Leftrightarrow$
		\begin{enumerate}[1)]
		\item $(R, +, 0, -, \cdot)$ je komutativní okruh
		\item $1$ je neutrální prvek vzhledem k $\cdot$.
	\end{enumerate}
\end{enumerate}
\end{mydef}

\begin{mydef}
Komutativní okruh s jednotkovým prvkem $(R, +, 0, -, \cdot, 1)$ se nazývá \emph{obor integrity}: $\Leftrightarrow$
\begin{enumerate}
	\item $R \backslash \{0\} \not= \emptyset$ (tj. $0 \not= 1$
	\item $\forall x,y \in R: x \not= 0 \land y \not= 0 \Rightarrow xy \not= 0$ (tj. neexistují dělitelé nuly).
\end{enumerate}
$\cdot$ je operace s krácením na $R \backslash \{0\}$.
$(R \backslash \{0\}, \cdot, 1)$ je komutativní monoid.
\end{mydef}

\begin{mydef}
\begin{enumerate}[a)]
	\item Okruh s jednotkovým prvkem $(R, +, 0, -, \cdot, 1)$ se nazývá těleso: $\Leftrightarrow$
	\begin{enumerate}[1)]
		\item $0 \not= 1$
		\item $(R \backslash \{0\}, \cdot)$ je grupa
	\end{enumerate}
	\item Komutativní těleso se nazývá \emph{pole}.
\end{enumerate}
\end{mydef}

\begin{veta}
Každé \emph{pole} je \emph{obor integrity}. Každý konečný obor integrity je pole.
\end{veta}

\begin{mydef}
Buď $(K, +, 0, -, \cdot, 1)$ pole, $I = \{a, b, c\} \cup K$, kde $a,b,c \not\in K$, $a,b,c$ po dvou různé. Algebra $(V, (\omega_i)_{i \in I}$ typu $(2, 0, 1, (1)_{\lambda \in K})$ se nazývá vektorový prostor nad $K$: $\Leftrightarrow$
\begin{enumerate}[1)]
	\item $(V, \omega_a, \omega_b, \omega_c) =: (V, +, 0, -)$ je abelovská grupa,
	\item $\forall x,y \in V, \lambda, \mu \in K$:\\
	$$\omega_\lambda(x + y) = \omega_\lambda(x) + \omega_\lambda(y) $$
	$$\omega_{\lambda + \mu}(x) = \omega_\lambda(x) + \omega_\mu(x) $$
	$$\omega_{\lambda \mu}(x) = \omega_\lambda( \omega_\mu(x)) $$
	$$\omega_1(x) = x $$
\end{enumerate}
\end{mydef}

\begin{mydef}
Algebra $(V, \cap, \cup)$ typu $(2, 2)$ se nazývá \emph{svaz}: $\Leftrightarrow$ pro všechna $a,b,c \in V$ platí:
\begin{enumerate}[1)]
	\item $a \cap b = b \cap a$,
		$a \cup b = b \cup a$
	\item $a \cap (b \cap c) = (a \cap b) \cap c$,
		$a \cup (b \cup c) = (a \cup b) \cup c$
	\item $a \cap (a \cup b) = a$,
		$a \cup (a \cap b) = a$
\end{enumerate}
Podle 1) a 2) jsou $\cap$, $\cup$ komutativní a asociativní, tj. $(V, \cap)$, $(V, \cup)$ jsou komutativní pologrupy. Zákony uvedení v bodě 3) se nazývají absorpční zákony.

$(V, \cap, \cup)$ je svaz $\Leftrightarrow$ $(V, \cup, \cap)$ je svaz -- princip duality pro svazy.
\end{mydef}

\begin{mydef}
Svaz $(V, \cap, \cup)$ se nazývá distributivní: $\Leftrightarrow$ pro všechna $a, b, c \in V$ platí 
\begin{enumerate}[4)]
	\item $a \cap (b \cup c) = (a \cap b) \cup (a \cap c)$,
		$a \cup (b \cap c) = (a \cup b) \cap (a \cup c)$
\end{enumerate}
\end{mydef}

\begin{mydef}
Buď $(V, \cap, \cup)$ svaz.
Prvek $0 \in V$ se nazývá \emph{nulový prvek svazu $V$}: $\Leftrightarrow \forall a \in V: a \cup 0 = a$ (tj. 0 je neutrální vzhledem k $\cup$).
Prvek $1 \in V$ se nazývá \emph{jednotkový prvek svazu $V$}: $\Leftrightarrow \forall a \in V: 1 \cap a = a$ (tj. neutrální vzhledem k $\cap$).
\end{mydef}

\begin{mydef}
Algebra $(V, \cap, \cup, 0, 1)$ typu (2, 2, 0, 0) se nazývá \emph{ohraničený svaz}: $\Leftrightarrow$
\begin{enumerate}[1)]
	\item $(V, \cap, \cup)$ je svaz,
	\item $0$ je nulový prvek svazu $V$,
	\item $1$ je jednotkový prvek svazu $V$.
\end{enumerate}
\end{mydef}

\begin{mydef}
Ohraničený svaz $(V, \cap, \cup, 0, 1)$ se nazývá komplementární: $\Leftrightarrow \forall x \in V \exists a' \in V: a \cap a' = 0 \land a \cup a' = 1$. Prvek $a'$ se nazývá \emph{komplement} prvku $a$.
\end{mydef}

\begin{mydef}
Distributivní a komplementární svaz $(V, \cap, \cup, 0, 1)$ se nazývá Boolův svaz.
\end{mydef}

\begin{veta}
Je-li $(V, \cap, \cup, 0, 1)$ Booleův svaz, pak existuje ke každému $a \in V$ přesně jeden komplement $a'$.
\end{veta}

\begin{mydef}
Algebra $(B, \cap, \cup, 0, 1, ')$ typu (2, 2, 0, 0, 1) se nazývá Booleova algebra: $\Leftrightarrow$
\begin{enumerate}[1)]
	\item $(B, \cap, \cup, 0, 1)$ je ohraničený svaz
	\item $\forall a \in B: a \cap a' = 0 \land a \cup a' = 1$.
\end{enumerate}
\end{mydef}

\subsection{Strom algeber}
\begin{itemize}
\item Booleův svaz
 \begin{itemize}
 \item Distributivní svaz
  \begin{itemize}
  \item Svaz $(V, \cap, \cup)$
   \begin{itemize}
   \item $(V, \cap)$, $(V, \cup)$ jsou komutativní pologrupy (komutativní a asociativní zákony)
   \item Absorpční zákony
   \end{itemize}
  \item Distributivní zákony
  \end{itemize}
 \item Komplementární svaz
  \begin{itemize}
  \item Ohraničený svaz
   \begin{itemize}
   \item Svaz $(V, \cap, \cup)$
   \item $0$ je nulový prvek svazu $V$
   \item $1$ je jednotkový prvek svazu $V$
   \end{itemize}
  \item Komplement $'$
  \end{itemize}
 \end{itemize}
\item Vektorový prostor nad $K$
 \begin{itemize}
 \item Pole $(K, +, 0, -, \cdot, 1)$
 \item Abelovská grupa: $(V, \omega_a, \omega_b, \omega_c) = (V, +, 0, -)$ je 
 \item ??? zákony
 \end{itemize}
\item Pole
 \begin{itemize}
 \item Konečný
 \item Obor Integrity
  \begin{itemize}
  \item Komutativní okruh s jednotkovým prvkem
   \begin{itemize}
   \item $(R, +, 0, -)$ je komutativní okruh
   \item 1 je neutrální prvek vzhledem k $\cdot$
   \end{itemize}
  \item $1 \not= 0$
  \item neexistují dělitelé nuly
  \end{itemize}
 \end{itemize}
\item Pole
 \begin{itemize}
 \item Komutativní
 \item Těleso
  \begin{itemize}
  \item Okruh s jednotkovým prvkem
   \begin{itemize}
   \item $(R, +, 0, -, \cdot)$ je okruh typu $(2, 0, 1, 2)$ nebo $(R, +, \cdot)$
    \begin{enumerate}
    \item $(R, +, 0, -)$ je abelovská grupa nebo $(R, +)$
     \begin{enumerate}
     \item \textbf{Grupa} $(G, \cdot, e, ^{-1})$ nebo $(G, \circ)$
     \item Komutativní zákony
     \end{enumerate}
    \item $(R, \cdot)$ je pologrupa
    \item $\cdot$ je distributivní nad $+$
    \end{enumerate}
   \item $1$ je neutrální prvek vzhledem k $\cdot$
   \end{itemize}
  \item $0 \not= 1$
  \item $(R \backslash \{0\}, \cdot)$ je grupa
  \end{itemize}
 \end{itemize}

\item Grupa
      \begin{itemize}
      \item Monoid
       \begin{itemize}
       \item Pologrupa $(H, \cdot)$, $(H, \cdot, e)$
        \begin{itemize}
        \item Grupoid
         \begin{enumerate}
		 \item Algebra $(A, \cdot)$ typu 2
		 \end{enumerate}
        \item $\cdot$ asociativní
        \end{itemize}
       \item neutrální prvek
       \item x(yz) = (xy)z
       \end{itemize}
      \item x(yz) = (xy)z
      \item neutrální prvek
      \item každý prvek je invertibilní
      \end{itemize}
\end{itemize}

\section{Základní pojmy teorie grup}

\paragraph{Grupoid}
Součin: $a_1, a \dots a_n := (a_1 \dots a_{n-1})a_n$

Mocniny: $a^1 = a$, $a^{n+1} = (a^n)a$

\paragraph{Grupa}
$(ab)^{-1} = b^{-1}a^{-1}$

$a^0 = e$;
$a^{-n} = (a^{-1})^n$;
$a^n a^m = a^{n+m}$;
$(a^m)^n = a^{mn}$;
$(a b)^n = a^n b^n$ (pokud je $\cdot$ komutativní)

Kardinální číslo ($a^n$) se nazává \emph{řád prvku}: $o(a) := |\{a^0 = e, a^1, a^{-1}, a^2, a^{-2}, \dots\}| = |\{ a^k | k \in Z\}$.

$|G|$ (mohutnost množiny) se nazývá řád grupy/algebry.

Dělení se zbytkem: $\forall k,l \in Z, l \not= 0 \exists q,r \in Z: 0 \leq  r < |l| \land k = lq + r$.

"$r$ je kongruentní s $s$ modulo $n$": $n|(r - s)$ ($n$ dělí $(r-s)$).

Je-li $o(a) = \infty$, pak jsou mocniny prvku $a$ navzájem různé. Je-li $o(a) = n$, potom $a^r = a^s \Leftrightarrow r \equiv s \text{ mod } n$



\section{Svazy a Booeovy algebry}
\subsection{(Částečně uspořádané množiny}

$M$ je množina, $R$ je relace na $M$. 

Částečné uspořádání $(M, R)$ = reflexivita, antisymetrie, tranzitivita.

$(M, R)$ Řetězec nebo Lineárně uspořádaná množina: navíc srovnatelnost: $\forall x,y \in M: xRy \lor yRx$.

\begin{mydef}
Buď $(M, \leq)$ uspořádaná množina. Potom se $k \in M$ nazývá nejmenší (resp. největší) prvek množiny $M :\Leftrightarrow \forall x \in M: k \leq x$ (resp. $k \geq x$).
\end{mydef}

Existuje vždy nejvýše jeden nejmenší resp. největší prvek.

\begin{mydef}
Buď $(M, \leq)$ uspořádaná množina. Potom se $m \in M$ nazývá minimální (resp. maximální) prvek množiny $M :\Leftrightarrow \forall x \in M: x \leq m$ (resp. $x \geq m$) $\Rightarrow x = m$.
\end{mydef}

\begin{veta}
\begin{enumerate}[a)]
	\item Buď $(M, \leq)$ uspořádaná množina a $N \subseteq M$. Potom je $(N, \leq)$ rovněž uspořádaná množina. Je-li $(M, \leq)$ řetězec, potom je také $(N, \leq)$ řetězec. Přitom $(N, \leq)$ zkráceně označuje $(N, \leq \cap (N \times N))$.
	\item Je-li $(M, \leq)$ uspořádaná množina, potom také $(M, \geq)$ je uspořádaná množina (tzv. \uv{princip duality uspořádané množiny}). Také maximální a minimální prvek.
\end{enumerate}
\end{veta}

\begin{mydef}
Buď $(M, \leq)$ uspořádaná množina a $N \subseteq M$. Potom se nazývá $u \in M$ dolní závora množiny $N : \Leftrightarrow \forall x \in N: u \leq x$. Největší prvek množiny všech dolních závor se nazývá \emph{infimum} množiny $N$, formálně inf $N$ nebo $\bigcap N$. Prvek $v \in M$ se nazývá horní závora množiny $N: \Leftrightarrow \forall x \in N: x \in N: x \leq v$. Nejmenší horní závora se nazývá suprémum množiny $N$, formálně sup $N$ nebo $\bigcup N$.
\end{mydef}

\paragraph{Hasseův diagram} Buď $(M, \leq)$ konečná uspořádaná množina a nechť relace "sousední" je definována takto:

a, b sousedí: $\Leftrightarrow \begin{cases}
a < b \text{nebo} b < a \\
\text{neexistuje c taková, že} a < c < b \text{nebo} b < c < a
\end{cases}$

Potom je Hasseův diagram $(M, \leq)$ dán grafy relace \uv{sousedí}. (Množina uzlů je M; je-li $a < b$, nakreslí se uzel $a$ \uv{níže} než uzel $b$ a $a$ se spojí s $b$ hranou, pokud jsou $a$ a $b$ sousední).

\subsection{(Částečná) uspořádaní a svazy}
\begin{mydef}
Buď $(V, \leq)$ uspořádaná množina. $(V, \leq)$ se nazývá svazově uspořádaná: $\Leftrightarrow sup\{a, b\}$ a $inf\{a, b\}$ existují pro všechna $a, b \in V$.
\end{mydef}

\begin{lemma}
Buď $(V, \cap, \cup)$ svaz, potom platí:
\begin{enumerate}[a)]
	\item $\forall a \in V: a \cap a = a = a \cup a$
	\item $\forall a, b \in V: a \cap b = a \Leftrightarrow a \cup b = b$
\end{enumerate}
\end{lemma}

\begin{veta}
\begin{enumerate}
	\item Buď $(V, \cap, \cup)$ svaz. Pokud definujeme relaci $\leq$ na $V$ pomocí vztahu $a \leq b: \Leftrightarrow a \cap b = a, a,b \in B$, potom je $(V, \leq)$ svazově uspořádaná množina.
	\item Buď $(V, \leq)$ svazově uspořádaná množina. Definujeme-li na $V$ binární operace $\cap, \cup$ pomocí vztahů $a \cap b := inf\{a, b\}$ a $a \cap b := sup\{a, b\}, a, b \in V$, potom je $(V, \cap, \cup)$ svaz.
	\item Přiřazení definovaná v a) a b) jsou navzájem inverzní.
\end{enumerate}
\end{veta}

Princip duality pro svazy:

$(V, \cap, \cup)$ svaz $\Leftrightarrow (V, \cup, \cap)$ svaz

$(V, \leq)$ svazově uspořádaný $\Leftrightarrow (V, \geq)$ svazově uspořádaný

\subsection{Booleovy algebry}

Dualita: $(B, \cap, \cup, 0, 1, ')$ Booleova algebra $\Leftrightarrow (B, \cup, \cap, 1, 0, ')$ Booleova algebra

\begin{lemma}
Buď $(V, \cap, \cup)$ svaz. Potom platí:
\begin{enumerate}[a)]
	\item $\forall a,b,c \in V: a \cap (b \cup c) = (a \cap b) \cup (a \cap c) \Leftrightarrow \forall a, b, c \in V: a \cup (b \cap c) = (a \cup b) \cap (a \cup c)$
	\item $\forall a \in V: 0 \cup a = a \Leftrightarrow \forall a \in V: 0 \cap a = 0$
	\item $\forall a \in V: 1 \cap a = a \Leftrightarrow \forall a \in V: 1 \cup a = 1$
\end{enumerate}
\end{lemma}

\begin{veta}
(Věta o komplementech) Buď $(B, \cap, \cup, 0, 1, ')$ Booleova algebra. Potom platí:
\begin{enumerate}[a)]
	\item Jsou-li $a, a^*$ prvky množiny $B$, kde $a \cup a^* = 1$ a $a \cap a^* = 0$, pak platí $a^* = a'$
	\item $(a')' = a$ pro všechna $a \in B$
	\item $0' = 1$ a $1' = 0$
	\item $(a \cup b)' = a' \cap b'$ a $(a \cap b) = a' \cup b'$ pro všechna $a, b \in B$ (De Morganovy zákony)
\end{enumerate}
\end{veta}

\begin{veta}
(Věta o homomorfizmech) Buďte $(B, \cap, \cup, 0, 1, ')$ a $(C, \cap, \cup, 0, 1, ')$ Booleovy algebry, $\varphi: B \to C$ surjektivní zobrazení. Potom platí $\varphi$ je homomorfizmus $(B, \cap, \cup, 0, 1, ')$ do $(C, \cap, \cup, 0, 1, ') \Leftrightarrow \varphi$ je homomorfizmus $(B, \cap, \cup)$ do $(C, \cap, \cup)$ (tj. stačí aby $\varphi$ bylo konzistentní se svazovými operacemi).
\end{veta}

\subsection{Stoneova věta o reprezentaci}
\begin{mydef}
Buď $(V, \cap, \cup, 0, 1)$ svaz s nulovým a jednotkovým prvkem. Potom se $a \in V$ nazývá \emph{atom}: $\Leftrightarrow$
\begin{enumerate}[1)]
	\item $0 < a$
	\item $0 \leq b \leq a \Rightarrow b = 0 \lor b = a$
\end{enumerate}
(tj. $a$ je horním sousedním prvkem nulového prvku.)
\end{mydef}

\begin{lemma}
Buď $(B, \cap, \cup, 0, 1, ')$ konečná Booleova algebra. Potom ke každému prvku $b \in B \backslash \{0\}$ existuje atom $a \in B$, kde $a \leq b$. (Toto platí i pro libovolní konečné svazy.).
\end{lemma}

\begin{veta}
(Stoneova věta) Buď $(B, \cap, \cup, 0, 1, ')$ konečná Booleova algebra a $M := \{a \in B | a \text{atom algebry} \}$. Potom platí:

$$ (B, \cap, \cup, 0, 1, ') \cong (\mathcal{P}(M), \cap, \cup, \emptyset, M, ')$$

přičemž  vztahem $\varphi(b) := \{a \in M | a \leq b\}$ je dán izomorfizmus $\varphi: B \to \mathcal{P}(M)$.
\end{veta}

$|M| = |M_1| \Rightarrow (\mathcal{P}(M), \cap, \cup, \emptyset, M, ') \cong (\mathcal{P}(M_1), \cap, \cup, \emptyset, M_1, ')$

$|M| = n \in \mathbb{N}_0 \Rightarrow |\mathcal{P}(M)| = 2^n$

Je-li $B$ konečná Booleova algebra, potom platí $|B| = 2^n$ pro libovolné $n \in \mathbb{N}_0$ pro libovolné $n \in \mathbb{N}_0$. Ke každému $n \in \mathbb{N}_0$ tak existuje -- až na izomorfizmus -- přesně jedna Booleova algebra s $2^n$ prvky, totiž $\mathcal{P}(\{0, 1, \dots, n-1\})$.

\begin{mydef}
Buď $M$ množina.  $\mathcal{K} \in \mathcal{P}(M)$ se nazývá množinový okruh: $\Leftrightarrow$ pro všechna $A, B \in \mathcal{K}$ platí:
\begin{enumerate}[1)]
	\item $A \cup B \in \mathcal{K}$
	\item $A \cap B \in \mathcal{K}$
	\item $A \cap B' = A \backslash B \in \mathcal{K}$
\end{enumerate}
\end{mydef}

\begin{mydef}
Buď $\mathcal{K} \in \mathcal{P}(M)$ množinový okruh a nechť $M \in \mathcal{K}$. Potom Booleova algebra $(\mathcal{K}, \cap, \cup, \emptyset, M, ')$ se nazývá algebra množinového okruhu.
\end{mydef}

Algebra množinového okruhu je tedy podalgebra $(\mathcal{P}(M), \cap, \cup, \emptyset, M, ')$

\begin{mydef}
(Stoneova věta) Každá Booleova algebra je izomorfní s nějakou algebrou množinového okruhu.
\end{mydef}

















%%%%%%%%%%%%%%%%%%%%%%%%%%%%%%%%%%%%%%%%%%%%%%%%%%%%%%%%%%%%%%%%%%%%%%%%%%%%%%%% 
%%%%%%%%%%%%%%%%%%%%%%%%%%%%%%%%%%%%%%%%%%%%%%%%%%%%%%%%%%%%%%%%%%%%%%%%%%%%%%%%
\chapter{Základní algebraické metody} \label{cha:9}

1. semestr, MAT, \texttt{Zaklady\_obecne\_algebry.pdf}, 2. kapitola

(podalgebry, homomorfismy, přímé součiny, kongruence a faktorové algebry, normální podgrupy a ideály okruhů)

\section{Podalgebry}

\begin{mydef}
Buď $A$ množina, $\omega: A^n \to A$ n-ární operace na $A$ ($n \in N_0)$, $T \subseteq A$. Potom se množina $T$ nazývá uzavřená vzhledem k $\omega: \Leftrightarrow \omega(T^n) \subseteq T$ (tj. $t_1, \dots, t_n \in T \Rightarrow \omega t_1 \dots t_n \in T$, v případě $n = 0: \omega \in T$.
\end{mydef}

\begin{mydef}
Buď $\mathcal{A} = (A, (\omega_i)_{i \in I}$ algebra typu $(n_i)_{i \in I}$, $T \subseteq A$. Potom se množina $T$ nazývá uzavřená vzhledem k $(\omega)_{i \in I}: \Leftrightarrow T$ je uzavřená vzhledem k $\omega_i$ pro všechna $i \in I$. V tomto případě se pomocí vztahu $\omega_i^*x_1 \dots x_{n_1}$, $(x_1, \dots x_n) \in T^{n_i}$, definuje $n_i$-ární operace $\omega_i^*$'na $T$, tj $\omega_i^* = \omega_i|T^{n_i}$. Algebra $(T, (\omega_i^*)_{i \in I}$ se nazývá podalgebra algebry $\mathcal{A}$. Většinou píšeme: $\omega_i^* =: \omega_i$.
\end{mydef}

$(\mathbb{N}, +)$ je podpologrupa $(\mathbb{Z}, +)$ (ale není grupa -- chybí inverzní prvek).

$(\mathbb{N}, +, \cdot)$ je podalgebrou $(\mathbb{Z}, +, \cdot)$ (ale není podokruhem).

$(\mathbb{R}, +, 0, -, \cdot, 1)$ je podpolem $(\mathbb{C}, +, 0, -,\cdot, 1)$ zatímco $(\mathbb{Z}, +, 0, -,\cdot, 1)$

\begin{veta}
Buď $(A, \Omega)$ algebra a $(T_j)_{j \in J}$ soubor podalgeber. Potom je $\bigcap_{\{j \in J\}} T_j$ rovněž podalgebra.
\end{veta}

\begin{veta}
Buď $(A, \Omega)$ algebra a $S \subseteq A$ podmnožina. Potom je

$$\left< S \right> := \bigcap\{T | T \supseteq S, T \text{je podalgebra algebry} (A, \Omega)\}$$

je nejmenší podalgebra algebry $(A, \Omega)$, která $S$ obsahuje.
\end{veta}

\begin{mydef}
$\left< S \right>$ se nazývá podalgebra algebry $(A, \Omega)$ generovaná množinou $S$. Množina $S$ se nazývá systém generátorů podalgebry $\left< S \right>$.
\end{mydef}

\begin{veta}
Buď $(G, \cdot, e, ^{-1})$ grupa, $x \in G, S = \{x\}$. Potom platí:

$$\left< x \right> := \left< S \right> = \{x^k | k \in \mathbb{Z}\}$$
\end{veta}

\begin{mydef}
$\left< x \right>$ se nazývá podgrupa grupy $(G, \cdot, e, ^{-1})$ generovaná prvkem $x$.
\end{mydef}

\begin{mydef}
Grupa $(G, \cdot, e, ^{-1})$ se nazývá cyklická: $\Leftrightarrow \exists x \in G: G = \left< x \right>$
\end{mydef}

\begin{itemize}
	\item pro $(\mathbb{Z}, +, 0, -)$ platí $\mathbb{Z} = \left< 1 \right> = \left< -1 \right>$
	\item pro $(\mathbb{Z}_m, +, 0, -)$ platí $\mathbb{Z}_m = \left< 1 \right> = \left< k \right>$, kde $NSD(m, k) = 1$
\end{itemize}

\subsection{Relace ekvivalence a rozklad na třídy ekvivalence?} % XXX
\begin{mydef}
Je-li $M$ množina, potom se podmnožina $R$ množiny $M \times M$ nazývá binární relace na $M$. Místo $(x, y) \in R$ píšeme většinou $xRy$. Speciální relace $\alpha_M := M \times M$ se nazývá univerzální relace, $\iota_M := \{(x, x) | x \in M\}$ se nazývá identická relace nebo relace rovnosti.
\end{mydef}

\begin{mydef}
Relace $R \subseteq M \times M$ se nazývá:
\begin{enumerate}[1)]
	\item \emph{reflexivní}: $\Leftrightarrow \iota_M \subseteq R$, tj. $\forall x \in M: xRx$
	\item \emph{symetrická}: $\Leftrightarrow \forall x, y \in M: xRy \Rightarrow yRx$
	\item \emph{antisymetrická}: $\Leftrightarrow \forall x, y \in M: xRy \land yRx \Rightarrow x = y$
	\item \emph{transitivní}: $\Leftrightarrow \forall x, y, z \in M: xRy \land yRz \Rightarrow xRz$
\end{enumerate}
Relace splňující 1), 2) a 4) se nazývá relace \emph{ekvivalence}, relace splňující 1), 3) a 4) se nazývá relace \emph{(částečného) uspořádání)}.
\end{mydef}

\begin{mydef}
Buď $M$ množina. $\mathcal{P} \subseteq \mathcal{P}(M)$ se nazývá rozklad množiny $M$ na třídy ekvivalence: $\Leftrightarrow$
\begin{enumerate}[1)]
	\item $\bigcup_{C \in \mathcal{P}} C = M$
	\item $\emptyset \not\in \mathcal{P}$
	\item $A, B \in \mathcal{P}  \Rightarrow A = B \lor A \cap B = \emptyset$ (tj. množiny v $\mathcal{P}$ jsou po dvou disjunktní.
\end{enumerate}
\end{mydef}

\begin{veta}
Buď $\pi$ relace ekvivalence na množině $M$, $a \in M$, $[a]_\pi := \{b \in M | b \pi a\}$, tzv. třída ekvivalence prvku $a$ a $M/\pi := \{[a]_\pi | a \in M\}$ tzv. faktorová množina množiny M podle ekvivalence $\pi$. Potom je $M/\pi$ rozklad množiny na třídy ekvivalence.

Je-i naopak $\mathcal{P}$ rozklad množiny $M$ na třídy ekvivalence a $\pi$ je definováno vztahem $a \pi b \Leftrightarrow \exists C \in \mathcal{P}: a, b \in C$, potom je $\pi$ relace ekvivalence na množině $M$, a platí $M/\pi' = \mathcal{P}$.

$\pi \mapsto M/\pi$ je bijektivní zobrazení množiny všech relací ekvivalence na množině M na množinu všech rozkladů množiny M na třídy ekvivalence. Inverzní zobrazení je dáno výše uvedeným předpisem $\mathcal{P} \mapsto \pi$.
\end{veta}

\begin{veta}
Buďte $M, N$ množiny, $f: M \to N$ zobrazení a $x \pi_f y: \Leftrightarrow f(x) = f(y)$. Potom platí:
\begin{enumerate}[a)]
	\item $\pi_f$ je relace ekvivalence na $M$, která se nazývá jádro $f$.
	\item Zobrazení
	$M/_{\pi_f} \to f(M) := \{f(x) | x \in M\} \subseteq N$\\
	$[x]_{\pi_f} \mapsto f(x)$\\
	je korektně definováno a bijektivní.
\end{enumerate}
\end{veta}

\subsection{Rozklad grupy na třídy podle podgrupy}

\begin{veta}
Buď $(G, \cdot, e, ^{-1})$ grupa a $(H, \cdot, e, ^{-1})$ podgrupa grupy $G$. Buď dále $\pi \subseteq G \times G$ podmnožina definovaná pomocí vztahu $x \pi y: \Leftrightarrow x^{-1}y \in H. x, y \in G$. Potom je $\pi$ relace ekvivalence na $G$.
\end{veta}

\begin{mydef}
Buď $(G, \cdot, e, ^{-1})$ grupa, $A, B \subseteq G$. Potom se nazývá $AB := \{ab | a \in A, b \in B\}$ složený součin A a B. Speciální případy: $A = \{a\}: AB := aB = \{ab | b \in B\}, B = \{b\}: AB := Ab = \{ab | a \in A\}$. Pro podgrupu H grupy G se nazývá $aH$ levá třída rozkladu grupy G podle H a $Ha$ se nazývá pravá třída rozkladu grupy G podle H ($a \in G$ pevné ale libovolné).
\end{mydef}

\begin{veta}
Buď $(G, \cdot, e, ^{-1})$ grupa, $H$ podgrupa grupy $G$, $a, b \in G$. Potom je vztahem $i: aH \to bH; ax \mapsto bx$ definováno bijektivní zobrazení.
\end{veta}


\section{Homomorfismy}

\begin{mydef}
Buďte $\mathcal{A} = (A, (\omega_i)_{i \in I})$ a $\mathcal{A}^* = (A^*, (\omega^*_i)_{i \in I}$ algebry téhož typu $(n_i)_{i \in I}$. Zobrazení $f: A \to A^*$ se nazývá homomorfizmus algebry $\mathcal{A}$ do algebry $\mathcal{A}^*: \Leftrightarrow$
\begin{enumerate}
	\item Pro $i \in I$, kde $n_i > 0$ platí $\forall x_1, \dots, x_{n_i} \in A: f(\omega_ix_1 \dots x_{n_i}) = \omega_i^*f(x_1)\dots f(x_{n_1})$
	\item pro $i \in I$, kde $n_i = 0$, platí $f(\omega_i) = \omega_i^*$
\end{enumerate}
\end{mydef}

\begin{lemma}
Buďte $(G, \cdot, e, ^{-1})$ a $(H, \cdot, e, ^{-1})$ grupy, $f: G \to H$. Potom platí $f$ je homomorfismus grupy $(G, \cdot, e, ^{-1})$ do grupy $(H, \cdot, e, ^{-1}) \Leftrightarrow f$ je homomofizmus grupy $(G, \cdot)$ do grupy $(H, \cdot)$.
\end{lemma}

Pro vektorové prostory a okruhy ...

\begin{mydef}
Buďte $\mathcal{A} = (A, (\omega_i)_{i \in I})$ a $\mathcal{A}^* = (A^*, (\omega^*_i)_{i \in I}$ algebry téhož typu $(n_i)_{i \in I}$ a $f: A \to A^*$ homomorfizmus algebry $\mathcal{A}$ do algebry $\mathcal{A}^*$. $f$ se nazývá:
\begin{enumerate}
	\item \emph{izomorfizmus}, pokud $f$ je bijektivní (v tomto případě říkáme, že $\mathcal{A}$ je izomorfní obraz $\mathcal{A}^*$, a píšeme $\mathcal{A} \cong \mathcal{A}^*$).
	\item \emph{endomorfizmus}, pokud $\mathcal{A} = \mathcal{A}^*$
	\item \emph{automorfismus}, pokud $\mathcal{A} = \mathcal{A}^*$ a $f$ je izomorfismus.
	\item \emph{epimorfizmus}, pokud $f$ je surjektivní (v tomto případě se nazývá $\mathcal{A}^*$ homomorfní obraz $\mathcal{A}$).
	\item \emph{monomorfizmus}, pokud je $f$ injektivní (v tomto případě se $\mathcal{A}$ nazývá izomorfně uzavřená v $\mathcal{A}^*$).
\end{enumerate}

$g \circ f$ zachovává homomorfismus a isomorfismus.

$f^{-1}$ zachovává homomorfismus.
\end{mydef}

\begin{veta}
Buď $(H, \cdot)$ pologrupa, $(H^*, \cdot)$ grupoid a $f: H \to H^*$ homomorfizmus. Potom je podalgebra $(f(H), \cdot)$ grupoidu $(H^*, \cdot)$ pologrupa.

Platí také pro: (abelovské) grupy, (komutativní) okruhy, okruhy s jednotkovým prvkem, svazy, Booleovy algebry, vektorové prostory nad K.
\end{veta}

Rozlišujeme fundamentální operace a odvozené operace (tvořené konečným počtem proměnných a symbolů operací).

Izomorfismus je pouze přeznačení. Algebry jsou "stejné", izomorfizmus zachovává algebraické vlastnosti.

\begin{veta}
(Cayleyova věta o reprezentaci) Buď $(G, \cdot, e, ^{-1})$ grupa. Potom je $G$ izomorfní s podgrupou symetrické grupy $(S_G, \circ, id_G, ^{-1})$. Krátce: Každá grupa je izomorfní s nějakou grupou permutací.
\end{veta}

\section{Kongruence a faktorové algebry}

\begin{mydef}
Buď $\mathcal{A} = (A, (\omega_i)_{i \in I})$ algebra typu $(n_i)_{i \in I}$ a $\pi$ relace ekvivalence na $A$. $\pi$ se nazývá (relace) kongruence na $\mathcal{A} : \Leftrightarrow$ pro všechna $i \in I$, kde $n_i > 0, a_1, \dots, a_{n_i}, b_1, \dots, b_{n_i} \in A$ platí

$$ a_1 \pi b_1 \land \dots \land a_{n_i} \pi b_{n_i} \Rightarrow \omega_i a_i \dots a_{n_i} \pi \omega_i b_1 \dots b_{n_i} $$
\end{mydef}

\begin{veta}
Buď $\mathcal{A} = (A, (\omega_i)_{i \in I})$ algebra a $\pi$ kongruence na $\mathcal{A}$. Potom jsou vztahy

$$ \omega_i^*[a_1)_\pi \dots [a_{n_i}]_\pi := [\omega_i a_1 \dots a_{n_i}]_\pi, n > 0, a_1, \dots a_{n_1} \in A$$

$$ \omega_i^* := [\omega_i]_\pi, n_i = 0 $$

definovány operace $(\omega_i^*)_{i \in I}$ na faktorové množině $A/\pi$
\end{veta}

\begin{mydef}
Algebra $\mathcal{A}/\pi := (A/\pi, (\omega_i^*)_{i \in I})$ se nazývá faktorová algebra algebry $\mathcal{A}$ podle kongruence $\pi$. Často klademe $\omega_i := \omega_i^*$.
\end{mydef}

$(\mathbb{Z}, +, 0, -, \cdot, 1)$ je komutativní okruh s jednotkovým prvkem, který se nazývá okruh zbytkových tříd modulo $n$.

\begin{veta}
Buď $\mathcal{A} := (A, (\omega_i)_{i \in I})$ algebra, $\pi$ kongruence na $\mathcal{A}$. Potom je zobrazení
$$
 \nu =
   \begin{dcases}
     A \to A / \pi \\
     a \mapsto [a]_\pi
   \end{dcases}
$$
surjektivní homomorfizmus algebry $\mathcal{A}$ na $\mathcal{A}/\pi$, který se nazývá přirozený homomorfizmus.
\end{veta}

$\mathcal{A}/\pi$ je homomorfní obraz $\mathcal{A}$ a každý zákon, který platí v $\mathcal{A}$ platí také v $\mathcal{A}/\pi$ (pologrupy, (abelovské grupy), vektorové prostory, (komutativní) okruhy, okruhy s jednotkovým prvkem, svazy, Booleovy algebry). Nemusí být u oboru integrity!

\begin{veta}
(O homomorfizmu) Buďte $\mathcal{A} := (A, (\omega_i)_{i \in I})$ a  $\mathcal{A^*} := (A^*, (\omega_i^*)_{i \in I})$ algebry téhož typu $(n_i)_{i \in I}$ a $f: A \to A^*$ homomorfizmus. Potom je jádro $\pi_j$ kongruencí na $\mathcal{A}$ a existuje přesně jeden injektivní homomorfizmus $g$ z $\mathcal{A}/\pi$ do $\mathcal{A}^*$ takový, že $f = g \circ \nu$ ($\nu$ je přirozené zobrazení).
\end{veta}

Pro podalgebry $(f(A), (\omega_i^*)_{i \in I})$ algebry $\mathcal{A}^*$ platí $(f(A), (\omega_i^*)_{i \in I}) \cong \mathcal{A}/\pi_f$, tedy je každý homomorfní obraz algebry izomorfní s nějakou faktorovou algebrou.

Relace rovnosti $\iota = \{(x, x) | (x \in A\}$ a univerzální relace $\alpha = A \times A$ jsou vždy kongruencemi na $\mathcal{A}$. Platí $\mathcal{A}/\iota \cong \mathcal{A}$ a $|\mathcal{A}/\alpha| \leq 1$. $\mathcal{A}/\iota$ a $\mathcal{A}/\alpha$ jsou triviální faktorové algebry.

\begin{mydef}
Algebra $\mathcal{A}$ se nazývá prostá, má-li pouze triviální kongruence.
\end{mydef}

\subsection{Relace kongruence na grupách a okruzích}

\begin{veta}
Buď $(G, \cdot, e, ^{-1})$ grupa a $\pi$ relace ekvivalence na $G$. Potom platí
\begin{enumerate}[a)]
	\item $\pi$ je kongruence na $(G, \cdot, e, ^{-1}) \Leftrightarrow \pi$ je kongruence na $(G, \cdot)$
	\item Je-li $\pi$ kongruence na $(G, \cdot)$ a $[e]_\pi =: N$, potom platí
	\begin{enumerate}[i)]
		\item $N$ je podgrupa $(G, \cdot, e, ^{-1})$
		\item $x N x^{-1} = \{x y x^{-1} | y \in N\} \subseteq N$ pro všechna $x \in G$ 
		\item $x \pi y \Leftrightarrow x^{-1} y \in N$ pro všechna $x, y \in G$ (tj. $[x]_\pi = x N$ pro všechna $x \in G$).
	\end{enumerate}
\end{enumerate}
\end{veta}

\begin{mydef}
Podgrupa $N$ grupy $(G, \cdot, e, ^{-1})$ se nazývá normální podgrupa grupy $G$ (symbolicky $N \triangleleft G$): $\Leftrightarrow x N x^{-1} \in N$ pro všechna $x \in G$.
\end{mydef}
(v abelovské grupě je každá podgrupa normální)

\begin{lemma}
Por podgrupu $N$ grupy $G$ jsou následující tvrzení ekvivalentní:
\begin{enumerate}
	\item $N$ je normální podgrupa grupy $G$
	\item $\forall x \in G: x N x^{-1} = N$
	\item $\forall x \in G: N x = x N$, tj. pravá třída rozkladu = levá třída rozkladu.
\end{enumerate}
\end{lemma}

\begin{veta}
Buď $(G, \cdot, e, ^{-1})$ grupa, $N \triangleleft G$ a $\pi$ buď binární relace na $G$ definovaná vztahem $x \pi y: \Leftrightarrow x^{-1}y \in N, x,y \in G$. Potom je $\pi$ relace kongruence na $G$, kde $[e]_\pi = N$.
\end{veta}

\begin{veta}
Vztahem $\pi \mapsto [e]_\pi$ je definování bijektivní zobrazení množiny kongruencí na grupě G na množinu všech normálních podgrup grupy G. Inverzní zobrazení je dáno pomocí vztahu $N \mapsto \pi$, kde $x \pi y: \Leftrightarrow x^{-1}y \in N$.
\end{veta}

Chceme-li najít všechny homomorfní obrazy -- až na izomorfizmus -- nějaké grupy $G$, můžeme tedy určit všechny normální podgrupy $N$ grupy $G$ a vytvořit faktorové algebry $G/\pi$ pomocí odpovídajících kongruencí. Pokud normální podgrupě $N$ odpovídá kongruence $\pi$, píšeme $G/N := G/\pi = \{x N | x \in G\}$. Takováto faktorová algebra se nazývá faktorgrupa grupy $G$.

Triviálním kongruencím $\iota = \{(x, x) | x \in G\}$ a $\alpha = G \times G$ odpovídají tzv. triviální normální podgrupy ${e}$ a $G$. $G$ je prostá $\Leftrightarrow G$ má pouze triviální podgrupy.

\begin{veta}
Buď $G$ grupa, $U$ podgrupa, kde $[G:U] = 2$ (index $U$ v $G$). Potom platí $U \triangleleft G$.
\end{veta}

\section{Normální podgrupy}
\section{Ideály okruhů}

\begin{mydef}
Buď $(R, +, 0, -, \cdot)$ okruh a $I$ podokruh okruhu $R$. Potom se $I$ nazývá:
\begin{itemize}
	\item \emph{levý ideál} okruhu R: $\Leftrightarrow \forall r \in R: rI := \{ri | i \in I\} \subseteq I$
	\item \emph{pravý ideál} okruhu R: $\Leftrightarrow \forall r \in R: Ir := \{ir | i \in I\} \subseteq I$
	\item \emph{ideál} okruhu R (formálně $I \triangleleft R$): $\Leftrightarrow \forall r \in R: Ir  \subseteq I \land rI \subseteq I$
\end{itemize}
\end{mydef}

$R$ a $\{0\}$ jsou vždy ideály okruhu $R$, takzvané triviální ideály.

\begin{lemma}
Buď $(R, +, 0, -, \cdot, 1)$ okruh s jednotkovým prvkem a $I$ ideál okruhu $R$. Potom platí $1 \in I \Leftrightarrow I = R$.
\end{lemma}

\begin{veta}
Každé těleso má pouze triviální ideály.
\end{veta}

\begin{veta}
Buď $(R, +, 0, -, \cdot, 1)$ komutativní okruh s jednotkovým prvkem, který má pouze triviální ideály. Potom je $R$ pole nebo $R = \{0\}$
\end{veta}

Komutativní okruh $R \not= \{0\}$ s jednotkovým prvkem je pole $\Leftrightarrow R$ má pouze triviální ideály.

\begin{veta}
Buď $(R, +, 0, -, \cdot)$ okruh.
\begin{enumerate}[a)]
	\item Je-li $\pi$ kongruence na $R$, potom je $I := [0]_\pi$ ideál okruhu R, a platí $R/\pi = R/I = \{x + I | x \in R\}$
	\item Je-li $I$ ideál okruhu $R$ a $\pi$ binární relace na $R$ definovaná vztahem $x \pi y :\Leftrightarrow y - z \in I, x,y \in R$, potom je $\pi$ kongruence na $R$ a $[0]_\pi = I$
	\item $\pi \mapsto [0]_\pi$ definuje bijektivní zobrazení množiny všech kongruencí na $R$ na množinu všech ideálů okruhu $R$. Inverzní zobrazení je dáno vztahem $I \mapsto \pi$, kde $\pi$ je kongruence definovaná v b). 
\end{enumerate}
\end{veta}

Okruh $R$ je prostý $\Leftrightarrow$ $R$ má pouze triviální kongruence $\Leftrightarrow$ $R$ má pouze triviální ideály $\{0\} := (0)$ a $R$.

\begin{veta}
Komutativní okruh $R \not= \{0\}$ s jednotkovým prvkem je prostý právě tehdy, když je pole
\end{veta}

\section{Přímé součiny}

\begin{mydef}
Buďte $\mathcal{A}_k = (A_k, \omega_i^{(k)_{i \in I}}, k \in K$ algebry téhož typu $(n_i)_{i \in I}$ a $A := \prod\limits_{k \in K} A_k = \{(a_k)_{k \in K} | a_k \in A_k\}$ kartézský součin všech množin $A_k$. Pro všechna $i \in I$ buď operace $\omega_i$ na $A$ definována vztahem:

$$ \omega_i(a_k^{(1)})_{k \in K} \dots (a_k^{n_i})_{k \in K} := (\underbrace{\omega_i^{(k)}a_k^{(1)} \dots a_k^{(n_i)}}_{\in A_k})_{k \in K} \text{pro} n_i > 0 $$

$$ \omega_i := (\omega_i^{(k)})_{k \in K} \text{ pro } n_i = 0 $$

Algebra $(A, (\omega_i)_{i \in I})$ se nazývá přímý součin algeber $\mathcal{A}_k$ a značí se $\prod\limits_{k \in K} \mathcal{A}_k$.
\end{mydef}

\begin{veta}
Pokud platí při vhodných termech $t_1, t_2$ zákon tvaru $\forall x_1, \dots x_n : t(x_1, \dots, x_n) = t_2(x_1, \dots, x_n)$ ve všech algebrách $\mathcal{A}_k, k \in K$, potom platí také v $\prod\limits_{k \in K}\mathcal{A}_k$.
\end{veta}

Důsledek: Přímé součiny pologrup (grup, vektorových prostorů, okruhů, Booleových algeber) jsou opět pologrupy (grupy, vektorové prostoru, okruhy, Booleovy algebry). Nikoliv však pro obory integrity!

Přímý součin je až na izomorfizmus komutativní ($A_1 \times A_2 \cong A_2 \times A_1$ a asociativní (uzávorkování).

\begin{veta}
Grupa $C_n \times C_m$ je cyklická $\Leftrightarrow \text{NSD}(m, n) = 1$.
\end{veta}

Je-li $n = p_1^{e_1} \dots p_k^{e_k}$ rozklad na prvočinitele čísla $n \in N$, potom platí $C_n \cong C_{p_1{e_1}} \times \dots \times C_{p_k{e_k}}$.

\begin{veta}
(Hlavní věta o konečně generovaných abelovských grupách) Je-li $G = \left< x_1, \dots, x_m \right>$ abelovská grupa generovaná prvky $x_1, \dots x_m$, potom platí:

$$ G \cong C_\infty^k \times C_{n_1} \times C_{n_r} $$

přičemž $k \geq 0 (C_\infty^0 := \{e\}), n_i = \mathbb{N}, r \geq 0$. Přitom platí $G$ je konečná $\Leftrightarrow k = 0$.
\end{veta}
($C_\infty$ označuje nekonečnou cyklickou grupu.)

















%%%%%%%%%%%%%%%%%%%%%%%%%%%%%%%%%%%%%%%%%%%%%%%%%%%%%%%%%%%%%%%%%%%%%%%%%%%%%%%%
%%%%%%%%%%%%%%%%%%%%%%%%%%%%%%%%%%%%%%%%%%%%%%%%%%%%%%%%%%%%%%%%%%%%%%%%%%%%%%%%
\chapter{Obory integrity a dělitelnost} \label{cha:10}

1. semestr, MAT, \texttt{Zaklady\_obecne\_algebry.pdf}, 4, 5. kapitola

(okruhy polynomů, pravidla dělitelnosti, Gaussovy a Eukleidovy okruhy)

\section{Polynomy}
\subsection{Konstrukce okruhů polynomů}

\begin{mydef}
Buď $(R, +, 0, -, \cdot, 1)$ komutativní okruh s jednotkovým prvkem. Výraz tvaru $\sum_{k=0}^\infty a_kx^k$, kde $a_k \in R$ pro všechna $k \in \mathbb{N}_0$ a množina $\{k \in \mathbb{N}_0 | a_k \not= 0\}$ je konečná, se nazývá polynom neurčití $x$ nad $R$. Množinu všech polynomů neurčité $x$ nad R označíme symbolem $R[x]$. Definujeme nyní operace $+, 0, -, \cdot, 1$ na $R[x]$ tak, aby $(R[x], +, 0, -, \cdot, 1)$ byl opět komutativní okruh s jednotkovým prvkem:

$ \sum_{k = 0}^\infty a_k x^k + \sum_{k = 0}^\infty b_k x^k  := \sum_{k = 0}^\infty (a_k + b_k) x^k$,
$0 := \sum_{k = 0}^\infty 0 x^k $,
$-(\sum_{k = 0}^\infty a_k x^k) := \sum_{k = 0}^\infty (-a_k) x^k$,
$\sum_{k = 0}^\infty a_k x^k \cdot \sum_{k = 0}^\infty b_k x^k := \sum_{k = 0}^\infty \sum_{l = 0}^n a_l k_{k-l} x^k$,
$1 := \sum_{k = 0}^\infty \delta_k x^k$
\end{mydef}

\begin{veta}
$(R[x], +, 0, -, \cdot, 1)$ je komutativní okruh s jednotkovým prvkem.
\end{veta}

\begin{mydef}
Je-li $p(x) = \sum_{l = 0}^n a_k x^k$, kde $a_n \not= 0$, pak se $n$ nazývá stupeň polynomu $p(x)$ (píšeme $n = \text{grad } p(x)$). 
\end{mydef}

\begin{mydef}
Je-li $p(x) = \sum_{k = 0}^n a_k x^k \in R[x]$, pak se prvky $a_x$ nazývají koeficienty polynomu $p(x)$. $0 \in R[x]$ je nulový polynom a $a \in R \subseteq R[x]$ se nazývá konstantní polynom. Platí-li $\text{grad } p(x) = n$ a $a_n = 1$, pak se $p(x)$ nazývá normalizovaný polynom. Polynomy tvaru $ax + b$, kde $a \not= 0$ se nazývají lineární polynomy.
\end{mydef}

\begin{veta}
Je-li $R$ obor integrity, potom je také $R[x]$ obor integrity, a pro $p(x), q(x) \in R[x] \backslash \{0\}$ platí $\text{grad } (p(x)q(x)) = \text{grad } p(x) + \text{grad } q(x)$.
\end{veta}

Není-li R obor integrity, pak ani $R[x]$ není obor integrity, neboť $r$ je podokruh  okruhu $R[x]$.

\subsection{Polynomy $n$ neurčitých $x_1, \dots x_n$}

Indukcí definujeme

$$R[x_1] := R[x], R[x_1, \dots, x_n] := (r[x_1, \dots, x_{n-1}])[x_n], n > 1$$

Potom platí

$$ R[x_1, \dots, x_n] = \{ \sum\limits_{0 \leq i_i, \dots, i_n \leq m} a_{i_i \dots i_n} x_1^{i_1} \dots x_n^{i_n} | m \in \mathbb{N}_0, a_{i_1} \dots a_{i_n} \in R \} $$

\subsection{Polynomy a funkce}
\paragraph{Princip dosazování} Buď $(R, +, 0, -, \cdot, 1)$ komutativní okruh s jednotkovým prvkem a $p(x) = a_n x^n + \dots + a_1 x + a_0 \in R[x]$. Pro $a \in R$ je potom $p(a) := a_n a^n + \dots + a_1 a + a_0$ opět prvkem z $R$ který se nazývá \emph{hodnota polynomu v $a$}. Funkce

$$ \begin{dcases}
R \to R \\
a \mapsto p(a)
\end{dcases} $$

se nazývá polynomiální funkce indukovaná polynomem $p(x)$ a často se taká označuje $p$.

\begin{veta}
Zobrazení
$$ \begin{dcases}
R[x] \to R \\
p(x) \mapsto p(a)
\end{dcases} $$
je pro pevně dané $a \in R$ surjektivní homomorfizmus $R[x]$ na $R$.
\end{veta}

\begin{mydef}
Buď $p(x) \in R[x]$ (komutativní okruh s jednotkovým prvkem). Potom se $a \in R$ nazývá kořen polynomu $p(x): \Leftrightarrow p(a) = 0$. Polynom $p(x)$ se nazývá dělitelný polynomem $q(x) \in R[x]$ (formálně $q(x) | p(x)$): $\Leftrightarrow p(x) = q(x)r(x)$ kde $r(x) \in R[x]$.
\end{mydef}

\begin{veta}
Je-li $a$ kořen polynomu $p(x)$, pak je $p(x)$ dělitelný lineárním polynomem $x - a$ (a opačně).
\end{veta}

Nechť je $R$ obor integrity (např. $R = \mathbb{Z}$ nebo $R$ pole). Je-li $\text{grad } p(x) = n$ a platí $(x - a)^k | p(x)$ , tj. $p(x) = (x -a)^k q(x)$, potom je $k + \text{ grad } q(x) = \text{ grad } p(x) = n$, z čehož plyne $k \leq n$.

\begin{mydef}
Buď $p(x) \in R[x] \backslash \{0\}$ a nechť $a \in R$ je kořenem $p(x)$. Potom největší číslo $k \in \mathbb{N}$ takové, že $(x - a)^k | p(x)$, se nazývá násobnost kořene $a$.

$l \leq \text{ grad } p(x)$
\end{mydef}

\begin{veta}
Nechť $a_1, \dots a_r$ jsou po dvou různé kořeny polynomu $p(x) \in R[x] \backslash \{0\}$ s násobností $k_1 ,\dots, k_r$. Potom platí

$$ (x - a)^{k_1} \dots (x - a)^{k_r} | p(x)$$
\end{veta}

Potom platí: $k_1 + \dots + k_r \leq \text{ grad } p(x)$

\begin{veta}
Buďte $p(x), q(x) \in R[x] \backslash \{0\}, \text{ grad } p(x) \leq n$ a $p(b_i) = q(b_i)$ pro $n+1$ po dvou různých prvků $b_0, \dots, b_n$ množiny $R$. Potom platí $p(x) = q(x)$.
\end{veta}

\begin{mydef}
Pole $K$ se nazývá algebraicky uzavření, jestliže každý polynom $p(x) \in K[x] \ K$ má alespoň jeden kořen.
\end{mydef}
Pokud má nad oborem integrity každý lineární polynom kořen, pak je tento obor integrity pole ($ax - 1 (a \not=0)$ má kořen $c \Rightarrow ac = 1 \Rightarrow c = a^{-1}$).

\begin{veta}
(Gaussova základní věta algebry) Pole $\mathbb{C}$ je algebraicky uzavřené.
\end{veta}

\begin{veta}
Je-li $K$ pole, potom jsou následující tvrzení ekvivalentní:
\begin{enumerate}[a)]
	\item $K$ je algebraicky uzavřené
	\item Pro všechna $p(x) \in K[x]$, kde $\text{ grad } p(x) = n > 0$, platí $p(x) = c(x - b_1)^{k_1} \dots (x - b_r)^{k_r}$, kde $b_1, \dots, b_r, c \in K$ a $k_1 + \dots + k_r = n$.
\end{enumerate}
\end{veta}

Výpočet kořenů nad poli:
\begin{enumerate}[1)]
	\item $\text{grad } p(x) = 1$: $ax + b$:  triviální
	\item $\text{grad } p(x) = 2$: $p(x) = ax^2 + bx + c (a \not= 0)$ má kořeny $\frac{-b \pm \sqrt{b^2 - 4ac}}{2a}$ (2 resp. 4 zde označuje 1 + 1 resp. 1+1+1+1; vyjádření kořenů musí existovat a musí být $1+1 \not= 0$.
	\item $\text{grad } p(x) = 3, 4$ Cardanovy vzorce
	\item $\text{grad } p(x) > 4$  nejsou obecné vzorce
\end{enumerate}

\subsection{Interpolace pomocí polynomů}
Buď $K$ pole a $f: K \to K$ funkce. \textbf{Zadáno}: $b_i = f(a_i)$  pro po dvou různá $a_i \in K, 1 \leq i \leq n$ (např. výsledek řady měření). \textbf{Hledá se}: $p(x) \in K[x]$, kde $p(a_i) = b_i = f(a_i), 1 \leq i \leq n$ a $\text{ grad } p(x) < n$. (Existuje nejvýše jeden takový polynom $p(x)$ z $p(a_i), 1 \leq i \leq n$, kde $\text{grad } p(x), \text{ grad } q(x) < n$ totiž plyne $p = q$.)

\subsubsection{Lagrangerovy interpolační vzorce}
Buď
$$q_i(x) := \prod\limits_{1 \leq j \leq n, j \not=i} (x - a_j) = \dots (x - a_{i-1}) (x - a_{i+1}) \dots (x - a_n)$$
Potom platí:
$$q_i(a_k) =  \begin{dcases}
0 \text{pro} i \not= k \\
\prod\limits_{1 \leq j \leq n, j \not=i} (a_k - a_j) \not= 0 \text{pro } i = k
\end{dcases}$$
Pro 
$$ p(x) := \sum\limits_{i = 1}^n b_i \frac{q_i(x)}{q i (a_i)} $$
Potom platí $$ p(a_j) = b_j, 1 \leq j \leq n $$

Je-li $K$ konečné pole (např. $K = \mathbb{Z}_p, p$ prvočíslo), $f: K \to K$, potom existuje polynom $p(x) \in K[x]$ takový, že $f(a) = p(a)$ pro všechna $a \in K$.

% \subsubsection{Newtonovy interpolačnı́ vzorce}
% XXX ???

\section{Obory integrity a dělitelnost}
\subsection{Jednoduchá pravidla dělitelnosti}


\begin{mydef}
Buď $(I, +, 0, -, \cdot, 1)$ obor integrity. Jsou-li $a, b \in I$, potom říkáme, že prvek $a$ je dělitelný prvkem $b$ a $b$ se nazývá dělitel prvku $a$ ($b$ \uv{dělí} $a$, formálně $b|a$): $\Leftrightarrow \exists c \in I: a = bc$.
\end{mydef}

Elementární pravidla dělitelnosti:
\begin{enumerate}[1)]
	\item $\forall a \in I: a | 0$
	\item $\forall a \in I: 1 | a$
	\item $\forall a \in I: a | a$
	\item $\forall a,b,c \in I: a|b \land b|c \Rightarrow a|c$
	\item $\forall a,b,c \in I: a|b \Rightarrow a|bc$
	\item $\forall a,b,c \in I: a|b \land a|c \Rightarrow a|b+c$
	\item $\forall a,b,c \in I, c \not= 0: a|b \Leftrightarrow ac|bc$
	\item $\forall a,b,c,d \in I: a|b \land c|d \Rightarrow ac|bd$
	\item $\forall a,b \in I, n \in \mathbb{N}: a|b \Rightarrow a^n|b^n$
\end{enumerate}

\begin{mydef}
Buď $(I, +, 0, -, \cdot, 1)$ obor integrity. Dělitel prvku 1 se nazývá jednotka oboru integrity $I$. Buď $E(I)$ množina všech jednotek $I$. Prvky $a, b \in I$ se nazývají asociované (formálně $a \sim b$): $\Leftrightarrow \exists e \in E(I): a = be$.
\end{mydef}

\begin{veta}
\begin{enumerate}[a)]
	\item $e \in I$ je jednotka oboru integrity $I \Leftrightarrow \exists f \in I: ef = 1$
	\item $(E(I(, \cdot))$ je abelovská grupa, která se nazývá grupa jednotek oboru integrity  $I$.
	\item $\sim$ je relace kongruence na $(I, \cdot)$
	\item $\forall a, b \in I: a \sim b \Leftrightarrow a|b \lor b|a$
\end{enumerate}
\end{veta}

\begin{mydef}
Buď $(I, +, 0, -, \cdot, 1)$ obor integrity, $a \in I$.
\begin{description}
	\item[Triviální dělitelé prvku $a$] jsou všechna $e \in E(I)$ a všechna $b$ taková, že $b \sim a$
	\item[Vlastní dělitelé prvku $a$] jsou všechna $b$ taková, že $b | a, b \not\in E(I)$ a $b \not\sim a$
\end{description}
\end{mydef}

\begin{mydef}
Prvek $a \in I \backslash E(I), a \not= 0$ se nazývá ireducibilní prvek: $\Leftrightarrow a$ má pouze triviální dělitele. (např. prvočísla v $\mathbb{Z}$.)
\end{mydef}

\begin{mydef}
$p \in I \backslash E(I), p \not= 0$ se nazývá prvočinitel: $\Leftrightarrow p|ab \Rightarrow p|a \lor p|b$
\end{mydef}

Prvočinitel $\Leftrightarrow$ je ireducibilní prvek.

\section{Gaussovy okruhy}

\begin{mydef}
Obor integrity $I$ se nazývá Gaussův okruh: $\Leftrightarrow$ Ke každému prvku $a \in I \backslash E(I), a \not= 0$, existují prvočinitelé $p_1, \dots, p_r$ (nikoliv nutně po dvou různí) tak, že platí $a = p_1 \dots p_r$
\end{mydef}

\begin{veta}
(Jednoznačnost rozkladu na prvočinitele) Buď $I$ Gaussův okruh, $a \in I \backslash E(I), a \not= 0, a = p_1^{(1)} \dots p_{r_1}^{(1)} = p_1^{(2)} \dots p_{r_2}^{(2)}$, kde $p_i^{(1)}, p_j^{(2)}$ jsou prvočinitelé. Potom je $r_1 = r_2 := r$ a existuje permutace množiny $\{1, \dots r\}$ taková, že $p_i^{(1)} \sim p_{\pi(i)}^{(2)}, i = 1, \dots, r$.
\end{veta}

$\mathbb{Z}$ a $K[x]$ (K pole) jsou Gaussovy okruhy.

\begin{mydef}
Buď $I$ obor integrity, $a_1, \dots, a_n \in I$.
\begin{enumerate}[1)]
	\item $d \in I$ se nazývá \emph{největší společný dělitel} (NSD) prvků $a_1, \dots a_n \in I: \Leftrightarrow$ (i) $d | a_i, i = 1, \dots, n$ a (ii) $\forall t \in I: t|a_i, i = 1, \dots, n \Rightarrow t | d$
	\item $v \in I$ se nazývá \emph{nejmenší společný násobek} (NSN) prvků $a_1, \dots a_n \in I: \Leftrightarrow$ (i) $a_i | v, i = 1, \dots, n$ a (ii) $\forall w \in I: a_i | w, i = 1, \dots, n \Rightarrow v | w$
\end{enumerate}
\end{mydef}

\begin{veta}
V Gaussově okruhu $I$ je každý ireducibilní prvek prvočinitelem.
\end{veta}

\begin{veta}
Je-li $I$ Gaussův okruh, $a \in I \backslash E(I), a \not=0$, potom platí $a = e p_1^{e_1} \dots p_r^{e_r}$, kde $e \in E(I), p_1, \dots p_r$ jsou normovaní navzájem různí prvočinitelé $e_i \in \mathbb{N}$. 
\end{veta}

\begin{lemma}
Buď $I$ Gaussův okruh, $a, b \in I \backslash \{0\}, a = fp_1^{f_1} \dots p_r^{f_r}, b = g p_1^{g_1} \dots p_r^{g_r}$ ($p_j$ normovaní navzájem různí prvočinitelé, $f_j, g_j \in \mathbb{N}_0, f,g \in E(I)$). Potom platí $a|b \Leftrightarrow f_j \leq g_j$ pro $j = 1, \dots, r$.
\end{lemma}

\begin{veta}
Buď $I$ Guassův okruh, $a_1, \dots, a_n \in I, a_i \not= 0, a_i = e_i p_1^{e_{1i}} \dots p_r^{e_{ri}}, e_i \in E(I), p_j$ navzájem různí normovaní prvočinitelé, $e_{ji} \in \mathbb{N_0}$. Potom platí:

$$ \text{NSD}(a_1, \dots, a_n) = p_1^{\text{min}_{1 \leq n \leq n}(e_{1i})} \dots p_r^{\text{min}_{1 \leq n \leq n}(e_{ri})}$$

a

$$ \text{NSN}(a_1, \dots, a_n) = p_1^{\text{max}_{1 \leq n \leq n}(e_{1i})} \dots p_r^{\text{max}_{1 \leq n \leq n}(e_{ri})}$$

Jsou-li některá $a_i = 0$, potom je NDS$(a_1, \dots, a_n) = $NSD$(a_i|a_i \not= 0)$; jsou-li všechna $a_i = 0$, potom je NDS$(a_1, \dots a_n) = 0$. Jsou-li některá $a_i = 0$, pak je $NSN(a_1, \dots, a_n) = 0$.
\end{veta}

\begin{veta}
Buď $I$ Gaussův okruh a $\cap, \cup$ binární operace na $I / \sim= \{[a]_\sim | a \in I\}$ definovaní vztahy:

$$[a]_\sim \cap [b]_\sim := [NSD(a, b)]_\sim $$
$$[a]_\sim \cup [b]_\sim := [NSN(a, b)]_\sim $$

Potom jsou $\cap, \cup$ korektně definovány (tj. nezávisle na volbě reprezentantů) a $(I / \sim, \cap, \cup)$ je svaz s nulovým prvkem $[1]_\sim = E(I)$ a jednotkovým prvkem $[0]_\sim = \{0\}$ (svaz dělitelů). Příslušné uspořádání $\leq$ je dáno vztahem: $[a]_\sim \leq [b]_\sim : \Leftrightarrow a|b$.
\end{veta}

$(\mathbb{Z}/ \sim, \cap, \cup) \equiv (\mathbb{N}_0, NSD, NSN)$

\section{Eukleidovy okruhy}
\begin{mydef}
Obor integrity $I$ se nazývá Eukleidův okruh: $\Leftrightarrow$ existuje zobrazení $H: I \backslash \{0\} \to \mathbb{N}_0$ (Eukleidovské ohodnocení) s následující vlastností: pro všechna $a \in I \backslash \{0\}, b \in I$ existují $q, r \in I$ tak, že $b=aq + r$, kde $r = 0 \lor H(r) < H(a)$ (dělení se zbytkem).
\end{mydef}

$\mathbb{Z}$ je Ekleidův okruh, kde $H(a) := |a|$. Každé pole je Eukleidův okruh ($q = a^{-1}b, r = 0$).

\begin{veta}
$K[x]$ (K pole) je Eukleidův okruh, kde $H(p(x)) := \text{ grad }p(x)$, tj. pro $p(x) \not= 0$, $p_1(x)$ libovolné, je $p_1(x) = p(x)q(x) + r(x)$, kde $r(x) = 0$ nebo grad $r(x) < \text{grad }p(x)$.
\end{veta}

Libovolný polynom $p(x) \in K[x]$ a libovolný prvek $a \in K$ existuje $r \in K$ tak, že $p(x) = (x - a)q(x) + p(a)$.

\begin{veta}
Každý Eukleidův okruh je Gaussův okruh.
\end{veta}













%%%%%%%%%%%%%%%%%%%%%%%%%%%%%%%%%%%%%%%%%%%%%%%%%%%%%%%%%%%%%%%%%%%%%%%%%%%%%%%%
%%%%%%%%%%%%%%%%%%%%%%%%%%%%%%%%%%%%%%%%%%%%%%%%%%%%%%%%%%%%%%%%%%%%%%%%%%%%%%%%
\chapter{Teorie polí} \label{cha:11}

1. semestr, MAT, \texttt{Zaklady\_obecne\_algebry.pdf}, 6. kapitola

(minimální pole, rozšíření pole, konečná pole a jejich konstrukce)

\section{Minimální pole}
\begin{mydef}
Pole $(K, +, 0, -, \cdot, 1)$ se nazývá minimální, pokud nemá žádná jiná podpole než sama sebe.
\end{mydef}

\begin{veta}
Každé pole má vždy jediné podpole, které je minimální.
\end{veta}

\begin{lemma}
Buď $(R, +, 0, -, \cdot, 1)$ okruh s jednotkovým prvkem. Pak $\{n \cdot 1 | n \in \mathbb{Z}\}$ je komutativní podokruh okruhu R s tímtéž jednotkovým prvkem 1, totiž podokruh generovaný prvkem 1.
\end{lemma}

\begin{mydef}
Buď $(R, +, 0, -, \cdot)$ okruh. Pak symbolem char $R$ označíme charakteristiku okruhu $R$, tj. nejmenší číslo $n \in \mathbb{N}$ takové, že pro každé $a \in R$ platí $n \cdot a = 0$ (kde $n \cdot a := \underbrace{a + a + \dots + a}_{n\text{-krát}}$. Pokud takové číslo neexistuje, pak klademe char $R = 0$.

Je-li $(R, +, 0, -, \cdot, 1)$ okruh s jednotkovým prvkem a $n \in \mathbb{N}$, pak pro každé $a \in R$ platí $n \cdot a = 0$, právě když platí $n \cdot 1 = 0$ (platí-li $n \cdot 1 = 0$ a je-li $a \in R$ libovolný prvek, pak máme $n \cdot a = \underbrace{a + a + \dots + a}_{n\text{-krát}} = (\underbrace{1 + 1 + \dots + 1}_{n\text{-krát}})\cdot a = (n \cdot 1) \cdot a = 0 \cdot a = 0$; opačná implikace je zřejmá). Je-li tedy R okruh s jednotkovým prvkem $1$, pak char R je nejmenší číslo $n \in N$ pro něž platí $n \cdot 1 = 0$, případně char $R = 0$ pokud takové číslo neexistuje. Odtud ihned plyne, že platí

$\text{char }R = \begin{dcases}
o(1), \text{ pokud } o(1) \in \mathbb{N}, \\
0, \text{ pokud } o(1) = \infty
\end{dcases}$

($o(1)$ značí řád prvku $1$ v abelovské grupě $(R, +)$, tedy $o(1) = |\{n \cdot 1 | n \in \mathbb{Z}\}|$ pokud je tato kardinalita konečná, jinak $\infty$.
\end{mydef}

$\text{char }R = \begin{dcases}
|\{n \cdot 1 | n \in \mathbb{Z}\}|, \text{ pokud se jedná o končnou kardinality }, \\
0, \text{ jinak }
\end{dcases}$

Okruh zbytkových tříd: $\mathbb{Z}_n$: char $\mathbb{Z}_n = n (n \in \mathbb{N}_0)$, char $\mathbb{Z} = 0$.

\begin{lemma}
Buď $(R, +, 0, -, \cdot, 1)$ okruh s jednotkovým prvkem a nechť $m = \text{ char }R$. Potom $\{n \cdot 1 | n \in \mathbb{Z}\} \cong \mathbb{Z}_m$
\end{lemma}

\begin{lemma}
\begin{enumerate}[1)]
	\item Je-li $R$ obor integrity a $m = \text{ char }R$, potom také $\{n \cdot 1 | n \in \mathbb{Z}\}$, a tedy i $\mathbb{Z}_m$ je obor integrity, takže platí $m=0$ nebo $m \in \mathbb{P}$ ($\mathbb{P}$ značí množinu všech prvočísel).
	\item Je-li $R$ obor integrity a char $R \in \mathbb{P}$, potom $\{n \cdot 1 | n \in \mathbb{Z}\}$ je pole.
\end{enumerate}
\end{lemma}

\begin{veta}
Buď $(K, +, 0, -, \cdot, 1)$ pole takové, že char $K \in \mathbb{P}$. Potom $\{n \cdot 1 | n \in \mathbb{Z}$ je minimální podpole pole K. V tomto případě tedy platí: minimální podpole pole K je izomorfní se $\mathbb{Z}_m$, kde $m = \text{ char } K$.
\end{veta}

\begin{veta}
Buď $(K, +, 0, -, \cdot, 1)$ pole, kde char $K = 0$. Potom je $\{ \frac{n \cdot 1}{m \cdot 1} | n \in \mathbb{Z}, m \in \mathbb{Z} \backslash \{0\}\}$ nejmenším podpolem  a tudíž minimálním podpolem pole $K$. Toto minimální podpole je izomorfní s $\mathbb{Q}$. Přitom jsme položili $\frac{n \cdot 1}{m \cdot 1} := (n \cdot 1)(m \cdot 1)^{-1}$.
\end{veta}

Každé minimální pole je izomorfní se $\mathbb{Z}_p (p \in \mathbb{P})$ nebo $\mathbb{Q}$.

\section{Rozšíření pole}
\begin{mydef}
Buďte $K, L$ pole a $K$ podpole pole $L$. Potom $L$ se nazývá nadpole nebo rozšíření pole $K$.
\end{mydef}

Pole $L$ se nazývá \emph{kořenové pole} polynomu $f(x)$ vzhledem ke $K$, pokud je rozšířením pole s právě $n$ kořeny.

Buď $L$ takové kořenové pole polynomu $f(x)$ vzhledem ke $K$. Potom je

$K(\alpha_1, \dots, \alpha_n) := \bigcap \{M \subseteq L | M \text{ podpole pole } L, K \subseteq M, \alpha_1, dots, \alpha_n \in M \}$

nejmenší podpole pole L, které obsahuje pole $K$ a prvky $\alpha_1, \dots, \alpha_n$. $K(\alpha_1, \dots, \alpha_n)$ se nazývá \emph{rozkladové pole} polynomu $f(x)$ vzhledem ke K.

\begin{veta}
(Kroneckerova věta) Ke každému $f(x) \in K[x], f(x) \not= 0$, existuje kořenové pole a tedy rozkladové pole polynomu $f(x)$ vzhledem ke K.
\end{veta}

Je-li $L$ nadpole pole $K$, potom je $L$ také vektorovým prostorem nad $K$. Vztahem $\text{dim}_K K =: [L : K]$ definujeme tzv. stupeň rozšíření $L$ pole $K$. Je-li $[L : K] < \infty$, pak se $L$ nazývá \emph{konečné rozšíření} pole $K$.

\begin{enumerate}
	\item $[ L : K] = 1 \Leftrightarrow L = K$
	\item Je-li $p(x) \in K[x]$ ireducibilní polynom stupně $k$, pak existuje rozšíření $L$ pole $K$ a prvek $\alpha \in L$ tak, že $p(x) = 0$ a $\{1, \alpha, \dots, \alpha^{k-1}\}$ je báze $L$ nad $K$. Tedy platí $ [L : K] = k$.
\end{enumerate}

\begin{mydef}
Buď $L$ nadpole pole $K$ a $\alpha \in L$.
$\alpha$ se nazývá \emph{algebraický} prvek nad $K: \Leftrightarrow \exists f(x) \in K[x] \backslash \{0\}: f(\alpha) = 0$. \\
$\alpha$ se nazývá \emph{transcendentní} prvek nad $K: \Leftrightarrow \not\exists f(x) \in K[x] \backslash \{0\}: f(\alpha) = 0$.
\end{mydef}

\begin{mydef}
Je-li $L$ nadpole pole $K$ a $S \subseteq L$, pak definujeme rozšíření $K(S)$ pole $K$ takto:

$$K(S) := \bigcap \{E \subseteq L | E \text{ je podpole pole L, které obsahuje } K \cup S\}$$

Je-li $S = \{u_1, \dots, u_r\}$ konečné, pak píšeme $K(S) =: K(u_1, \dots, u_r)$. Je-li speciálně $S = \{\alpha\}$ jednoprvkové, pak píšeme $K(S) =: K(\alpha)$ (\uv{jednoduché rozšíření} pole K).
\end{mydef}

\section{Konečná pole (Galoisova pole)}

\begin{veta}
Řád každého konečného pole je mocnina prvočísla $p^n (p \in \mathbb{P}, n \in \mathbb{N})$. Obráceně ke každé mocnině čísla $p^n$ existuje až na izomorfizmus jediné pole $K$ takové, že $|K| = p^n$.
\end{veta}
Způsob zápisu pro $K$, kde $|K| = p^n: K = GF(p^n)$ (Galoisovo pole).

\begin{veta}
Je-li $K$ konečné pole, pak je grupa $(K \backslash \{0\}, \cdot)$ cyklická.
\end{veta}

\subsection{Konstrukce konečného pole $K$}

Každý generátor grupy $(K \ \{0\}, \cdot)$ se nazývá primitivní prvek $K$. Je-li $\alpha$ primitivní prvek $K$, pak $K = \{0, 1, \alpha, \alpha^2, \dots, \alpha^{|K|-2}\}$. Buď $\mathbb{Z}_q, q \in \mathbb{P}$, minimální podpole pole $K$. Pak pro libovolná primitivní prvek $\alpha$ z $K$ platí $K \cong \mathbb{Z}_q(\alpha)$ a $\alpha$ je algebraický prvek nad $\mathbb{Z}_q$ (neboť je kořenem polynomu $x^{|K|-1} - 1 \in \mathbb{Z}_q[x]$). Buď $f(x)$ minimální polynom kořene $\alpha$ vzhledem k $\mathbb{Z}_q$. Potom je $f(x)$ ireducibilní a platí

$$ \mathbb{Z}_q(\alpha) = \{a_0 + a_1 \alpha + \dots + a_{n-1}\alpha^{m-1} | a_i \in \mathbb{Z}_q\}$$

kde $m = \text{ grad } f(x)$. Odtud dostáváme $|\mathbb{Z_q}(\alpha)| = q^m$ a z podmínky $|\mathbb{Z}_q(\alpha)| = |K| = p^n$ nyní vyplývá $q = p$ a $m = n$.

Při určování konečného pole $K = GF(p^n)$, tj. při sestavování tabulek jeho operací, lze proto postupovat následujícím způsobem:
\begin{enumerate}
	\item Za minimální podpole pole $K$ se vezme $\mathbb{Z}_p$.
	\item Zvolíme normovaný ireducibilní polynom $q(x) \in \mathbb{Z}_p[x]$ stupně $n$. Nechť např. $q(x) = x^n - a_{n-1}x^{n-1} - \dots - a_1 x - x_0$ kde $a_i \in \mathbb{Z}_p$.
	\item Položíme $q(\alpha) = 0$ a uvažujeme bází $\{1, \alpha, \dots, \alpha^{n-1}\}$ vektorového prostoru $GF(p^n)$ nad $\mathbb{Z}_p$ (víme, že $[GF(p^n): \mathbb{Z}_p] = n$). Spočítáme použitím $q(\alpha = 0)$ (tj. $a^n = a_0 + x_1 \alpha + \dots + a_{n-1}x^{n-1}$ mocniny $\alpha$. Platí-li $\alpha^{p^n-1} = 1$ pro $1 \leq j < p^n - 1$, je $\alpha$ primitivní prvek $GF(p^n)$. Jinak učiníme další pokus s novým polynomem $q(x)$.
\end{enumerate}













%%%%%%%%%%%%%%%%%%%%%%%%%%%%%%%%%%%%%%%%%%%%%%%%%%%%%%%%%%%%%%%%%%%%%%%%%%%%%%%%
%%%%%%%%%%%%%%%%%%%%%%%%%%%%%%%%%%%%%%%%%%%%%%%%%%%%%%%%%%%%%%%%%%%%%%%%%%%%%%%%
\chapter{Metrické prostory} \label{cha:12}

1. semestr, MAT, \texttt{Zaklady\_funkcionalni\_analyzy\_opr.pdf}, 1. -  6. kapitola

(příklady, konvergence posloupností, spojitá a izometrická zobrazení, úplnost, Banachova věta o pevném bodu)

\section{Definice, příklady}
\begin{mydef}
Metrickým prostorem budeme rozumět dvojici $\mathcal{X} = (X, \varrho)$, kde $X$ je množina, jejíž prvky nazýváme bodu a $varrho$ je tzv. vzdálenost (metrika), což je nezáporná reální funkce $varrho(x, y)$, která je definována pro každou dvojicí $x, y \in X$ (tedy $varrho: X \times X \to \mathbb{R}_0^+$ je zobrazení) a která splňuje tyto podmínky:
\begin{enumerate}[1)]
	\item $\forall x, y \in X: varrho(x, y) = 0 \Leftrightarrow x = y$
	\item $\forall x, y \in X: varrho(x, y) = varrho(y, x)$ (symetrie)
	\item $\forall x, y, z \in X: varrho(x, y) + varrho(y, z) \geq varrho(x, z)$ (trojúhelníková nerovnost)
\end{enumerate}
\end{mydef}

Příklady:
\begin{itemize}
	\item Diskrétní metrický prostor $(X, varrho)$:
	$$varrho(x, y) = \begin{dcases}
	0 \text{v případě} x = y\\
	1 \text{v případě} x \not= y
	\end{dcases}$$
	\item Metrický prostor $\mathbb{R}^1$ se vzdáleností $varrho(x, y) = |x-y|$
	\item $n$-rozměrný euklidovský prostor $\mathbb{R}^n$ se vzdáleností $varrho(x, y) = \sqrt{\sum\limits_{k = 1}^n}(y_k - x_k)^2$. Tento metrický prostor $(\mathbb{R}^n, varrho)$ značíme $\mathbb{R}^n_1$.
	\item Metrický prostor $\mathbb{R}^n_0$ používá vzdálenost $varrho_0(x, y) = \max\limits_{1 \leq k \leq n} |y_k - x_k|$
	\item Množina $C(\left<a, b \right>)$ (často též značená $C^0\left<a,b\right>$) všech spojitých reálných funkcí, které jsou definovány na intervalu $\left<a, b\right>$, se vzdáleností: $varrho(f, g) = \max\limits_{a \leq t \leq b} | g(t) - f(t)$ také tvoří metrický prostor.
	\item Označme symbolem $l_2$ metrický prostor, jehož body jsou posloupnosti reálných čísel
	$x = (x_1, x_2, \dots, x_n, \dots)$
	splňující podmínku
	$\sum\limits_{k=1}^\infty x_k^2 < +\infty$
	a v němž se vzdálenost definuje vztahem
	$varrho(x, y) = \sqrt{\sum\limits_{k=1}^\infty(y_k - x_k)^2}$
	\item Uvažujeme množinu všech spojitých funkcí na intervalu $\left<a, b\right>$. Vzdálenost definujeme jako $varrho(x, y) = \left( \int_a^b (x(t) - y(t))^2 dt \right)^\frac{1}{2}$. Tento metrický prostor budeme značit $C_2^0\left<a, b\right>$ a nazývat prostorem spojitých funkcí s kvadratickou metrikou.
	\item Uvažujme množinu všech ohraničených posloupností $x = (x_1, x_2, \dots, x_n, \dots)$ reálných čísel. Položíme-li $varrho(x, y) = sup |x_k - y_k|$ dostaneme metrický prostor $M^\infty$.
\end{itemize}

\begin{mydef}
Nechť $\mathcal{X} = (X, varrho)$ je libovolný metrický prostor. Metrický prostor $\mathcal{M} = (M, varrho)$ s toutéž metrikou $varrho$ uvažovanou pouze na množině $M \subset X$ se nazývá podprostorem metrického prostoru $X$.
\end{mydef}

\section{Konvergence posloupností}
Buď $\mathcal{X} = (X, varrho)$ metrický prostor, $x_0 \in X, r \in \mathbb{R}^+$. \emph{Otevřenou koulí} $S(x_0, r)$ v metrickém prostoru $\mathcal{X}$ budeme nazývat množinu bodů $x \in X$, které vyhovují podmínce

$$varrho(x, x_0) < r$$

Bod $x_0$ se nazývá \emph{středem} a číslo $r$ \emph{poloměrem} této koule.

\emph{Uzavřenou koulí} poloměru $\epsilon$ se středem $x_0$ budeme také nazývat $\epsilon$-okolím bodu $x_0$ a značit symbolem $0_\epsilon(x_0)$.

Bod $x$ nazýváme bodem \emph{uzávěru množiny} $M$, jestliže jeho libovolné okolí obsahuje alespoň jeden bod z $M$. Množina všech bodů uzávěru množiny $M$ se označuje $\overline{M}$ a nazývá se \emph{uzávěrem} množiny.

Protože každý bod, který náleží $M$ je bodem uzávěru množiny $M$ (tento bod totiž leží v každém svém okolí), platí $M \subseteq \overline{M}$.

Množinu $M$, pro kterou platí $M = \overline{M}$, nazýváme \emph{uzavřenou}.

Bod $x$ nazýváme \emph{vnitřním bodem množiny $M$}, existuje-li okolí $O_\epsilon(x)$ tohoto bodu, které je celé obsažené v množině $M$. Množinu, jejíž všechny body jsou vnitřní nazýváme \emph{otevřenou}.

Zřejmě je tedy množina $M$ otevřená (uzavřená), právě když její doplněk je uzavřená (otevřená) množina.

\begin{veta}
Uzávěr uzávěru $M$ je roven uzávěru $M$:
$$ \overline{\overline{M}} = \overline{M} $$
\end{veta}

\begin{veta}
Jestliže $M_1 \subseteq M$, potom $\overline{M_1} \subseteq \overline{M}$
\end{veta}

\begin{veta}
Uzávěr sjednocení je roven sjednocení uzávěrů:
$$ \overline{M_1 \cup M_2} = \overline{M_1} \cup \overline{M_2} $$
\end{veta}

Nechť $x_1, x_2, \dots$ je posloupnost bodů v metrickém prostoru $\mathcal{X} = (X, varrho)$. Říkáme, že tato \emph{posloupnost konverguje k bodu} $x \in X$, jestliže ke každému reálnému číslu $\epsilon > 0$ lze najít takové přirozené čislo $N(\epsilon)$, že okolí $O_\epsilon(x)$ obsahuje všechny body $x_n$, kde $n \geq N(\epsilon)$. Bod $x$ se nazývá \emph{limita posloupnosti} $\{x_n\}$.

Nebo: Posloupnost $\{x_n\}$ konverguje k bodu $x$, jestliže

$$\lim\limits_{n\to \infty} varrho(x, x_n) = 0 $$

Žádná posloupnost nemůže mít dvě různé limity. Jestliže posloupnost $\{x_n\}$ konverguje k bodu $x$, potom každá posloupnost vybraná z této posloupnosti konverguje k témuž bodu.

\begin{veta}
Aby bod $x$ byl bodem uzávěru množiny $M$, je nutné a stačí, aby existovala posloupnost $\{x_n\}$ bodů množiny $M$ konvergující k $x$.
\end{veta}

Nechť $A$ je podmnožina metrického prostoru $\mathcal{X}$. Množinu $A$ nazýváme hustotou v $\mathcal{X}$, jestliže $\overline{A} = \mathcal{X}$.

Například množina racionálních čísel je hustá na číselní přímce.

Metrický prostor $\mathcal{X}$, který obsahuje \emph{spočetnou} množinu, se nazývá \emph{separabilní}.

Příklady separabilních metrických prostorů:
\begin{itemize}
	\item Diskrétní metrická prostor je separabilní, právě když je spočetný. V tomto prostoru je $\overline{M}$ libovolné množiny $M$ totožný s množinou $M$.
	\item V metrickém prostoru $\mathbb{R}^1$ je hustou množinou množina racionálních čísel.
	\item V metrickém prostorech $\mathbb{R}^n$, $\mathbb{R}^n_0$ je hustou množinou množina vektorů s racionálními souřadnicemi.
	\item V metrickém prostorech $C^0\left<a, b\right>$ a $C_2^0\left<a, b\right>$ je hustou množinou množina polynomů s racionálními koeficienty.
	\item V metrickém prostorech $l_2$ je hustou množinou množina všech posloupností $\{x_n\}$. kde $x_n$ jsou racionální čísla, přičemž počet čísel $x_n$ různých od nuly je pouze konečný a pro různé posloupnost obecně různý.
\end{itemize}

Příklad neseparabilního prostoru: Metrický prostor $M^\infty$ všech ohraničených posloupností není separabilní.

\section{Spojitá zobrazení. Homeomorfismus. Izometrické zobrazení}
\begin{mydef}
Nechť $\mathcal{X} = (X, varrho)$ a $\mathcal{Y} = (Y, varrho^*)$ jsou dva různé metrické prostory. Zobrazení $f: \mathcal{X} \to \mathcal{Y}$ prostoru $\mathcal{X}$ do prostoru $\mathcal{Y}$ se nazývá spojité v bodě $x_0 \in X$, jestliže k libovolnému $\epsilon > 0$ lze najít takové $\delta > 0$, že pro všechny body $x$ s vlastností $varrho(x, x_0) < \delta$ platí
$$varrho^*(f(x), f(x_0)) < \epsilon$$
\end{mydef}

Zobrazení $f: \mathcal{X} \to \mathcal{Y}$ se nazývá spojité, jestliže je spojité ve všech bodech prostoru $\mathcal{X}$.

Je-li $\mathcal{Y}$ číselná přímka, potom spojité zobrazení $\mathcal{X}$ do $\mathcal{Y}$ se nazývá spojitou funkcí na $\mathcal{X}$.

\begin{veta}
Zobrazení $f: \mathcal{X} \to \mathcal{Y}$ je spojité v bodě $x$, právě když pro libovolnou posloupnost $\{x_n\}$, která konverguje k bodu $x$, konverguje posloupnost $\{f(x_n)\}$ k bodu $y = f(x)$.
\end{veta}

\begin{veta}
Aby zobrazení $f: \mathcal{X} \to \mathcal{Y}$ bylo spojité, je nutné a stačí, aby vzor vzhledem k $f$ každé uzavřené množiny z $\mathcal{Y}$ byla uzavřená množina v $\mathcal{X}$.
\end{veta}

\begin{veta}
Aby zobrazení $f: \mathcal{X} \to \mathcal{Y}$ bylo spojíté, je nutné a stačí, aby vzor každé otevřené množiny z $\mathcal{Y}$ byla otevřená množina.
\end{veta}

\begin{veta}
Nechť $\mathcal{X}, \mathcal{Y}, \mathcal{Z}$ jsou metrické prostory a $f: \mathcal{X} \to \mathcal{Y}$, $\varphi: \mathcal{Y} \to \mathcal{Z}$ spojitá zobrazení. Potom zobrazení $z = \varphi(f(x))$ metrického prostoru $\mathcal{X}$ do metrického prostoru $\mathcal{Z}$ je spojité.
\end{veta}

\begin{mydef}
Zobrazení $f$ se nazývá homeomorfismus, je-li vzájemně jednoznačné a jak zobrazení tak k němu inverzní zobrazení $f^{-1}$ jsou spojitá.

Metrické prostory $\mathcal{X}$ a $\mathcal{Y}$ se nazývají homeomorfní, existuje-li mezi nimi homeomorfní zobrazení.
\end{mydef}

\begin{veta}
Aby vzájemně jednoznačné zobrazení bylo homeomorfní, je nutné a stačí, aby uzavřené (otevřené) množiny odpovídaly uzavřeným (otevřeným) množinám.
\end{veta}

\begin{mydef}
Říkáme, že vzájemně jednoznačné zobrazení $y = f(x)$ metrického prostoru $\mathcal{X} = (X, varrho)$ na metrický prostor $\mathcal{Y} = (Y, varrho^*)$ je izometrické, jestliže

$$ varrho(x_1, x_2) = varrho^*(f(x_1), f(x_2)) \quad \forall x_1, x_2 \in X $$

Samotné metrické prostory $\mathcal{X}$ a $\mathcal{Y}$ mezi kterými je možno stanovit izometrické zobrazení se nazývají izometrickými mezi sebou.
\end{mydef}

\section{Úplné metrické prostory}

Číselná osa

\begin{mydef}
Posloupnost $\{x_n\}$ bodů metrického prostoru $\mathcal{X}$ budeme nazývat \emph{cauchyovskou} (nebo \emph{fundamentální}), jestliže splňuje Caughyovo kritérium, tj. jestliže ke každému $\epsilon > 0$ existuje takové kladné celé číslo $N(\epsilon)$, že
$$ \forall m,n \geq N(\epsilon) \quad varrho(x_m, x_n) < \epsilon $$
\end{mydef}

\begin{mydef}
Jestliže v metrickém prostoru $\mathcal{X}$ libovolná cauchyovská posloupnost konverguje (tj. existuje $x \in \mathcal{X}$ tak, že $varrho(x_n, x) \to 0$), potom nazýváme tento prostor \emph{úplný}.
\end{mydef}

Příklady:
\begin{enumerate}
	\item V diskrétním metrickém prostoru jsou cauchyovské pouze stacionární posloupnosti, tj. takové, v nichž se od určitého indexu stále opakuje tentýž bod. Každá taková posloupnost ovšem konverguje, tj. tento prostor jeúplný.
	\item Úplnost prostoru $\mathbb{R}^1$, tj. úplnost množiny všech reálných čísel je známa z matematická analýzy.
	\item Úplnost euklidovského prostoru $\mathbb{R}^n$ plyne snadno z úplnost prostoru $\mathbb{R}^1$: Nechť $\{x^{(p)}\}$ je cauchyovská posloupnost bodů
	$$ x^{(p)} = (x^{(p)}_1, x^{(p)}_2, \dots, x^{(p)}_n) \in \mathbb{R}^n, p = 1, 2, \dots $$
	\item Úplnost prostorů $\mathbb{R}^n_0, \mathbb{R}^n_1$ lze ukázat obdobně.
	\item Také metrické prostory $C^0 \left<a, b \right>, l_2$ a $M^\infty$ jsou úplné.
	\item Prostor $C_2^0 \left<a, b \right>$ není úplný.
\end{enumerate}

Není-li metrický prostor $\mathcal{X}$ úplný, pak jej lze vždy vnořit do úplného prostoru $\mathcal{X}^*$. Přesněji, existuje (až na izometrii) jediný úplný prostor $\mathcal{X}^*$ takový, že $\mathcal{X}$ je podprostor prostoru $\mathcal{X}^*$ a $\mathcal{X}$ je v $\mathcal{X}^*$ hustý.

\section{Banachův princip pevného bodu (BPPB}

\begin{mydef}
Nechť $\mathcal{X} = (X, varrho)$ je metrický prostor. Zobrazení $A$ prostoru $\mathcal{X}$ do prostoru $\mathcal{X}$ se nazývá \emph{kontraktivní} (neboli kontrakce), existuje-li takové číslo $\alpha < 1$, ž epro libovolné dva body $x, y \in X$ platí nerovnost
$$varrho(Ax, Ay) \leq \alpha varrho(x, y)$$
\end{mydef}

\begin{veta}
Každé kontraktivní zobrazení je spojité.
\end{veta}

\begin{mydef}
Bod $x$ se nazývá \emph{pevný bod} zobrazení $A$, jestliže $Ax$ = x. Jinak řečeno, pevné body jsou řešení rovnice $Ax = x$.
\end{mydef}

\begin{veta}
(BPPB) Každé kontraktivní zobrazení definované v neprázdném úplném metrickém prostoru $\mathcal{X}$ má právě jeden pevný bod.
\end{veta}

\section{Aplikace BPPB}
BPPB lze použít k důkazu vět o existenci a jednoznačnosti řešení pro rovnice různých typů. Kromě důkazu existence a jednoznačnosti řešení rovnice $Ax = x$ dává BPPB také praktickou metodu přibližného výpočtu tohoto řešení (nazývanou metoda postupných aproximací).




%%%%%%%%%%%%%%%%%%%%%%%%%%%%%%%%%%%%%%%%%%%%%%%%%%%%%%%%%%%%%%%%%%%%%%%%%%%%%%%%
%%%%%%%%%%%%%%%%%%%%%%%%%%%%%%%%%%%%%%%%%%%%%%%%%%%%%%%%%%%%%%%%%%%%%%%%%%%%%%%%
\chapter{Normované a unitární prostory} \label{cha:13}

1. semestr, MAT, \texttt{Zaklady\_funkcionalni\_analyzy\_opr.pdf}, 7. -  11. kapitola

(základní vlastnosti a příklady, normované prostory konečné dimenze, uzavřené ortonormální systémy a Fourierovy řady)

\section{Definice, příklady}

\begin{mydef}
Nechť $\mathcal{L}$ je neprázdná množina prvků $x, y, z, \dots$ a nechť je splněno těchto osm podmínek:
\begin{enumerate}[I.]
	\item $\mathcal{L}$ je komutativní grupa, tj. ke každým dvěma prvkům $x, y \in \mathcal{L}$ je jednoznačně přiřazen třetí prvek ležící v $\mathcal{L}$, který je nazývaný jejich součet a označovaný $x + y$, přičemž platí tyto čtyři axiomy:
	\begin{enumerate}[1.]
		\item $x + y = y + x$ (komutativita)
		\item $x + (y + z) = (x + y) + z$ (asociativita)
		\item v $\mathcal{L}$ existuje takový prvek (značíme jej $\theta$), že $x + \theta = x$ pro všechny prvky $x \in \mathcal{L}$ (existence nulového prvku)
		\item Ke každému prvku $x \in \mathcal{L}$ existuje prvek, který značíme $-x$, takový, že $x + (-x) = \theta$ (existence opačného prvku)
	\end{enumerate}
	\item Ke každému číslu $\alpha$ nějakého číselného tělesa $T$ a ke každému prvku $x \in \mathcal{L}$ je jednoznačně přiřazen prvek $\alpha x \in \mathcal{L}$ (tzv. součin prvku $x$ a čísla $\alpha$), přičemž platí tyto dva axiomy:
	\begin{enumerate}[1.]
		\item $\alpha(\beta x) = (\alpha \beta) x \quad \alpha, \beta \in T, x \in \mathcal{L}$
		\item $1 \cdot x = x \quad 1 \in T, x \in \mathcal{L}$
	\end{enumerate}
	\item Obě operace (tj. sčítání prvků a násobení prvku číslem) jsou svázány těmito dvěma zákony:
	\begin{enumerate}[1.]
		\item $(\alpha + \beta) x = \alpha x + \beta x, \quad \alpha, \beta \in T, x \in \mathcal{L}$
		\item $\alpha (x + y) = \alpha x + \beta x, \quad \alpha \in T, x, y \in \mathcal{L}$
	\end{enumerate}
\end{enumerate}
Množinu $\mathcal{L}$ potom nazýváme \emph{lineárním nebo vektorovým prostorem} nad číselným tělesem $T$. Podle toho, zde čísly $\alpha, \beta, \dots$ rozumíme komplexní čísla, resp. reálná čísla, mluvíme krátce o \emph{komplexním}, resp. \emph{reálném prostoru}. Prvky lineárního prostoru $\mathcal{L}$ nazýváme \emph{body} nebo \emph{vektory}, kdežto čísla $\alpha, \beta, \dots$ nazýváme \emph{skaláry}.
\end{mydef}

\begin{mydef}
Lineární prostor $\mathcal{L}$ se nazývá \emph{normovaný}, jestliže každému prvku $x \in \mathcal{L}$ je přiřazeno reálné nezáporné číslo $||x||$, které se nazývá \emph{norma} prvku $x$, přičemž pro každé $x, y \in \mathcal{L}$ a $ \alpha \in T$ platí:
\begin{enumerate}
	\item $||x|| = 0$ když a jen když $x = \theta$
	\item $||\alpha x|| = |\alpha| \cdot ||x||$ (homogenita)
	\item $||x + y|| \leq ||x|| + ||y||$ (trojúhelníková nerovnost)
\end{enumerate}
Protože se zabýváme pouze lineárními prostory, budeme lineární normované prostory stručně nazývat \emph{normovanými prostory}.
\end{mydef}

Snadno je vidět, že každý lineární normovaný prostor i každá jeho podmnožina je současně metrickým prostorem.

\begin{mydef}
\emph{Úplný} lineární normovaný prostor se nazývá \emph{Banachovým prostorem}.
\end{mydef}

Příklady lineárních normovaných prostorů:
\begin{enumerate}
	\item Číselné těleso $T$ je normovaný prostor (nad $T$) s normou danou absolutní hodnotou. Např. reálná osa $\mathbb{R}^1$, tj. množina všech reálných čísel s obvyklými aritmetickými operacemi sčítání a násobení, je lineárním prostorem. Prostor $\mathbb{R}^1$ se stane normovaným prostorem jestliže pro každé číslo $x \in \mathbb{R}^1$ položíme $||x|| := |x|$.
	\item Množina všech uspořádaných $n$-tic reálných, popřípadě komplexních čísel $x = (x_1, x_2, \dots)$, kde sčítání $n$-tic a násobení $n$-tic konstantou je definováno vztahy
	$$(x_1, x_2, \dots, x_n) + (y_1, y_2, \dots, y_n) = (x_1 + y_1, x_2 + y_2, \dots, x_n + y_n) $$
	$$\alpha (x_1, x_2, \dots, x_n) = (\alpha x_1, \alpha x_2, \dots, \alpha x_n) $$
	je lineárním prostorem, který budeme nazývat \emph{n-rozměrným prostorem}. Jde-li o n-tice reálných čísel a jsou-li multiplikativní konstanty také reálná čísla, budeme mluvit o \emph{reálném n-rozměrném prostoru} a používat označení $\mathbb{R}^n$.
	Normu v reálném n-rozměrném prostoru $\mathbb{R}^n$ s prvky $x = (x_1, x_2, \dots, x_n) $ definujeme předpisem
	$$||x||_2 = \sqrt{\sum\limits_{k = 1}^n x_k^2}$$
	V lineárním prostoru $\mathbb{R}^n$ lze definovat normu také předpisem
	$$||x||_1 = \sum\limits_{k = 1}^n |x_k|$$
	nebo
	$$||x||_0 = \max\limits_{1 \leq k \leq n}^n |x_k|$$
	V komplexním n-rozměrném prostoru $\mathbb{C}^n$ je potřeba  použít explicitní absolutní hodnotu. Nebo obecně:
	$$||x||_p = \left( \sum\limits_{k = 1}^n |x_k|^p \right)^\frac{1}{p}$$
	\item Množina všech spojitých reálných funkcí reálné proměnné na intervalu $\left<a, b\right>$ (popř. spojitých komplexních funkcí reálné proměnné) s běžnými operacemi sčítání funkcí a násobení funkce číslem je reálný (popř. komplexní) lineární prostor, která je jedním z nejdůležitějších v matematické analýze. Značíme jej $\mathbb{C}^0\left<a,b\right>$ nebo zkráceně $\mathbb{C}\left<a,b\right>$. V prostoru $\mathbb{C}^0\left<a,b\right>$ definujeme normu vztahem
	$$||f|| = \max\limits_{a \leq t \leq b} |f(t)|$$
	\item Nechť $\mathbb{C}_2^0\left<a,b\right>$ opět sestává ze všech spojitých funkcí, ale norma je definována vztahem
	$$||f|| = \left( \int_a^b [f(t)]^2 dt \right)^\frac{1}{2}$$
	\item Prostor $l_2$ všech posloupností $x = \{x_k\}_{k=1}^\infty$, které splňují podmínku
	$$\sum\limits_{k=1}^\infty |x_k|^2 < + \infty$$
	se stane lineárním normovaným prostorem, definujeme-li, že součet dvou prvků $x = (x_1, x_2, \dots, x_n, \dots)$ a $y = (y_1, y_2, \dots, y_n, \dots)$ z $l_2$ je roven
	$$x + y = (x_1 + y_1, x_2 + y_2, \dots, x_n + y_n, \dots)$$
	a že součin čísla $\alpha$ a prvku $x \in l_2$je dán vztahem
	$$\alpha x = (\alpha x_1, \alpha x_2, \dots, \alpha x_n, \dots)$$
	Normu v $l_2$ definujeme vztahem
	$$||x|| = \sqrt{\sum\limits_{k = 1}^\infty |x_k|^2}$$
	\item Prostor $c$ se sestává ze všech konvergentních posloupností. Prostor $c_0$ se sestává z posloupností $x = (x_1, x_2, \dots, x_n, \dots)$ reálných čísel, které splňují podmínku $\lim\limits{n \to \infty} = 0$. Sčítání prvků a násobení prvků skalárem se definují v prostorech $c$ a $c_0$ stejně jako v předchozím příkladě a norma je dána $||x|| = \max\limits_{1 \leq n \infty} |x_n|$
	\item Množina $M^\infty$ všech ohraničených posloupností $x = \{x_k\}_{k=1}^\infty$ reálných (popř. komplexních) čísel s týmž operacemi jako v předchozích příkladech je lineární prostor. Normu v něm můžeme zavést vztahem $||x|| = \sup\limits_{1\leq n\leq \infty} |x_n|$
\end{enumerate}
Všechny příklady s výjimkou prostoru $\mathbb{C}_2^0\left<a,b\right>$ jsou Banachovými prostory.

\begin{mydef}
(Lineární nezávislost) Množina vektorů $\{x_1, x_2, \dots, x_n\}$ se nazývá lineárně závislá, jestliže existují konstanty $\alpha_1, \dots, \alpha_n$, z nichž aspoň jedna je různá od nuly a platí

$$\alpha_1 x_1 + \alpha_2 x_2 + \dots + \alpha_n x_n = \theta$$

V opačném případě se tato množina nazývá lineárně nezávislá. Jinak řečeno, množina se nazývá lineárně nezávislá, jestliže z předchozí rovnosti plyne, že $\alpha_1 = \alpha_2 = \dots = 0$.

Nekonečná podmnožina prostoru $\mathcal{L}$ se nazývá \emph{lineárně nezávislá}, jestliže každá její konečná množina je lineárně nezávislá.

Výraz $\alpha_1 x_1 + \alpha_2 x_2 + \dots + \alpha_n x_n$ nazýváme lineární kombinací prvků $x_1, x_2, \dots, x_n$.

Jestliže v prostoru $\mathcal{L}$ lze najít $n$ lineárně nezávislých prvků, ale libovolné $n+1$ prvky jsou již lineárně závislé, říkáme, že prostor $\mathcal{L}$ má \emph{dimenzi (rozměr)} $n$. Jestliže v prostoru $\mathcal{L}$ lze nalézt nekonečný systém lineárně nezávislých prvků, říkáme, že prostor $\mathcal{L}$ má nekonečnou dimenzi.

\emph{(Lineární) bází v n-rozměrném prostoru $\mathcal{L}$} nazýváme libovolný systém $n$ lineárně nezávislých prvků. Prostory $\mathbb{R}^n$ v reálném případě a $\mathbb{C}^n$ v komplexním případě mají dimenzi $n$.
\end{mydef}

Každý z předchozích příkladů má nekonečnou dimenzi.

\begin{mydef}
(Podprostory lineárního prostoru) Neprázdná podmnožina $\mathcal{L}^*$ lineárního prostoru $\mathcal{L}$ se nazývá \emph{podprostor} prostoru $\mathcal{L}$, jestliže sama tvoří lineární prostor vzhledem k operacím sčítání prvků a násobení skalárem, které jsou definovány v prostoru $\mathcal{L}$. Jinak řečeno, $\mathcal{L}^* \subset \mathcal{L}^*$ je podprostor, jestliže
$x,y \in \mathcal{L}^* \Rightarrow \alpha x + \beta y \in \mathcal{L}^*$
pro libovolná čísla $\alpha, \beta$.

V každém lineárním prostoru $\mathcal{L}$ existuje podprostor, který se skládá pouze z nulového prvku $\theta$ a nazývá se \emph{nulový podprostor}. Také celý prostor $\mathcal{L}$ lze považovat za podprostor prostoru $\mathcal{L}$. Podprostor různý od prostoru $\mathcal{L}$ a obsahující aspoň jeden nenulový prvek se nazývá \emph{vlastní podprostor}.
\end{mydef}

\begin{mydef}
Průnik libovolného systému $\{\mathcal{L}_\gamma\}$ podprostorů lineárního prostoru $\mathcal{L}$ je zřejmě opět podprostor.
(Jestliže $\mathcal{L}^* = \bigcap\limits_\gamma \mathcal{L}_\gamma$ a $x, y \in \mathcal{L}^*$, potom také $\alpha x + \beta y \in \mathcal{L}^*$ pro všechna čísla $\alpha, \beta$.
Nechť  $\{x_\alpha\}$ je libovolná neprázdná množina prvků lineárního prostoru $\mathcal{L}$. Průnik všech podprostorů obsahující množinu $\{x_\alpha\}$ nazveme \emph{podprostorem vytvořeným (generovaným) množinou} $\{x_\alpha\}$ nebo \emph{lineárním obalme množiny} $\{x_\alpha\}$. Tento podprostor budeme označovat $L\{x_\alpha\}$. Je to nejmenší podprostor obsahující množinu $\{x_\alpha\}$.
\end{mydef}

\section{Normované prostory konečné dimenze}

Každý lineární prostor $X(n)$ konečné dimenze $n$ je izomorfní s euklidovským prostorem $\mathbb{R}^n$, a proto můžeme považovat prvky uvažovaného prostoru $X(n)$ za n-tice reálných čísel.

\begin{veta}
(Riesz) Nechť $X(n)$ je normovaný prostor konečné dimenze $n$. Aby posloupnost
$\{x_\nu = (\xi_1^{(\nu)}, \xi_2^{(\nu)}, \dots, \xi_n^{(\nu)})$,
konvergovala k prvku
$x_0 = (\xi_1^{(0)}, \xi_2^{(0)}, \dots, \xi_n^{(0)}) \in X(n)$, je nutné a stačí, aby
$\xi_k^{(\nu)} \to \xi_k^{(0)} (k = 1, 2, \dots, n).$
\end{veta}

\begin{lemma}
Je-li posloupnost $\{x_\nu\}$ ohraničená, potom je ohraničená i každá posloupnost $\{\xi_k^{(\nu)} (k = 1, 2, \dots, n)$.
\end{lemma}

Důsledek: Každý normovaný prostor konečné dimenze je úplný.

\section{Hilbertovy prostory}
\subsection{Unitární prostory a jejich základní vlastnosti}

\begin{mydef}
Skalárním součinem v reálném lineárním prostoru $R$ nazýváme zobrazení $(-, -): R \to \mathbb{R}$, tedy reálnou funkci $(x, y)$ definovanou pro každou dvojici prvků $x, y \in R$, která splňuje tyto čtyři podmínky ($x, x_1, x_2, y \in R, \lambda $ je reálné číslo):
\begin{enumerate}
	\item $(x, y) = (y, x)$
	\item $(x_1 + x_2, y) = (x_1, y) + (x_2, y)$
	\item $(\lambda x, y) = \lambda(x, y)$
	\item $(x, x) \geq 0$, přičemž $(x,x) = 0$, když a jen když $x = \theta$.
\end{enumerate}
\end{mydef}

\begin{mydef}
Lineární prostor, v němž je definován skalární součin se nazývá \emph{unitární prostor}. V unitárním prostoru $R$ se norma zavádí vztahem
$||x|| = \sqrt{(x, x)}$
\end{mydef}

Množina $\{ x_\alpha \}$ nenulových vektorů $x_\alpha \in R$ se nazývá \emph{ortogonální} soustava, jestliže
$(x_\alpha, x_\beta) = 0$ pro $\alpha \not= \beta$

Soustava nenulových vektorů $x_\alpha \in R$ se nazývá \emph{ortonormální}, jestliže
$$(x_\alpha, x_\beta) = \begin{dcases}
0 \text{ pro } \alpha \not= \beta \\
1 \text{ pro } \alpha = \beta
\end{dcases}$$

Je zřejmé, že je-li $\{x_\alpha\}$ ortogonální soustava, potom $\{x_\alpha / ||x_\alpha|| \}$ je ortonormální soustava. Úplná ortonormální soustava se nazývá ortonormální báze.

\begin{veta}
Je-li $\{x_\alpha\}$ ortogonální soustava, jsou vektory $x_\alpha$ lineárně nezávislé.
\end{veta}

\begin{mydef}
Množina $\{x_\alpha\} \in R$ se nazývá úplný systém vektorů, jestliže uzávěr podprostoru generovaného množinou $\{x_\alpha\}$ je celý prostor $R$. Je-li $\{x_\alpha\}$ navíc ortogonální (ortonormální) soustavou, nazýváme ji \emph{ortogonální (ortonormální) báze}.
\end{mydef}

Pokud má prostor $R$ konečnou dimenzi, pak je každá jeho ortogonální báze (lineární) bází. Ovšem toto neplatí pro prostory nekonečné dimenze, neboť báze vygeneruje pouze hustou podmnožinu v prostoru $R$, což nemusí být celý prostor $R$.

Příklad: Konečně rozměrný prostor $\mathbb{R}^n$, jehož prvky jsou všechny n-tice reálných čísel
$x = (x_1, x_2, \dots, x_n)$,
s obvyklými operacemi sčítání n-tic a násobení konstantou a se skalárním součinem
$(x, y) = \sum\limits_{i = 1}^n x_i y_i$
je dobře známým příkladem unitárního prostoru (Takto zavedený skalární součin definuje v prostoru $\mathbb{R}^n$ normu $||x||_2 = \sqrt{\sum\limits_{k=1}^n x_k^2}$,
a tedy tvoří euklidovskou metriku.). Ortonormální bázi tohoto prostoru (jednu z nekonečně mnoha možných) tvoří vektory
$ e_1 = (1, 0, \dots, 0) $,
$ e_2 = (0, 1, \dots, 0) $,
\dots,
$ e_n = (0, 0, \dots, 1) $.

Prostor $l_2$ s prvky
$x = (x_1, x_2, \dots, x_n, \dots)$, kde $\sum\limits_{i=1}^\infty x_i^2 < + \infty$
a se skalárním součinem $(x, y) = \sum\limits_{i=1}^\infty x_i y_i$
je unitární prostor. Nejjednodušší ortonormální bázi v prostoru $l_2$ tvoří vektory
$ e_1 = (1, 0, 0, \dots) $,
$ e_2 = (0, 1, 0, \dots) $,
$ e_n = (0, 0, 1, \dots) $,
\dots.
Ortogonálnost a normovanost této soustavy jsou zřejmé; soustava je úplná. Nechť $x = (x_1, x_2, \dots, x_n, \dots)$ je libovolný vektor prostoru $l_2$ a $x^{(n)} = (x_1, x_2, \dots, x_n, 0, 0, \dots)$. Potom $x^{(n)}$ je lineární kombinace vektorů $e_1, \dots, ,e_n$ a $||x^{(n)} - x|| \to 0$ pro $n \to \infty$.
Tedy každý prvek v $l_2$ leží v uzávěru vygenerovaného prostoru. Ovšem soustava vektorů $e_n$ netvoří bázi prostoru $l_2$, neboť prostor generovaný touto soustavou obsahuje pouze posloupnosti, které mají pouze konečný počet členů (jedná se o prostor polynomů $\mathbb{R}[X]$).

Prostor $C_2^0 \left<a,b\right>$ všech spojitých reálných funkcí definovaných na intervalu $\left<a, b\right>$ se skalárním součinem definovaným vztahem $(f,g) = \int_a^b f(t)g(t) dt$ je také unitární prostor, ale neúplný. Mezi různými ortogonálními bázemi, které lze v něm uvést je nejdůležitější soustava trigonometrických funkcí
$1$, $\cos\frac{2\pi nt}{b-a}$, $\sin\frac{2\pi nt}{b-a}$, $n=1, 2, \dots$.
Má-li interval $\left<a,b\right>$ délku $2\pi$, např. je-li $a=-\pi, b = \pi$, potom příslušná trigonometrická soustava je 
$1$, $\cos nt$, $\sin nt$, $n=1, 2, \dots$.

\section{Existence ortogonálních bází, ortogonalizace}
Každý z prostorů uvedených v předchozí kapitole je separabilní.
\begin{veta}
Nechť $R$ je separabilní unitární prostor. V takovém prostoru je každá ortogonální systém nejvýše spočetný.
\end{veta}

\begin{veta}
(Schmidtova věta o ortogonalizaci). Nechť
$f_1, f_2, \dots, f_n, \dots$
je lineárně nezávislý systém prvku v unitárním prostoru $R$. Potom v prostoru $R$ existuje systém prvků
$\varphi_1, \varphi_2, \dots, \varphi_n, \dots$,
který splňuje tyto podmínky:
\begin{enumerate}
	\item Systém je ortonormální
	\item Každý prvek $\varphi_n$ je lineární kombinace prvků $f_1, f_2, \dots, f_n$, že
		$\varphi_n = a_{n1}f_1 + \dots + a_{nn} f_n$, přičemž $a_{nn} \not= 0$.
	\item Každý prvek $f_n$ lze vyjádřit ve tvaru $f_n = n_{n1} \varphi_1 + \dots + b_{nn} \varphi_n$, přičemž $b_{nn} \not= 0$
\end{enumerate}
Každý prvek systému je určen podmínkami 1 - 3 až na znaménko jednoznačně.
\end{veta}

Přechod od systému $f_n$ k systému $\varphi_n$, který splňuje podmínky 1 - 3 se nazývá \emph{ortonormalizace}.
Je zřejmé, že prostory vytvořené výše zmíněnými systémy jsou totožné, takže je-li jeden z těchto systémů úplný v prostoru $R$, je úplný v prostoru $R$ i druhý.

Důsledek: V separabilním unitárním prostoru R existuje ortonormální báze.

\section{Besselova nerovnost. Uzavřené ortogonální systémy}

Je-li $e_1, e_2, \dots, e_n$ ortonormální báze n-rozměrného unitárního prostoru $\mathbb{R}^n$, potom vektor $x \in \mathbb{R}^n$ lze zapsat ve tvaru
$x = \sum\limits_{k=1}^n c_k e_k$, kde $c_k = (x, e_k)$.

Nechť $\varphi_1, \varphi_2, \dots, \varphi_n, \dots$
je ortonormální systém v unitárním prostoru $R$ a $f$ je libovolný prvek v prostoru $R$. Přiřadíme prvku $f \in R$ posloupnost čísel
$c_k = (f, \varphi_k), k = 1, 2, \dots$
které budeme nazývat \emph{souřadnicemi} nebo \emph{Fourierovými koeficienty prvku $f$ vzhledem k systému $\{\varphi_k\}$},
a funkční řadu $\sum\limits_{k=1}^n c_k \varphi_k$ kterou nazveme \emph{Fourierovou řadou prvku $f$ vzhledem k systému $\{\varphi_k\}$}.

Dá se ukázat, že řada $\sum\limits_{k=1}^\infty c_k \varphi_k^2$  konverguje a platí
$\sum\limits_{k=1}^\infty c_k^2 \leq ||f||^2$.

Uvedenou nerovnost nazýváme \emph{Besselovou nerovností}.

\begin{mydef}
Ortonormální systém se nazývá \emph{uzavřený}, jestliže mezi každým vektorem $f \in R$ a jeho Fourierovými koeficienty $c_k$ platí tzv. \emph{Parsevalova nerovnost}:

$$\sum\limits_{k=1}^\infty c_k^2 = ||f||^2$$
\end{mydef}

Uzavřenost systému je ekvivalentní s tím, že pro každý vektor $f \in R$ konverguje posloupnost částečných součtů Fourierovy řady $\sum{k=1}^n c_k \varphi_k$ k prvku $f$. Pojem uzavřenosti ortonormálního systému úzce souvisí s dříve zavedeným pojmem úplnosti systému:

\begin{veta}
V separabilním unitárním prostoru $R$ je každý úplný ortonormální systém uzavřený a naopak.
\end{veta}

Nechť $\{\varphi_n\}$ je libovolný ortogonální systém. Vzhledem k němu lze sestrojit normovaný systém, vytvořený z prvků $\psi_n = \varphi_n/||\varphi_n||$. Pro libovolný prvek $f \in R$ platí
$c_n = (f, \psi_n) = \frac{1}{||\varphi_n||}(f, \varphi_n)$ a 
$\sum_{n=1}^\infty c_n \psi_n = \sum_{n=1}^\infty \frac{c_n}{||\varphi_n||} = \sum_{n=1}^\infty a_n \varphi_n$, kde
$$a_n = \frac{c_n}{||\varphi_n||} = \frac{(f,\varphi_n)}{||\varphi_n||^n}$$

Koeficienty $a_n$ nazveme \emph{Fourierovými koeficienty v ortogonálním (nenormovaném) systému $\{\varphi_n\}$}. Dosadíme-li do Besselovy nerovnosti za $c_n$, dostaneme
$$\sum\limits_{n=1}^\infty ||\varphi_n||^2a_n^2 \leq ||f||^2$$
což je Besselova nerovnost pro libovolný ortogonální systém.

Příklad: Uvažujme v unitárním prostoru $C_2^0\left<-\pi, \pi\right>$ všech spojitých funkcí na intervalu $\left<-\pi, \pi\right>$ se skalárním součinem daným vztahem $(f, g) = \int_{-\pi}^\pi f(t) g(t) dt$ úplný ortogonální systém goniometrických funkcí, tedy systém
$1, \cos nt, \sin nt, n = 1, 2, \dots$

Tento systém není ortonormální, k němu příslušný ortonormální systém tvoří funkce
$\frac{1}{\sqrt{2\pi}}, \frac{\cos nt}{\sqrt{\pi}}, \frac{\sin nt}{\sqrt{\pi}}, n = 1, 2, \dots$

Nechť $f$ je funkce z prostoru $C_2^0\left<-\pi, \pi\right>$. Fourierovy koeficienty této funkce vzhledem k systému $1, \cos nt, \sin nt$ se většinou značí $\frac{a_0}{2}, a_n$ a $b_n$.
V souladu s obecnými vzorci pro Fourierovy koeficienty tedy máme
$$\frac{a_0}{2} = \frac{1}{2 \pi} \int_{-\pi}^\pi f(t) dt \dots a_0 = \frac{1}{\pi} \int_{-\pi}^\pi f(t) dt$$
$$a_n = \frac{1}{\pi} \int_{-\pi}^\pi f(t) \cos nt dt, b_n = \frac{1}{\pi} \int_{-\pi}^\pi f(t) \sin nt dt$$

Fourierova řada funkce $f$ vzhledem k systému goniometrických funkcí má tvar

$$\frac{a_0}{2} + \sum\limits_{n=1}^\infty (a_n \cos nt + b_n \sin nt)$$

a konverguje k $f$.

\begin{comment}
\section{Úplné unitární prostoru. Rieszova-Fischerova věta}

\begin{veta}
(Riesz-Fischer). Nechť $\{\varphi_n\}$ je libovolný ortonormální systém v úplném unitárním prostoru R a nechť čísla $c_1, c_2, \dots, c_n, \dots$ jsou taková, že řada
$$\sum\limits_{k=1}^\infty c_k^2$$
konverguje. Potom existuje takový prvek $f \in R$, že $c_k = (f, \varphi_k)$ a
$$\sum\limits_{k=1}^\infty (f, f) = ||f||^2$$
\end{comment}




















%%%%%%%%%%%%%%%%%%%%%%%%%%%%%%%%%%%%%%%%%%%%%%%%%%%%%%%%%%%%%%%%%%%%%%%%%%%%%%%%
%%%%%%%%%%%%%%%%%%%%%%%%%%%%%%%%%%%%%%%%%%%%%%%%%%%%%%%%%%%%%%%%%%%%%%%%%%%%%%%%
\chapter{Obyčejné grafy} \label{cha:14}

1. semestr, MAT, \texttt{Grafy\_1\_141212.pdf}, \texttt{Grafy\_2\_141220.pdf}

(stupně uzlů, sledy, souvislost, izomorfismy, stromy, kostry, Kruskalův a Primův algoritmus pro hledání minimální kostry ohodnoceného grafu, eulerovské a hamiltonovské grafy, planarita a obarvitelnost)

\begin{mydef}
\textbf{Obyčejný graf} je dvojice $G=(U, H)$, kde $U$ je konečná množina uzlů (vrcholů) a $H \subseteq \left\{ \left\{u, v\right\}: u, v \in U \land u \not= v \right\}$ je (konečná) množina hran. O hraně $h = \{u, v\}$ říkáme, že je incidentní s uzly $u$ a $v$ nebo že je mezi uzly $u$ a $v$, spojuje uzly $u$ a $v$ a podobně.
\end{mydef}

\begin{mydef}
Je-li $G=(U, H)$ obyčejný graf, definujeme \textbf{sled} mezi uzly $u, v$ o délce $n$ jako posloupnost $(u=w_0, h_1, w_1, h_2, \dots, w_{n-1}, h_n, w_n = v$ takovou, že  $w_0, w_1, \dots, w_n \in U, h_1, h_2, \dots, h_n \in H$ a $h_1 = \{w_{i-1}m w_i\}, 1 \leq i \leq n$.
\end{mydef}
(ve sledu se mohou opakovat jak uzly, tak hrany)

\begin{mydef}
Je-li $G=(U, H)$ obyčejný graf, potom \textbf{tahem} mezi uzly $u, v$ o délce $n$ rozumíme sled $(u=w_0, h_1, w_1, h_2, \dots, w_{n-1}, h_n, w_n = v$ takový, že platí $i \not= j \Rightarrow h \not= h_j, 1 \leq i, j \leq n$.

Je-li navíc $w_0 = w_n$, pak se tento tah nazývá uzavřený.
\end{mydef}

(uzly se mohou opakovat, ale ne hrany)

\begin{mydef}
Je-li $G=(U, H)$ obyčejný graf, potom \textbf{cesta} mezi uzly  $u, v$ o délce $n$ je sled $(u=w_0, h_1, w_1, h_2, \dots, w_{n-1}, h_n, w_n = v$ mezi uzly $u, v$, takový, že platí $i \not= j \Rightarrow w_i \not= w_j, 0 \leq i, j \leq n$.
\end{mydef}

(v cestě jsou všechny uzly různé)

\begin{mydef}
Graf $G=(U, H)$ se nazývá \textbf{diskrétní}, resp. \textbf{úplný}, jestliže $H = \emptyset$, resp. $H=\{\{u, v\}, u, v \in U \land u \not= v\}$.
\end{mydef}

\begin{mydef}
Je-li $G=(U, H)$ obyčejný graf, kružnice v grafu $G$ o délce $n$ je sled $(w_0, h_1, w_1, h_2, \dots, w_{n-1}, h_n, w_n$ takový, že platí
$$ i \not= j \Rightarrow w_i \not=w_j, 0 \leq i,j \leq n -1 \land w_0 = w_n$$
\end{mydef}

\begin{veta}
Nechť $G=(U, H)$ obsahuje dvě různé kružnice
$C_1 = (u_0, g_1, u_1, g_2, \dots, u_{n-1}, g_n, u_n$ a
$C_2 = (v_0, h_1, v_1, h_2, \dots, v_{n-1}, h_n, v_n$,
kde $u_0 = v_0, u_1 = v_1, h_1 = g_1$. Potom tento podgraf obsahuje i kružnici $C_3$ neobsahující hranu $h_1 = g_1$.
\end{veta}

\begin{mydef}
Je-li $G=(U, H)$ obyčejný graf, řekneme, že je \textbf{souvislý}, když pro libovolné uzly $u, v \in U$ existuje sled $(u=w_0, h_1, w_1, h_2, \dots, w_{n-1}, h_n, w_n = v$.
\end{mydef}

\begin{mydef}
Jsou-li $G=(U, H)$ a $G'=(U', H')$ obyčejné grafy, řekneme, že $G'$ je \textbf{podgrafem grafu} $G$, když $U' \subseteq U \land H' \subseteq H$. Pokud navíc platí $(u, v \in U' \land \{u, v \} \in H) \Rightarrow \{u, v\} \in H'$, říkáme, že podgraf $G'$ je \textbf{indukovaný} (množinou uzlů $U'$). \textbf{Faktorem grafu} $G=(U, H)$ nazýváme takový jeho podgraf $G'=(U', H')$, pro který platí $U=U'$. 
\end{mydef}

\begin{mydef}
Jsou-li $G=(U, H)$ a $G'=(U', H')$ obyčejné grafy, řekneme, že $G'$ je \textbf{komponentou} grafu $G$, když $G'$ je souvislým indukovaným podgrafem grafu $G$ a pro libovolný obyčejný graf $G''=(U'', H'')$ platí: ($U' \subset U''$ a $G''$ je podgraf G) $\Rightarrow G''$ není souvislý.
\end{mydef}

\begin{mydef}
Je-li $G=(U, H)$ obyčejný graf a $h \in H$, pak řekneme, že hrana $h$ je \textbf{mostem}, když jejím odstraněním se zvýší počet komponent grafu.
\end{mydef}
(hrana je jedinou hranou mezi uzly $u, v$)

\begin{mydef}
Je-li $G = (u, H)$ obyčejný graf a $u \in U$, definujeme číslo $\text{deg}(u)$, tzv. \textbf{stupeň uzlu} $u$ jako počet hran incidentních s uzlem $u$.
\end{mydef}
Nechť $G=(U, H)$ je obyčejný graf, $|H|=m$. Snadno se dokáže, že platí $\sum\limits_{u \in U} \text{deg}(u) = 2 m$.

\begin{mydef}
Jsou-li $G_1=(U_1, H_1)$ a $G_2=(U_2, H_2)$ dva obyčejné grafy a $\varphi: U_1 \to U_2$ bijekce mezi množinami uzlů, řekneme, že $\varphi$ je \textbf{isomorfismus} $G_1$ na $G_2$, jestliže pro každé dva uzly $u, v \in H_1$ platí
$$ \{u, v\} \in H_1 \Leftrightarrow \{\varphi(u), \varphi(v)\} \in H_2$$
\end{mydef}

Isomorfismus sama na sebe se nazývá \textbf{automorfismus}.

\begin{mydef}
Obyčejný graf, jehož žádný podgraf není kružnice se nazývá \textbf{les}.
\end{mydef}

\begin{mydef}
Obyčejný \emph{souvislý} graf, jehož žádný podgraf není kružnicí, se nazývá \textbf{strom}.
\end{mydef}

\begin{veta}
Nechť $S=(U, H)$ je les, který má alespoň jednu hranu. Pak existují dva uzly $u, v \in U$ takové, že $\text{deg}(u) = \text{deg}(v) = 1$.
\end{veta}

\begin{veta}
Nechť $G=(U, H)$ obyčejný graf a $|U| = n, |H| = m$. Pak jsou následující podmínky ekvivalentní:
\begin{enumerate}[(a)]
	\item $G$ je strom
	\item $G$ je souvislý a $m = n-1$
	\item $G$ neobsahuje jako podgraf kružnici a $m=n -1$
	\item $G$ je souvislý a každá hrana je mostem
	\item mezi každou dvojicí různých uzlů v $G$ existuje jediná cesta
	\item $G$ neobsahuje kružnici a vznikne-li graf $G'$ přidáním libovolné hrany ke grafu $G$, $G'$ kružnici obsahuje.
	\item $G$ je souvislý, pro $n > 2$ je $G$ neúplný a vznikne-li graf $G'$ přidáním libovolné hrany ke grafu $G$, pak $G'$ obsahuje právě jednu kružnici.
\end{enumerate}
\end{veta}

\begin{mydef}
Je-li dán obyčejný graf $G=(U, H)$, pak jeho faktor $K = (U, H')$ nazveme \textbf{kostrou grafu} $G$, pokud je $K$ strom.
\end{mydef}

\begin{veta}
Nechť $G$ je obyčejný graf. $G$ je souvislý, právě když má kostru.
\end{veta}

\begin{veta}
Nechť $G=(U, H)$ je obyčejný graf a $|U|=n$. Pokud faktor $K=(U, H')$ grafu $G$ splňuje kterékoliv dvě z následujících podmínek, pak je \textbf{kostrou grafu}:
\begin{enumerate}
	\item $K$ je souvislý
	\item $|H'| = n-1$
	\item $K$ neobsahuje jako podgraf kružnici
\end{enumerate}
\end{veta}

\section{Oceněný graf}
\begin{mydef}
Nechť $G=(U, H)$ je obyčejný graf. Je-li navíc dáno zobrazení $c: H\to R$, potom trojici $G=(U, H, c)$ nazýváme \textbf{oceněným grafem}. Každé hraně $h$ grafu $G$ je tak přiřazeno reálné číslo $c(h)$, které se nazývá cenou hrany $h$. Je-li $G'=(U', H')$ podgraf grafu $G$, potom $c(G') = \sum\limits_{u \in H'} c(u)$ se nazývá cennou podgrafu $G'$.
\end{mydef}

\begin{mydef}
Nechť $G=(U, H, c)$ je obyčejný oceněný graf. Nechť $K=(U, H')$ je kostra grafu $G$. Řekneme, že $K$ je minimální kostra grafu $G$, jestliže platí $c(K) \leq c(L)$ pro každou kostru $L$ grafu $G$.
\end{mydef}

\begin{veta}
Nechť $G=(U, H, c)$ je obyčejný souvislý oceněný graf a nechť
$$C=(v, h_1, u_1, h_2, u_2, \dots, u_{p-1}, h_p, v), p \geq 3$$
je kružnice v grafu $G$. Jestliže platí $c(h_1) > c(h_i), 2 \leq i \leq p$, potom hrana $h_1$ není obsažena v žádní minimální kostře grafu $G$.
\end{veta}

Důsledek: Nechť $G=(U, H, c)$ je obyčejný souvislý oceněný graf a nechť
$$C=(v, h_1, u_1, h_2, u_2, \dots, u_{p-1}, h_p, v), p \geq 3$$
je kružnice v grafu $G$. Jestliže platí $c(h_1) \geq c(h_i), 2 \leq i \leq p$, potom existuje aspoň jedna minimální kostra, ve které není hrana $h_1$ obsažena.

(pokud hrany mohou mít stejné ceny. Pokud jsou všechny ceny hran rozdílné, existuje pouze jedna minimální kostra)

\paragraph{Kruskalův algoritmus}
Je dá oceněný obyčejný souvislý graf $G=(U, H, c)$, kde $|U| = n$ a $H = \{h_1, h_2, \dots, h_k\}$. Setřiďme hrany z $H$ do posloupnosti $S=(s_1, s_2, \dots, s_k)$ tak, že platí $c(s_i) \leq c(s_{j}$ pro $i<j$. Budeme nyní postupně vytvářet grafy $K_1=(U, Q_1), K_2=(U, Q_2), \dots K_{n-1} = (U, Q_{n-1})$ tak, aby platilo:
\begin{enumerate}[(a)]
	\item $Q_1 = \{s_1\}$
	\item Jestliže $Q_i = \{s_{j_1}, s_{j_2}, \dots, s_{j_i}\}$, kde $1 < i < n-1, c(s_{j_1}) \leq c(s_{j_2}) \leq \dots \leq c(s_{j_i})$, potom $Q_{i+1} = \{s_{j_1}, s_{j_2}, \dots, s_{j_i}, s_q\}$, kde $s_q$ je hrana z posloupnosti $S$ s nejmenším indexem $q$ taková, že $s_q \not= s_{j_k}, 1 \leq k \leq i$ a $K_{i+1}$ neobsahuje kružnici.
\end{enumerate}

\paragraph{Primův algoritmus}
Je-dán oceněný obyčejný souvislý graf $G=(U, H, c)$. Pro podgraf $K=(V, J)$ grafu $G$, který neobsahuje kružnici, označme $K^+ = (V^+, J^+)$ graf, který vznikne z grafu $K$ přidáním uzlu $u$ do $V$ a hrany $h$ do $J$ takové, že $h$ je incidentním uzlem $u$ a s nějakým uzlem ve $V$, nevytvoří v $K$ kružnici a přitom $h$ je hranou nejmenší ceny s takovouto vlastností. Sestrojíme postupně podgrafy $K_1, K_2, \dots, K_{n-1}$ následovně:
\begin{enumerate}[(a)]
	\item $K_1 = (\{u, v\}, \{\{u, v\}\})$ kde $c(\{u, v\}) \leq c(\{u, w\})$ pro všechna $w \in U$
	\item $K_{i+1} \overset{def}{=} K^+_i$ pro každé $i = 1, 2, \dots, n-2$
\end{enumerate}


\begin{veta}
Nechť $G = (U, H, c)$ je obyčejný oceněný souvislý graf a $\{u,v\} \in H$ hrana taková, že $c(\{u, v\}) < c(\{u, v\})$ pro každý uzel $w \in U$. Potom hrana $\{u, v\}$ leží v každé minimální kostře grafu $G$.
\end{veta}

\paragraph{Důsledek}: Nechť $G = (U, H, c)$ je obyčejný oceněný souvislý graf a $\{u, v \} \in H$ hrana taková, že $c(\{u, v\}) \leq c(\{u, w\})$ pro každý uzel $w \in U$. Potom existuje minimální kostra grafu $G$, ve které hrana $\{u, v\}$ leží.

\paragraph{Důsledek}: Nechť $G = (U, H, c)$ je obyčejný oceněný souvislý graf, $V \subseteq U, S = (V, H')$ je strom a nechť tento strom je podgrafem minimální kostry grafu $G$. Pak existuje minimální kostra grafu $G$, která obsahuje $S$ jako podgraf a navíc hranu $h$ s nejmenší cenou takovou, že $h = \{u, v\}, u \in V, v \in (U-V)$.

\paragraph{Maticová forma Primova algoritmu}
Pokud jsou ceny hran grafu $G=(U, H)$, kde $U=\{u_1, u_2, \dots, u_n\}$, zadány ve formě matice
$$
 \begin{pmatrix}
  c_{11} & c_{12} & \cdots & c_{1n} \\
  c_{21} & c_{22} & \cdots & c_{2n} \\
  \vdots  & \vdots  & \ddots & \vdots  \\
  c_{n1} & c_{n2} & \cdots & c_{nn} 
 \end{pmatrix}
$$

kde prvek na $i$-tém řádku a v $j$-tém sloupci označuje cenu hrany incidentní s uzly $u_i, u_j$, je možno Primův algoritmus vyjádřit v následující formě:
\begin{itemize}
	\item Krok 1: Vyškrtnou se všechny prvky v 1. sloupci a 1. řádek se označí.
	\item Krok 2: Pokud v označených řádcích neexistuje žádný nepodtržený prvek, algoritmus končí a podtržené prvky označují hrany v minimální kostře. Jinak se vybere minimální takový prvek.
	\item Je-li vybraný prvek $c_{ij}$, podtrhne se a označí se j-tý řádek a vymažou se nepodtržené prvky j-tého sloupce. Přechod ke kroku 2.
\end{itemize}

\begin{mydef}
Graf $G$ se nazývá \textbf{eulerovský}, existuje-li v něm uzavřený tah, který obsahuje každou hranu v $G$.
\end{mydef}

\begin{mydef}
Graf $G$ se nazývá \textbf{poloeulerovský}, existuje-li v něm tah, který obsahuje každou hranu v $G$.
\end{mydef}

\begin{veta}
Nechť $G$ je souvislý graf. potom je $G$ eulerovský, právě když každý jeho uzel má sudů stupeň.
\end{veta}

\paragraph{Důsledek}: Souvislý graf je poloeulerovský, právě když každý jeho uzel má sudý stupeň, nebo existují právě dva uzly lichého stupně.

\begin{mydef}
\textbf{Hamiltonovskou kružnicí grafu $G$} nazveme kružnici, která prochází každým uzlem grafu právě jednou. Graf nazveme hamiltonovským, má-li hamiltonovskou kružnici.
\end{mydef}

\begin{veta}
(Ore) Nechť $G$ je graf s $n$ uzly $n \geq 3$ a nechť platí $\text{deg}(u) + \text{deg}(v) \leq n$ pro každé dva uzly $u$ a $v$ grafu $G$, které nejsou spojeny hranou. Potom je graf $G$ hamiltonovský.
\end{veta}

\begin{veta}
(Dirac) Nechť $G$ je obyčejný graf s $n$ uzly a nechť platí $\text{deg}(u) \geq \frac{n}{2}$ pro každý uzel $u$. Potom je graf $G$ hamiltonovský.
\end{veta}

(poznámka: Domeček jedním tahem je hamiltonovský, ale nesplňuje Diracovu podmínku)

\subsection{Barvení uzlů}
Graf je \emph{obarvený}, když se každému uzlu přiřadí barva tak, že dvěma uzlům spojeným hranou jsou přiřazeny různé barvy.

Pokud je možno graf obarvit pomocí $k$ barev, aniž bychom nutně užili všechny z nich, nazývá \emph{k-obarvitelným}.

Nejmenší možná hodnata $k$ pro kterou je graf $G$ k-obarvitelným, se nazývá \emph{chromatické číslo} grafu $G$, formálně $\mathcal{X}(G)$.

\section{Typy grafů}
\begin{description}
	\item[$K_n$] úplný graf s $n$ uzly
	\item[$D_n$] diskrétní graf s $n$ uzly
	\item[$K$] bipartitní graf, tj graf, jehož množinu uzlů lze rozdělit na dvě disjunktní množiny $V_1, V_2$ tak, že každá jeho hrana spojuje některý uzel z množiny $V_1$ s některým uzlem množiny $V_2$. Jestliže je navíc každý uzel z množiny $V_1$ spojen hranou s každým uzlem z množiny $V_2$, pak se tento graf nazývá úplný bipartitní graf a značí se $K_{m,n}$, kde $m = |V_1|, n=|V_2|$.
\end{description}
Bipratitní graf neobsahuje kružnici liché délky.

Platí:
$$\mathcal{X}(G) = 1 \leftrightarrow G = D_n$$
$$\mathcal{X}(K_n) = n$$
$$\mathcal{X}(K) = 2$$

Kružnice je 2-obarvitelná, právě když má sudý počet uzlů.

Graf je 2-obarvitelný, když neobsahuje kružnici s lichým počtem uzlů.  Strom, který žádné kružnice nemá, je 2-obarvitelný.

\begin{mydef}
Graf $G$ se nazývá \textbf{planární (rovinný)}, když je možno jej nakreslit v rovině tak, aby se jeho hrany nekřížily. Části roviny vymezené hranami planárního grafu nakresleného v rovině bez křížení hran se nazývají \textbf{buňky} a hrany kolem nich jsou jejich \textbf{hranice}.
\end{mydef}

\begin{veta}
Má-li souvislý planární graf $n$ uzlů a $m$ hran a tvoří $p$ buněk, platí $n-m+p = 2$.
\end{veta}

\begin{veta}
Nechť $G$ je souvislý planární graf s $n \geq 3$ uzly a $m$ hranami. Potom $m \leq 3n - 6$.
\end{veta}

\begin{mydef}
Dva grafy $G_1$ a $G_2$ se nazývají \textbf{homeomorfní} (nebo shodné až na uzly stupně 2), je-li možno $G_1$ i $G_2$ získat z nějakého grafu $G_3$ postupným rozpůlením některých hran vložením nového uzlu.
\end{mydef}

\begin{veta}
Graf je planární, právě když neobsahuje podgraf homeomorfní s grafem $K_5$ ani podgraf homeomorfní s $K_{3,3}$.
\end{veta}

\begin{veta}
Každý planární graf je 5-obarvitelný.
\end{veta}

\begin{veta}
Každý planární graf je 4-obarvitelný.
\end{veta}




%%%%%%%%%%%%%%%%%%%%%%%%%%%%%%%%%%%%%%%%%%%%%%%%%%%%%%%%%%%%%%%%%%%%%%%%%%%%%%%%
%%%%%%%%%%%%%%%%%%%%%%%%%%%%%%%%%%%%%%%%%%%%%%%%%%%%%%%%%%%%%%%%%%%%%%%%%%%%%%%%
\chapter{Orientované grafy} \label{cha:15}

1. semestr, MAT, \texttt{Grafy\_3\_141220.pdf}

(orientované sledy, souvislost a silná souvislost, turnaje, eulerovské a hamiltonovské grafy, Dijkstrův a Floyd-Warshallův algoritmus pro hledání cesty minimální délky)

\begin{mydef}
\textbf{Orientovaný graf} je dvojice $G=(U, H)$, kde $U$ je neprázdná konečná množina vrcholů nebo uzlů a $H=\{(u, v)|u, v \in U\}$ je konečná množina orientovaných hran.
\end{mydef}
(nebo také neprázdná konečná množina s binární relací)

\begin{mydef}
Nechť $G=(U, U)$ je orientovaný graf. Pro uzel $u \in U$ grafu $G$ definujeme čísla $\text{deg}_+(u) = |M|, \text{deg}_-(u) = |N|$, kde
$M = \{h \in H| \exists v \in U: h=(v, u)\}$ a
$N = \{h \in H| \exists v \in U: h=(u, v)\}$.

Číslo $\text{deg}_+(u)$ se rovná počtu hran, kteér vedou z nějakého uzlu do uzlu $u$, a nazývá se \textbf{vstupním stupněm uzlu $u$}. Číslo $\text{deg}_-(u)$ se rovná počtu hran, které vedou z uzlu $u$ do nějakého uzlu, a nazývá se \textbf{výstupním stupněm uzlu $u$}. Pokud platí $\text{deg}_-(u) = 0$, $u$ se nazývá koncový uzel, a pokud $\text{deg}_+(u) = 0$, $u$ se nazývá počáteční uzel grafu $G$.
\end{mydef}

Analogicky k obyčejným grafům definujeme (uzavřený) orientovaný sled, orientovaný tah, orientovanou cestu a orientovanou kružnici. Hrany v příslušných posloupnostech jsou přitom nahrazeny orientovanými hranami tak, aby směřovaly od předchozího k následujícímu uzlu.

\begin{mydef}
Máme-li zadán obyčejný graf $G=(U, H)$, je k němu možno definovat orientovaný graf $G'=(U', H')$ tak, že pro každou hranu $\{u, v\} \in H$ existují v $H'$ právě dvě hrany $h, h'$ takové, že $h=(u,v) \land h' = (v,u)$. Přitom v $H'$ žádné jiné hrany nejsou. Takovýto graf se nazývá \textbf{symetrickou orientací grafu $G$}. jinými slovy, hrana v obyčejném grafu mezi uzly $u$ a $v$ se nahradí oběma orientovanými hranami mezi těmito uzly v novém grafu.
\end{mydef}

\begin{mydef}
Máme-li zadán obyčejný graf $G=(U, H)$, je k němu možno definovat orientovaný graf $G'=(U', H')$ tak, že pro každou hranu $\{u, v\} \in H$ existuje v $H'$ jediná orientovaná hrana $h$ takové, že $h=(u, v)$ nebo $h = (v, u)$ a přitom $H'$ žádné jiné hrany neobsahuje. Tento graf se nazývá \textbf{orientací grafu $G$}.
\end{mydef}

\begin{mydef}
Máme-li zadán orientovaný graf $G=(U, H)$, potom k němu můžeme sestrojit jednoznačně obyčejný graf $G'=(U', H')$, který se nazývá \textbf{symetrizací grafu $G$}. Položíme
$H' = \{\{u, v\} | u, v \in U, u \not= v, \exists h \in H: h=(u, v) \lor h = (v, u)\}$
\end{mydef}
Jinými slovy, symetrizace vznikne zanedbáním šipek a smyček v původním grafu.

\begin{mydef}
Řekneme, že orientovaný graf $G=(U,H)$ je \textbf{souvislý}, jestliže jeho symetrizace $G=(U, H')$ je souvislý graf.
\end{mydef}

\begin{mydef}
Řekneme, že orientovaný graf $G=(U,H)$ je \textbf{silně souvislý}, jestliže pro libovolné dva uzly $u, v \in U$ existuje orientovaná cesta z uzlu $u$ do uzlu $v$.
\end{mydef}

\begin{mydef}
Orientovaný graf $T = (U, H)$ bez smyček se nazývá \textbf{turnajem}, když pro každou množinu uzlů $\{u,v\}, u, v \in U, u \not= v$ existuje právě jedna hrana $h \in H$ taková, že platí $h = (u, v) \lor h = (v,u)$.
\end{mydef}
V turnaji tedy existuje pro každou dvojici různých uzlů jediná orientovaná hrana jdoucí z jednoho uzlu do druhého.

\begin{veta}
Buď $T = (U, H)$ turnaj a $v \in U$ uzel s maximálním výstupním stupněm. Pak pro každý uzel $w \in U$ existuje orientovaná cesta z uzlu $v$ do uzlu $w$ délky nejvýše 2.
\end{veta}

\begin{mydef}
Orientovaný graf $G=(U, H)$ se nazývá eulerovský, jestliže v něm existuje uzavřený orientovaný tah obsahující všechny jeho hrany.
\end{mydef}

\begin{veta}
Souvislý orientovaný graf $G=(U, H)$ je eulerovský, právě tehdy když platí $\text{deg}_+(u) = \text{deg}_-(u)$ pro každý uzel $u \in U$.
\end{veta}

\begin{mydef}
\textbf{Orientovanou hamiltonovskou cestou v grafu $G$} nazveme orientovanou cestu, která prochází každým uzlem grafu $G$.

\textbf{Orientovanou hamiltonovskou kružnicí grafu $G$} nazveme orientovanou kružnici, která prochází každým uzlem grafu $G$.

\textbf{Orientovaný graf nazveme hamiltonovským}, má-li orientovanou hamiltonovskou kružnici.
\end{mydef}

\begin{veta}
V každém turnaji $T=(U, H)$ existuje orientovaná hamiltonovská cesta.
\end{veta}

\begin{veta}
Je-li $T_n$ silně souvislý turnaj s $n$ uzly, $n \geq 3$, potom $T_n$ obsahuje orientované kružnice o délkách $3, 4, \dots, n$.
\end{veta}

Důsledek: Turnaj $T$ s alespoň třemi uzly je hamiltonovský, právě když je silně souvislý.

Graf bude dále vždy znamenat orientovaný graf bez smyček, hrana orientovanou hranu a cesta bude vždy znamenat orientovanou cestu.
\begin{mydef}
Nechť $G=(U, H)$ je graf a každé hraně $h \in H$ nechť je přiřazeno reálné číslo $l(h)$. Potom tomuto číslu budeme říkat \textbf{délka hrany $h$}.

\textbf{Délka $l(p)$ sledu $p$ v grafu $G$} se definuje jako součet délek všech hran obsažených v sledu $p$. Je-li $p$ tvořena jediným uzlem, klademe $l(p) = 0$.
\end{mydef}

\begin{mydef}
Nechť je dán graf $G=(U, H)$ a $u, v \in U$. Pokud existuje mezi uzly $u$ a $v$ \textbf{cesta minimální délky}, definujeme číslo $d(u, v)$ jako délku této cestu. Pokud z uzlu $u$ do uzlu $v$ vůbec žádná cesta neexistuje, klademe $d(u, v) = \infty$.
\end{mydef}
Poznámka: Pro kružnice se zápornou délkou nemá pojem cesta minimální délky žádný význam.

\subsection{Dijkstrův algoritmus pro stanovení minimální cesty}
\begin{mydef}
\textbf{Horní odhad vzdálenosti uzlu $s$ a uzlu $v$} je číslo $D(v)$ takové, že platí $D(v) \geq d(s,v)$.

Pro každý uzel $v \in U$ bude symbol $\pi(v)$ označovat uzel, který \textbf{bezprostředně předchází uzlu $v$ v cestě minimální délky z uzlu $s$ do uzlu $v$} zkonstruované Dijkstrovým algoritmem. Pokud $v=s$ nebo pokud taková cesta dosud nebyla zkonstruována, položíme $\pi(v) = \emptyset$.
\end{mydef}

V grafu $G = (U, H)$, kde každé hraně $h$ je přiřazeno kladné reálné číslo $l(h)$ a kde je vyznačen výchozí uzel $s$, najděte ke každému uzlu $v \not= s$ cestu $p(s,v)$ minimální délky a tuto minimální délku $d(s, v)$.

\begin{enumerate}
	\item \textbf{Inicializace}: Pro každý uzel $u \in U$ položíme $\pi(u) = \emptyset, D(s) = 0, D(u) = \infty$, jestliže $u \not= s, S= \emptyset, Q = U$.
	\item \textbf{Test na ukončení algoritmu}: Pokud $S = U$, přechod na bod 5.
	\item \textbf{Nalezení uzlu s definitivní cestou}: Z množiny $Q$ přesuneme do množiny $S$ uzel $v$ s minimální hodnotou $D(v)$. Jestliže pro všechny $u \in Q$ platí $D(u) = \infty$, přechod na bod 5.
	\item \textbf{Zlepšení horních odhadů}: Pro každý uzel $w \in N(v) \cap Q$ takový, že $D(w) \geq D(v) + l((v, w))$ položíme $D(w) = D(v) + l((v, w))$ a $\pi(w) = v$. Přechod na bod 2.
	\item \textbf{Konstrukce výstupu}: Do uzlů, které zůstaly v množině $Q$ žádná cesta z uzlu $s$ neexistuje. Pro všechny ostatní uzly $v$ položíme $d(s, v) = D(v)$ a cestu minimální délky sestrojíme obrácením cesty $v \to \pi(v) \to \pi(\pi(v)) \to \dots \to s$.
\end{enumerate}

\begin{veta}
Dijkstrův algoritmus nalezne cestu minimální délky a vzdálenost z výchozího uzlu $s$ do každého jiného uzlu $v \in V$.
\end{veta}

\subsection{Floyd-Warshallův algoritmus}
Dijkstrův algoritmus nelze použít, pokud se v grafu vyskytují hrany se zápornou délkou. Floyd-Warshallův algoritmus při každém zadání délek hran nalezne cestu minimální délky z každého uzlu do každého jiného uzlu a pokud taková cesta, která by byla současně minimálním sledem, neexistuje kvůli kružnici se zápornou délkou, tuto kružnici odhalí.

Uvažujeme graf $G = (U, H), U = \{1, 2, \dots, n\}$ v němž jsou délky hran zadány maticí
A = $$
 \begin{pmatrix}
  a_{11} & a_{12} & \cdots & a_{1n} \\
  a_{21} & a_{22} & \cdots & a_{2n} \\
  \vdots & \vdots & \ddots & \vdots \\
  a_{n1} & a_{n2} & \cdots & a_{nn}
 \end{pmatrix}
$$
kde $a_{ij}$ značí délku hrany $(i, j)$ pro libovolné $i, j \in \{1, 2, \dots, n\}$. Dále budeme používat matici
P = $$
 \begin{pmatrix}
  p_{11} & p_{12} & \cdots & p_{1n} \\
  p_{21} & p_{22} & \cdots & p_{2n} \\
  \vdots & \vdots & \ddots & \vdots \\
  p_{n1} & p_{n2} & \cdots & p_{nn}
 \end{pmatrix}
$$
kde na začátku platí $p_{ij} = j$. Algoritmus má vždy $n$ iterací.

Začneme s maticí $A^0 = A, P^0 = P$ a v $i$-té iteraci vytvoříme matice $A^i, P^i$ pomocí $A^{i-1}, P^{i-1}$. Nakonec tedy dostaneme matice $A^n, P^n$. Prvky matic $A^j, P^j, j = 1, 2, \dots, n$ se vypočítají následujícím způsobem:
$$a_{ik}^j = a_{ik}^{j-1}, p_{ik}^j = p_{ik}^{j-1} \text{ jestliže } a_{ik}^{j-1} \leq a_{ij}^{j-1} + a_{ij}^{j-1}$$
$$a_{ik}^j = a_{ij}^{j-1} + a_{jk}^{j-1}, p_{ik}^j = p_{ij}^{j-1} \text{ jestliže } a_{ik}^{j-1} > a_{ij}^{j-1} + a_{ij}^{j-1}$$

Indukcí lze dokázat, že po skončení algoritmu má prvek $a_{ij}^n$ hodnotu minimální vzdálenosti z uzlu $i$ do uzlu $j$. Dá se též ověřit, že pokud $p_{ij}^n = k$, potom $(i,k)$ je první hrana v minimální cestě z uzlu $i$ do uzlu $j$, což se dá využít při konstrukci této trasy.

Existence kružnice se zápornou délkou je indikována pomocí záporného diagonálního prvku.





%%%%%%%%%%%%%%%%%%%%%%%%%%%%%%%%%%%%%%%%%%%%%%%%%%%%%%%%%%%%%%%%%%%%%%%%%%%%%%%%
%%%%%%%%%%%%%%%%%%%%%%%%%%%%%%%%%%%%%%%%%%%%%%%%%%%%%%%%%%%%%%%%%%%%%%%%%%%%%%%%
\chapter{Klasifikace gramatik, formálních jazyků a automatů přijímajících jazyky.} \label{cha:16}

1. semestr, TIN, \texttt{opora.pdf}, 2., 3. kapitola

\subsection{Jazyky}
\begin{mydef}
Abecedou rozumíme neprázdnou množinu prvků, které nazýváme \emph{symboly abecedy}.
\end{mydef}

\begin{mydef}
\emph{Řetězcem} (také slovem) nad danou abecedou rozumíme každou konečnou posloupnost symbolů abecedy. Prázdnou posloupnost symbolů, tj posloupnost, která neobsahuje žádný symbol, nazýváme \emph{prázdný řetězec}. Prázdný řetězec budeme označovat písmenem $\epsilon$.

Formálně lze definovat řetězec nad abecedou $\Sigma$ takto:
\begin{enumerate}[(1)]
	\item prázdný řetězec $\epsilon$ je řetězec nad abecedou $\Sigma$
	\item je-li $x$ řetězec nad $\Sigma$ a $a \in \Sigma$, pak $xa$ je řetězec nad $\Sigma$.
	\item $y$ je řetězec nad $\Sigma$, když a jen když lze $y$ získat aplikací pravidel 1 a 2.
\end{enumerate}
\end{mydef}

Velká řecká písmena pro abecedy, Malá latinská písmena ($a, b, c, \dots$) pro symboly, ($t, u, v, \dots$) pro řetězce.

\begin{mydef}
Nechť $x$ a $y$ jsou řetězce nad abecedou $\Sigma$. \emph{Konkatenací} (zřetězením) řetězce $x$ s řetězcem $y$ vznikne řetězec $xy$ přípojením řetězce $y$ za řetězec $x$. Operace konkatenace je zřejmě asociativní, tj. $x(yz) = (xy)z$, ne však komutativní $xy \not= yx$.
\end{mydef}

\begin{mydef}
Nechť $x = a_1 a_2 \dots a_n$ je řetězec nad abecedou $\Sigma, a_i \in \Sigma$ pro $i = 1, \dots, n$. \emph{Reverzí} (zrcadlovým obrazem) řetězce $x$ rozumíme řetězec $x^R = a_n a_{n-1} \dots a_2 a_1$; tj. symboly řetězce $x^R$ jsou vzhledem k řetězci $x$ zapsány v opačném pořadí.
\end{mydef}

\begin{mydef}
Nechť $w$ je řetězec nad abecedou $\Sigma$. Řetězec $z$ se nazývá \emph{řetězcem} řetězce $w$, jestliže existují řetězce $x$ a $y$ takové, že $w = xzy$.
Řetězec $x_1$ se nazývá \emph{prefixem} (předponou) řetězce $w$, jestliže existuje řetězec $y_1$ takový, že $w = x_1 y_1$. Analogicky řetězec $y_2$ se nazývá \emph{sufixem} (příponou) řetězce $w$, jestliže existuje řetězec $x_2$ takový, že $w = x_2 y_2$. Je-li $y_1 \not= \epsilon$, resp. $x_2 \not= \epsilon$, pak $x_1$ je \emph{vlastní prefix}, resp. $y_2$ je vlastní sufix řetězce $w$.
\end{mydef}

\begin{mydef}
\emph{Délka řetězce} je nezáporné celé číslo udávající počet symbolů řetězce. Délku řetězce $x$ značíme symbolicky $|x|$. Je-li $x = a_1 a_2 \dots a_n, a_i \in \Sigma$ pro $i = 1, \dots, n$, pak $|x| = n$. Délka prázdného řetězce je nulová, tj. $|\epsilon| = 0$.
\end{mydef}

$a^3 = aaa, \quad b^2 = bb, \quad a^0 = \epsilon$

\begin{mydef}
Nechť $\Sigma$ je abeceda. Označme symbolem $\Sigma^*$ množinu všech řetězců nad abecedou $\Sigma$ včetně řetězce prázdného, symbolem $\Sigma^+$ množinu všech řetězců nad $\Sigma$ vyjma řetězce prázdného, tj. $\Sigma^* = \Sigma^+ \cup \{\epsilon\}$. Množinu $L$, pro niž platí $L \subseteq \Sigma^*$ (případně $L \subseteq \Sigma^+$, pokud $\epsilon \not\in L$) nazýváme \emph{jazykem} $L$ nad abecedou $\Sigma$. Jazykem tedy může být libovolná podmnožina řetězců nad danou abecedou. Řetězec $x, x \in L$ nazýváme \emph{větou} (někdy také slovem) jazyka L.
\end{mydef}

\begin{mydef}
Nechť $L_1$ je jazyk nad abecedou $\Sigma_1$, $L_2$ jazyk nad abecedou $\Sigma_2$. \emph{Součinem} (konkatenací) jazyků $L_1$ a $L_2$ je jazyk $L_1 \cdot L_2$ nad abecedou $\Sigma_1 \cup \Sigma_2$, jenž je definován takto: $L_1 \cdot L_2 = \{ xy | x \in L_1, y \in L_2\}$.
\end{mydef}
Operace je definovaná prostřednictvím operace konkatenace řetězců a má také stejné vlastnosti: je asociativní a nekomutativní.

\begin{mydef}
Nechť $L$ je jazyk nad abecedou $\Sigma$. \emph{Iteraci} $L^*$ jazyka $L$ a \emph{pozitivní iteraci} $L^+$ jazyka $L$ definujeme takto:
$$L^0 = \{\epsilon\}$$
$$L^n = L \cdot L^{n-1}, \text{ pro } n \geq 1$$
$$L^* = \bigcup\limits_{n \geq 0} L^n, \quad L^+ = \bigcup\limits_{n \geq 1} L^n$$
\end{mydef}

\begin{veta}
Je-li $L$ jazyk, pak platí:
$$L^* = L^+ \cup \{\epsilon\}$$
$$L^+ = L \cdot L^* = L^* \cdot L$$
\end{veta}

\begin{mydef}
Algebraická struktura $\left< A, +, \cdot, 0, 1 \right>$ se nazývá polokruh s aditivním jednotkovým prvkem 0 a multiplikativním jednotkovým prvkem 1, jestliže:
\begin{enumerate}[(1)]
	\item $\left< A, +, 0 \right>$ je komutativní monoid
	\item $\left< A, \cdot, 1 \right>$ je monoid
	\item pro operaci $\cdot$ platí distributivní zákon vzhledem k $+$:
	$ a \cdot (b + c) = a \cdot b + a \cdot c$ pro $a, b, c \in A$.
\end{enumerate}
\end{mydef}

\begin{veta}
Algebra jazyků $\left< 2^{\Sigma^*}, \cup, \cdot, \emptyset, \{\epsilon\} \right>$, kde $\cup$ je sjednocení a $\cdot$ konkatenace jazyků tvoří polokruh.
\end{veta}

\subsection{Gramatika}
Nejznámější prostředek pro reprezentaci jazyka (konečné, nekonečné). Používá dvou konečných disjunktních abeced: Množina $N$ nonterminálních symbolů a množina $\Sigma$ terminálních symbolů.

\emph{Noterminální symboly}, krátce \emph{nonterminály}, mají roli pomocných proměnných označující určité syntaktické celky -- syntaktické kategorie.

Množina \emph{terminálních symbolů}, krátce \emph{terminálů} je identická s abecedou, nad níž je definován jazyk. Sjednocení obou množin, tj $N \cup \Sigma$, nazýváme \emph{slovníkem gramatiky}.

$a, b, c$ terminální symboly; $A, B, C, S$ nonterminální symboly; $U, V, Z$ terminální nebo nonterminální symboly; $\alpha, \beta$ řetězce terminálních a nonterminálních symbolů, $u,v,$ řetězce terminálních symbolů.

Gramatika představuje generativní systém. Jádrem gramatiky je konečná množina $P$ \emph{přepisovacích pravidel} (nazývaných také produkce). Každé přepisovací pravidlo má tvar uspořádané dvojce $(\alpha, \beta)$ řetězců stanovuje množinu substitucí řetězce $\beta$ namísto řetězce $\alpha$, který se vyskytuje jako podřetězec generovaného řetězce. Řetězec $\alpha$ obsahuje alespoň jeden nonterminální symbol.
$$P \subseteq (N \cup \Sigma)^* N (N \cup \Sigma)^* \times (N \cup \Sigma)^*$$

\begin{mydef}
Gramatika $G$ je čtveřice $G = (N, \Sigma, P, S)$, kde
\begin{itemize}
	\item $N$ je konečná množina nonterminálních symbolů
	\item $\Sigma$ je konečná množina terminálních symbolů
	\item $P$ je konečná podmnožina kartézského součinu $(N \cup \Sigma)^* N (N \cup \Sigma)^* \times (N \cup \Sigma)^*$
	\item $S \in N$ je výchozí (také počáteční) symbol gramatiky
\end{itemize}
\end{mydef}

Prvek $(\alpha, \beta)$ množiny $P$ nazýváme \emph{přepisovacím pravidlem} (krátce pravidlem) a budeme jej zapisovat ve tvaru $\alpha \to \beta$. Řetězec $\alpha$ reps. $\beta$ nazýváme levou resp. pravou stranou přepisovacího pravidla.

Obsahuje-li gramatika pravidla se stejnou levou stranou, můžeme zjednodušeně zapsat $\alpha \to \beta_1 | \beta_2 | \dots | \beta_n$.

\begin{mydef}
Nechť $G = (N, \Sigma, P, S)$ je gramatika a nechť $\lambda$ a $\mu$ jsou řetězce z $(N \cup \Sigma)^*$. Mezi řetězci $\lambda$ a $\mu$ platí binární relace $\underset{G}{\Rightarrow}$, nazývaná \emph{přímá defivace}, jestliže můžeme řetězce $\lambda$ a $\mu$ vyjádřit ve tvaru:
$$\lambda = \gamma \alpha \delta$$
$$\mu = \gamma \beta \delta$$
kde $\gamma$ a $\delta$ jsou libovolné řetězce z $(N \cup \Sigma)^*$ a $\alpha \to \beta$ je nějaké přepisovací pravidlo z $P$.

Platí-li mezi řetězci $\lambda$ a $\mu$ relace přímé derivace, pak píšeme $\lambda \underset{G}{\Rightarrow} \mu$ a říkáme, že řetězec $\mu$ lze přímo generovat z řetězce $\lambda$ v gramatice $G$. Je-li z kontextu zřejmé, že jde o derivaci v gramatice $G$, pak nemusíme specifikaci gramatiky pod symbolem $\Rightarrow$ uvádět.
\end{mydef}

\begin{mydef}
Nechť $G = (N, \Sigma, P, S)$ je gramatika a $\lambda$ a $\mu$ jsou řetězce z $(N \cup \Sigma)^*$. Mezi řetězci $\lambda$ a $\mu$ platí relace $\Rightarrow^+$ nazývaná \emph{derivace}, jestliže existuje taková posloupnost přímých derivací $\nu_{i-1} \Rightarrow \nu_i, i = 1, \dots, n, n \geq 1$ taková, že platí:
$$\lambda = \nu_0 \Rightarrow \nu_1 \Rightarrow \dots \Rightarrow \nu_{n-1} \Rightarrow \nu_n = \mu$$

Tuto posloupnost nazýváme \emph{derivací délky $n$}. Platí-li $\lambda \Rightarrow^+ \mu$, pak říkáme, že řetězec $\mu$ lze generovat z řetězce $\lambda$ v gramatice $G$. Relace $\Rightarrow^+$ je zřejmě tranzitivním uzávěrem relace přímé derivace $\Rightarrow$. Symbolem $\Rightarrow^n$ značíme $n$-tou mocninu relace $\Rightarrow$.
\end{mydef}

\begin{mydef}
Jestliže v gramatice $G$ platí pro řetězce $\lambda$ a $\mu$ relace $\lambda \Rightarrow^+ \mu$ nebo identita $\lambda = \mu$, pak píšeme $\lambda \Rightarrow^* \mu$. Relace $\Rightarrow^*$ je tranzitivním a reflexivním uzávěrem relace přímé derivace $\Rightarrow$.
\end{mydef}

\begin{mydef}
Nechť $G=(N, \Sigma, P, S)$ je gramatika. Řetězec $\alpha \in (N \cup \Sigma)^*$ nazýváme \emph{větnou formou}, jestliže platí $S \Rightarrow^* \alpha$, tj. řetězec $\alpha$ je generovaný z výchozího symbolu $S$.
Větná forma, která obsahuje pouze terminální symboly se nazývá \emph{věta}. Jazyk $L(G)$, generovaný gramatikou $G$, je definován množinou všech vět $L(G) = \{ w | S \Rightarrow^* w \land w \in \Sigma^*\}$.
\end{mydef}

\subsection{Chomského klasifikace gramatik}
\subsubsection{Typ 0: Neomezené}
Gramatika typu 0 obsahuje pravidla v nejobecnějším tvaru

$$\alpha \to \beta, \alpha \in (N \cup \Sigma)^* N (N \cup \Sigma)^*, \beta \in (N \cup \Sigma)^*$$

\subsubsection{Typ 1: Kontextové}
Gramatika typu 1 obsahuje pravidla ve tvaru

$$ \alpha A \beta \to \alpha \gamma \beta, A \in N, \alpha, \beta in (N \cup \Sigma)^*, \gamma \in (N \cup \Sigma)^+$$
nebo 
$$S \to \epsilon$$
pokud se $S$ neobjevuje na pravé straně žádného pravidla.

\subsubsection{Typ 2: Bezkontextové}
Gramatika typu 2 obsahuje pravidla tvaru

$A \to \gamma, A \in N, \gamma \in (N \cup \Sigma)^*$

\subsubsection{Typ 3: Lineární (pravé lineární)}
Gramatika typu 3 obsahuje pravidla tvaru

$A \to xB$ nebo $A \to x; A, B \in N, x \in \Sigma^*$

Regulární (pravé regulární)

$A \to aB$ nebo $A \to a; A, B \in N, a \in \Sigma$ nebo $S \to \epsilon$, pokud se $S$ neobjevuje na pravé straně žádného pravidla.


\begin{mydef}
Jazyk generovaný gramatikou typu $i$, $i = 0, 1, 2, 3$ nazýváme jazykem typu $i$. Podle názvu gramatik mluvíme také o jazycích \emph{rekurzivně vyčíslitelných $(i=0)$} a analogicky ke gramatikám jazycích \emph{kontextových $(i=1)$}, \emph{bezkontextových $(i=2)$} a \emph{regulárních $(i=3)$}.
\end{mydef}

\begin{veta}
Nechť $\mathcal{L}_i, i = 0, 1, 2, 3$ značí třídu všech jazyků typu $i$. Pak platí $\mathcal{L}_0 \supseteq \mathcal{L}_1 \supseteq \mathcal{L}_2 \supseteq \mathcal{L}_3$.
\end{veta}

\begin{veta}
Nechť $\mathcal{L}_i, i = 0, 1, 2, 3$ jsou třídy jazyků typu $i$. Pak platí $\mathcal{L}_0 \supset \mathcal{L}_1 \supset \mathcal{L}_2 \supset \mathcal{L}_3$.
\end{veta}

%TODO XXX??












%%%%%%%%%%%%%%%%%%%%%%%%%%%%%%%%%%%%%%%%%%%%%%%%%%%%%%%%%%%%%%%%%%%%%%%%%%%%%%%%
%%%%%%%%%%%%%%%%%%%%%%%%%%%%%%%%%%%%%%%%%%%%%%%%%%%%%%%%%%%%%%%%%%%%%%%%%%%%%%%%
\chapter{Vlastnosti formálních jazyků} \label{cha:17}

1. semestr, TIN, \texttt{opora.pdf}, kapitola 3.5, TODO

(typické vlastnosti a jejich rozhodnutelnost)

\section{Vlastnosti regulárních jazyků}

Vlastnosti strukturální, uzávěrové, rozhodnutelné problémy.

\subsection{Strukturální vlastnosti regulárních jazyků}
\begin{veta}
Každý konečný jazyk je regulární.
\end{veta}

\subsubsection{Pumping lemma}
V každé dostatečně dlouhé větě každého regulárního jazyka jsme schopni najít poměrně krátkou sekvenci, kterou je možné vypustit, resp. zopakovat libovolný počet krát, přičemž stále dostáváme věty daného jazyka.

\begin{veta}
Nechť $L$ je nekonečný regulární jazyk. Pak existuje celočíselná konstanta $p > 0$ taková, že platí:

$$w \in L \land |w| \geq p \Rightarrow w = xyz \land 0 < |y| \leq p \land x y^i z \in L \text{ pro } i \geq 0$$
\end{veta}

Většinou k důkazu, že daný jazyk není regulární. Vyhodnocujeme všechny možnosti.

\subsubsection{Myhill-Nerodova věta}

ekvivalence je binární relace, která je reflexivní, symetrická a tranzitivní.

\begin{mydef}
Nechť $\Sigma$ je abeceda a $\sim$ je ekvivalence na $\Sigma^*$.
Ekvivalence $\sim$ je \emph{pravou kongruencí} (je zprava invariantní), pokud pro každé $u,v,w \in \Sigma^*$ platí

$$u \sim v \Rightarrow uw \sim vw$$
\end{mydef}

\begin{veta}
Ekvivalence $\sim$ na $\Sigma^*$ je pravá kongruence právě tehdy, když pro každé $u,v \in \Sigma^*, a \in \Sigma$ platí $u \sim v \Rightarrow ua \sim va$.
\end{veta}

\begin{mydef}
Nechť $L$ je libovolný jazyk (ne nutně regulární) jazyk nad abecedou $\Sigma$. Na množině $\Sigma^*$ definujeme relaci $\sim_L$ zvanou prefixová ekvivalence pro $L$ takto:
$$u \sim_L v \xLeftrightarrow{def} \forall w \in \Sigma^*: uw \in L \Leftrightarrow vw \in L$$
\end{mydef}

\begin{veta}
Nechť $L$ je jazyk nad $\Sigma$. Pak následující tvrzení jsou ekvivalentní:
\begin{enumerate}
	\item $L$ je jazyk přijímaný deterministickým konečným automatem
	\item $L$ je sjednocení některých tříd rozkladu určeného pravou kongruencí na $\Sigma^*$ s konečným indexem
	\item Relace $\sim_L$ má konečný index
\end{enumerate}
\end{veta}

Slouží opět pro dokazování, zda je jazyk regulární (resp. není).

\subsection{Uzávěrové vlastnosti regulárních jazyků}

\begin{veta}
Třída regulárních jazyků \emph{je uzavřena} (mino jiné) vzhledem k operacím $\cup$ sjednocení, $\cdot$ (konkatenace) a $^*$ (iterace).
\end{veta}
Důkaz z definice regulárních množina a ekvivalence regulárních množina a regulárních jazyků.

\begin{veta}
Třída regulárních jazyků tvoří množinovou \emph{Booleovu algebru}.
\end{veta}

\begin{veta}
Nechť $L \in \mathcal{L}_3$ a nechť $L^R = \{w^R | w \in L\}$. Pak $L^R \in \mathcal{L}_3$
\end{veta}

\subsection{Rozhodnutelné problémy regulárních jazyků}

neprázdnost $L \not= \emptyset$, náležitost $w \in L$ a ekvivalence $L(G_1) = L(G_2)$

\begin{veta}
Ve třídě $\mathcal{L}_3$ je rozhodnutelný problém \emph{neprázdnosti} jazyka i problém \emph{náležitosti} řetězce (do jazyka).
\end{veta}

\begin{veta}
Nechť $L_1 = L(G)$ a $L_2 = L(G_2)$ jsou dva jazyky generované regulárními gramatikami $G_1$ a $G_2$.
Pak je rozhodnutelný problém \emph{ekvivalence}, tj. $L(G_1) = L(G_2)$ nebo $L(G_1) \not= L(G_2)$.
\end{veta}
Podle relace nerozlišitelnosti stavů $\equiv$\$















%%%%%%%%%%%%%%%%%%%%%%%%%%%%%%%%%%%%%%%%%%%%%%%%%%%%%%%%%%%%%%%%%%%%%%%%%%%%%%%%
%%%%%%%%%%%%%%%%%%%%%%%%%%%%%%%%%%%%%%%%%%%%%%%%%%%%%%%%%%%%%%%%%%%%%%%%%%%%%%%%
\chapter{Konečné automaty} \label{cha:18}

1. semestr, TIN, \texttt{opora.pdf}, 3. kapitola 1. a 2. sekce

(jazyky přijímané jazyky KA, varianty KA, minimalizace KA)

\subsection{Jazyky přijímané konečnými automaty a deterministický konečný automat}

\subsubsection{Konečný automat}
\begin{mydef}
Konečný automat (KA) je 5-tice $M = (Q, \Sigma, \delta, q_0, F)$, kde
\begin{enumerate}[(1)]
	\item $Q$ je konečná množina stavů
	\item $\Sigma$ je konečná vstupní abeceda
	\item $\delta$ je zobrazení $Q \times \Sigma \to 2^Q$, které nazýváme \emph{funkcí přechodu} ($2^Q$ je množina podmnožin množiny $Q$).
	\item $q_0 \in Q$ je počáteční stav
	\item $F \subseteq Q$ je množina koncových stavů
\end{enumerate}

Je-li $\forall q \in Q \forall a \in \Sigma: |\delta(q, a)| \leq 1$, pak $M$ nazýváme \emph{deterministickým konečným automatem} (zkráceně DKA), v případě, že $\exists q \in Q \exists a \in \Sigma : | \delta(q, a)| > 1$ pak \emph{nedeterministickým konečným automatem (NKA)}.

\emph{Deterministický konečný automat} často také definujeme jako 5-tici $M = (Q, \Sigma, \delta, q_0, F)$, kde $\delta$ je \emph{parciální} přechodová funkce tvaru $\delta: Q \times \Sigma \to Q$. Je-li přechodová funkce $\delta$ totální, pak $M$ nazýváme \emph{úplně definovaným deterministickým konečným automatem}.
\end{mydef}

\begin{lemma}
Ke každému DKA $M$ existuje \uv{ekvivalentní} úplně definovaný DKA $M'$.
\end{lemma}
(nový nekoncový SINK stav, do kterého povedou všechny chybějící přechody.)

\begin{mydef}
Je-li $M = (Q, \Sigma, \delta, q_0, F)$ konečný automat, pak dvojici $C = (q, w)$ z $Q \times \Sigma^*$ nazýváme konfigurací automatu $M$. Konfigurace tvaru $(q_0, w)$ je počáteční konfigurace, konfigurace tvaru $(q, \epsilon), q \in F$ je koncová konfigurace.
Přechod automatu $M$ je reprezentován binární relací $\vdash_M$ na množině konfigurací $C$.
Pro všechna $q, q' \in Q$ a $w, w' \in \Sigma^*$ definujeme, že platí $(q, w) \vdash_M (q', w')$ tehdy a jen tehdy když $w = aw'$ pro nějaké $a \in \Sigma$ a $q' \in \delta(q, a)$ (tj. $\delta(q, a) = Q_j, Q_j \in 2^Q, q' \in Q_j$).
Označíme symbolem $\vdash_M^k$, $k \geq 0$ k-tou mocninu ($C \vdash^0 C'$ právě když $C = C'$), symbolem $\vdash_M^+$ tranzitivní uzávěr a symbolem $\vdash_M^*$ tranzitivní a reflexivní uzávěr relace $\vdash_M$.
Bude-li zřejmé, že jde o automat $M$, pak uvedené relace zapíšeme pouze jako $\vdash, \vdash^k, \vdash^+, \vdash^*$.
\end{mydef}

\begin{mydef}
Říkáme, že vstupní řetězec $w$ je \emph{přijímán} konečným automatem $M$, jestliže $(q_0, w) \vdash^* (q, \epsilon), q \in F$. Jazyk přijímaný konečným automatem $M$ označujeme symbolem $L(M)$ a definujeme ho jako množinu všech řetězců přijímaných automatem $M$:
$L(M) = \{w | (q_0, w) \vdash^* (q, \epsilon) \land q \in F\}$
\end{mydef}

\begin{veta}
Každý \emph{nedeterministický} konečný automat $M$ lze převést na \emph{deterministický} konečný automat $M'$ tak, že $L(M) = L(M')$.
\end{veta}

\begin{alg}
Převod nedeterministického KA na ekvivalentní DKA\\
Vstup: NKA $M = (Q, \Sigma, \delta, q_0, F)$\\
Výstup: NKA $M = (Q', \Sigma, \delta', q_0', F')$\\
Metoda:
\begin{enumerate}
	\item Polož $Q' = 2^Q \backslash \{0\}$
	\item Polož $q_0' = \{q_0\}$
	\item Polož $F' = \{S | S \in 2^Q \land S \cap F \not= \emptyset\}$
	\item Pro všechna $S \in 2^Q \backslash \{\emptyset\}$ a pro všechna $a \in \Sigma$ polož:
	\begin{itemize}
		\item $\delta'(S, a) = \bigcup_{q \in S} \delta(q, a)$, je-li $\bigcup_{q \in S} \delta(q, a) \not= \emptyset$
		\item Jinak $\delta'(S, a)$ není definována.
	\end{itemize}
\end{enumerate}
\end{alg}

\subsection{Lineární a regulární gramatiky}

\begin{mydef}
\begin{itemize}
	\item Gramatika $G = (N, \Sigma, P, S)$ s pravidly tvaru:
	$A \to xB, \quad A, B \in N, x \in \Sigma^*$ nebo
	$A \to x, \quad x \in \Sigma^*$

	respektive tvaru
	$A \to Bx, \quad A, B \in N, x \in \Sigma^*$ nebo
	$A \to x, \quad x \in \Sigma^*$
	se nazývá \emph{pravá lineární}, resp. \emph{levá lineární} gramatika.
	\item Gramatika $G = (N, \Sigma, P, S)$ s pravidly tvaru:
	$A \to xB, \quad A, B \in N, x \in \Sigma^*$,
	$A \to x, \quad x \in \Sigma^*$, případně
	$S \to \epsilon, \quad a \in \Sigma$

	respektive tvaru
	$A \to Bx, \quad A, B \in N, x \in \Sigma^*$,
	$A \to x, \quad x \in \Sigma^*$, případně
	$S \to \epsilon, \quad a \in \Sigma$
	se nazývá \emph{pravá regulární}, resp. \emph{levá regulární} gramatika. Souhrnně \emph{regulární}.
\end{itemize}
\end{mydef}

\emph{Lineární gramatika} je gramatika s pravidly tvaru $A \to xBy$.
\begin{itemize}
	\item $\mathcal{L}_{PL}$ všechny jazyky generované pravými lineárními gramatikami
	\item $\mathcal{L}_{LL}$ všechny jazyky generované levými lineárními gramatikami
	\item $\mathcal{L}_{L}$ všechny jazyky generované lineárními gramatikami
\end{itemize}
$\mathcal{L}_{PL} = \mathcal{L}_{LL}$ a $\mathcal{L}_{PL} \subset \mathcal{L}_{L}$.

\begin{veta}
Každá pravá lineární gramatika $G = (N, \Sigma, P, S)$ (pravidla typu $A \to xB$ nebo $A \to x$, kde $A, B \in N, x \in \Sigma^*$), může být transformována na (pravou regulární) gramatiku $G' = (N', \Sigma, P', S')$, která obsahuje pravidla $A \to aB$ nebo $A \to \epsilon$, přičemž $L(G) = L(G')$.
\end{veta}

\begin{veta}
Každý jazyk typu 3 lze generovat pravou regulární gramatikou.
\end{veta}

\begin{veta}
Každý jazyk typu 3 může být generován levou lineární gramatikou.
\end{veta}

\begin{veta}
Každý jazyk typu 3 lze generovat levou regulární gramatikou.
\end{veta}

\subsection{Ekvivalence třídy $\mathcal{L}_3$ a třídy jazyků přijímaných konečnými automaty}

$\mathcal{L}_M$ třída jazyků přijímaných konečnými automaty.

\begin{veta}
Nechť $L$ je jazyk typu 3. Pak existuje konečný automat $M$ takový, že $L = L(M)$, tj. $\mathcal{L}_3 \subseteq \mathcal{L}_M$.
\end{veta}

\begin{veta}
Nechť $L = L(M)$ pro nějaký konečný automat $M$. Pak existuje gramatika $G$ typu 3 taková, že $L = L(G)$, tj. $\mathcal{L}_M \subseteq \mathcal{L}_3$.
\end{veta}

\begin{veta}
Třída jazyků, jež jsou přijímány konečnými automaty je totožná s třídou jazyků typu 3, Chomského hierarchie.
\end{veta}

\begin{alg}
Konstrukce nedeterministického KA k pravé regulární gramatice\\
Vstup: Pravá regulární gramatika  $G = (N, \Sigma, P, S)$, jejíž pravidla mají tvar $A \to aB$ a $A \to a$, kde $A,B \in N$ a $a \in \Sigma$, případně $S \to \epsilon$ za předpokladu, že $S$ se nevyskytuje na pravé straně žádného pravidla.\\
Výstup: NKA $M = (Q, \Sigma, \delta, q_0, F)$, pro který je $L(M) = L(G)$\\
Metoda:
\begin{enumerate}
	\item Polož $Q = N \cup \{q_F\}$
	\item Množina vstupních symbolů automatu $M$ je identická s množinou terminálů gramatiky $G$
	\item Funkci přechodů $\delta$ definujeme takto:
	\begin{enumerate}
		\item Je-li $A \to aB$ pravidlo z $P$, pak $\delta(A, a)$ obsahuje stav $B$.
		\item Je-li $A \to a$ pravidlo z $P$, pak $\delta(A, a)$ obsahuje stav $q_F$.
	\end{enumerate}
	\item $q_0 = S$
	\item Je-li $S \to \epsilon$ pravidlo z $P$, pak $F = \{S, q_F\}$ v opačném případě $F = \{q_F\}$.
\end{enumerate}
\end{alg}

\begin{veta}
Nechť $G$ je pravá regulární gramatika a $M$ je konečný automat z předchozího algoritmu. Pak $L(M) = L(G)$.
\end{veta}

\begin{alg}
Konstrukce nedeterministického KA k levé regulární gramatice \\
Vstup: Levá regulární gramatika  $G = (N, \Sigma, P, S)$, jejíž pravidla mají tvar $A \to Ba$ a $A \to a$, kde $A,B \in N$ a $a \in \Sigma$, případně $S \to \epsilon$ za předpokladu, že $S$ se nevyskytuje na pravé straně žádného pravidla.\\
Výstup: NKA $M = (Q, \Sigma, \delta, q_0, F)$, pro který je $L(M) = L(G)$\\
Metoda:
\begin{enumerate}
	\item Polož $Q = N \cup \{q_0\}$
	\item Množina vstupních symbolů automatu $M$ je identická s množinou terminálů gramatiky $G$
	\item Funkci přechodů $\delta$ definujeme takto:
	\begin{enumerate}
		\item Je-li $A \to Ba$ pravidlo z $P$, pak $\delta(B, a)$ obsahuje stav $A$.
		\item Je-li $A \to a$ pravidlo z $P$, pak $\delta(q_0, a)$ obsahuje stav $A$.
	\end{enumerate}
	\item $q_0$ je počáteční stav automatu $M$
	\item Je-li $S \to \epsilon$ pravidlo z $P$, pak $F = \{S, q_0\}$ v opačném případě $F = \{S\}$.
\end{enumerate}
\end{alg}

\begin{veta}
Nechť $G$ je pravá regulární gramatika a $M$ je konečný automat z předchozího algoritmu. Pak $L(M) = L(G)$.
\end{veta}

\section{Minimalizace deterministického konečného automatu}

\begin{mydef}
Deterministický konečný automat $M = (Q, \Sigma, \delta, q_0, F)$ nazýváme \emph{úplný deterministický konečný automat}, pokud pro všechna $q \in Q$ a všechna $a \in \Sigma$ platí, že $\delta(q, a) \in Q$, tj. $\delta$ je totální funkcí na $Q \times \Sigma$.
\end{mydef}

\begin{mydef}
Nechť $M=(Q, \Sigma, \delta, q_0, F)$ je konečný automat. Stav $q \in Q$ nazveme \emph{dosažitelný}, pokud existuje $w \in \Sigma^*$ takové, že $(q_0, w) \underset{M}{\overset{*}{\vdash}} (q, \epsilon)$.
Stav je nedosažitelný, pokud není dosažitelný.
\end{mydef}

\begin{alg}
Eliminace nedosažitelných stavů \\
Vstup: Deterministická KA $M = (Q, \Sigma, \delta, q_0, F)$\\
Výstup: Deterministický KA $M'$ bez nedosažitelných stavů, $L(M) = L(M')$\\
Metoda:
\begin{enumerate}
	\item $i := 0$
	\item $S_I := \{q_0\}$
	\item \texttt{repeat}
	\begin{enumerate}
		\item $S_{i+1} := S_i \cup \{q | \exists p \in S_i \exists a \in \Sigma: \delta(p, a) = q \}$
		\item $i := i + 1$
	\end{enumerate}
	\item \texttt{until} $S_i = S_{i-1}$
	\item $M' := (S_i, \Sigma, \delta_{S_i}, q_0, F \cap S_i)$.
\end{enumerate}
\end{alg}

\begin{mydef}
Nechť $M = (Q, \Sigma, \delta, q_0, F)$ je úplný DKA. Říkáme, že řetězec $w \in \Sigma^*$ rozlišuje stavy z $Q$, jestliže
$(q_1, w) \underset{M}{\overset{*}{\vdash}} (q_3, \epsilon) \land (q_2, w) \underset{M}{\overset{*}{\vdash}} (q_4, \epsilon)$ pro nějaké $q_3, q_4$  a právě jeden ze stavů $q_3, q_4$ je v $F$. Říkáme, že stavy $q_1, q_2 \in Q$ jsou k-nerozlišitelné a píšeme $q_1 \overset{k}{\equiv} q_2$, právě když neexistuje $w \in \Sigma^*, |w| \leq k$, který rozlišuje $q_1$ a $q_2$. Stavy jsou nerozlišitelné, značíme $q_1 \equiv q_2$, jsou-li pro každé $k \geq 0$, $k$-nerozlišitelné.
\end{mydef}

\begin{mydef}
Úplně definovaný DKA nazýváme \emph{redukovaný}, jestliže žádný stav $Q$ není nedostupný a žádné dva stavy nejsou nerozlišitelné.
\end{mydef}

\begin{veta}
Nechť $M=(Q,\Sigma,\delta,q_0,F)$ je úplný DKA a $|Q| = n, n \geq 2$. Platí $\forall q_1, q_2 \in Q: q_1 \equiv q_2 \Leftrightarrow q_1 \overset{n-2}{\equiv} q_2$.
\end{veta}

\begin{alg}
Převod na redukovaná deterministický konečný automat\\
Vstup: Úplně definovaný DKA $M = (Q, \Sigma, \delta, q_0, F)$\\
Výstup: Redukovaný DKA $M'$ bez nedosažitelných stavů, $L(M) = L(M')$\\
Metoda:
\begin{enumerate}
	\item Odstraň nedosažitelné stavy využitím předchozího algoritmu
	\item $i := 0$
	\item $\overset{0}{\equiv} := \{(p, q)|p \in F \Leftrightarrow q \in F\}$
	\item \texttt{repeat}
	\begin{enumerate}
		\item $\overset{i+1}{\equiv} := \{(p, q)|p \overset{i}{\equiv} q \land \forall a \in \Sigma: \delta(p,a) \overset{i}{\equiv} \delta(q, a)\}$
		\item $i := i + 1$
	\end{enumerate}
	\item \texttt{until} $\overset{i+1}{\equiv} = \overset{i}{\equiv}$
	\item $Q' := Q/\overset{i}{\equiv}$
	\item $\forall p,q \in Q \forall a \in \Sigma: \delta'([p], a) = [q] \Leftrightarrow \delta(p,a) = q$
	\item $q_0' = [q_0]$
	\item $F' = \{[q] | q \in F\}$
\end{enumerate}
(výraz $[x]$ značí ekvivalenční třídu určenou prvkem $x$)
\end{alg}







%%%%%%%%%%%%%%%%%%%%%%%%%%%%%%%%%%%%%%%%%%%%%%%%%%%%%%%%%%%%%%%%%%%%%%%%%%%%%%%%
%%%%%%%%%%%%%%%%%%%%%%%%%%%%%%%%%%%%%%%%%%%%%%%%%%%%%%%%%%%%%%%%%%%%%%%%%%%%%%%%
\chapter{Regulární množiny, regulární výrazy a rovnice nad regulárními výrazy.} \label{cha:19}
1. semestr, TIN, \texttt{opora.pdf}, 3. kapitola 3. sekce

\section{Regulární množiny}

\begin{mydef}
Nechť $\Sigma$ je konečná abeceda. \emph{Regulární množinu} nad abecedou $\Sigma$ definujeme rekurzivně takto:
\begin{enumerate}
	\item $\emptyset$ (prázdná množina) je regulární množina nad $\Sigma$
	\item $\{ \epsilon \}$ (množina obsahující pouze prázdný řetězec) je regulární množina nad $\Sigma$
	\item $\{ a \}$ pro všechna $a \in \Sigma$ je regulární množina nad $\Sigma$
	\item Jsou-li $P$ a $Q$ regulární množiny nad $\Sigma$, pak také $P \cup Q, P \cdot Q$ a $P^*$ jsou regulární množiny nad $\Sigma$.
	\item regulárními množinami jsou právě ty množiny, které lze získat aplikací 1 -- 4.
\end{enumerate}
\end{mydef}
Nejmenší třída jazyků, uzavřená vzhledem k operacím sjednocení, součinu a iterace.

\begin{veta}
Nechť $\Sigma$ je konečná abeceda. Pak $\emptyset, \{\epsilon\}, \{a\}$ (pro všechna $a \in \Sigma$ jsou jazyky typu 3 nad abecedou $\Sigma$.
\end{veta}

\begin{veta}
Nechť $L_1$ a $L_2$ jsou jazyky typu 3 nad abecedou $\Sigma$. Pak $L_1 \cup L_2$, $L_1 \cdot L_2$, $L_1^*$ jsou jazyky typu 3 nad abecedou $\Sigma$.
\end{veta}

\section{Regulární výrazy}

Notace pro reprezentaci regulárních množin.

\begin{mydef}
Regulární výrazy nad $\Sigma$ a regulární množiny, které označují, jsou rekurzivně definovány takto:
\begin{enumerate}
	\item $\emptyset$ je regulární výraz označující regulární množinu $\emptyset$
	\item $\epsilon$ je regulární výraz označující regulární množinu $\{ \epsilon \}$
	\item $a$ je regulární výraz označující regulární množinu $\{ a \}$ pro všechna $a \in \Sigma$
	\item Jsou-li $p$ a $q$ regulární výrazy označující regulární množiny $P$ a $Q$, pak
	\begin{enumerate}
		\item $(p+q)$ je regulární výraz označující regulární množinu $P \cup Q$
		\item $(pq)$ je regulární výraz označující regulární množinu $P \cdot Q$
		\item $(p*)$ je regulární výraz označující regulární množinu $P^*$
	\end{enumerate}
	\item Žádní jiné regulární výrazy na $\Sigma$ neexistují
\end{enumerate}
\end{mydef}

Regulární výraz $p^+$ značí regulární výraz $pp^*$; Abychom minimalizovali počet závorek, stanovujeme priority operátorů $^*$ a $^+$ (iterace -- nejvyšší priorita), $\cdot$ (konkatenace), $+$ (alternativa).

\begin{mydef}
Kleeneho algebra sestává z neprázdné množiny se dvěma význačnými konstantami $0$ a $1$, dvěma binárními operacemi $+$ a $\cdot$ a unární operací $^*$, které splňují následující axiomy:

\begin{subequations}
\begin{align*}
	& a + (b + c) = (a + b) + c		& \text{ asociativita } + 				\quad & [A.1] \\
	& a + b = b + a					& \text{ komutativita } +				\quad & [A.2] \\
	& a + a = a						& \text{ idempotence } + 				\quad & [A.3] \\
	& a + 0 = a						& 0 \text{ je identitou pro } + 		\quad & [A.4] \\
	& a(bc) = (ab)c					& \text{ asociativita } \cdot 			\quad & [A.5] \\
	& a1 = 1a = a					& 1 \text{ je identitou pro } \cdot		\quad & [A.6] \\
	& a0 = 0a = 0					& 0 \text{ je anihilátorem pro } \cdot	\quad & [A.7] \\
	& a(b + c) = ab + ac			& \text{ distributivita zleva }			\quad & [A.8] \\
	& (a + b)c = ac + bc			& \text{ distributivita zprava }		\quad & [A.9] \\
	& 1 + aa^* = a^*				&										\quad & [A.10] \\
	& 1 + a^*a = a^*				&										\quad & [A.11] \\
	& b + ac \leq c \Rightarrow a^*b \leq c &								\quad & [A.12] \\
	& b + ca \leq c \Rightarrow ba^* \leq c &								\quad & [A.13] \\
\end{align*}
\end{subequations}
(poslední dva axiomy reprezentují $\leq$ uspořádání definované takto: $a \leq b \xLeftrightarrow{def} a+b=b$.
\end{mydef}

Některé teorémy Kleeneho algebry:
\begin{subequations}
\begin{align*}
	& 0^* = 1						& \\
	& 1 + a^* = a^*					& \\
	& a^* = a + a^*					& \\
	& a^* a^* = a^*					& \\
	& a^{*^*} = a^*					& \\
	& (a^*b)^*a* = (a + b)^*		& \text{ pravidlo \uv{vynořování} }		\quad & [R.16] \\
	& a (ba)^* = (ab)^* a			& \text{ pravidlo posuvu }				\quad & [R.14] \\
	& a^* = (aa)^* + a (aa)^*		& \\
\end{align*}
\end{subequations}

\section{Rovnice nad regulárními výrazy}

\begin{mydef}
Rovnice, jejímiž složkami jsou koeficienty a neznámí, které reprezentují (dané a hledaní) regulární výrazy, nazýváme \emph{rovnicemi nad regulárními výrazy}.
\end{mydef}

Nemusí existovat jediné řešení, obvykle hledáme nejmenší řešení, nejmenší pevný bod.

\begin{veta}
Nejmenším pevným bodem rovnice $X = pX + q$ je $X = p^*q$
\end{veta}

\section{Soustavy rovnic nad regulárními výrazy}

\begin{mydef}
Soustava rovnic nad regulárními výrazy je ve \emph{standardním tvaru} vzhledem k neznámým $\Delta = \{X_1, X_2, \dots, X_n\}$, má-li soustava tvar

$$\bigwedge_{i \in \{1, \dots, n} X_i = \alpha_{i0} + \alpha_{i1}X_1 + \alpha_{i2}X_2 + \dots + \alpha_{in}X_n$$

kde $\alpha_{ij}$ jsou regulární výrazy nad nějakou abecedou $\Sigma, \Sigma \cap \Delta = \emptyset$.
\end{mydef}

\begin{veta}
Je-li soustava rovnic nad regulárními výrazy ve standardním tvaru, pak existuje její minimální pevný bod a algoritmus k jeho nalezení.
\end{veta}
Algoritmus: Vyjadřujeme hodnotu jednotlivých proměnných pomocí řešení rovnice $X=pX + q$ jako regulární výraz s proměnnými, jejichž počet se postupně snižuje:
Z rovnice pro $X_n$ vyjádříme např. $X_n$ jako regulární výraz nad $\Sigma$ a $X_1, \dots, X_{n-1}$.
Dosadíme za $X_n$ do rovnice $X_{n-1}$ a postup opakujeme.
Jsou přitom možné (ale ne nutné) různé optimalizace tohoto pořadí.

\begin{veta}
Každý jazyk generovaný gramatikou typu 3 je regulární množinou $\mathcal{L}_3 \subseteq \mathcal{L}_R$.
\end{veta}
Vytvoříme soustavu rovnic nad regulárními výrazy s proměnnými $X_1, \dots, X_n$ ve standardním tvaru popisující množinu řetězců přijímaných ze stavu $Q_i$. Koncové stavy přijímají $\epsilon$.

Regulární přechodový graf je zobecněním konečného automatu s regulárními výrazy na hranách.

\section{Převod regulárních výrazů na konečné automaty}

\begin{mydef}
\emph{Rozšířený konečný automat} (RKA) je pětice $M=(Q, \Sigma, \delta, q_0, F)$, kde:
\begin{enumerate}
	\item $Q$ je konečná množina stavů
	\item $\Sigma$ konečná vstupní abeceda
	\item $\delta$ je zobrazení $Q \times (\Sigma \cup \{\epsilon\}) \to 2^Q$
	\item $q_0 \in Q$ je počáteční stav
	\item $F \subseteq Q$ je množina koncových stavů
\end{enumerate}
\end{mydef}

\begin{mydef}
Klíčovou funkci v algoritmu převodu RKA na DKA má výpočet funkce, která k danému stavu určí množinu všech stavů, jež jsou dostupné po $\epsilon$ hranách diagramu přechodů funkce $\delta$. Označme tuto funkci $\epsilon$-uzávěr:
$$\epsilon\text{-uzávěr}(q) = \{p | \exists w \in \Sigma^*: (q, w) \overset{*}{\vdash} (p, w)\}$$
Funkci $\epsilon$-uzávěr zobecníme tak, aby argumentem mohla být množina $T \in Q$:
$$\epsilon\text{-uzávěr}(q) = \bigcup\limits_{s \in T} \epsilon\text{-uzávěr}(s)$$
\end{mydef}

\begin{alg}
Převod rozšířeného konečného automatu na deterministický konečný automat\\
Vstup: Rozšířený konečný automat $M = (Q, \Sigma, \delta, q_0, F)$\\
Výstup: Deterministický konečný automat $M = (Q', \Sigma, \delta', q_0', F')$, $L(M) = L(M')$\\
Metoda:
\begin{enumerate}
	\item $Q := 2^Q \backslash \{\emptyset\}$
	\item $q' := \epsilon\text{-uzávěr}(q_0)$
	\item $\delta': Q' \times \Sigma \to Q'$ je vypočtena takto:
	\begin{enumerate}
		\item Nechť $\forall T \in Q', a \in \Sigma: \overline{\delta}(T, a) = \bigcup_{q \in T}(q, a)$
		\item Pak pro každé $T \in Q', a \in \Sigma$:
		\begin{enumerate}
			\item pokud $\overline{\delta}(T, a) \not= \emptyset$, pak $\delta'(T, a) = \epsilon\text{-uzávěr}(\overline{\delta}(T, a))$
			\item jinak $\delta'(T, a)$ není definována
		\end{enumerate}
	\end{enumerate}
	\item $F' := \{S | S \ in Q' \land S \cap F \not= \emptyset\}$
\end{enumerate}
\end{alg}

\begin{alg}
Převod regulárního výrazu na rozšířený konečný automat\\
Vstup: Regulární výraz $r$ popisující regulární množinu $R$ nad $\Sigma$\\
Výstup: Rozšířený konečný automat $M$, takový, že $L(M) = R$\\
Metoda:
\begin{enumerate}
	\item Rozložíme $r$ na jeho primitivní složky podle rekurzivní definice regulární množiny/výrazu
	\begin{enumerate}
		\item Pro výraz $\epsilon$ zkonstruujeme automat 
\begin{tikzpicture}[shorten >=1pt,node distance=1.5cm,on grid,auto]
   \node[state,initial] (s)   {$s$};
   \node[state,accepting](f) [right=of s] {$f$};
   \path[->] (s) edge  node {$\epsilon$} (f);
\end{tikzpicture}
		\item Pro výraz $a$, $a \in \Sigma$ zkonstruujeme automat
\begin{tikzpicture}[shorten >=1pt,node distance=1.5cm,on grid,auto]
   \node[state,initial] (s)   {$s$};
   \node[state,accepting](f) [right=of s] {$f$};
   \path[->] (s) edge  node {$a$} (f);
\end{tikzpicture}
		\item Pro výraz $\emptyset$ zkonstruujeme automat
\begin{tikzpicture}[shorten >=1pt,node distance=1.5cm,on grid,auto]
   \node[state,initial] (s)   {$s$};
   \node[state,accepting](f) [right=of s] {$f$};
\end{tikzpicture}
		\item Nechť $N_1$ je automat přijímající jazyk specifikovaný výrazem $r_1$ a $N_2$ je automat přijímající jazyk specifikovaný výrazem $r_2$.
		\begin{enumerate}
			\item Pro výraz $r_1 + r_2$ zkonstruujeme automat
\begin{tikzpicture}[shorten >=1pt,node distance=1.5cm,on grid,auto]
	\node[state,initial] (s)   {$s$};
	\node[state] (n1) [above right=of s]  {};
	\node[state] (n2) [right=of s]  {};
	\node[state] (n3) [right=of n1]  {};
	\node[state] (n4) [right=of n2]  {};
	\node[state,accepting](f) [right=of n4] {$f$};
	\path[->]
		(s) edge  node {$\epsilon$} (n1)
		(s) edge  node {$\epsilon$} (n2)
		(n3) edge  node {$\epsilon$} (f)
		(n4) edge  node {$\epsilon$} (f)
		;
	\node [draw=red, fit= (n1) (n3)] {$N_1$};
	\node [draw=blue, fit= (n2) (n4)] {$N_2$};
\end{tikzpicture}
			\item Pro výraz $r_1 r_2$ zkonstruujeme automat
\begin{tikzpicture}[shorten >=1pt,node distance=1.5cm,on grid,auto]
	\node[state,initial] (n1) {};
	\node[state] (n2) [right=of n1]  {};
	\node[state,accepting] (n3) [right=of n2]  {};
	\node [draw=red, fit= (n1) (n2)] {$N_1$};
	\node [draw=blue, fit= (n2) (n3)] {$N_2$};
\end{tikzpicture}
			\item Pro výraz $r_1^*$ zkonstruujeme automat
\begin{tikzpicture}[shorten >=1pt,node distance=1.5cm,on grid,auto]
	\node[state,initial] (s)   {$s$};
	\node[state] (n1) [right=of s]  {};
	\node[state] (n2) [right=of n1]  {};
	\node[state,accepting](f) [right=of n2] {$f$};
	\path[->]
		(s) edge  node {$\epsilon$} (n1)
		(s) edge [bend right,above] node {$\epsilon$} (f)
		(n2) edge [bend right,below] node {$\epsilon$} (n1)
		(n2) edge  node {$\epsilon$} (f)
		;
	\node (N) [draw=red, fit= (n1) (n2)] {};
	\node [yshift=2.0ex] at (N.south) {$N_1$};
\end{tikzpicture}
		\end{enumerate}
	\end{enumerate}
\end{enumerate}
\end{alg}

Konečné automaty, gramatiky typu 3 a regulární výrazy mají ekvivalentní vyjadřovací sílu.





























%%%%%%%%%%%%%%%%%%%%%%%%%%%%%%%%%%%%%%%%%%%%%%%%%%%%%%%%%%%%%%%%%%%%%%%%%%%%%%%%
%%%%%%%%%%%%%%%%%%%%%%%%%%%%%%%%%%%%%%%%%%%%%%%%%%%%%%%%%%%%%%%%%%%%%%%%%%%%%%%%
\chapter{Transformace a normální formy bezkontextových gramatik.} \label{cha:20}
1. semestr, TIN, \texttt{opora.pdf}, 4. kapitola, sekce 1. - 8.

\begin{mydef}
Bezkontextová gramatika $G$ je čtveříce $G = (N, \Sigma, P, S)$
\begin{enumerate}
	\item $N$ je konečná množina nonterminálních symbolů
	\item $\Sigma$ je konečná množina terminálních symbolů
	\item $P$ je konečná množina přepisovacích pravidel tvaru $A \to \alpha, A \in N$ a $\alpha \in (N \cup \Sigma)$.
	\item $S$ je výchozí symbol gramatiky.
\end{enumerate}
\end{mydef}

\section{Derivační strom}
Strom je orientovaný acyklický graf s těmito vlastnostmi:
\begin{enumerate}
	\item Existuje jediný uzel, tzv. \emph{kořen stromu}, do něhož nevstupuje žádná hrana
	\item Do všech ostatních uzlů vstupuje právě jedna hrana.
\end{enumerate}
Uzly z nichž žádná hrana nevystupuje, se nazývají \emph{koncové uzly stromu} (listy).

\begin{mydef}
Nechť $\delta$ je věta nebo větná forma generovaná v gramatice $G=(N,\Sigma,P,S)$ a nechť $S=\nu_0 \Rightarrow \nu_1 \Rightarrow \dots \Rightarrow \nu_k = \delta$ její derivace v $G$. Derivační strom příslušející této derivaci je strom s těmito vlastnostmi:
\begin{enumerate}
	\item Uzly derivačního stromu jsou (ohodnoceny) symboly z množiny $N \cup \Sigma$; kořen stromu je označen výchozím symbolem $S$.
	\item Přímé derivací $\nu_{i-1} \Rightarrow \nu_i, i \in 1,2 \dots, k$, kde
	
	$\nu_{i-1} = \mu A \lambda, \mu, \lambda \in (N \cup \Sigma)^*, A \in N\\
	\nu_i = \mu \alpha \lambda \\
	A \to \alpha, \alpha = X_1, \dots X_n \text{ je pravidlo z } P$
	
	odpovídá právě $n$ hran $(A, X_j), j =  1, \dots, n$ vycházejících z uzlu $A$ jež jsou uspořádány zleva doprava v pořadí $(A, X_1), (A, X_2), \dots, (A, X_n)$.
	
	\item Označení koncových uzlů derivačního stromu vytváří zleva doprava větnou formu nebo větu $\delta$.
\end{enumerate}
\end{mydef}

\begin{mydef}
Nechť $S = \alpha_1 \Rightarrow \alpha_2 \Rightarrow \dots \Rightarrow \alpha_n = \alpha$ je derivace větné formy $\alpha$. Jestliže byl v každém řetězci $\alpha_i, i = 1, \dots, n-1$ přepsán nelevější (nejpravější) nonterminál, pak tuto derivaci nazýváme levou (pravou) derivací větné formy $\alpha$.
\end{mydef}

\section{Fráze větné formy}
\begin{mydef}
Nechť $G = (N, \Sigma, P, S)$ je gramatika a nechť řetězec $\lambda = \alpha \beta \gamma$ je větná forma. Podřetězec $\beta$ se nazývá \emph{frází větné formy} $\lambda$, vzhledem k nonterminálu $A$ z $N$, jestliže platí:
$$S \Rightarrow^* \alpha A \gamma$$
$$A \Rightarrow^+ \alpha \beta$$
Podřetězec $\beta$ je \emph{jednoduchou frází větné formy} $\lambda$, jestliže platí
$$S \Rightarrow^* \alpha A \gamma$$
$$A \Rightarrow \alpha \beta$$
\end{mydef}

Nejlevější jednoduchá fráze se nazývá \emph{l-fráze}.

\section{Víceznačnost gramatik}

\begin{mydef}
Nechť $G$ je gramatika. Říkáme, že věta $w$ generovaná gramatikou $G$ je víceznačná, existují-li aspoň dvě různé derivační stromy s koncovými uzly tvořící větu $w$.
\emph{Gramatika $G$ je víceznačná}, jestliže generuje alespoň jednu víceznačnou větu. V opačném případě mluvíme o jednoznačné gramatice.
\end{mydef}

Jazyky, které nelze generovat jednoznačnou gramatikou se nazývají jazyky s \emph{inherentní víceznačností}.

Pro bezkontextové gramatiky je dokázáno, že problém, zda daná gramatika je nebo není víceznačná, je nerozhodnutelný, tj. neexistuje algoritmus, který by v konečném čase odhalil víceznačnost každé bezkontextové gramatiky.

Gramatika s pravidlem $A \to A$ je zřejmě víceznačná. Tot pravidlo můžeme vypustit, aniž bychom změnili jazyk generovaný takto zredukovanou gramatikou.

Nejednoznačná gramatika podmíněných příkazů: \texttt{if (x) if (y) S else S}. Nežádoucí interpretace bývá zajištěna dodatečným sémantickým pravidlem.

\section{Rozklad věty}
Konstrukci derivace či derivačního stromu pro danou větu nebo větnou formu nazýváme \emph{rozkladem} nebo \emph{syntaktickou analýzou} této věty nebo větné formy. Pro gram, který provádí rozklad vět určitého jazyka se nazývá \emph{syntaktický analyzátor} (parser).

\begin{itemize}
	\item Syntaktická analýza shora dolů (od výchozího symbolu gramatiky)
	\item Syntaktická analýza zdola nahoru (od koncových uzlů)
\end{itemize}

Hledání prvního podřetězce věty (v dalších krocích větných forem), nazývaného \emph{l-fráze}. Problém: Které pravidlo použít? Jak určit začátek a konec l-fráze? Možnosti:
\begin{itemize}
	\item Syntaktická analýza s návratem: Náhodný výběr některého z pravidel a návrat, pokud alternativa nebyla správná. 
	\item Syntaktická analýza bez návratu (Deterministické gramatiky): které dokáží na základě kontextu zpracovávaného řetězce určit správnou alternativu v každém kroku analýzy.
\end{itemize}

\section{Transformace bezkontextových gramatik}

\begin{mydef}
Říkáme, že gramatiky $G_1$ a $G_2$ jsou ekvivalentní, jestliže platí $L(G_1) = L(G_2)$, tj. jestliže jimi generované jazyky jsou totožné.
\end{mydef}

\begin{mydef}
Nechť $G=(N,\Sigma,P,S)$ je gramatika. Říkáme, že symbol $X \in (N \cup \Sigma)$ je v gramatice $G$ zbytečný, jestliže neexistuje derivace tvaru $S \Rightarrow^* w X y \Rightarrow^* wxy$, kde řetězce $w, x, y \in \Sigma^*$.
\end{mydef}

\begin{alg}
Je $L(G)$ neprázdný?\\
Vstup: Gramatika $G = (N, \Sigma, P, S)$\\
Výstup: ANO je-li $L(G) \not= \emptyset$, NE v opačném případě.\\
Metoda:
\begin{enumerate}
	\item $N_0 = \emptyset, i = 1$
	\item $N_1 = \{A | A \to \alpha \in P \land \alpha \in (N_{i-1} \cup \Sigma)^*\}$
	\item Je-li $N_i \not= N_{i-1}$, polož $i = i + 1$ a vrať se ke kroku 2.
	\item Je-li $N_i = N_{n-1}$ polož $N_t = N_i$
	\item Jestliže výchozí symbol $S$ je v $N_t$, pak je výstup ANO, jinak NE
\end{enumerate}
\end{alg}
Algoritmus skončí po max. $n+1$ krocích, kde $n$ je počet nonterminálů.

\begin{veta}
Předchozí algoritmus má výstup ANO, právě když $S \Rightarrow^* w$ pro nějaké $w \in \Sigma^*$.
\end{veta}

\begin{mydef}
Říkáme, že symbol $X \in (N \cup \Sigma)$ je \emph{nedostupný} v gramatice $G = (N, \Sigma, T, S)$, jestliže $X$ se nemůže objevit v žádné větné formě.
\end{mydef}

\begin{alg}
Odstranění nedostupných symbolů\\
Vstup: Gramatika $G = (N, \Sigma, P, S)$\\
Výstup: Gramatika $G' = (N', \Sigma', P', S')$, pro kterou platí:
\begin{itemize}
	\item $L(G) = L(G')$
	\item Pro všechna $X \in (X' \cup \Sigma')$ existují řetězce $\alpha, \beta \in (N' \cup \Sigma')^*$ tak, že $S \Rightarrow^* \alpha X \beta \in G'$.
\end{itemize}
Metoda:
\begin{enumerate}
	\item Položíme $V_0 = \{S\}$ a $i = 1$
	\item Konstruujeme $V_1 = \{X | A \to \alpha X \beta \in P \land A \in V_{i-1}\} \cup V_{i-1}$
	\item Je-li $V_i \not= V_{i-1}$, polož $i = i + 1$ a vrať se ke kroku 2.
	\item Je-li $V_i = N_{n-1}$, pak:
	\begin{itemize}
		\item $N' = V_i \cap N$
		\item $\Sigma' = V_i \cap \Sigma$
		\item $P' \subseteq P$ obsahuje ta pravidla, která jsou tvořena pouze symboly z $V_i$.
	\end{itemize}
\end{enumerate}
\end{alg}

\begin{alg}
Odstranění zbytečných symbolů\\
Vstup: Gramatika $G = (N, \Sigma, P, S)$ generující neprázdný jazyk\\
Výstup: Gramatika $G' = (N', \Sigma', P', S')$, pro kterou platí:
\begin{itemize}
	\item $L(G) = L(G')$
	\item Žádný symbol z $X \in (X' \cup \Sigma')$ není zbytečný
\end{itemize}
Metoda:
\begin{enumerate}
	\item Na gramatiky $G$ aplikuj předchozí algoritmus 1 a získej množinu $N_t$ obsahující symboly generující věty jazyka.\\
	Polož $\overline{G} = (N_t \cup \{S\}, \Sigma, P_1, S)$, kde $P_1$ obsahuje pravidla tvořená pouze symboly z $N_t \cup \Sigma$.
	\item Algoritmus 2 odstraňující nedostupné symboly aplikuj na gramatiku $\overline{G}$. Výsledkem je gramatika $G' = (N', \Sigma', P', S)$, která neobsahuje zbytečné symboly.
\end{enumerate}
\end{alg}

\begin{veta}
Gramatika $G'$ z předchozího algoritmu je ekvivalentní gramatice  $G$ a nemá zbytečné symboly.
\end{veta}

\begin{mydef}
Říkáme, že gramatika $G = (N, \Sigma, P, S)$ je \emph{gramatikou bez $\epsilon$-pravidel}, jestliže buď $P$ neobsahuje žádné $\epsilon$-pravidlo, nebo v případě $\epsilon \in L(G)$, existuje jediné $\epsilon$-pravidlo tvaru $S \to \epsilon$ a výchozí symbol $S$ se nevyskytuje na pravé straně žádného pravidla z $P$.
\end{mydef}

\begin{alg}
Transformace na gramatiku bez $\epsilon$-pravidel\\
Vstup: Gramatika $G = (N, \Sigma, P, S)$ generující neprázdný jazyk\\
Výstup: Ekvivalentní gramatika $G' = (N', \Sigma', P', S')$ bez $\epsilon$-pravidel\\
Metoda:
\begin{enumerate}
	\item Sestroj $N_\epsilon = \{ A | A \in N \land A \Rightarrow^* \epsilon \}$. Konstrukce analogická konstrukci $N_t$ v prvním algoritmu.
	\item Nechť $P'$ je množina pravidel, kterou konstruujeme takto:
	\begin{enumerate}
		\item Jestliže $A \to \alpha_0 B_1 \alpha_1 B_2 \dots B_k \alpha_k \in P, k \geq 0$ a každé $B_i \in N_\epsilon, 1 \leq i \leq k$, avšak žádný ze symbolů řetězců $\alpha_j$ není v $N_\epsilon, 0 \leq j \leq k$, pak k $P'$ přidej všechna nová pravidla tvaru
		$$A \to \alpha_0 X_1 \alpha_1 X_2 \dots X_k \alpha_k$$
		kde $X_i$ je buď $B_i$ nebo $\epsilon$. Nepřidávej $\epsilon$-pravidlo $A \to \epsilon$, které se objeví, jsou-li všechna $\alpha_i = \epsilon$.
		\item Jestliže $S \in N_\epsilon$ pak k $P'$ přidej pravidla
		$$S' \to \epsilon | S $$
		kde $S'$ je nový výchozí symbol. Poté polož $N' = N \cup \{S'\}$. Jestliže $S \not\in N_\epsilon$, pak $N' = N$ a $S' = S$
	\end{enumerate}
	\item Výsledná gramatika má tvar $G' = (N', \Sigma', P', S')$
\end{enumerate}
\end{alg}

\begin{veta}
Předchozí algoritmus převádí vstupní gramatiku $G$ na ekvivalentní gramatiku $G'$ bez $\epsilon$-pravidel.
\end{veta}

\begin{mydef}
Přepisovací pravidlo $A \to B, A,B \in N$ se nazývá jednoduché pravidlo.
\end{mydef}

\begin{alg}
Odstranění jednoduchých pravidel\\
Vstup: Gramatika $G = (N, \Sigma, P, S)$ bez $\epsilon$-pravidel\\
Výstup: Ekvivalentní gramatika $G' = (N, \Sigma, P', S)$ bez jednoduchých pravidel\\
Metoda:
\begin{enumerate}
	\item Pro každé $A \in N$ sestroj $N_A = \{ B | A \Rightarrow^* B \}$ takto:
	\begin{enumerate}
		\item $N_0 = \{A\}, i = 1$
		\item $N_1 = \{C | B \to C \in P \land B \in N_{i-1}\}$
		\item Jestliže $N_i \not= N_{i-1}$ polož $i = i + 1$ a opakuj krok b)
		\item V opačném případě je $N_A = N_i$.
	\end{enumerate}
	\item Sestroj $P'$ takto: Jestliže $B \to \alpha \in P$ a není jednoduchým pravidlem, pak pro všechna $A$ pro něž platí $B \in N_A$ přidej k $P'$ pravidla $A \to \alpha$
	\item Výsledná gramatika je $G' = (N, \Sigma, P', S)$
\end{enumerate}
\end{alg}

\begin{veta}
Předchozí algoritmus převádí vstupní gramatiku $G$ na ekvivalentní gramatiku $G'$ bez jednoduchých pravidel.
\end{veta}

\begin{mydef}
Říkáme, že $G$ je \emph{gramatika bez cyklů}, jestliže v ní neexistuje derivace tvaru $A \Rightarrow^+ A$ pro žádné $A \in N$. Jestliže $G$ je gramatika bez cyklů a bez $\epsilon$-pravidel a nemá žádné zbytečné symboly, pak říkáme, že $G$ je \emph{vlastní gramatika}.
\end{mydef}

\begin{veta}
Je-li $L$ bezkontextový jazyk, pak existuje jeho vlastní gramatika $G$ taková, že $L = L(G)$.
\end{veta}

Gramatika, která obsahuje cykly je víceznačná a nemůže být použita v konstrukci deterministického syntaktického analyzátoru.

\begin{mydef}
Nechť $G=(N, \Sigma, P, S)$. Přepisovací pravidlo z $P$ se nazývá \emph{rekurzivní zleva (rekurzivní zprava)}, jestliže je tvaru $A \to A \alpha (A \to \alpha A), A \in N, \alpha \in (N \cup \Sigma)^*$. Jestliže v $G$ existuje derivace $A \Rightarrow^+ \alpha A \beta$ pro nějaké $A \in N$, říkáme, že gramatika $G$ je rekurzivní. Je-li $\alpha = \epsilon$, pak mluvíme o \emph{gramatice rekurzivní zleva}, je-li $\beta = \epsilon$, pak říkáme, že $G$ je \emph{rekurzivní zprava}.
\end{mydef}

(je-li jazyk $L(G)$ nekonečný, pak $G$ musí být rekurzivní.)

\begin{mydef}
Pravidlo $A \to \alpha$, s levou stranou tvořenou nonterminálem $A$, budeme nazývat $A$-pravidlo.
\end{mydef}

\begin{veta}
Nechť $G = (N, \Sigma, P, S)$ je gramatika a nechť $A \to A \alpha_1 | A \alpha_2 | \dots | A \alpha_m | \beta_1 | \beta_2 | \dots | \beta_n$ jsou všechna $A$-pravidla. Žádný z řetězců $\beta_i$ nezačíná nonterminálem $A$. Gramatika $G' = (N \cup \{A'\}, \Sigma, P', S)$ kde $P'$ obsahuje namísto uvedených pravidel pravidla
$$A \to \beta_1 | \beta_2 | \dots | \beta_n | \beta_1 A' | \beta_2 A' | \dots | \beta_n A' $$
$$A' \to \alpha_1 | \alpha_2 | \dots | \alpha_n | \alpha_1 A' | \alpha_2 A' | \dots | \alpha_n A' $$
je ekvivalentní s gramatikou $G$, tj. $L(G) = L(G')$.
\end{veta}
(náhrada pravidel rekurzivních zleva, pravidly rekurzivními zprava.)

\begin{veta}
Nechť $G = (N, \Sigma, P, S)$ je gramatika. $A \to \alpha B \beta, B \in N, \alpha, \beta \in (N \cup \Sigma)^*$ je pravidlo z $P$ a $B \to \gamma_1 | \gamma_2 | \dots | \gamma_n$ jsou všechna $B$-pravidla v P. Nechť $G = (N, \Sigma, P', S)$, kde
$$P' = P - \{A \to \alpha B \beta\} \cup \{A \to \alpha \gamma_1 \beta | \alpha \gamma_2 \beta | \dots | \alpha \gamma_n \beta\}$$
Pak $L(G) = L(G')$.
\end{veta}

\begin{alg}
Odstranění levé rekurze\\
Vstup: Vlastní gramatika $G = (N, \Sigma, P, S)$\\
Výstup: Gramatika $G'$ bez levé rekurze\\
Metoda:
\begin{enumerate}
	\item Nechť $N = \{A_1, A_2, \dots, A_n\}$. Gramatiku budeme transformovat tak, že je-li $A_i \to \alpha$ pravidlo, pak $\alpha$ začíná buď terminálem, nebo nonterminálem $A_j, j > i$. K tomuto účelu položíme $i = 1$.
	\item Nechť $A_i \to A_i \alpha | \dots | A_i \alpha_m | \beta_1 | \dots | \beta_p$ jsou všechna $A_I$-pravidla a nechť žádné $\beta_i$ nezačíná nonterminálem $A_k$, je-li $k \leq i$.

	Nahraď všechna $A_i$-pravidla těmito pravidly:
	$$A_i \to \beta_1 | \dots | \beta_p | \beta_1 A'_i | \dots | \beta_p A'_i$$
	$$A'_i \to \alpha_1 | \dots | \alpha_m | \alpha_1 A'_i | \dots | \alpha_m A'_i$$
	kde $A'_i$ je nový nonterminál. Takto všechna $A_i$-pravidla začínají buď terminálem, nebo nonterminálem $A_k, k > i$.
	\item Je-li $i = n$, pak jsme získali výslednou gramatiku $G'$. V opačném případě polož $i = i +1$ a $j = 1$.
	\item Každé pravidlo tvaru $A_i \to A_j \alpha$ nahraď pravidly $A_i \to \beta_1 \alpha | \dots | \beta_p \alpha$ kde $A_j \to \beta_1 | \dots | \beta_p$ jsou všechna $A_j$-pravidla. Po této transformaci budou všechna $A_j$-pravidla začínat buď terminálem, nebo nonterminálem $A_k, k > j$, takže také všechna $A_i$ -pravidla budou mít tuto vlastnost.
	\item Je-li $j = i - 1$, pak přejdi ke kroku (2). Jinak $j = j +1 $ a opakuj krok (4).
\end{enumerate}
\end{alg}

\begin{mydef}
Každý bezkontextový jazyk lze generovat gramatikou bez levé rekurze.
\end{mydef}

\section{Chomského normální forma}

\begin{mydef}
Gramatika $G = (N, \Sigma, P, S)$ je v \emph{Chomského normální formě} (CNF), jestliže každé pravidlo z $P$ má jeden z těchto tvarů:
\begin{enumerate}
	\item $A \to BC, A, B, C \in N$ nebo
	\item $A \to a, a \in \Sigma$ nebo
	\item Je-li $\epsilon \in L(G)$, pak $S \to \epsilon$ je pravidlo z $P$ a výchozí symbol $S$ se neobjeví na pravé straně žádného pravidla.
\end{enumerate}
\end{mydef}

\begin{alg}
Převod do Chomského normální formy\\
Vstup: Vlastní gramatika $G = (N, \Sigma, P, S)$\\
Výstup: Gramatika $G' = (N', \Sigma, P', S')$ v CNF taková, že $L(G) = L(G')$\\
Metoda:
\begin{enumerate}
	\item Množina pravidel $P'$ obsahuje všechna pravidla $A \to a$ z $P$.
	\item Množina pravidel $P'$ obsahuje všechna pravidla $A \to BC$ z $P$.
	\item Je-li pravidlo $S \to \epsilon$ v $P$, pak $S \to \epsilon$ je také v $P'$.
	\item Pro každé pravidlo tvaru $A \to X_1 \dots X_k$, kde $k > 2$ z $P$ přidej k $P'$ tuto množinu pravidel. Symbolem $X'_i$ značíme nonterminál $X_i$, je-li  $X_i \in N$, nebo nový nonterminál, je-li $X_i \in \Sigma$:
	\begin{align*}
		A								&\to X'_1 \left<X_2 \dots X_k \right> \\
		\left<X_2 \dots X_k \right>		&\to X'_2 \left<X_3 \dots X_k \right> \\
										& \hdots \\
		\left<X_{k-1} \dots X_k \right>	&\to X'_{k-1} X'_k \\
	\end{align*}
	kde každý symbol $\left<X_i \dots X_k \right>$ značí nový nonterminální symbol.
	\item Pro každé pravidlo tvaru $A \to X_1 X_2$, kde nějaký ze symbolů $X_1$ nebo $X_2$ leží v $\Sigma$ přidej k $P'$ pravidlo $A \to X'_1 X'_2$, kde $X'_i$ značíme nonterminál $X_i$, je-li $X_i \in N$, nebo nový nonterminál, je-li $X_i \in \Sigma$.
	\item Pro každý nový nonterminál tvaru $a'$ přidej k $P'$ pravidlo $a' \to a$. Výsledná gramatika je $G' = (N', \Sigma, P', S')$; Množina $N'$ obsahuje všechny nonterminály tvaru $\left<X_i \dots X_k \right>$ a $a'$.
\end{enumerate}
\end{alg}

\begin{veta}
Nechť $L$ je bezkontextový jazyk. Pak existuje bezkontextová gramatika $G$ v CNF taková, že $L = L(G)$
\end{veta}

Počet derivací v CNF pro libovolný řetězec je vždy lichý.

\section{Geribachové normální forma}

\begin{mydef}
Gramatika $G = (N, \Sigma, P, S)$ je v Greibachové normální formě (GNF), je-li $G$ gramatikou bez $\epsilon$-pravidel a jestliže každé pravidlo (s výjimkou případného pravidla $S \to \epsilon$) má tvar $a \to a \alpha$, kde $a \in \Sigma, \alpha \in N^*$.
\end{mydef}

\begin{lemma}
Nechť $G = (N, \Sigma, P, S)$ je gramatika bez levé rekurze. Pak existuje lineární uspořádání $<$ definované na množině nonterminálních symbolů $N$ takové, že $A \to B\alpha$ v P, paj $A < B$.
\end{lemma}

\begin{alg}
Převod do Greibachové normální formy\\
Vstup: Vlastní gramatika bez levé rekurze $G = (N, \Sigma, P, S)$\\
Výstup: Ekvivalentní gramatika $G'$ v GNF\\
Metoda:
\begin{enumerate}
	\item Podle předchozí lemmy vytvoř lineární uspořádání $<$ na $N$ takové, že každé $A$-pravidlo začíná buď terminálem, nebo nějakým nonterminálem $B$ takovým, že $A < B$. Nechť
	$$N = \{A_1, A_2, \dots, A_n\} \text{ a } A_1 < A_2 < \dots < A_n$$
	\item Polož $i = n -1$
	\item Je-li $i = 0$ přejdi k bodu (5), je-li $i \not=0$ nahraď každé pravidlo tvaru $A_i \to A_j \alpha$, kde $j > i$ pravidly $A_i \to \beta_1 \alpha | \dots \beta_m \alpha$, kde $A_j \to \beta_1 | \dots | \beta_m$ jsou všechna $A_j$-pravidla. (každý z řetězců $\beta_1, \dots, \beta_m$ začíná terminálem.)
	\item Polož $i = i - 1$ a opakuj krok (3)
	\item V tomto okamžiku všechna pravidla (s výjimkou pravidla $S \to \epsilon$) začínají terminálním symbolem. V každém pravidle $A \to aX_1 \dots X_k$ nahraď ty symboly $X_j$, které jsou terminálními symboly, novými nonterminálem $X'_j$.
	\item Pro všechna $X'_k$ z bodu (5) přidej pravidla $X'_j \to X_j$
\end{enumerate}
\end{alg}

\begin{veta}
Nechť $L$ je bezkontextový jazyk. Pak existuje gramatika $G$ v GNF taková, že $L(G) = L$.
\end{veta}

Nevýhodou je velké množství nových pravidel nebo nonterminálů.




%%%%%%%%%%%%%%%%%%%%%%%%%%%%%%%%%%%%%%%%%%%%%%%%%%%%%%%%%%%%%%%%%%%%%%%%%%%%%%%%
%%%%%%%%%%%%%%%%%%%%%%%%%%%%%%%%%%%%%%%%%%%%%%%%%%%%%%%%%%%%%%%%%%%%%%%%%%%%%%%%
\chapter{Zásobníkové automaty} \label{cha:21}
1. semestr, TIN, \texttt{opora.pdf}, 4. kapitola, sekce 9. - 12.

(jazyky přijímané ZA, varianty ZA)

\section{Základní definice zásobníkových automatů}
\begin{mydef}
\emph{Zásobníkový automat} $P$ je sedmice
$$P = (Q, \Sigma, \Gamma, \delta, q_0, Z_0, F)$$
kde
\begin{enumerate}
	\item $Q$ je konečná množina stavových symbolů reprezentujících vnitřní stavy řídicí jednotky.
	\item $\Sigma$ je konečná vstupní abeceda; jejími prvky jsou vstupní symboly.
	\item $\Gamma$ je konečná abeceda zásobníkových symbolů
	\item $\delta$ je zobrazení množiny $Q \times (\Sigma \cup \{\epsilon\}) \times \Gamma$ do konečné množiny podmnožin množiny $Q \times \Gamma^*$ popisující funkci přechodů
	\item $q_0 \in Q$ je počáteční stav řídicí jednotky
	\item $Z_0 \in \Gamma$ je symbol, který je na počátku uložen do zásobníku -- tzn. \emph{startovací symbol} zásobníku.
	\item $F \in Q$ je množina koncových stavů.
\end{enumerate}
\end{mydef}

\begin{mydef}
\emph{Konfigurací} zásobníkového automatu $P$ nazveme trojici
$$(q, w, \alpha) \in Q \times \Sigma^* \times \Gamma^*$$
kde
\begin{enumerate}
	\item $q$ je přítomný stav řídicí jednotky
	\item $w$ je doposud nepřečtená část vstupního řetězce; první symbol řetězce $w$ je pod čtecí hlavou. Je-li $w = \epsilon$, pak byly všechny symboly ze vstupní pásky přečteny.
	\item $\alpha$ je obsah zásobníku. Pokud nebude uvedeno jinak, budeme zásobník reprezentovat řetězcem, jehož nejlevější symbol koresponduje s vrcholem zásobníku. Je-li $\alpha = \epsilon$, pak je zásobník prázdny.
\end{enumerate}
\end{mydef}

\begin{mydef}
\emph{Přechod} zásobníkového automatu $P$ budeme reprezentovat binární relací $\vdash_P$ (nebo $\vdash$, bude-li zřejmé, že jde o automat $P$), která je definována na množině konfigurací zásobníkového automatu $P$. Relace
$$(q, w, \beta) \vdash (q', w', \beta')$$
platí pro $q, q' \in Q, w, w' \in \Sigma^*, \beta, \beta' \in \Gamma^*$, jestliže $w = aw'$ pro nějaké $a \in (\Sigma \cup \{\epsilon\}), \beta = Z \alpha$ a $\beta' = \gamma \alpha$ pro nějaké $Z \in \Gamma, a, \gamma \in \Gamma^*$ a $\delta(q, a, Z)$ obsahuje prvek $(q', \gamma)$.
\end{mydef}

Relace $\vdash^i, \vdash^+, \vdash^*$ jsou definovány obvyklém způsobem.

\emph{Počáteční konfigurace} zásobníkového automatu má tvar $(q_0, w, Z_0)$ pro $w \in \Sigma^*$, tj. automat je v počátečním stavu $q_0$, na vstupní pásce je řetězec $w$ a v zásobníku je startovací symbol $Z_0$. \emph{Koncová konfigurace} má tvar $(q, \epsilon, \alpha)$, kde $q \in F$ je koncový stav a $\alpha \in \Gamma^*$.

\begin{mydef}
Platí-li pro řetězec $w \in \Sigma^*$ relace $(q_0, w, Z_0) \vdash^* (q, \epsilon, \alpha)$ pro nějaké $q \in F$ a $\alpha \in \Gamma^*$, pak říkáme, že řetězec $w$ je přijímán zásobníkovým automatem $P$. Množinu $L(P)$ všech řetězců přijímaných zásobníkovým automatem $P$, který nazýváme
\emph{jazykem přijímaným zásobníkovým automatem}.
\end{mydef}

\section{Varianty zásobníkových automatů}

\begin{mydef}
\emph{Rozšířeným zásobníkovým automatem} rozumíme sedmici
$$P = (Q, \Sigma, \Gamma, \delta, q_0, Z_0, F)$$
kde $\delta$ je zobrazení z konečné podmnožiny $Q \times (\Sigma \cup \{\epsilon\}) \times \Gamma^*$ do množiny podmnožin množiny $Q \times \Gamma^*$. Ostatní symboly mají stejný význam jako v základní definici.

Podobně, relace přechodu $\vdash$ je definovaná jako \emph{nejmenší} relace taková, že $(q, aw, \alpha \gamma) \vdash (q', w, \beta \gamma)$ platí jestliže $\delta(q, a, \alpha)$ obsahuje $(q', \beta)$ pro $q, q' \in Q, a \in \Sigma \cup \{\epsilon\}$ a $\alpha, \beta, \gamma \in \Gamma^*$. Tato relace odpovídá přechodu, v němž je vrcholový řetězec odstraněn a nahrazen řetězcem $\beta$ (v základní definici je pouze znak abecedy -- $\alpha \in \Gamma^*$).
\end{mydef}

Jazyk definovaný automatem $P$ je
$$L(P) = \{w | (q_0, w, Z_0) \vdash^* (q, \epsilon, \alpha), q \in F, \alpha \in \Gamma^*\}$$

Rozšířený zásobníkový automat může, na rozdíl od základní definice provádět přechody, i když je zásobník prázdný.

\begin{veta}
Nechť $P = (Q, \Sigma, \Gamma, \delta, q_0, Z_0, F)$ je rozšířený zásobníkový automat. Pak existuje zásobníkový automat $P_1$ takový, že $L(P_1) = L(P)$.
\end{veta}

\begin{mydef}
Nechť $P = (Q, \Sigma, \Gamma, \delta, q_0, Z_0, F)$ je zásobníkový nebo rozšířený zásobníkový automat. Řetězec $w$ je přijímán s \emph{vyprázdněním zásobníku}, jestliže platí $(q_0, w, Z_0) \vdash^* (q, \epsilon, \epsilon), q \in Q$. Označme $L_\epsilon(P)$ množinu všech řetězců, které jsou přijímány zásobníkovým automatem $P$ s vyprázdněním zásobníku.
\end{mydef}

\begin{veta}
Nechť $L$ je jazyk přijímaný zásobníkovým automatem $P = (Q, \Sigma, \Gamma, \delta, q_0, Z_0, F), L = L(P)$. Lze zkonstruovat zásobníkový automat $P'$ takový, že $L_\epsilon(P) = L$.
\end{veta}
Stav vyprazdňující zásobník a znak dna zásobníku, který je odebrán pouze v tomto stavu.

\begin{veta}
Nechť $P = (Q, \Sigma, \Gamma, \delta, q_0, Z_0, \emptyset)$ je zásobníkový automat přijímající vyprázdněním zásobníku. Lze zkonstruovat zásobníkový automat $P'$ takový, že $L(P') = L_\epsilon(P)$.
\end{veta}
Speciální symbol na dně zásobníku. Po jeho odstranění přechází automat do nového koncového stavu.

\section{Ekvivalence bezkontextových jazyků a jazyků přijímaných zásobníkovými automaty}

\begin{veta}
Nechť $G = (N, \Sigma, P, S)$ je bezkontextová gramatika. Z gramatiky $G$ můžeme zkonstruovat zásobníkový automat $R$ takový, že $L_\epsilon(R) = L(G)$.
\end{veta}

Zásobníkový automat $R = (\{q\}, \Sigma, N \cup \Sigma, \delta, S, \emptyset\}$. Pravidla $A \to \alpha$ převedeme na $(q, \alpha \in \delta(q, \epsilon, A)$. Pravidla $\delta(q, a, a) = \{(q, \epsilon)\}$ pro všechna $a \in \Sigma$.

\begin{veta}
Nechť $G = (N, \Sigma, P, S)$ je bezkontextová gramatika. Rozšířený zásobníkový automat $R = (\{q,r\}, \Sigma, N \cup \Sigma \cup \{\$\}, \delta, q, \$, \{r\})$, kde zobrazení $\delta$ je definováno takto:
\begin{enumerate}
	\item $\delta(q, a, \epsilon) = \{(q, a)\}$ pro všechna $a \in \Sigma$.
	\item Je-li $A \to \alpha$ pravidlo v $P$, pak $\delta(q, \emptyset, \alpha)$ obsahuje $(q, A)$.
	\item $\delta(q, \epsilon, S\$) = \{(r, \epsilon)\}$.
\end{enumerate}
přijímá jazyk $L(G)$, tj. $L(R) = L(G)$.
\end{veta}

Syntaktická analýza zdola nahoru (na zásobník ukládáme symboly na vstupu a když tvoří $l$-frázi, redukujeme je na "levé strany" pravidel až ke kořenu. Když je na zásobníku výchozí symbol, přejdeme do koncového stavu.

\begin{veta}
Nechť $R = (Q, \Sigma, \Gamma, \delta, q_0, Z_0, F)$ je zásobníkový automat. Pak existuje bezkontextová gramatika $G$, pro kterou platí $L(G) = L(R)$.
\end{veta}
Levá derivace terminálního řetězce $w$ bude přímo korespondovat s posloupností přechodů $R$. Nonterminální symboly tvaru $[qZr], q,r \in Q, Z \in \Gamma$.

\section{Deterministický zásobníkový automat}

\begin{mydef}
Zásobníkový automat $P = (Q, \Sigma, \Gamma, \delta, q_0, Z_0, F)$ nazýváme deterministickým zásobníkovým automatem, jestliže pro každé $q \in Q$ a $Z \in \Gamma$ platí buď pro každé $a \in \Sigma$ obsahuje $\delta(q, a, Z)$ nanejvýš jeden prvek a $\gamma(q, \epsilon, Z) = \emptyset$, nebo $\delta(q, a, Z) = \emptyset$ pro všechna $a \in \Sigma$ a $\delta(q, \epsilon, Z)$ obsahuje nanejvýš jeden prvek.
\end{mydef}

\begin{mydef}
Jazyk $L$ se nazývá \emph{deterministický bezkontextový jazyk}, jestliže existuje deterministický zásobníkový automat $P$ takový, že $L(P) = L$.
\end{mydef}

\begin{mydef}
Zásobníkový automat $P = (Q, \Sigma, \Gamma, \delta, q_0, z_0, F)$ nazýváme \emph{deterministický} RZA (DRZA), jestliže platí:
\begin{enumerate}
	\item $\forall q \in Q \forall x \in \Sigma \cup \{\epsilon\} \forall \gamma \in \Gamma: |\delta(q, a, \gamma) \leq 1$
	\item Je-li $\delta(q, a, \alpha) \not= \emptyset, \delta(q, a, \beta) \not= \emptyset$ a $\alpha \not='\beta$, pak ani $\alpha$ není předponou $\beta$, ani $\beta$ není předponu $\alpha$.
	\item Je-li $\delta(q, a, \alpha) \not= \emptyset, \delta(q, \epsilon, \beta) \not= \emptyset$, pak ani $\alpha$ není předponou $\beta$, ani $\beta$ není předponu $\alpha$.
\end{enumerate}
\end{mydef}

\begin{veta}
Deterministické rozšířené zásobníkové automaty mají ekvivalentní vyjadřovací sílu jako deterministické zásobníkové automaty.
\end{veta}

\begin{veta}
Deterministické zásobníkové automaty mají striktně menší vyjadřovací sílu než nedeterministické zásobníkové automaty.
\end{veta}







%%%%%%%%%%%%%%%%%%%%%%%%%%%%%%%%%%%%%%%%%%%%%%%%%%%%%%%%%%%%%%%%%%%%%%%%%%%%%%%%
%%%%%%%%%%%%%%%%%%%%%%%%%%%%%%%%%%%%%%%%%%%%%%%%%%%%%%%%%%%%%%%%%%%%%%%%%%%%%%%%
\chapter{Turingovy stroje (jazyky přijímané TS, varianty TS, lineárně omezené automaty, univerzální TS).} \label{cha:22}
1. semestr, TIN, \texttt{opora.pdf}, ?. kapitola

\begin{mydef}
\end{mydef}


\chapter{Nerozhodnutelnost (problém zastavení TS, princip diagonalizace a redukce, Postův korespondenční problém).} \label{cha:23}
\chapter{Parciální rekurzivní funkce.} \label{cha:24}
\chapter{Časová a paměťová složitost (třídy složitosti, úplnost, SAT problém).} \label{cha:25}
\chapter{Ukazatele a zákony paralelního zpracování. Funkce konst. účinnosti a škálovatelnost.} \label{cha:26}
\chapter{Paralelizace programů: vzory programových a datových struktur, podpůrné struktury.} \label{cha:27}
\chapter{Paralelní zpracování v OpenMP, SPMD, smyčky, sekce a tasky. Synchronizační prostředky.} \label{cha:28}
\chapter{Architektury se sdílenou pamětí, UMA i NUMA, zajištění koherence pamětí cache.} \label{cha:29}
\chapter{Architektury distribuovaných systémů se zasílání zpráv.} \label{cha:30}
\chapter{Blokující a neblokující párové (point-to-point) komunikace v MPI.} \label{cha:31}
\chapter{Kolektivní komunikace v MPI, paralelní vstup a výstup.} \label{cha:32}
\chapter{Propojovací sítě: Topologie a směrovací algoritmu, přepínání a řízení toku.} \label{cha:33}
\chapter{Klasifikace metod komprese dat (ztrátové, bezeztrátové, intuitivní, algoritmické, četnost výskytu, pravděpodobnost výskytu), princip kódování délek sledů, kódování „přesuň na začátek.‘‘} \label{cha:34}
\chapter{Kódy s proměnnou délkou - princip, zdůvodnění, Huffmanovy kódy - různé typy, kanonický Huffmanův kód, adaptivní Huffmanův kód, aritmetický kód.} \label{cha:35}
\chapter{Slovníkové metody (LZ77, LZ78, práce se slovníkem, pohyblivé okno, prodlužování položek).} \label{cha:36}
\chapter{Informace a entropie, Shannova věta o kódování.} \label{cha:37}
\chapter{Bezpečnostní kódy: lineární, Hammingovy, cyklické, konvoluční. Detekce a oprava chyb.} \label{cha:38}
\chapter{Základní architektury přepínačů, algoritmy pro plánování, řešení blokování, vícestupňové přepínací sítě.} \label{cha:39}
\chapter{Základní funkce směrovače, zpracování paketů ve směrovači, typy architektur.} \label{cha:40}
\chapter{Sítě Peer-to-Peer (P2P), Milgramův problém malého světa, model sítě P2P, směrování v P2P sítích, strukturované a nestrukturované sítě.} \label{cha:41}
\chapter{Základní principy softwarově definovaných sítí SDN, architektura, technologie OpenFlow.} \label{cha:42}
\chapter{Formální metody v počítačových sítích.} \label{cha:43}
\chapter{Konflikty a závislosti při řetězovém zpracování instrukcí a jejich HW/SW ošetření.} \label{cha:44}
\chapter{Architektura superskalárních procesorů a algoritmy OOO zpracování instrukcí.} \label{cha:45}
\chapter{Procesory VLIW a používané optimalizační techniky s HW podporou.} \label{cha:46}
\chapter{Multivláknové procesory, hrubý, jemný a simultánní MT.} \label{cha:47}
\chapter{Datový paralelismus SIMD a SIMT, HW implementace a SW podpora.} \label{cha:48}
\chapter{Architektura grafických procesorů, odlišnosti od superskalárních procesorů.} \label{cha:49}
\chapter{Programovací jazyk CUDA, model vláken a paměťový model.} \label{cha:50}
\chapter{Základní rysy nízkopříkonových procesorů (požadavky, architektura, výkonnost).} \label{cha:51}
\chapter{Jazyky pro popis obvodů (VHDL, syntetizovatelné konstrukce).} \label{cha:52}
\chapter{Logická syntéza obvodů (návrh pro technologie FPGA a ASIC, fáze syntézy, optimalizace, mapování, techniky zřetězení a vyvážení).} \label{cha:53}
\chapter{Moderní přístupy k syntéze číslicových obvodů (reprezentace obvodu pomocí AIG, techniky odstraňování funkční redundance v AIG, tradiční mapování AIG do LUT).} \label{cha:54}
\chapter{Aplikace omezujících podmínek (časová a fyzická omezení).} \label{cha:55}
\chapter{Verifikace číslicových obvodů (metodologie OVM).} \label{cha:56}
\chapter{Technologie programovatelného hardware (architektura FPGA, struktura konfigurovatelných bloků a vestavěných bloků, propojovací sít, způsoby konfigurace, srovnání s technologií ASIC).} \label{cha:57}
\chapter{Vestavěný počítačový systém (shody a odlišnosti s běžným univerzálním počítačovým systémem).} \label{cha:58}
\chapter{Implementace funkcí vestavěného systému SW a HW prostředky (výhody a nevýhody - dopady SW a HW implementace konkrétní funkce na vlastnosti systému, příklad).} \label{cha:59}
\chapter{Číslicové vstupy a výstupy vestavěných systémů (problémy a jejich řešení, přizpůsobení napěťových úrovní, snímání stavu mechanického kontaktu, ovládání zátěže, posílení výstupu, H-můstek).} \label{cha:60}
\chapter{Architektura SW pro vestavěné systémy (hlavní smyčka, implementace stavového automatu, obsluha přerušení).} \label{cha:61}
\chapter{Konstrukce adaptéru systémové sběrnice: návrh adresového dekodérů, obsluha čtecí a zápisové transakce.} \label{cha:62}
\chapter{Architektura sběrnice PCI-Express a USB: typy transakcí, způsob komunikace a směrování transakcí, detekce chyb a způsob zotavení.} \label{cha:63}

% 2. semestr, IPR, zdroje % XXX


\end{document}

