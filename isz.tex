\documentclass[a4paper, 11pt]{report}
\usepackage[czech]{babel}
\usepackage[utf8]{inputenc}
\usepackage{multirow}
\usepackage{amsmath}
\usepackage{amsfonts}
\usepackage{enumerate}
\usepackage{verbatim}
\usepackage{tikz-qtree}
\usepackage{mathtools}

\usepackage{amsthm}
\newtheorem{mydef}{Definice}[chapter]
\newtheorem{veta}{Věta}
\newtheorem{lemma}{Lemma}

\usepackage{geometry}
\usepackage{layout}

\geometry{
  includeheadfoot,
  hmargin=2.0cm,
  vmargin={0cm, 1.0cm}
}

\usepackage{color}
\usepackage[unicode,colorlinks,hyperindex,plainpages=false,pdftex]{hyperref}

\usepackage{listings}  
\definecolor{mygreen}{rgb}{0,0.6,0}
\lstset{language=VHDL,commentstyle=\color{mygreen},tabsize=4}

\usepackage{fancyhdr}
\pagestyle{fancyplain}
\fancyhf{}
\renewcommand{\headrulewidth}{0pt}

\cfoot{\hfill © Jakuje \hfill \thepage }


\begin{document}

\ref{cha:1}
\ref{cha:2}
\ref{cha:3}
\ref{cha:4}
\ref{cha:5}
\ref{cha:6}
\ref{cha:7}
\ref{cha:8}
\ref{cha:9}
\ref{cha:10}
\ref{cha:11}
\ref{cha:12}
\ref{cha:13}
\ref{cha:14}
\ref{cha:15}
\ref{cha:16}
\ref{cha:17}
\ref{cha:18}
\ref{cha:19}
\ref{cha:20}

\ref{cha:21}
\ref{cha:22}
\ref{cha:23}
\ref{cha:24}
\ref{cha:25}
\ref{cha:26}
\ref{cha:27}
\ref{cha:28}
\ref{cha:29}
\ref{cha:30}
\ref{cha:31}
\ref{cha:32}
\ref{cha:33}
\ref{cha:34}
\ref{cha:35}
\ref{cha:36}
\ref{cha:37}
\ref{cha:38}
\ref{cha:39}
\ref{cha:40}

\ref{cha:41}
\ref{cha:42}
\ref{cha:43}
\ref{cha:44}
\ref{cha:45}
\ref{cha:46}
\ref{cha:47}
\ref{cha:48}
\ref{cha:49}
\ref{cha:50}
\ref{cha:51}
\ref{cha:52}
\ref{cha:53}
\ref{cha:54}
\ref{cha:55}
\ref{cha:56}
\ref{cha:57}
\ref{cha:58}
\ref{cha:59}
\ref{cha:60}

\ref{cha:61}
\ref{cha:62}
\ref{cha:63}
\newpage

\tableofcontents

%%%%%%%%%%%%%%%%%%%%%%%%%%%%%%%%%%%%%%%%%%%%%%%%%%%%%%%%%%%%%%%%%%%%%%%%%%%%%%%%
%%%%%%%%%%%%%%%%%%%%%%%%%%%%%%%%%%%%%%%%%%%%%%%%%%%%%%%%%%%%%%%%%%%%%%%%%%%%%%%%
\chapter{Metodika návrhu HW/SW codesign, platformy, programovatelné obvody.} \label{cha:1}
X. semestr, XXX, ??
\chapter{Výpočetní modely} \label{cha:2}
X. semestr, XXX, ??

(StateCharts, Kahnova síť procesů, synchronní dataflow)
\chapter{Specifikace (chování, struktura), syntéza (alokace, přidělení, plánování) a integrace systémů (rozhraní, synchronizace, komunikace).} \label{cha:3}
X. semestr, XXX, ??
\chapter{Syntéza HW z vyšších programovacích jazyků (reprezentace, alokace, plánování, přiřazení) a jazyk Catapult C.} \label{cha:4}
X. semestr, XXX, ??
\chapter{Odhady (přesnost, věrnost, metriky, metody) a optimalizace vlastností systému (příkon, energie).} \label{cha:5}

%%%%%%%%%%%%%%%%%%%%%%%%%%%%%%%%%%%%%%%%%%%%%%%%%%%%%%%%%%%%%%%%%%%%%%%%%%%%%%%%
%%%%%%%%%%%%%%%%%%%%%%%%%%%%%%%%%%%%%%%%%%%%%%%%%%%%%%%%%%%%%%%%%%%%%%%%%%%%%%%%
\chapter{Jazyk a sémantika predikátové logiky} \label{cha:6}
1. semestr, MAT, \texttt{logikaaktual3.pdf}, 3., 4. kapitola

(termy, formule, realizace jazyka, pravdivost formulí)

\section{Jazyk predikátové logiky}

\begin{itemize}
	\item Logické symboly
	\begin{itemize}
		\item proměnné: $x, y, z, \dots, x_1, x_2, \dots$
		\item logické spojky: $ \lnot, \land, \lor, \to, \leftrightarrow$
		\item kvantifikátory: $\exists, \forall$
		\item závorky, čárka: $(,)$
		\item predikátový symbol rovnosti $=$
	\end{itemize}
	\item Speciální symboly
	\begin{itemize}
		\item funkční symboly $f, g, h, \dots, f_1, f_2, f_3$, nezáporné celé číslo -- četnost
		\item predikátové symboly $p, q, r, \dots, p_1, p_2, \dots$, kladné celé číslo -- jeho četnost.
	\end{itemize}
\end{itemize}

Obsahuje-li jazyk symbol $=$ pro rovnost, mluvíme o \emph{jazyku s rovností}.
Specifiku jazyka určují jeho funkční a predikátové symboly (určující oblast kterou jazyk popisuje)

\subsection{Termy}
\begin{enumerate}[(i)]
	\item Každá proměnná je term
	\item Je-li $f$ funkční symbol s četností $n$ a jsou-li $t_1, \dots, t_n$ termy, pak také $f(t_1, \dots, t_n)$ je term
	\item Každý term vznikne konečným počtem užití (i), (ii)
\end{enumerate}
(ze (ii) plyne, že každá konstanta je term)

\subsection{Atomické formule}
Je-li $p$ predikátový symbol s četností $n$ a jsou-li $t_1, \dots t_n$ termy, pak $p(t_1, \dots, t_n)$ je \emph{atomická formule}.

Speciální, máme-li jazyk s rovností a jsou-li $t_1, t_2$ termy, pak $(t_1 = t_2)$ je atomická formule.
Píšeme $(t_1 = t_2)$ místo $=(t_1, t_2)$. Podobný zápis používáme i pro jiné binární predikátové operátory, např. místo $< (t_1, t_2)$ píšeme $(t_1 < t_2)$.

\subsection{Formule}
\begin{enumerate}[(i)]
	\item Každá atomická formule je formule
	\item Jsou-li $\varphi, \psi$ formule, pak také $(\lnot \varphi), (\varphi \land \psi), (\varphi \lor \psi), (\varphi \to \psi), (\varphi \leftrightarrow \psi)$ jsou formule.
	\item Je-li $x$ proměnná a $\varphi$ formule, pak také $(\forall x \varphi)$, $(\exists x \varphi)$ jsou formule.
	\item Každá formule vznikne konečným počtem užití (i), (ii), (iii)
\end{enumerate}

\tikzset{every tree node/.style={align=center,anchor=north}}
\Tree[.{Formule}
	[.{Atomická formule}
		{Predikátový symbol\\ + Term\\ $p(t_1, \dots, t_n)$}
		[.{Term}
			{Proměnná\\ $x$}
			{Funkční symbol\\ + Term\\ $f(t_1, \dots, t_n)$}
			]
		]
	{Logické spojky\\ + Formule\\ $\varphi \land \psi$}
	{Kvantifikátory\\ + Proměnné\\ + Formule\\ $\forall x \varphi$}
]

\begin{description}
	\item[Vázaný výskyt proměnné] nachází-li se v nějaké podformuli tvaru $\forall x \varphi$ nebo $\exists x \varphi$.
	\item[Obor kvantifikátoru] $\varphi$
	\item[Volný výskyt proměnné] není vázaný
	\item[Volná (Vázaná) proměnná] existuje-li volný (vázaný) výskyt proměnné v této formuli
	\item[Uzavřená formule (Výrok)] Formule neobsahující žádnou volnou proměnnou
	\item[Otevřená formule (Výrok)] Formule neobsahující žádnou vázanou proměnnou
	\item[Formule s čistými proměnnými] Otevřené a uzavřené formule
\end{description}

\section{Sémantika predikátové logiky}

\begin{mydef}
Nechť $L$ je jazyk 1. řádu. \emph{Realizací jazyka} $L$ rozumíme algebraickou strukturu $\mathcal{M}$, která se skládá z
\begin{enumerate}[(i)]
	\item neprázdné množiny $M$, kterou nazveme \emph{univerzum}
	\item pro každý funkční symbol $f$ četností $n$ je dáno zobrazení $f_\mathcal{M} : M^n \to M$
	\item pro každý predikátový symbol $p$ četnosti $n$, kromě rovnosti, je dána relace $p_\mathcal{M} \subset M^n$
\end{enumerate}
Poznamenejme, že pro nulární funkční symbol, tj. pro konstantu, je $M^0 = \{0\}$ a příslušné zobrazení $M^0 \to M$ lze chápat jako vyznačení určitého prvku z $M$ odpovídajícího daného konstantě.
\end{mydef}

\paragraph{Ohodnocení proměnných}: Libovolné zobrazení $e$ množiny všech proměnných do univerza $M$ dané realizace $\mathcal{M}$ jazyka $L$. Pokud proměnné $x$ přiřazuje prvek $m$, budeme značit $e(x/m)$.

\begin{mydef}
\emph{Hodnota termu} $t$ v realizaci $\mathcal{M}$ jazyka $L$ při daném ohodnocení $e$ proměnných, označovaná $t[e]$, se definuje indukcí následovně:
\begin{enumerate}[(i)]
	\item Je-li $t$ proměnná $x$, potom $t[e]$ je $e(x)$
	\item je-li $t$ term tvaru $f(t_1, \dots, t_n)$, kde $f$ je funkční symbol četnosti $n$ a $t_1, \dots, t_n$ jsou termy, potom $t[e]$ je $f_\mathcal{M}(t_1[e], \dots, t_n[e])$
\end{enumerate}

\end{mydef}
\begin{mydef}
Nechť $\mathcal{M}$ je realizace jazyka $L$, nechť $e$ je ohodnocení proměnných a nechť $\varphi$ je formule jazyka $L$. Indukcí podle složitosti formule $\varphi$ definujeme, co znamená, že \emph{formule $\varphi$ je pravdivá v $\mathcal{M}$ při ohodnocení $e$}. Tuto skutečnost budeme značit $\mathcal{M} \models \varphi[e]$
\begin{enumerate}[(i)]
	\item Je-li $\varphi$ atomická formule tvaru $p(t_1, \dots, t_n)$, kde $p$ je predikátový symbol četnosti $n$ a $t_1, \dots, t_n$ jsou termy, pak $\mathcal{M} \models \varphi[e]$ právě když $(t_1[e], \dots, t_n[e]) \in p_\mathcal{M}$.
	\item Je-li $\varphi$ atomická formule tvaru $t_1 = t_2$, kde $t_1, t_2$ jsou termy, pak $\mathcal{M} \models \varphi[e]$ právě když $t_1[e]$ je tentýž prvek jako $t_2[e]$ v $M$
	\item Je-li $\varphi$ tvaru $\lnot \psi$, kde je $\psi$ je formule jazyka $L$, pak $\mathcal{M} \models \varphi[e]$ je právě když $\mathcal{M} \not\models \psi[e]$.
	\item Je-li $\varphi$ některého z tvarů $(\eta \land \psi), (\eta \lor \psi), (\eta \to \psi), (\eta \leftrightarrow \psi)$, kde $\eta$, $\psi$ jsou formule, klademe: \\
	$\mathcal{M} \models (\eta \land \psi)[e]$ právě když současně $\mathcal{M} \models \eta[e]$ a $\mathcal{M} \models \psi[e]$. \\
	$\mathcal{M} \models (\eta \lor \psi)[e]$ právě platí alespoň jedno z $\mathcal{M} \models \eta[e]$ a $\mathcal{M} \models \psi[e]$ a podobně další logické spojky.
	\item Je-li $\varphi$ tvaru $(\forall x \psi)$, kde $\psi$ je formule jazyka $L$, pak $\mathcal{M} \models \varphi[e]$ právě když pro každý prvek $m \in M$ je $\mathcal{M} \models \psi[e(x/m)]$.
	\item Je-li $\varphi$ tvaru $(\exists x \psi)$, kde $\psi$ je formule jazyka $L$, pak $\mathcal{M} \models \varphi[e]$ právě když existuje $m \in M$ taková, že $\mathcal{M} \models \psi[e(x/m)]$.
\end{enumerate}
\end{mydef}

\begin{comment}
\section{Ostatní}

Formalizovaná axiomatická teorie je dána
\begin{itemize}
	\item symboly -- tvoří abecedu
	\item formulemi -- určitá slova této abecedy, která tvoří jazyk této teorie
	\item axiomy -- výchozí tvrzení této teorie zapsaná pomocí abecedy jako jisté formule
	\item odvozovací pravidla --  pravidla pro manipulaci s formulemi, pomocí kterých odvozujeme z axiomů důsledky.
\end{itemize}

\begin{description}
	\item[Proměnné] Označení libovolného prvku z daného oboru $(x, y, z, \dots, x_1, x_2, \dots)$
	\item[Konstanty] Význačné objekty $(0, 1, \dots)$
	\item[Funkční symboly] Operace $(f, g, h, \dots, f_1, f_2, f_3)$
	\item[Četnost funkčního symbolu] počet argumentů dané operace
	\item[Predikáty] Vlastnosti a vztahy mezi objekty
	\item[Predikátové symboly] Vyjádření predikátů? $(p, q, r, \dots, p_1, p_2, \dots)$.
	\item[Četnost predikátového symbolu] počet argumentů predikátu
	\item[Atomické formule] Nejjednodušší tvrzení, složená z \emph{Proměnných}, \emph{konstant}, \emph{funkčních symbolů}, \emph{predikátových symbolů}.
	\item[Složitější formule] Atomické formule + logické spojky + kvantifikace proměnných ($\forall, \exists$)
	\item[Abeceda predikátové logiky 1. řádu] výše uvedené symboly s logickými spojkami a pomocnými symboly (závorky, čárka).
\end{description}


\section{Predikátová logika 1. řádu}

Matematické teorie pracují s celými soubory objektů (čísla, body v prostoru, prvky algebraických struktur).
Pro označení lib. prvků z daného oboru používáme \emph{proměnné} $(x, y, z, \dots, x_1, x_2, \dots)$

Mezi prvky z daného oboru mohou být některé význačné objekty (0, neutrální prvek grupy, \dots), pro než užíváme zvláštní symboly -- \emph{konstanty} (např. 0, 1, \dots).

S objekty daného oboru lze provádět různé operace (sčítání a násobení čísel, násobení v grupách, \dots).
K označení operace užíváme \emph{funkční symboly} $(f, g, h, \dots, f_1, f_2, f_3)$.
Ke každému funkčnímu symbolu je přiřazeno přirozené číslo, které vyjadřuje jeho \emph{četnost}, tj. počet argumentů dané operace.
Je-li četnost symbolu rovna $n$, říkáme, že symbol je $n$-ární.
Je přirozené chápat konstanty jako nulární funkční symboly.

Matematika zkoumá vlastnosti objektů a vztahy mezi objekty.
Vlastnosti  a vztahy mezi objekty daného oboru, tzv. \emph{predikáty} (\uv{být záporným číslem} (vlastnost), \uv{být menší než}, \uv{být prvkem} (vztahy)) vyjadřujeme pomocí \emph{predikátových symbolů} $(p, q, r, \dots, p_1, p_2, \dots)$.
Predikát znamená vztah mezi užitým počtem objektů.
Tím je každému predikátovému symbolu přiřazeno přirozené číslo, jeho četnost udávající počet jeho argumentů.
Je-li četnost rovna $n$, říkáme, že symbol je \emph{n-ární}.
V mnoha případech používáme zvláštní označení = pro binární predikátový symbol označující rovnost, tj. totožnost objektů z daného oboru.

Z proměnných, konstant, funkčních symbolů a predikátových symbolů sestavujeme jistým způsobem nejjednodušší tvrzení, vyjádřená tzv. \emph{atomickými formulemi}.
Z nich vytváříme složitější formule pomocí \emph{logických spojek} (stejných jako ve výrokové logice) a pomocí \emph{kvantifikace proměnných}.\\
\emph{Univerzální (obecný) kvantifikátor} $\forall$ vyjadřuje platnost pro všechny objekty z daného oboru.\\
\emph{Existenční kvantifikátor} $\exists$ vyjadřuje existenci požadovaného objektu v daném oboru.

Uvedené symboly spolu s logickými spojkami a \emph{pomocnými symboly} (závorka, čárka) tvoří abecedu jazyka \emph{predikátové logiky 1. řádu}.
Proměnné jazyka prvního řádu jsou obecná jména pro objekty daného oboru, tj. pro individua (např. čísla).
Jazyk neobsahuje proměnné pro množiny individuí (např. množiny čísel, relací, \dots), vyšších řádů, které dovolují kvantifikovat např. množiny, relace.

%%%%%%%%%%%%%%%%%%%%%%%%%%%%%%%%%%
\section{Jazyk predikátové logiky}

\begin{itemize}
	\item Logické symboly
	\begin{itemize}
		\item proměnné: $x, y, z, \dots, x_1, x_2, \dots$
		\item logické spojky: $ \lnot, \land, \lor, \to, \leftrightarrow$
		\item kvantifikátory: $\exists, \forall$
		\item závorky, čárka: $(,)$
		\item predikátový symbol rovnosti $=$
	\end{itemize}
	\item Speciální symboly
	\begin{itemize}
		\item funkční symboly $f, g, h, \dots, f_1, f_2, f_3$, nezáporné celé číslo -- četnost
		\item predikátové symboly $p, q, r, \dots, p_1, p_2, \dots$, kladné celé číslo -- jeho četnost.
	\end{itemize}
\end{itemize}

Obsahuje-li jazyk symbol $=$ pro rovnost, mluvíme o \emph{jazyku s rovností}.
Specifiku jazyka určují jeho funkční a predikátové symboly (určující oblast kterou jazyk popisuje)

\subsection{Termy}
\begin{enumerate}[(i)]
	\item Každá proměnná je term
	\item Je-li $f$ funkční symbol s četností $n$ a jsou-li $t_1, \dots, t_n$ termy, pak také $f(t_1, \dots, t_n)$ je term
	\item Každý term vznikne konečným počtem užití (i), (ii)
\end{enumerate}
(ze (ii) plyne, že každá konstanta je term)

\subsection{Atomické formule}
Je-li $p$ predikátový symbol s četností $n$ a jsou-li $t_1, \dots t_n$ termy, pak $p(t_1, \dots, t_n)$ je \emph{atomická formule}.

Speciální, máme-li jazyk s rovností a jsou-li $t_1, t_2$ termy, pak $(t_1 = t_2)$ je atomická formule.
Píšeme $(t_1 = t_2)$ místo $=(t_1, t_2)$. Podobný zápis používáme i pro jiné binární predikátové operátory, např. místo $< (t_1, t_2)$ píšeme $(t_1 < t_2)$.

\subsection{Formule}
\begin{enumerate}[(i)]
	\item Každá atomická formule je formule
	\item Jsou-li $\varphi, \psi$ formule, pak také $(\lnot \varphi), (\varphi \land \psi), (\varphi \lor \psi), (\varphi \to \psi), (\varphi \leftrightarrow \psi)$ jsou formule.
	\item Je-li $x$ proměnná a $\varphi$ formule, pak také $(\forall x \varphi)$, $(\exists x \varphi)$ jsou formule.
	\item Každá formule vznikne konečným počtem užití (i), (ii), (iii)
\end{enumerate}

Poznamenejme, že píšeme $x \not= y$ místo $\lnot (x, y)$ a také, pokud to nemůže narušit srozumitelnost, vynecháváme některé dvojice závorek.

Při tvorbě formule $\varphi$ podle předchozí definice vytváříme určitou posloupnost formulí, která začíná atomickými formulemi a končí formulí $\varphi$ a každá formule v této posloupnosti vzniká z některých předcházejících pomocí logických spojek a kvantifikátorů. Každá z těchto formulí se nazývá podformule $\varphi$.

Každá formule je konečnou posloupností symbolů. Každý symbol, zejména každá proměnná, se může ve formuli vyskytovat na jednom nebo více místech.
Řekněme, že daný \emph{výskyt} proměnné $x$ ve formuli $\varphi$ je \emph{vázaný}, nenachází-li se v nějaké podformuli tvaru $\forall x \psi$ nebo $\exists x \psi$. V tomto případě se proměnná $x$ vyskytuje v kvantifikátoru samém nebo ve formuli $\psi$ (podformule $\psi$ se nazývá \emph{obor kvantifikátoru} $\forall x$ nebo $\exists x$.
V opačném případě (výskyt není vázaný) řekneme, že daný výskyt proměnné $x$ ve formuli $\phi$ je \emph{volný}.
Proměnná $x$ se nazývá \emph{volnou (vázanou) proměnnou} ve formuli $\phi$, existuje-li její volný (vázáný) výskyt v této formuli.
Proměnná tedy může být ve formuli volná i vázaná. Formule neobsahující žádnou volnou proměnnou se nazývá \emph{uzavřená formule} nebo též \emph{výrok}.
Naopak, formule neobsahující žádnou vázanou proměnnou se nazývá \emph{otevřenou formulí}. Uzavření a otevřené formule nazýváme \emph{formulemi s čistými proměnnými}.

%%%%%%%%%%%%%%%%%%%%%%%%%%%%%%%%%%%%%%
\section{Sémantika predikátové logiky}

Chceme dát interpretaci symbolům jazyka predikátové logiky 1. řádu.
Nejprve vymezíme obor, který budeme určovat možné hodnoty proměnných, bude to určitý soubor $M$ uvažovaných objektů.
Funkčním symbolům budou odpovídat operace na $M$ příslušných četností.
Predikátovým symbolům budou odpovídat vztahy mezi objekty z $M$, které lze popsat jako relace na $M$ příslušných četností. Máme-li jazyk s rovností, interpretujeme symbol $=$ jako rovnost objektů z $M$.

\begin{mydef}
Nechť $L$ je jazyk 1. řádu. \emph{Realizací jazyka} $L$ rozumíme algebraickou strukturu $\mathcal{M}$, která se skládá z
\begin{enumerate}[(i)]
	\item neprázdné množiny $M$, kterou nazveme \emph{univerzum}
	\item pro každý funkční symbol $f$ četností $n$ je dáno zobrazení $f_\mathcal{M} : M^n \to M$
	\item pro každý predikátový symbol $p$ četnosti $n$, kromě rovnosti, je dána relace $p_\mathcal{M} \subset M^n$
\end{enumerate}
\end{mydef}

Poznamenejme, že pro nulární funkční symbol, tj. pro konstantu, je $M^0 = \{0\}$ a příslušné zobrazení $M^0 \to M$ lze chápat jako vyznačení určitého prvku z $M$ odpovídajícího daného konstantě.

Chceme-li zkoumat pravdivost formulí jazyka $L$ v nějaké jeho realizaci $\mathcal{M}$ musíme volným proměnným přiřadit hodnoty, jimiž budou nějaké prvky množiny $M$.

\begin{mydef}
Libovolné zobrazení $e$ množiny všech proměnných do univerza $M$ dané realizace $\mathcal{M}$ jazyka $L$ budeme nazývat \emph{ohodnocení proměnných}.

Je-li $x$ proměnná a $e$ ohodnocení proměnných a $m \in M$, potom ohodnocení proměnných, které proměnné $x$ přiřazuje prvek $m$ a pro všechny ostatní proměnné splývá s ohodnocením $e$, budeme značit $e(x/m)$.
\end{mydef}

\begin{mydef}
\emph{Hodnota termu} $t$ v realizaci $\mathcal{M}$ jazyka $L$ při daném ohodnocení $e$ proměnných, označovaná $t[e]$, se definuje indukcí následovně:
\begin{enumerate}[(i)]
	\item Je-li $t$ proměnná $x$, potom $t[e]$ je $e(x)$
	\item je-li $t$ term tvaru $f(t_1, \dots, t_n)$, kde $f$ je funkční symbol četnosti $n$ a $t_1, \dots, t_n$ jsou termy, potom $t[e]$ je $f_\mathcal{M}(t_1[e], \dots, t_n[e])$
\end{enumerate}
\end{mydef}

\begin{mydef}
Nechť $\mathcal{M}$ je realizace jazyka $L$, nechť $e$ je ohodnocení proměnných a nechť $\varphi$ je formule jazyka $L$. Indukcí podle složitosti formule $\varphi$ definujeme, co znamená, že \emph{formule $\varphi$ je pravdivá v $\mathcal{M}$ při ohodnocení $e$}. Tuto skutečnost budeme značit $\mathcal{M} \models \varphi[e]$
\begin{enumerate}[(i)]
	\item Je-li $\varphi$ atomická formule tvaru $p(t_1, \dots, t_n)$, kde $p$ je predikátový symbol četnosti $n$ a $t_1, \dots, t_n$ jsou termy, pak $\mathcal{M} \models \varphi[e]$ právě když $(t_1[e], \dots, t_n[e]) \in p_\mathcal{M}$.
	\item Je-li $\varphi$ atomická formule tvaru $t_1 = t_2$, kde $t_1, t_2$ jsou termy, pak $\mathcal{M} \models \varphi[e]$ právě když $t_1[e]$ je tentýž prvek jako $t_2[e]$ v $M$
	\item Je-li $\varphi$ tvaru $\lnot \psi$, kde je $\psi$ je formule jazyka $L$, pak $\mathcal{M} \models \varphi[e]$ je právě když $\mathcal{M} \not\models \psi[e]$.
	\item Je-li $\varphi$ některého z tvarů $(\eta \land \psi), (\eta \lor \psi), (\eta \to \psi), (\eta \leftrightarrow \psi)$, kde $\eta$, $\psi$ jsou formule, klademe: \\
	$\mathcal{M} \models (\eta \land \psi)[e]$ právě když současně $\mathcal{M} \models \eta[e]$ a $\mathcal{M} \models \psi[e]$. \\
	$\mathcal{M} \models (\eta \lor \psi)[e]$ právě platí alespoň jedno z $\mathcal{M} \models \eta[e]$ a $\mathcal{M} \models \psi[e]$ a podobně další logické spojky.
	\item Je-li $\varphi$ tvaru $(\forall x \psi)$, kde $\psi$ je formule jazyka $L$, pak $\mathcal{M} \models \varphi[e]$ právě když pro každý prvek $m \in M$ je $\mathcal{M} \models \psi[e(x/m)]$.
	\item Je-li $\varphi$ tvaru $(\exists x \psi)$, kde $\psi$ je formule jazyka $L$, pak $\mathcal{M} \models \varphi[e]$ právě když existuje $m \in M$ taková, že $\mathcal{M} \models \psi[e(x/m)]$.
\end{enumerate}
\end{mydef}

\begin{mydef}
Řekneme, že formule $\varphi$ jazyka $L$ je \emph{logicky platná}, jestliže pro každou realizaci $\mathcal{M}$ jazyka $L$ je $\mathcal{M} \models \varphi$, píšeme $\models \varphi$.
\end{mydef}
\end{comment}
















%%%%%%%%%%%%%%%%%%%%%%%%%%%%%%%%%%%%%%%%%%%%%%%%%%%%%%%%%%%%%%%%%%%%%%%%%%%%%%%%
%%%%%%%%%%%%%%%%%%%%%%%%%%%%%%%%%%%%%%%%%%%%%%%%%%%%%%%%%%%%%%%%%%%%%%%%%%%%%%%%
\chapter{Formální systém predikátové logiky} \label{cha:7}

1. semestr, MAT, \texttt{logikaaktual3.pdf}, 5., 6., 7., 8. kapitola

(axiomy a odvozovací pravidla, dokazatelnost, model a důsledek teorie, věty o úplnosti a kompaktnosti, prenexní tvar formulí)

\section{Axiomy}

%Budujeme predikátovou logiku jako formální axiomatický systém. Jazyk $L$ predikátové logiky přebíráme z předchozího s tím, že z logických spojek bereme jako základní $\lnot$ a $\to$ (ostatní mohou být definovány jako ve výrokovém počtu). Z kvantifikátorů bereme jako základní $\forall$, kvantifikátor $\exists$ je možno zavést takto: Je-li $\varphi$ formule, pak $\exists x \varphi$ je zkratka pro $\lnot(\forall x (\lnot \varphi))$. Omezíme se tedy pouze na ty formule, které jsou vytvořeny z atomických formulí jen pomocí spojek $\lnot$, $\to$ a kvantifikátoru $\forall$.. Axiomy predikátové logiky lze rozdělit do čtyř skupin.

\subsection{Schémata výrokových axiomů}

Jsou-li $\varphi, \psi, \eta$ formule jazyka $L$, pak
\begin{eqnarray*}
&\varphi \to (\psi \to \varphi) & \\
&(\varphi \to (\psi \to \eta)) \to ((\varphi \to \psi) \to (\varphi \to \eta)) & \\
&((\lnot \psi) \to (\not \varphi)) \to (\varphi \to \psi) & 
\end{eqnarray*}
jsou axiomy predikátové logiky.

\subsection{Schéma axiomu kvantifikátoru}
Jsou-li $\varphi, \psi$ formule a je-li $x$ proměnná, která nemá volný výskyt ve formuli $\varphi$, pak
$$ (\forall x (\varphi \to \psi)) \to (\varphi \to (\forall x \psi)) $$
je axiom predikátové logiky.

\subsection{Schéma axiomu substituce}
Je-li $\varphi$ formule, $x$ proměnná a $t$ term substituovatelný za $x$ do $\varphi$, pak
$$ (\forall x \varphi) \to \varphi_x[t] $$
je axiom predikátové logiky.

Jestliže $t = x$, pak schéma axiomu substituce má tvar
$$ (\forall x \varphi) \to \varphi $$


\subsection{Schémata axiomů rovnosti}
Je-li $x$ proměnná, pak $x = x$ je axiom. Jsou-li $x_1, \dots, x_n, y_1, \dots, y_n$ proměnné a je-li $f$ funkční symbol s četností $n$, pak
$$(x_1 = y_1 \to (x_2 = y_2 \to ( \dots (x_n = y_n \to f(x_1, \dots, x_n) = f(y_1, \dots, y_n)) \dots ))) $$
je axiom. Jsou-li $x_1, \dots, x_n, y_1, \dots, y_n$ proměnné, je-li $p$ predikátový symbol s četností $n$, pak
$$(x_1 = y_1 \to (x_2 = y_2 \to ( \dots (x_n = y_n \to p(x_1, \dots, x_n) = p(y_1, \dots, y_n)) \dots ))) $$
je axiom.

\section{Odvozovací pravidla predikátové logiky}

\subsection{Pravidlo odloučení (modus ponens}
Z formulí $\varphi, \varphi \to \psi$ se odvodí formule $\psi$.

\subsection{Pravidlo zobecnění (generalizace)}
Pro libovolnou proměnnou $x$ se z formule $\varphi$ odvodí formule $\forall x \varphi$.

%Spolu se schématy výrokových axiomů a pravidle odloučení přechází do predikátové logiky celá výroková logika.

\section{Dokazatelnost?}
\begin{veta}
(O korektnosti) Libovolná formule jazyka $L$ dokazatelná v predikátové logice 1. řádu je logicky platnou formulí, tj. je splněna v každé realizaci jazyka L.
\end{veta}

\begin{lemma}
(Pravidlo $\forall$) Je-li $\vdash \varphi \to \psi$ a proměnná $x$ nemá volný výskyt ve $\varphi$, pak $\vdash \varphi \to (\forall x \psi)$.
\end{lemma}

\begin{lemma}
(Pravidlo $\exists$) Je-li $\vdash \varphi \to \psi$ a proměnná $x$ nemá volný výskyt ve $\psi$, pak $\vdash (\exists x \varphi) \to \psi$.
\end{lemma}

\begin{lemma}
Je-li $\varphi$ formule, $x$ proměnná, $t$ term substituovatelný za $x$ do $\varphi$, pak $\vdash \varphi_x[t] \to (\exists x \varphi)$
\end{lemma}

\begin{lemma}
Nechť $\varphi'$ je instancí formule $\varphi$, tj. nechť $\varphi'$ je tvaru $\varphi_{x_1, \dots, x_n}[t_1, \dots t_n]$ pro nějaké termy $t_1, \dots, t_n$ substituovatelné za $x_1, \dots, x_n$ do $\varphi$. Jestliže $\vdash \varphi$, pak $\vdash \varphi'$.
\end{lemma}

\subsection{Uzávěr formule}
\begin{mydef}
Jsou-li $x_1, \dots, x_n$ všechny volné proměnné ve formuli $\varphi$ v nějakém pořadí, pak formuli $(\forall x_1 \dots \forall x_n \varphi$ nazveme uzávěrem formule $\varphi$.
\end{mydef}

\begin{veta}
(O uzávěru) Je-li $T$ množina formulí a $\varphi'$ uzávěr formule $\varphi$, pak $T \vdash \varphi$ právě když $T \vdash \varphi'$.
\end{veta}

\begin{lemma}
(Distribuce kvantifikátorů) Je-li $\vdash \varphi \to \psi$, potom $\vdash (\forall x \varphi ) \to (\forall x \psi), \vdash (\exists x \varphi) \to (\exists x \psi)$.
\end{lemma}

\begin{veta}
(O dedukci) Nechť $T$ je množina formulí jazyka L, nechť $\varphi$ je uzávřená formule, $\psi$ je libovolná formule jazyka $L$. Potom $T \vdash \varphi \to \psi$, právě když $T, \varphi \vdash \psi$.
\end{veta}

\begin{veta}
(O konstantách) Nechť $T$ je množina formulí jazyka $L$, nechť $\varphi$ je formule. Nechť $x_1, \dots, x_n$ jsou proměnné a nechť $c_1, \dots, c_n$ jsou nové konstanty, jejichž přidáním k $L$ vznikne jazyk $L'$. Potom $T \vdash \varphi_{x_1, \dots x_n}[c_1, \dots, c_n]$, právě když $T \vdash \varphi$.
\end{veta}

\begin{lemma}
Je-li $L$ jazyk s rovností, pak

$$ \vdash x = y \to y = x $$
$$ \vdash x = y \to (y = z \to x = z) $$
\end{lemma}

\begin{lemma}
Je-li $f$ funkční symbol četnosti $n$, je-li $p$ predikátová symbol četnosti $m$ a jsou-li $u$, $v$, $w$, $s_1, \dots, s_n$, $t_1, \dots, t_n$ termy jazyka $L$, pak
\begin{enumerate}[(i)]
	\item $\vdash u = u $
	\item $\vdash u = v \to v = u $
	\item $\vdash u = v \to (v = w \to u = w) $
	\item $\vdash s_1 = t_1 \to (s_2 = t_2 \to \dots (s_n = t_n \to f(s_1, \dots, s_n) = f(t_1, \dots, t_n)) \dots ) $
	\item $\vdash s_1 = t_1 \to (s_2 = t_2 \to \dots (s_n = t_n \to p(s_1, \dots, s_n) = p(t_1, \dots, t_n)) \dots ) $
\end{enumerate}
\end{lemma}

\section{Prenexní tvar formulí}
Základní tvar formulí.

\begin{comment}
\begin{lemma}
Buď $i_1, \dots, i_n$ libovolná permutace čísel $\{1, \dots, n\}$. Nechť $x_1, \dots, x_n$ jsou proměnné a $A$ formule predikátové logiky. Pak platí:
\begin{enumerate}
	\item $\vdash (\forall x_1) \dots (\forall x_n) A \leftrightarrow (\forall x_{i_1}) \dots (\forall x_{i_n}) A $
	\item $\vdash (\exists x_1) \dots (\exists x_n) A \leftrightarrow (\exists x_{i_1}) \dots (\exists x_{i_n}) A $
\end{enumerate}
\end{lemma}

\begin{veta}
Buď $A$ formule taková, že proměnné $x_1, \dots x_n$ jsou jediné proměnné s volným výskytem v $A$. Pak $\vdash A$, právě když $\vdash \forall x_1 \dots \forall x_n A$
\end{veta}

\begin{veta}
(O ekvivalenci) Nechť formule $A'$ vznikne s formule $A$ nahrazením některých výskytů podformulí $B_1 \dots B_n$ pro řadě formulemi $B'_1, \dots, B'_n$. Je-li $\vdash B_i \leftrightarrow B'_i$ pro všechna $i = 1, \dots, n$, pak platí $\vdash A \leftrightarrow A'$.
\end{veta}

\begin{veta}
Buďte $A, B$ formule a $x$ proměnná. Pak

$$ \vdash (\exists x) \lnot A \leftrightarrow \lnot (\forall x) A $$
$$ \vdash (\forall x) \lnot A \leftrightarrow \lnot (\exists x) A $$

Jestliže $x$ není volná ve formuli $A$ a $\circ$ značí některou z výrokových spojek $\land, \lor, \to$, pak platí

$$ \vdash \forall x (A \circ B) \leftrightarrow (A \circ \forall x B) $$
$$ \vdash \exists x (A \circ B) \leftrightarrow (A \circ \exists x B) $$

pro opačnou implikaci $B \to A$ platí:

$$ \vdash \forall x (B \to A) \leftrightarrow (\exists x B \to A)$$
$$ \vdash \exists x (B \to A) \leftrightarrow (\forall x B \to A)$$
\end{veta}
\end{comment}

\begin{mydef}
Nechť $A$ je formule predikátové logiky. Formule $A'$ je \emph{variantou} formule $A$, jestliže vznikne z $A$ postupným nahrazením podformulí tvaru $(Qx) B$ podformulemi $(Q y) B_x[y]$, kde $Q$ je obecný nebo existenční kvantifikátor a $y$ je proměnná, která není volná v $B$.
\end{mydef}

Důsledek: Je-li $A'$ variantou formule $A$, pak je dokazatelné, že obě formule jsou ekvivalentní: $\vdash A \leftrightarrow A'$

\begin{mydef}
Formule $A$ je v \emph{prenexním tvaru}, jestliže má tvar $Q_1 x_1 \dots Q_n x_n B$, kde
\begin{enumerate}[(i)]
	\item $n \geq 0$ a pro každé $i = 1, \dots n$ je $Q_i$ buď $\forall$ nebo $\exists$,
	\item $x_i, \dots, x_n$  jsou navzájem různé proměnné,
	\item $B$ je otevřená formule (neobsahuje kvantifikátory).
\end{enumerate}
\end{mydef}

\begin{veta}
Ke každé formuli $A$ lze sestrojit formuli $A'$ v prenexním tvaru tak, že $\vdash A \leftrightarrow A'$.
\end{veta}

\subsection{Převedení formule na prenexní tvar}
\begin{description}
	\item[Vyloučení zbytečných kvantifikátorů] vynecháme všechny kvantifikátory $\forall x$, resp. $\exists x$ v podformulích tvaru $\forall x B$ nebo $\exists x B$, pokud se proměnná $x$ nevyskytuje volně v $B$.

	\item[Přejmenování proměnných] Vyhledáme podformuli $Q x A$ nejvíce vlevo takovou, že proměnná $x$ se vyskytuje volně v $A$. Pokud $x$ má ještě další výskyt ve výchozí formuli, nahradíme podformuli $Q x A$ její variantou $Q x' A'$, kde $x'$ je proměnná různá od všech proměnných vyskytujících se v převáděné formuli. Tento proces opakujeme do té doby, až všechny kvantifikátory mají různé proměnné a žádná proměnná není v získané formule současně volná i vázaná (formule s čistými proměnnými).

	\item[Eliminace spojky $\leftrightarrow$] provede se podle následujícího schématu:
	$$ A \leftrightarrow B \dots (A \to B) \land (B \to A)$$

	\item[Přesun negace dovnitř] - provádíme postupně náhrady podformulí podle schémat
	\begin{eqnarray*}
	& \lnot (\forall x A)	\dots \exists x \lnot A & \\
	& \lnot (\exists x A)	\dots \forall x \lnot A & \\
	& \lnot (A \to B)		\dots A \land \lnot B & \\
	& \lnot (A \lor B)	\dots \lnot A \land \lnot B & \\
	& \lnot (A \land B)	\dots \lnot A \lor \lnot B & \\
	& \lnot (\lnot A )	\dots A &
	\end{eqnarray*}

	\item[Přesun kvantifikátoru doleva] pro podformuli $B$, ve které se nevyskytuje proměnná $x$, provádíme náhrady podle schémat
	\begin{eqnarray*}
	& (QxA) \lor B	\dots Qx(A \lor B) &\\
	& (QxA) \land B	\dots Qx(A \land B) &\\
	& (QxA) \to B		\dots \bar{Q}x(A \to B) &\\
	& B \to (QxA)		\dots Qx(B \to A) &\\
	& (\exists x A) \lor (\exists y B) 	\dots \exists x (A \lor B_y[x]) &\\
	& (\forall x A) \land (\forall y B) 	\dots \forall x (A \land B_y[x]) &
	\end{eqnarray*}
	kde $\bar{Q}$ je kvantifikátor "opačný" ke $Q$. N+kde lze snížit počet kvantifikátorů pomocí schémat
\end{description}

\section{Věta o úplnosti}

\begin{mydef}
Je-li $L$ jazyk 1. řádu a $T$ množina formulí jazyka $L$, říkáme, že $T$ je \emph{teorie 1. řádu} s jazykem L.
\end{mydef}

\begin{mydef}
Říkáme, že teorie je \emph{sporná}, jestliže pro každou formuli $\varphi$ jazyka $L$ platí $T \vdash \varphi$. V opačném případě je teorie \emph{bezesporná}.
\end{mydef}
(tedy platí $T \vdash \varphi$ a zároveň $T \vdash \lnot \varphi$)

Důsledek: Nechť $T$ je množina formulí a nechť $\varphi'$ je uzávěr formule $\varphi$. Potom $T \vdash \varphi$, právě když $T \cup \{\lnot \varphi'\}$ je sporná teorie.

\subsection{Model a důsledek teorie}

\begin{mydef}
Buď $T$ teorie s jazykem $L$ a nechť $\mathcal{M}$ je nějaká realizace jazyka $L$. Řekněme, že $\mathcal{M}$ je model teorie $T$, jestliže $\mathcal{M} \models \varphi$ pro každou formuli $\varphi \in T$. Pak píšeme $\mathcal{M} \models T$.
\end{mydef}

\begin{mydef}
Řekneme, že formule $\varphi$ je \emph{důsledkem teorie $T$}, jestliže pro každý model teorie $\mathcal{M}$ teorie $T$ je $\mathcal{M} \models \varphi$. Pak píšeme $T \models \varphi$.
\end{mydef}

\begin{veta}
(O korektnosti) Je-li teorie s jazykem $L$ a $\varphi$ formule taková, že $T \vdash \varphi$, pak $T \models \varphi$.
\end{veta}

Důsledek: Má-li teorie $T$ s jazykem $L$ nějaký model, potom je bezesporná.

\begin{veta}
(Gödelova věta o úplnosti) Je-li $T$ teorie s jazykem $L$ a je-li $\varphi$ libovolná formule jazyka $L$, pak $T \vdash \varphi$ právě když $T \models \varphi$.
\end{veta}

\begin{veta}
(Gödelova věta o úplnosti) Teorie $T$ je bezesporná, právě když má nějaký model.
\end{veta}

\begin{mydef}
Řekneme, že teorie $T$ s jazykem L je \emph{úplná}, jestliže $T$ je bezesporná a pro každou uzavřenou formuli $\varphi$ platí $T \vdash \varphi$ nebo $T \vdash \lnot \varphi$ (v důsledku bezespornosti nemůže platit $T \vdash \varphi$ i $T \vdash \lnot \varphi$ současně), V opačném případě říkáme, že $T$ je \emph{neúplná}.
\end{mydef}

\begin{mydef}
Řekneme, že teorie $T$ s jazykem $L$ je \emph{Henkinova}, jestliže pro libovolnou uzavřenou formuli tvaru $\exists x \psi$ jazyka $L$ existuje konstanta $c$ jazyka $L$ taková, že $T \vdash (\exists x \psi) \to \psi_x[x]$.
\end{mydef}

\begin{lemma}
Libovolná Henkinova teorie má model.
\end{lemma}

\begin{mydef}
Jazyk $L'$ je rozšířením jazyka $L$ jestliže každý speciální symbol jazyka $L$ je obsažen v jazyce $L'$. Teorie $T'$ jazyka $L'$ je rozšířením teorie $T$ jazyka $L$, jestliže pro libovolnou formuli $\varphi$ jazyka $L$ takovou, že $T \vdash \varphi$, je také $T' \vdash \varphi$. Teorie $T'$ je konzervativním rozšířením teorie $T$, jestliže  navíc pro každou formuli $\psi$ jazyka $L$ takovou, že $T' \vdash \psi$, je již $T \vdash \psi$.
\end{mydef}

\begin{lemma}
(Heinkin) K libovolné teorii lze sestrojit Heinkinovu teorii $T_H$, která je konzervativním rozšířením teorie $T$.
\end{lemma}

\begin{lemma}
Teorie $T$ je bezesporná, právě když každá její konečná podmnožina $Q \subseteq T$ je bezesporná.
\end{lemma}

\begin{veta}
(Lindenbaum) Je-li $T$ bezesporná teorie s jazykem $L$, pak existuje rozšíření $T'$ teorie $T$ se stejným jazykem $L$.
\end{veta}

\section{Věta o kompaktnosti a věta Herbrandova}

\begin{veta}
(O kompaktnosti) Nechť $T$ je množina formulí jazyka $L$. Pak teorie $T$ má nějaký model, právě když každá její konečná podmnožina $Q \subseteq T$ má model.
\end{veta}

\begin{veta}
(Löwenheim,Skolem) Má-li teorie $T$ s jazykem $L$ nekonečný model, pak má model libovolné mohutnosti $n \geq max\{\aleph_0, |L|\}$
\end{veta}

Is there more?





%%%%%%%%%%%%%%%%%%%%%%%%%%%%%%%%%%%%%%%%%%%%%%%%%%%%%%%%%%%%%%%%%%%%%%%%%%%%%%%%
%%%%%%%%%%%%%%%%%%%%%%%%%%%%%%%%%%%%%%%%%%%%%%%%%%%%%%%%%%%%%%%%%%%%%%%%%%%%%%%%
\chapter{Algebraické struktury} \label{cha:8}

1. semestr, MAT, \texttt{Zaklady\_obecne\_algebry.pdf}, 1. kapitola?

(grupy, okruhy, obory integrity a tělesa, svazy a Boolovy algebry, univerzální algebry)

\section{Operace a zákony}

\begin{mydef}
Buď $A$ množina, $n \in \mathbb{N}_0$. Potom zobrazení $\omega : A^n \to A$ se nazývá n-ární operace na $A$. Tedy pro $n \in \mathbb{N}$:

$$ \omega : A^n \to A $$
$$ \omega : (x_1, \dots, x_n) \to \omega x_1 \dots x_n$$
\end{mydef}

\begin{mydef}
Buď $A$ množina, $n \in \mathbb{N}_0, D \subseteq A^n$. Potom zobrazení $\omega : D \to A$ se nazývá n-ární parciální operace na $A$.
\end{mydef}

\begin{mydef}
Buď $A$ množina $I$ množina (indexů). Pro $i \in I$ buď $\omega_i$ n-ární operace na $A$, $n_i \in \mathbb{N}$. Potom $\mathcal{A} := (A, (\omega_i)_{i \in I}$ označujeme (univerzální) algebru s nosnou množinou $A$ a souborem operací $(\omega_i)_{i \in I} =: \Omega$.
\end{mydef}

\begin{mydef}
Buď $A$ množina, $\circ$ binární operace na $A$. Prvek $e \in A$ se nazývá
a) \emph{levý neutrální prvek} vzhledem k $\circ: \Leftrightarrow x \in A : e \circ x = x$,
b) \emph{pravý neutrální prvek} vzhledem k $\circ: \Leftrightarrow x \in A : x \circ e = x$,
c) \emph{neutrální prvek} vzhledem k $\circ: \Leftrightarrow x \in A : e \circ x = x \circ e = x$
\end{mydef}

\begin{veta}
Buď $\circ$ binární operace na $A$, $e_1$ levý neutrální prvek a $e_2$ pravý neutrální prvek. Potom platí: $e_1 = e_2$ a $e_1 (= e_2)$ je neutrální prvek.
\end{veta}
Existuje nanejvýše jeden neutrální prvek.

\begin{mydef}
Buď $A$ množina $\circ$ binární operace, $e$ neutrální prvek, $x \in A$. Potom nazýváme prvek $y \in A$
a) \emph{levým inverzním prvkem} k $x: \Leftrightarrow y \circ x = e$
b) \emph{pravým inverzním prvkem} k $x: \Leftrightarrow x \circ y = e$
c) \emph{inverzním prvkem} k $x: \Leftrightarrow x \circ y = y \circ x = e$
\end{mydef}

\begin{mydef}
Prvek $x$ se nazývá invertibilní: $\Leftrightarrow$ existuje inverzní prvek k $x$.
\end{mydef}

\begin{mydef}
Buď $A$ množina, $\circ$ binární operace na $A$. $\circ$ se nazývá \emph{asociativní}: $\Leftrightarrow \forall x,y,z \in A: (x \circ y) \circ = x \circ (y \circ z)$ (asociativní zákon). 
\end{mydef}

\begin{veta}
Buď $\circ$ \textbf{asociativní} binární operace na $A$, $x \in A$, $y_1$ levý inverzní prvek k $x$, $y_2$ pravý inverzní prvek k $x$. Potom platí $y_1 = y_2$.
\end{veta}

Je-li operace asociativní, existuje ke každému prvku nejvýše jeden inverzní prvek.

\begin{mydef}
Binární operace $\circ$ se nazývá operace s dělením na $A$: $\Leftrightarrow \forall (a,b) \in A^2 \exists (x,y) \in A^2: a \circ x = b$ (levý zákon o dělení) $\land y \circ a = b$ (pravý zákon o dělení).
\end{mydef}

\begin{veta}
Buď $A \not= \emptyset$ a $\circ$ asociativní binární operace na $A$. Potom jsou následující tvrzení ekvivalentní:
\begin{enumerate}[a)]
	\item $\circ$ je operace s dělením na $A$
	\item Existuje neutrální prvek $e$ (vzhledem k $\circ$) a každý prvek $x \in A$ je invertibilní, tzn. $\exists y \in A: x \circ y = y \circ x = e$.
\end{enumerate}
\end{veta}

\begin{mydef}
Binární operace $\circ$ na $A$ se nazývá operace s krácením: $\Leftrightarrow \forall a, x_1, x_2, y_1, y_2 \in A: (a \circ x_1 = a \circ x_2 \Rightarrow x_1 = x_2)$ (levý zákon o krácení) $\land (y_1 \circ a = y_2 \circ a \Rightarrow y_1 = y_2$ (pravý zákon o krácení).
\end{mydef}

\begin{mydef}
Binární operace $\circ$ na $A$ se nazývá komutativní: $\Leftrightarrow \forall x,y \in A: x \circ y = y \circ x$ (komutativní zákon).
\end{mydef}

\begin{mydef}
Pokud jsou $+, \cdot$ binární operace na $A$, potom se $\cdot$ nazývá distributivní nad $+$: $\Leftrightarrow \forall x,y,z \in A: x \cdot (y + z) = x \cdot y + x \cdot z$ (levý komutativní zákon) $\land (y + z) \cdot x = y \cdot x + z \cdot x$ (pravý komutativní zákon).
\end{mydef}

\section{Důležité typy algeber}
\begin{mydef}
Algebra $(A, \cdot)$ typu (2) se nazývá \emph{grupoid}.
\end{mydef}

\begin{mydef}
Grupoid $(H, \cdot)$ se nazývá \emph{pologrupa}: $\Leftrightarrow \cdot$ je asociativní.
\end{mydef}

\begin{mydef}
\begin{enumerate}[a)]
	\item Pologrupa $(H, \cdot)$ se nazývá \emph{monoid} typu (2): $\Leftrightarrow$ existuje neutrální prvek $e$.
	\item Algebra $(H, \cdot, e)$ typu (2,0) se nazývá \emph{monoid} typu (2, 0): $\Leftrightarrow$ platí následující zákony pro všechna $x,y,z \in H$:
	\begin{enumerate}[1)]
		\item $x(yz) = (xy)z$
		\item $ex = x$, $xe = x$
	\end{enumerate}
\end{enumerate}
\end{mydef}

\begin{mydef}
\begin{enumerate}[a)]
	\item Monoid $(G, \circ)$ s neutrálním prvkem $e$ se nazývá \emph{grupa} typu (2): $\Leftrightarrow$ každý prvek $x \in G$ je invertibilní, tj., $\forall x \in G \exists x^{-1} \in G: x x^{-1} = x^{-1} x = e$.
	\item Algebra $(G, \cdot, e, ^{-1}$ typu (2, 0, 1) se nazývá \emph{grupa} typu (2, 0, 1): $\Leftrightarrow$ platí následující zákony pro všechna $x, y, z \in G$:
	\begin{enumerate}[1)]
		\item $x(yz) = (xy)z$
		\item $ex = x$, $xe = x$
		\item $x x^{-1} = e$, $x^{-1} x = e$
	\end{enumerate}
	\item Grupa $(G, \cdot)$, resp. $(G, \cdot, e, ^{-1})$ se nazývá \emph{komutativní nebo abelovská}ů $\Leftrightarrow \forall x,y \in G: xy = yx$.
\end{enumerate}
\end{mydef}

\begin{mydef}
\begin{enumerate}[a)]
	\item Algebra $(R, +, \cdot)$ typu (2, 2) se nazývá \emph{okruh} typu (2, 2): $\Leftrightarrow$
	\begin{enumerate}[1)]
		\item $(R, +)$ je abelovská grupa
		\item $(R, \cdot)$ je pologrupa
		\item $\cdot$ je distributivní nad $+$
	\end{enumerate}
	\item Algebra $(R, +, 0, -, \cdot)$ typu (2, 0, 1, 2) se nazývá \emph{okruh} typu (2, 0, 1, 2): $\Leftrightarrow$
		\begin{enumerate}[1)]
		\item $(R, +, 0, -)$ je abelovská grupa
		\item $(R, \cdot)$ je pologrupa
		\item $\cdot$ je distributivní nad $+$
	\end{enumerate}
	Prvek $0$ se nazývá "nulový prvek" okruhu. Budeme psát $x - y:= x + (-y)$
\end{enumerate}
\end{mydef}

\begin{mydef}
\begin{enumerate}[a)]
	\item Algebra $(R, +, 0, -, \cdot, 1)$ typu (2, 0, 1, 2, 0) se nazývá \emph{okruh s jednotkovým prvkem}: $\Leftrightarrow$
	\begin{enumerate}[1)]
		\item $(R, +, 0, -, \cdot)$ je okruh
		\item $1$ je neutrální prvek vzhledem k $\cdot$, tj. $\forall x \in R: 1 \cdot x = x \cdot 1 = x$ (1 se nazývá \emph{jednotkový prvek} okruhu).
	\end{enumerate}
	\item Okruh $(R, +, 0, -, \cdot)$ se nazývá \emph{komutativní}: $\Leftrightarrow \forall x ,y \in R: xy = yx$
	\item Algebra $(R, +, 0, -, \cdot, 1)$ se nazývá \emph{komutativní okruh s jednotkovým prvkem}: $\Leftrightarrow$
		\begin{enumerate}[1)]
		\item $(R, +, 0, -, \cdot)$ je komutativní okruh
		\item $1$ je neutrální prvek vzhledem k $\cdot$.
	\end{enumerate}
\end{enumerate}
\end{mydef}

\begin{mydef}
Komutativní okruh s jednotkovým prvkem $(R, +, 0, -, \cdot, 1)$ se nazývá \emph{obor integrity}: $\Leftrightarrow$
\begin{enumerate}
	\item $R \backslash \{0\} \not= \emptyset$ (tj. $0 \not= 1$
	\item $\forall x,y \in R: x \not= 0 \land y \not= 0 \Rightarrow xy \not= 0$ (tj. neexistují dělitelé nuly).
\end{enumerate}
$\cdot$ je operace s krácením na $R \backslash \{0\}$.
$(R \backslash \{0\}, \cdot, 1)$ je komutativní monoid.
\end{mydef}

\begin{mydef}
\begin{enumerate}[a)]
	\item Okruh s jednotkovým prvkem $(R, +, 0, -, \cdot, 1)$ se nazývá těleso: $\Leftrightarrow$
	\begin{enumerate}[1)]
		\item $0 \not= 1$
		\item $(R \backslash \{0\}, \cdot)$ je grupa
	\end{enumerate}
	\item Komutativní těleso se nazývá \emph{pole}.
\end{enumerate}
\end{mydef}

\begin{veta}
Každé \emph{pole} je \emph{obor integrity}. Každý konečný obor integrity je pole.
\end{veta}

\begin{mydef}
Buď $(K, +, 0, -, \cdot, 1)$ pole, $I = \{a, b, c\} \cup K$, kde $a,b,c \not\in K$, $a,b,c$ po dvou různé. Algebra $(V, (\omega_i)_{i \in I}$ typu $(2, 0, 1, (1)_{\lambda \in K})$ se nazývá vektorový prostor nad $K$: $\Leftrightarrow$
\begin{enumerate}[1)]
	\item $(V, \omega_a, \omega_b, \omega_c) =: (V, +, 0, -)$ je abelovská grupa,
	\item $\forall x,y \in V, \lambda, \mu \in K$:\\
	$$\omega_\lambda(x + y) = \omega_\lambda(x) + \omega_\lambda(y) $$
	$$\omega_{\lambda + \mu}(x) = \omega_\lambda(x) + \omega_\mu(x) $$
	$$\omega_{\lambda \mu}(x) = \omega_\lambda( \omega_\mu(x)) $$
	$$\omega_1(x) = x $$
\end{enumerate}
\end{mydef}

\begin{mydef}
Algebra $(V, \cap, \cup)$ typu $(2, 2)$ se nazývá \emph{svaz}: $\Leftrightarrow$ pro všechna $a,b,c \in V$ platí:
\begin{enumerate}[1)]
	\item $a \cap b = b \cap a$,
		$a \cup b = b \cup a$
	\item $a \cap (b \cap c) = (a \cap b) \cap c$,
		$a \cup (b \cup c) = (a \cup b) \cup c$
	\item $a \cap (a \cup b) = a$,
		$a \cup (a \cap b) = a$
\end{enumerate}
Podle 1) a 2) jsou $\cap$, $\cup$ komutativní a asociativní, tj. $(V, \cap)$, $(V, \cup)$ jsou komutativní pologrupy. Zákony uvedení v bodě 3) se nazývají absorpční zákony.

$(V, \cap, \cup)$ je svaz $\Leftrightarrow$ $(V, \cup, \cap)$ je svaz -- princip duality pro svazy.
\end{mydef}

\begin{mydef}
Svaz $(V, \cap, \cup)$ se nazývá distributivní: $\Leftrightarrow$ pro všechna $a, b, c \in V$ platí 
\begin{enumerate}[4)]
	\item $a \cap (b \cup c) = (a \cap b) \cup (a \cap c)$,
		$a \cup (b \cap c) = (a \cup b) \cap (a \cup c)$
\end{enumerate}
\end{mydef}

\begin{mydef}
Buď $(V, \cap, \cup)$ svaz.
Prvek $0 \in V$ se nazývá \emph{nulový prvek svazu $V$}: $\Leftrightarrow \forall a \in V: a \cup 0 = a$ (tj. 0 je neutrální vzhledem k $\cup$).
Prvek $1 \in V$ se nazývá \emph{jednotkový prvek svazu $V$}: $\Leftrightarrow \forall a \in V: 1 \cap a = a$ (tj. neutrální vzhledem k $\cap$).
\end{mydef}

\begin{mydef}
Algebra $(V, \cap, \cup, 0, 1)$ typu (2, 2, 0, 0) se nazývá \emph{ohraničený svaz}: $\Leftrightarrow$
\begin{enumerate}[1)]
	\item $(V, \cap, \cup)$ je svaz,
	\item $0$ je nulový prvek svazu $V$,
	\item $1$ je jednotkový prvek svazu $V$.
\end{enumerate}
\end{mydef}

\begin{mydef}
Ohraničený svaz $(V, \cap, \cup, 0, 1)$ se nazývá komplementární: $\Leftrightarrow \forall x \in V \exists a' \in V: a \cap a' = 0 \land a \cup a' = 1$. Prvek $a'$ se nazývá \emph{komplement} prvku $a$.
\end{mydef}

\begin{mydef}
Distributivní a komplementární svaz $(V, \cap, \cup, 0, 1)$ se nazývá Boolův svaz.
\end{mydef}

\begin{veta}
Je-li $(V, \cap, \cup, 0, 1)$ Booleův svaz, pak existuje ke každému $a \in V$ přesně jeden komplement $a'$.
\end{veta}

\begin{mydef}
Algebra $(B, \cap, \cup, 0, 1, ')$ typu (2, 2, 0, 0, 1) se nazývá Booleova algebra: $\Leftrightarrow$
\begin{enumerate}[1)]
	\item $(B, \cap, \cup, 0, 1)$ je ohraničený svaz
	\item $\forall a \in B: a \cap a' = 0 \land a \cup a' = 1$.
\end{enumerate}
\end{mydef}

\subsection{Strom algeber}
\begin{itemize}
\item Booleův svaz
 \begin{itemize}
 \item Distributivní svaz
  \begin{itemize}
  \item Svaz $(V, \cap, \cup)$
   \begin{itemize}
   \item $(V, \cap)$, $(V, \cup)$ jsou komutativní pologrupy (komutativní a asociativní zákony)
   \item Absorpční zákony
   \end{itemize}
  \item Distributivní zákony
  \end{itemize}
 \item Komplementární svaz
  \begin{itemize}
  \item Ohraničený svaz
   \begin{itemize}
   \item Svaz $(V, \cap, \cup)$
   \item $0$ je nulový prvek svazu $V$
   \item $1$ je jednotkový prvek svazu $V$
   \end{itemize}
  \item Komplement $'$
  \end{itemize}
 \end{itemize}
\item Vektorový prostor nad $K$
 \begin{itemize}
 \item Pole $(K, +, 0, -, \cdot, 1)$
 \item Abelovská grupa: $(V, \omega_a, \omega_b, \omega_c) = (V, +, 0, -)$ je 
 \item ??? zákony
 \end{itemize}
\item Pole
 \begin{itemize}
 \item Konečný
 \item Obor Integrity
  \begin{itemize}
  \item Komutativní okruh s jednotkovým prvkem
   \begin{itemize}
   \item $(R, +, 0, -)$ je komutativní okruh
   \item 1 je neutrální prvek vzhledem k $\cdot$
   \end{itemize}
  \item $1 \not= 0$
  \item neexistují dělitelé nuly
  \end{itemize}
 \end{itemize}
\item Pole
 \begin{itemize}
 \item Komutativní
 \item Těleso
  \begin{itemize}
  \item Okruh s jednotkovým prvkem
   \begin{itemize}
   \item $(R, +, 0, -, \cdot)$ je okruh typu $(2, 0, 1, 2)$ nebo $(R, +, \cdot)$
    \begin{enumerate}
    \item $(R, +, 0, -)$ je abelovská grupa nebo $(R, +)$
     \begin{enumerate}
     \item \textbf{Grupa} $(G, \cdot, e, ^{-1})$ nebo $(G, \circ)$
     \item Komutativní zákony
     \end{enumerate}
    \item $(R, \cdot)$ je pologrupa
    \item $\cdot$ je distributivní nad $+$
    \end{enumerate}
   \item $1$ je neutrální prvek vzhledem k $\cdot$
   \end{itemize}
  \item $0 \not= 1$
  \item $(R \backslash \{0\}, \cdot)$ je grupa
  \end{itemize}
 \end{itemize}

\item Grupa
      \begin{itemize}
      \item Monoid
       \begin{itemize}
       \item Pologrupa $(H, \cdot)$, $(H, \cdot, e)$
        \begin{itemize}
        \item Grupoid
         \begin{enumerate}
		 \item Algebra $(A, \cdot)$ typu 2
		 \end{enumerate}
        \item $\cdot$ asociativní
        \end{itemize}
       \item neutrální prvek
       \item x(yz) = (xy)z
       \end{itemize}
      \item x(yz) = (xy)z
      \item neutrální prvek
      \item každý prvek je invertibilní
      \end{itemize}
\end{itemize}

\section{Základní pojmy teorie grup}

\paragraph{Grupoid}
Součin: $a_1, a \dots a_n := (a_1 \dots a_{n-1})a_n$

Mocniny: $a^1 = a$, $a^{n+1} = (a^n)a$

\paragraph{Grupa}
$(ab)^{-1} = b^{-1}a^{-1}$

$a^0 = e$;
$a^{-n} = (a^{-1})^n$;
$a^n a^m = a^{n+m}$;
$(a^m)^n = a^{mn}$;
$(a b)^n = a^n b^n$ (pokud je $\cdot$ komutativní)

Kardinální číslo ($a^n$) se nazává \emph{řád prvku}: $o(a) := |\{a^0 = e, a^1, a^{-1}, a^2, a^{-2}, \dots\}| = |\{ a^k | k \in Z\}$.

$|G|$ (mohutnost množiny) se nazývá řád grupy/algebry.

Dělení se zbytkem: $\forall k,l \in Z, l \not= 0 \exists q,r \in Z: 0 \leq  r < |l| \land k = lq + r$.

"$r$ je kongruentní s $s$ modulo $n$": $n|(r - s)$ ($n$ dělí $(r-s)$).

Je-li $o(a) = \infty$, pak jsou mocniny prvku $a$ navzájem různé. Je-li $o(a) = n$, potom $a^r = a^s \Leftrightarrow r \equiv s \text{ mod } n$



\section{Svazy a Booeovy algebry}
\subsection{(Částečně uspořádané množiny}

$M$ je množina, $R$ je relace na $M$. 

Částečné uspořádání $(M, R)$ = reflexivita, antisymetrie, tranzitivita.

$(M, R)$ Řetězec nebo Lineárně uspořádaná množina: navíc srovnatelnost: $\forall x,y \in M: xRy \lor yRx$.

\begin{mydef}
Buď $(M, \leq)$ uspořádaná množina. Potom se $k \in M$ nazývá nejmenší (resp. největší) prvek množiny $M :\Leftrightarrow \forall x \in M: k \leq x$ (resp. $k \geq x$).
\end{mydef}

Existuje vždy nejvýše jeden nejmenší resp. největší prvek.

\begin{mydef}
Buď $(M, \leq)$ uspořádaná množina. Potom se $m \in M$ nazývá minimální (resp. maximální) prvek množiny $M :\Leftrightarrow \forall x \in M: x \leq m$ (resp. $x \geq m$) $\Rightarrow x = m$.
\end{mydef}

\begin{veta}
\begin{enumerate}[a)]
	\item Buď $(M, \leq)$ uspořádaná množina a $N \subseteq M$. Potom je $(N, \leq)$ rovněž uspořádaná množina. Je-li $(M, \leq)$ řetězec, potom je také $(N, \leq)$ řetězec. Přitom $(N, \leq)$ zkráceně označuje $(N, \leq \cap (N \times N))$.
	\item Je-li $(M, \leq)$ uspořádaná množina, potom také $(M, \geq)$ je uspořádaná množina (tzv. \uv{princip duality uspořádané množiny}). Také maximální a minimální prvek.
\end{enumerate}
\end{veta}

\begin{mydef}
Buď $(M, \leq)$ uspořádaná množina a $N \subseteq M$. Potom se nazývá $u \in M$ dolní závora množiny $N : \Leftrightarrow \forall x \in N: u \leq x$. Největší prvek množiny všech dolních závor se nazývá \emph{infimum} množiny $N$, formálně inf $N$ nebo $\bigcap N$. Prvek $v \in M$ se nazývá horní závora množiny $N: \Leftrightarrow \forall x \in N: x \in N: x \leq v$. Nejmenší horní závora se nazývá suprémum množiny $N$, formálně sup $N$ nebo $\bigcup N$.
\end{mydef}

\paragraph{Hasseův diagram} Buď $(M, \leq)$ konečná uspořádaná množina a nechť relace "sousední" je definována takto:

a, b sousedí: $\Leftrightarrow \begin{cases}
a < b \text{nebo} b < a \\
\text{neexistuje c taková, že} a < c < b \text{nebo} b < c < a
\end{cases}$

Potom je Hasseův diagram $(M, \leq)$ dán grafy relace \uv{sousedí}. (Množina uzlů je M; je-li $a < b$, nakreslí se uzel $a$ \uv{níže} než uzel $b$ a $a$ se spojí s $b$ hranou, pokud jsou $a$ a $b$ sousední).

\subsection{(Částečná) uspořádaní a svazy}
\begin{mydef}
Buď $(V, \leq)$ uspořádaná množina. $(V, \leq)$ se nazývá svazově uspořádaná: $\Leftrightarrow sup\{a, b\}$ a $inf\{a, b\}$ existují pro všechna $a, b \in V$.
\end{mydef}

\begin{lemma}
Buď $(V, \cap, \cup)$ svaz, potom platí:
\begin{enumerate}[a)]
	\item $\forall a \in V: a \cap a = a = a \cup a$
	\item $\forall a, b \in V: a \cap b = a \Leftrightarrow a \cup b = b$
\end{enumerate}
\end{lemma}

\begin{veta}
\begin{enumerate}
	\item Buď $(V, \cap, \cup)$ svaz. Pokud definujeme relaci $\leq$ na $V$ pomocí vztahu $a \leq b: \Leftrightarrow a \cap b = a, a,b \in B$, potom je $(V, \leq)$ svazově uspořádaná množina.
	\item Buď $(V, \leq)$ svazově uspořádaná množina. Definujeme-li na $V$ binární operace $\cap, \cup$ pomocí vztahů $a \cap b := inf\{a, b\}$ a $a \cap b := sup\{a, b\}, a, b \in V$, potom je $(V, \cap, \cup)$ svaz.
	\item Přiřazení definovaná v a) a b) jsou navzájem inverzní.
\end{enumerate}
\end{veta}

Princip duality pro svazy:

$(V, \cap, \cup)$ svaz $\Leftrightarrow (V, \cup, \cap)$ svaz

$(V, \leq)$ svazově uspořádaný $\Leftrightarrow (V, \geq)$ svazově uspořádaný

\subsection{Booleovy algebry}

Dualita: $(B, \cap, \cup, 0, 1, ')$ Booleova algebra $\Leftrightarrow (B, \cup, \cap, 1, 0, ')$ Booleova algebra

\begin{lemma}
Buď $(V, \cap, \cup)$ svaz. Potom platí:
\begin{enumerate}[a)]
	\item $\forall a,b,c \in V: a \cap (b \cup c) = (a \cap b) \cup (a \cap c) \Leftrightarrow \forall a, b, c \in V: a \cup (b \cap c) = (a \cup b) \cap (a \cup c)$
	\item $\forall a \in V: 0 \cup a = a \Leftrightarrow \forall a \in V: 0 \cap a = 0$
	\item $\forall a \in V: 1 \cap a = a \Leftrightarrow \forall a \in V: 1 \cup a = 1$
\end{enumerate}
\end{lemma}

\begin{veta}
(Věta o komplementech) Buď $(B, \cap, \cup, 0, 1, ')$ Booleova algebra. Potom platí:
\begin{enumerate}[a)]
	\item Jsou-li $a, a^*$ prvky množiny $B$, kde $a \cup a^* = 1$ a $a \cap a^* = 0$, pak platí $a^* = a'$
	\item $(a')' = a$ pro všechna $a \in B$
	\item $0' = 1$ a $1' = 0$
	\item $(a \cup b)' = a' \cap b'$ a $(a \cap b) = a' \cup b'$ pro všechna $a, b \in B$ (De Morganovy zákony)
\end{enumerate}
\end{veta}

\begin{veta}
(Věta o homomorfizmech) Buďte $(B, \cap, \cup, 0, 1, ')$ a $(C, \cap, \cup, 0, 1, ')$ Booleovy algebry, $\varphi: B \to C$ surjektivní zobrazení. Potom platí $\varphi$ je homomorfizmus $(B, \cap, \cup, 0, 1, ')$ do $(C, \cap, \cup, 0, 1, ') \Leftrightarrow \varphi$ je homomorfizmus $(B, \cap, \cup)$ do $(C, \cap, \cup)$ (tj. stačí aby $\varphi$ bylo konzistentní se svazovými operacemi).
\end{veta}

\subsection{Stoneova věta o reprezentaci}
\begin{mydef}
Buď $(V, \cap, \cup, 0, 1)$ svaz s nulovým a jednotkovým prvkem. Potom se $a \in V$ nazývá \emph{atom}: $\Leftrightarrow$
\begin{enumerate}[1)]
	\item $0 < a$
	\item $0 \leq b \leq a \Rightarrow b = 0 \lor b = a$
\end{enumerate}
(tj. $a$ je horním sousedním prvkem nulového prvku.)
\end{mydef}

\begin{lemma}
Buď $(B, \cap, \cup, 0, 1, ')$ konečná Booleova algebra. Potom ke každému prvku $b \in B \backslash \{0\}$ existuje atom $a \in B$, kde $a \leq b$. (Toto platí i pro libovolní konečné svazy.).
\end{lemma}

\begin{veta}
(Stoneova věta) Buď $(B, \cap, \cup, 0, 1, ')$ konečná Booleova algebra a $M := \{a \in B | a \text{atom algebry} \}$. Potom platí:

$$ (B, \cap, \cup, 0, 1, ') \cong (\mathcal{P}(M), \cap, \cup, \emptyset, M, ')$$

přičemž  vztahem $\varphi(b) := \{a \in M | a \leq b\}$ je dán izomorfizmus $\varphi: B \to \mathcal{P}(M)$.
\end{veta}

$|M| = |M_1| \Rightarrow (\mathcal{P}(M), \cap, \cup, \emptyset, M, ') \cong (\mathcal{P}(M_1), \cap, \cup, \emptyset, M_1, ')$

$|M| = n \in \mathbb{N}_0 \Rightarrow |\mathcal{P}(M)| = 2^n$

Je-li $B$ konečná Booleova algebra, potom platí $|B| = 2^n$ pro libovolné $n \in \mathbb{N}_0$ pro libovolné $n \in \mathbb{N}_0$. Ke každému $n \in \mathbb{N}_0$ tak existuje -- až na izomorfizmus -- přesně jedna Booleova algebra s $2^n$ prvky, totiž $\mathcal{P}(\{0, 1, \dots, n-1\})$.

\begin{mydef}
Buď $M$ množina.  $\mathcal{K} \in \mathcal{P}(M)$ se nazývá množinový okruh: $\Leftrightarrow$ pro všechna $A, B \in \mathcal{K}$ platí:
\begin{enumerate}[1)]
	\item $A \cup B \in \mathcal{K}$
	\item $A \cap B \in \mathcal{K}$
	\item $A \cap B' = A \backslash B \in \mathcal{K}$
\end{enumerate}
\end{mydef}

\begin{mydef}
Buď $\mathcal{K} \in \mathcal{P}(M)$ množinový okruh a nechť $M \in \mathcal{K}$. Potom Booleova algebra $(\mathcal{K}, \cap, \cup, \emptyset, M, ')$ se nazývá algebra množinového okruhu.
\end{mydef}

Algebra množinového okruhu je tedy podalgebra $(\mathcal{P}(M), \cap, \cup, \emptyset, M, ')$

\begin{mydef}
(Stoneova věta) Každá Booleova algebra je izomorfní s nějakou algebrou množinového okruhu.
\end{mydef}

















%%%%%%%%%%%%%%%%%%%%%%%%%%%%%%%%%%%%%%%%%%%%%%%%%%%%%%%%%%%%%%%%%%%%%%%%%%%%%%%% 
%%%%%%%%%%%%%%%%%%%%%%%%%%%%%%%%%%%%%%%%%%%%%%%%%%%%%%%%%%%%%%%%%%%%%%%%%%%%%%%%
\chapter{Základní algebraické metody} \label{cha:9}

1. semestr, MAT, \texttt{Zaklady\_obecne\_algebry.pdf}, 2. kapitola

(podalgebry, homomorfismy, přímé součiny, kongruence a faktorové algebry, normální podgrupy a ideály okruhů)

\section{Podalgebry}

\begin{mydef}
Buď $A$ množina, $\omega: A^n \to A$ n-ární operace na $A$ ($n \in N_0)$, $T \subseteq A$. Potom se množina $T$ nazývá uzavřená vzhledem k $\omega: \Leftrightarrow \omega(T^n) \subseteq T$ (tj. $t_1, \dots, t_n \in T \Rightarrow \omega t_1 \dots t_n \in T$, v případě $n = 0: \omega \in T$.
\end{mydef}

\begin{mydef}
Buď $\mathcal{A} = (A, (\omega_i)_{i \in I}$ algebra typu $(n_i)_{i \in I}$, $T \subseteq A$. Potom se množina $T$ nazývá uzavřená vzhledem k $(\omega)_{i \in I}: \Leftrightarrow T$ je uzavřená vzhledem k $\omega_i$ pro všechna $i \in I$. V tomto případě se pomocí vztahu $\omega_i^*x_1 \dots x_{n_1}$, $(x_1, \dots x_n) \in T^{n_i}$, definuje $n_i$-ární operace $\omega_i^*$'na $T$, tj $\omega_i^* = \omega_i|T^{n_i}$. Algebra $(T, (\omega_i^*)_{i \in I}$ se nazývá podalgebra algebry $\mathcal{A}$. Většinou píšeme: $\omega_i^* =: \omega_i$.
\end{mydef}

$(\mathbb{N}, +)$ je podpologrupa $(\mathbb{Z}, +)$ (ale není grupa -- chybí inverzní prvek).

$(\mathbb{N}, +, \cdot)$ je podalgebrou $(\mathbb{Z}, +, \cdot)$ (ale není podokruhem).

$(\mathbb{R}, +, 0, -, \cdot, 1)$ je podpolem $(\mathbb{C}, +, 0, -,\cdot, 1)$ zatímco $(\mathbb{Z}, +, 0, -,\cdot, 1)$

\begin{veta}
Buď $(A, \Omega)$ algebra a $(T_j)_{j \in J}$ soubor podalgeber. Potom je $\bigcap_{\{j \in J\}} T_j$ rovněž podalgebra.
\end{veta}

\begin{veta}
Buď $(A, \Omega)$ algebra a $S \subseteq A$ podmnožina. Potom je

$$\left< S \right> := \bigcap\{T | T \supseteq S, T \text{je podalgebra algebry} (A, \Omega)\}$$

je nejmenší podalgebra algebry $(A, \Omega)$, která $S$ obsahuje.
\end{veta}

\begin{mydef}
$\left< S \right>$ se nazývá podalgebra algebry $(A, \Omega)$ generovaná množinou $S$. Množina $S$ se nazývá systém generátorů podalgebry $\left< S \right>$.
\end{mydef}

\begin{veta}
Buď $(G, \cdot, e, ^{-1})$ grupa, $x \in G, S = \{x\}$. Potom platí:

$$\left< x \right> := \left< S \right> = \{x^k | k \in \mathbb{Z}\}$$
\end{veta}

\begin{mydef}
$\left< x \right>$ se nazývá podgrupa grupy $(G, \cdot, e, ^{-1})$ generovaná prvkem $x$.
\end{mydef}

\begin{mydef}
Grupa $(G, \cdot, e, ^{-1})$ se nazývá cyklická: $\Leftrightarrow \exists x \in G: G = \left< x \right>$
\end{mydef}

\begin{itemize}
	\item pro $(\mathbb{Z}, +, 0, -)$ platí $\mathbb{Z} = \left< 1 \right> = \left< -1 \right>$
	\item pro $(\mathbb{Z}_m, +, 0, -)$ platí $\mathbb{Z}_m = \left< 1 \right> = \left< k \right>$, kde $NSD(m, k) = 1$
\end{itemize}

\subsection{Relace ekvivalence a rozklad na třídy ekvivalence?} % XXX
\begin{mydef}
Je-li $M$ množina, potom se podmnožina $R$ množiny $M \times M$ nazývá binární relace na $M$. Místo $(x, y) \in R$ píšeme většinou $xRy$. Speciální relace $\alpha_M := M \times M$ se nazývá univerzální relace, $\iota_M := \{(x, x) | x \in M\}$ se nazývá identická relace nebo relace rovnosti.
\end{mydef}

\begin{mydef}
Relace $R \subseteq M \times M$ se nazývá:
\begin{enumerate}[1)]
	\item \emph{reflexivní}: $\Leftrightarrow \iota_M \subseteq R$, tj. $\forall x \in M: xRx$
	\item \emph{symetrická}: $\Leftrightarrow \forall x, y \in M: xRy \Rightarrow yRx$
	\item \emph{antisymetrická}: $\Leftrightarrow \forall x, y \in M: xRy \land yRx \Rightarrow x = y$
	\item \emph{transitivní}: $\Leftrightarrow \forall x, y, z \in M: xRy \land yRz \Rightarrow xRz$
\end{enumerate}
Relace splňující 1), 2) a 4) se nazývá relace \emph{ekvivalence}, relace splňující 1), 3) a 4) se nazývá relace \emph{(částečného) uspořádání)}.
\end{mydef}

\begin{mydef}
Buď $M$ množina. $\mathcal{P} \subseteq \mathcal{P}(M)$ se nazývá rozklad množiny $M$ na třídy ekvivalence: $\Leftrightarrow$
\begin{enumerate}[1)]
	\item $\bigcup_{C \in \mathcal{P}} C = M$
	\item $\emptyset \not\in \mathcal{P}$
	\item $A, B \in \mathcal{P}  \Rightarrow A = B \lor A \cap B = \emptyset$ (tj. množiny v $\mathcal{P}$ jsou po dvou disjunktní.
\end{enumerate}
\end{mydef}

\begin{veta}
Buď $\pi$ relace ekvivalence na množině $M$, $a \in M$, $[a]_\pi := \{b \in M | b \pi a\}$, tzv. třída ekvivalence prvku $a$ a $M/\pi := \{[a]_\pi | a \in M\}$ tzv. faktorová množina množiny M podle ekvivalence $\pi$. Potom je $M/\pi$ rozklad množiny na třídy ekvivalence.

Je-i naopak $\mathcal{P}$ rozklad množiny $M$ na třídy ekvivalence a $\pi$ je definováno vztahem $a \pi b \Leftrightarrow \exists C \in \mathcal{P}: a, b \in C$, potom je $\pi$ relace ekvivalence na množině $M$, a platí $M/\pi' = \mathcal{P}$.

$\pi \mapsto M/\pi$ je bijektivní zobrazení množiny všech relací ekvivalence na množině M na množinu všech rozkladů množiny M na třídy ekvivalence. Inverzní zobrazení je dáno výše uvedeným předpisem $\mathcal{P} \mapsto \pi$.
\end{veta}

\begin{veta}
Buďte $M, N$ množiny, $f: M \to N$ zobrazení a $x \pi_f y: \Leftrightarrow f(x) = f(y)$. Potom platí:
\begin{enumerate}[a)]
	\item $\pi_f$ je relace ekvivalence na $M$, která se nazývá jádro $f$.
	\item Zobrazení
	$M/_{\pi_f} \to f(M) := \{f(x) | x \in M\} \subseteq N$\\
	$[x]_{\pi_f} \mapsto f(x)$\\
	je korektně definováno a bijektivní.
\end{enumerate}
\end{veta}

\subsection{Rozklad grupy na třídy podle podgrupy}

\begin{veta}
Buď $(G, \cdot, e, ^{-1})$ grupa a $(H, \cdot, e, ^{-1})$ podgrupa grupy $G$. Buď dále $\pi \subseteq G \times G$ podmnožina definovaná pomocí vztahu $x \pi y: \Leftrightarrow x^{-1}y \in H. x, y \in G$. Potom je $\pi$ relace ekvivalence na $G$.
\end{veta}

\begin{mydef}
Buď $(G, \cdot, e, ^{-1})$ grupa, $A, B \subseteq G$. Potom se nazývá $AB := \{ab | a \in A, b \in B\}$ složený součin A a B. Speciální případy: $A = \{a\}: AB := aB = \{ab | b \in B\}, B = \{b\}: AB := Ab = \{ab | a \in A\}$. Pro podgrupu H grupy G se nazývá $aH$ levá třída rozkladu grupy G podle H a $Ha$ se nazývá pravá třída rozkladu grupy G podle H ($a \in G$ pevné ale libovolné).
\end{mydef}

\begin{veta}
Buď $(G, \cdot, e, ^{-1})$ grupa, $H$ podgrupa grupy $G$, $a, b \in G$. Potom je vztahem $i: aH \to bH; ax \mapsto bx$ definováno bijektivní zobrazení.
\end{veta}


\section{Homomorfismy}

\begin{mydef}
Buďte $\mathcal{A} = (A, (\omega_i)_{i \in I})$ a $\mathcal{A}^* = (A^*, (\omega^*_i)_{i \in I}$ algebry téhož typu $(n_i)_{i \in I}$. Zobrazení $f: A \to A^*$ se nazývá homomorfizmus algebry $\mathcal{A}$ do algebry $\mathcal{A}^*: \Leftrightarrow$
\begin{enumerate}
	\item Pro $i \in I$, kde $n_i > 0$ platí $\forall x_1, \dots, x_{n_i} \in A: f(\omega_ix_1 \dots x_{n_i}) = \omega_i^*f(x_1)\dots f(x_{n_1})$
	\item pro $i \in I$, kde $n_i = 0$, platí $f(\omega_i) = \omega_i^*$
\end{enumerate}
\end{mydef}

\begin{lemma}
Buďte $(G, \cdot, e, ^{-1})$ a $(H, \cdot, e, ^{-1})$ grupy, $f: G \to H$. Potom platí $f$ je homomorfismus grupy $(G, \cdot, e, ^{-1})$ do grupy $(H, \cdot, e, ^{-1}) \Leftrightarrow f$ je homomofizmus grupy $(G, \cdot)$ do grupy $(H, \cdot)$.
\end{lemma}

Pro vektorové prostory a okruhy ...

\begin{mydef}
Buďte $\mathcal{A} = (A, (\omega_i)_{i \in I})$ a $\mathcal{A}^* = (A^*, (\omega^*_i)_{i \in I}$ algebry téhož typu $(n_i)_{i \in I}$ a $f: A \to A^*$ homomorfizmus algebry $\mathcal{A}$ do algebry $\mathcal{A}^*$. $f$ se nazývá:
\begin{enumerate}
	\item \emph{izomorfizmus}, pokud $f$ je bijektivní (v tomto případě říkáme, že $\mathcal{A}$ je izomorfní obraz $\mathcal{A}^*$, a píšeme $\mathcal{A} \cong \mathcal{A}^*$).
	\item \emph{endomorfizmus}, pokud $\mathcal{A} = \mathcal{A}^*$
	\item \emph{automorfismus}, pokud $\mathcal{A} = \mathcal{A}^*$ a $f$ je izomorfismus.
	\item \emph{epimorfizmus}, pokud $f$ je surjektivní (v tomto případě se nazývá $\mathcal{A}^*$ homomorfní obraz $\mathcal{A}$).
	\item \emph{monomorfizmus}, pokud je $f$ injektivní (v tomto případě se $\mathcal{A}$ nazývá izomorfně uzavřená v $\mathcal{A}^*$).
\end{enumerate}

$g \circ f$ zachovává homomorfismus a isomorfismus.

$f^{-1}$ zachovává homomorfismus.
\end{mydef}

\begin{veta}
Buď $(H, \cdot)$ pologrupa, $(H^*, \cdot)$ grupoid a $f: H \to H^*$ homomorfizmus. Potom je podalgebra $(f(H), \cdot)$ grupoidu $(H^*, \cdot)$ pologrupa.

Platí také pro: (abelovské) grupy, (komutativní) okruhy, okruhy s jednotkovým prvkem, svazy, Booleovy algebry, vektorové prostory nad K.
\end{veta}

Rozlišujeme fundamentální operace a odvozené operace (tvořené konečným počtem proměnných a symbolů operací).

Izomorfismus je pouze přeznačení. Algebry jsou "stejné", izomorfizmus zachovává algebraické vlastnosti.

\begin{veta}
(Cayleyova věta o reprezentaci) Buď $(G, \cdot, e, ^{-1})$ grupa. Potom je $G$ izomorfní s podgrupou symetrické grupy $(S_G, \circ, id_G, ^{-1})$. Krátce: Každá grupa je izomorfní s nějakou grupou permutací.
\end{veta}

\section{Kongruence a faktorové algebry}

\begin{mydef}
Buď $\mathcal{A} = (A, (\omega_i)_{i \in I})$ algebra typu $(n_i)_{i \in I}$ a $\pi$ relace ekvivalence na $A$. $\pi$ se nazývá (relace) kongruence na $\mathcal{A} : \Leftrightarrow$ pro všechna $i \in I$, kde $n_i > 0, a_1, \dots, a_{n_i}, b_1, \dots, b_{n_i} \in A$ platí

$$ a_1 \pi b_1 \land \dots \land a_{n_i} \pi b_{n_i} \Rightarrow \omega_i a_i \dots a_{n_i} \pi \omega_i b_1 \dots b_{n_i} $$
\end{mydef}

\begin{veta}
Buď $\mathcal{A} = (A, (\omega_i)_{i \in I})$ algebra a $\pi$ kongruence na $\mathcal{A}$. Potom jsou vztahy

$$ \omega_i^*[a_1)_\pi \dots [a_{n_i}]_\pi := [\omega_i a_1 \dots a_{n_i}]_\pi, n > 0, a_1, \dots a_{n_1} \in A$$

$$ \omega_i^* := [\omega_i]_\pi, n_i = 0 $$

definovány operace $(\omega_i^*)_{i \in I}$ na faktorové množině $A/\pi$
\end{veta}

\begin{mydef}
Algebra $\mathcal{A}/\pi := (A/\pi, (\omega_i^*)_{i \in I})$ se nazývá faktorová algebra algebry $\mathcal{A}$ podle kongruence $\pi$. Často klademe $\omega_i := \omega_i^*$.
\end{mydef}

$(\mathbb{Z}, +, 0, -, \cdot, 1)$ je komutativní okruh s jednotkovým prvkem, který se nazývá okruh zbytkových tříd modulo $n$.

\begin{veta}
Buď $\mathcal{A} := (A, (\omega_i)_{i \in I})$ algebra, $\pi$ kongruence na $\mathcal{A}$. Potom je zobrazení
$$
 \nu =
   \begin{dcases}
     A \to A / \pi \\
     a \mapsto [a]_\pi
   \end{dcases}
$$
surjektivní homomorfizmus algebry $\mathcal{A}$ na $\mathcal{A}/\pi$, který se nazývá přirozený homomorfizmus.
\end{veta}

$\mathcal{A}/\pi$ je homomorfní obraz $\mathcal{A}$ a každý zákon, který platí v $\mathcal{A}$ platí také v $\mathcal{A}/\pi$ (pologrupy, (abelovské grupy), vektorové prostory, (komutativní) okruhy, okruhy s jednotkovým prvkem, svazy, Booleovy algebry). Nemusí být u oboru integrity!

\begin{veta}
(O homomorfizmu) Buďte $\mathcal{A} := (A, (\omega_i)_{i \in I})$ a  $\mathcal{A^*} := (A^*, (\omega_i^*)_{i \in I})$ algebry téhož typu $(n_i)_{i \in I}$ a $f: A \to A^*$ homomorfizmus. Potom je jádro $\pi_j$ kongruencí na $\mathcal{A}$ a existuje přesně jeden injektivní homomorfizmus $g$ z $\mathcal{A}/\pi$ do $\mathcal{A}^*$ takový, že $f = g \circ \nu$ ($\nu$ je přirozené zobrazení).
\end{veta}

Pro podalgebry $(f(A), (\omega_i^*)_{i \in I})$ algebry $\mathcal{A}^*$ platí $(f(A), (\omega_i^*)_{i \in I}) \cong \mathcal{A}/\pi_f$, tedy je každý homomorfní obraz algebry izomorfní s nějakou faktorovou algebrou.

Relace rovnosti $\iota = \{(x, x) | (x \in A\}$ a univerzální relace $\alpha = A \times A$ jsou vždy kongruencemi na $\mathcal{A}$. Platí $\mathcal{A}/\iota \cong \mathcal{A}$ a $|\mathcal{A}/\alpha| \leq 1$. $\mathcal{A}/\iota$ a $\mathcal{A}/\alpha$ jsou triviální faktorové algebry.

\begin{mydef}
Algebra $\mathcal{A}$ se nazývá prostá, má-li pouze triviální kongruence.
\end{mydef}

\subsection{Relace kongruence na grupách a okruzích}

\begin{veta}
Buď $(G, \cdot, e, ^{-1})$ grupa a $\pi$ relace ekvivalence na $G$. Potom platí
\begin{enumerate}[a)]
	\item $\pi$ je kongruence na $(G, \cdot, e, ^{-1}) \Leftrightarrow \pi$ je kongruence na $(G, \cdot)$
	\item Je-li $\pi$ kongruence na $(G, \cdot)$ a $[e]_\pi =: N$, potom platí
	\begin{enumerate}[i)]
		\item $N$ je podgrupa $(G, \cdot, e, ^{-1})$
		\item $x N x^{-1} = \{x y x^{-1} | y \in N\} \subseteq N$ pro všechna $x \in G$ 
		\item $x \pi y \Leftrightarrow x^{-1} y \in N$ pro všechna $x, y \in G$ (tj. $[x]_\pi = x N$ pro všechna $x \in G$).
	\end{enumerate}
\end{enumerate}
\end{veta}

\begin{mydef}
Podgrupa $N$ grupy $(G, \cdot, e, ^{-1})$ se nazývá normální podgrupa grupy $G$ (symbolicky $N \triangleleft G$): $\Leftrightarrow x N x^{-1} \in N$ pro všechna $x \in G$.
\end{mydef}
(v abelovské grupě je každá podgrupa normální)

\begin{lemma}
Por podgrupu $N$ grupy $G$ jsou následující tvrzení ekvivalentní:
\begin{enumerate}
	\item $N$ je normální podgrupa grupy $G$
	\item $\forall x \in G: x N x^{-1} = N$
	\item $\forall x \in G: N x = x N$, tj. pravá třída rozkladu = levá třída rozkladu.
\end{enumerate}
\end{lemma}

\begin{veta}
Buď $(G, \cdot, e, ^{-1})$ grupa, $N \triangleleft G$ a $\pi$ buď binární relace na $G$ definovaná vztahem $x \pi y: \Leftrightarrow x^{-1}y \in N, x,y \in G$. Potom je $\pi$ relace kongruence na $G$, kde $[e]_\pi = N$.
\end{veta}

\begin{veta}
Vztahem $\pi \mapsto [e]_\pi$ je definování bijektivní zobrazení množiny kongruencí na grupě G na množinu všech normálních podgrup grupy G. Inverzní zobrazení je dáno pomocí vztahu $N \mapsto \pi$, kde $x \pi y: \Leftrightarrow x^{-1}y \in N$.
\end{veta}

Chceme-li najít všechny homomorfní obrazy -- až na izomorfizmus -- nějaké grupy $G$, můžeme tedy určit všechny normální podgrupy $N$ grupy $G$ a vytvořit faktorové algebry $G/\pi$ pomocí odpovídajících kongruencí. Pokud normální podgrupě $N$ odpovídá kongruence $\pi$, píšeme $G/N := G/\pi = \{x N | x \in G\}$. Takováto faktorová algebra se nazývá faktorgrupa grupy $G$.

Triviálním kongruencím $\iota = \{(x, x) | x \in G\}$ a $\alpha = G \times G$ odpovídají tzv. triviální normální podgrupy ${e}$ a $G$. $G$ je prostá $\Leftrightarrow G$ má pouze triviální podgrupy.

\begin{veta}
Buď $G$ grupa, $U$ podgrupa, kde $[G:U] = 2$ (index $U$ v $G$). Potom platí $U \triangleleft G$.
\end{veta}

\section{Normální podgrupy}
\section{Ideály okruhů}

\begin{mydef}
Buď $(R, +, 0, -, \cdot)$ okruh a $I$ podokruh okruhu $R$. Potom se $I$ nazývá:
\begin{itemize}
	\item \emph{levý ideál} okruhu R: $\Leftrightarrow \forall r \in R: rI := \{ri | i \in I\} \subseteq I$
	\item \emph{pravý ideál} okruhu R: $\Leftrightarrow \forall r \in R: Ir := \{ir | i \in I\} \subseteq I$
	\item \emph{ideál} okruhu R (formálně $I \triangleleft R$): $\Leftrightarrow \forall r \in R: Ir  \subseteq I \land rI \subseteq I$
\end{itemize}
\end{mydef}

$R$ a $\{0\}$ jsou vždy ideály okruhu $R$, takzvané triviální ideály.

\begin{lemma}
Buď $(R, +, 0, -, \cdot, 1)$ okruh s jednotkovým prvkem a $I$ ideál okruhu $R$. Potom platí $1 \in I \Leftrightarrow I = R$.
\end{lemma}

\begin{veta}
Každé těleso má pouze triviální ideály.
\end{veta}

\begin{veta}
Buď $(R, +, 0, -, \cdot, 1)$ komutativní okruh s jednotkovým prvkem, který má pouze triviální ideály. Potom je $R$ pole nebo $R = \{0\}$
\end{veta}

Komutativní okruh $R \not= \{0\}$ s jednotkovým prvkem je pole $\Leftrightarrow R$ má pouze triviální ideály.

\begin{veta}
Buď $(R, +, 0, -, \cdot)$ okruh.
\begin{enumerate}[a)]
	\item Je-li $\pi$ kongruence na $R$, potom je $I := [0]_\pi$ ideál okruhu R, a platí $R/\pi = R/I = \{x + I | x \in R\}$
	\item Je-li $I$ ideál okruhu $R$ a $\pi$ binární relace na $R$ definovaná vztahem $x \pi y :\Leftrightarrow y - z \in I, x,y \in R$, potom je $\pi$ kongruence na $R$ a $[0]_\pi = I$
	\item $\pi \mapsto [0]_\pi$ definuje bijektivní zobrazení množiny všech kongruencí na $R$ na množinu všech ideálů okruhu $R$. Inverzní zobrazení je dáno vztahem $I \mapsto \pi$, kde $\pi$ je kongruence definovaná v b). 
\end{enumerate}
\end{veta}

Okruh $R$ je prostý $\Leftrightarrow$ $R$ má pouze triviální kongruence $\Leftrightarrow$ $R$ má pouze triviální ideály $\{0\} := (0)$ a $R$.

\begin{veta}
Komutativní okruh $R \not= \{0\}$ s jednotkovým prvkem je prostý právě tehdy, když je pole
\end{veta}

\section{Přímé součiny}

\begin{mydef}
Buďte $\mathcal{A}_k = (A_k, \omega_i^{(k)_{i \in I}}, k \in K$ algebry téhož typu $(n_i)_{i \in I}$ a $A := \prod\limits_{k \in K} A_k = \{(a_k)_{k \in K} | a_k \in A_k\}$ kartézský součin všech množin $A_k$. Pro všechna $i \in I$ buď operace $\omega_i$ na $A$ definována vztahem:

$$ \omega_i(a_k^{(1)})_{k \in K} \dots (a_k^{n_i})_{k \in K} := (\underbrace{\omega_i^{(k)}a_k^{(1)} \dots a_k^{(n_i)}}_{\in A_k})_{k \in K} \text{pro} n_i > 0 $$

$$ \omega_i := (\omega_i^{(k)})_{k \in K} \text{pro} n_i = 0 $$

Algebra $(A, (\omega_i)_{i \in I})$ se nazývá přímý součin algeber $\mathcal{A}_k$ a značí se $\prod\limits_{k \in K} \mathcal{A}_k$.
\end{mydef}

\begin{veta}
Pokud platí při vhodných termech $t_1, t_2$ zákon tvaru $\forall x_1, \dots x_n : t(x_1, \dots, x_n) = t_2(x_1, \dots, x_n)$ ve všech algebrách $\mathcal{A}_k, k \in K$, potom platí také v $\prod\limits_{k \in K}\mathcal{A}_k$.
\end{veta}

Důsledek: Přímé součiny pologrup (grup, vektorových prostorů, okruhů, Booleových algeber) jsou opět pologrupy (grupy, vektorové prostoru, okruhy, Booleovy algebry). Nikoliv však pro obory integrity!

Přímý součin je až na izomorfizmus komutativní ($A_1 \times A_2 \cong A_2 \times A_1$ a asociativní (uzávorkování).

\begin{veta}
Grupa $C_n \times C_m$ je cyklická $\Leftrightarrow \text{NSD}(m, n) = 1$.
\end{veta}

Je-li $n = p_1^{e_1} \dots p_k^{e_k}$ rozklad na prvočinitele čísla $n \in N$, potom platí $C_n \cong C_{p_1{e_1}} \times \dots \times C_{p_k{e_k}}$.

\begin{veta}
(Hlavní věta o konečně generovaných abelovských grupách) Je-li $G = \left< x_1, \dots, x_m \right>$ abelovská grupa generovaná prvky $x_1, \dots x_m$, potom platí:

$$ G \cong C_\infty^k \times C_{n_1} \times C_{n_r} $$

přičemž $k \geq 0 (C_\infty^0 := \{e\}), n_i = \mathbb{N}, r \geq 0$. Přitom platí $G$ je konečná $\Leftrightarrow k = 0$.
\end{veta}
($C_\infty$ označuje nekonečnou cyklickou grupu.)

















%%%%%%%%%%%%%%%%%%%%%%%%%%%%%%%%%%%%%%%%%%%%%%%%%%%%%%%%%%%%%%%%%%%%%%%%%%%%%%%%
%%%%%%%%%%%%%%%%%%%%%%%%%%%%%%%%%%%%%%%%%%%%%%%%%%%%%%%%%%%%%%%%%%%%%%%%%%%%%%%%
\chapter{Obory integrity a dělitelnost} \label{cha:10}

1. semestr, MAT, \texttt{Zaklady\_obecne\_algebry.pdf}, 4, 5. kapitola

(okruhy polynomů, pravidla dělitelnosti, Gaussovy a Eukleidovy okruhy)

\section{Polynomy}
\subsection{Konstrukce okruhů polynomů}

\begin{mydef}
Buď $(R, +, 0, -, \cdot, 1)$ komutativní okruh s jednotkovým prvkem. Výraz tvaru $\sum_{k=0}^\infty a_kx^k$, kde $a_k \in R$ pro všechna $k \in \mathbb{N}_0$ a množina $\{k \in \mathbb{N}_0 | a_k \not= 0\}$ je konečná, se nazývá polynom neurčití $x$ nad $R$. Množinu všech polynomů neurčité $x$ nad R označíme symbolem $R[x]$. Definujeme nyní operace $+, 0, -, \cdot, 1$ na $R[x]$ tak, aby $(R[x], +, 0, -, \cdot, 1)$ byl opět komutativní okruh s jednotkovým prvkem:

$ \sum_{k = 0}^\infty a_k x^k + \sum_{k = 0}^\infty b_k x^k  := \sum_{k = 0}^\infty (a_k + b_k) x^k$,
$0 := \sum_{k = 0}^\infty 0 x^k $,
$-(\sum_{k = 0}^\infty a_k x^k) := \sum_{k = 0}^\infty (-a_k) x^k$,
$\sum_{k = 0}^\infty a_k x^k \cdot \sum_{k = 0}^\infty b_k x^k := \sum_{k = 0}^\infty \sum_{l = 0}^n a_l k_{k-l} x^k$,
$1 := \sum_{k = 0}^\infty \delta_k x^k$
\end{mydef}

\begin{veta}
$(R[x], +, 0, -, \cdot, 1)$ je komutativní okruh s jednotkovým prvkem.
\end{veta}

\begin{mydef}
Je-li $p(x) = \sum_{l = 0}^n a_k x^k$, kde $a_n \not= 0$, pak se $n$ nazývá stupeň polynomu $p(x)$ (píšeme $n = \text{grad } p(x)$). 
\end{mydef}

\begin{mydef}
Je-li $p(x) = \sum_{k = 0}^n a_k x^k \in R[x]$, pak se prvky $a_x$ nazývají koeficienty polynomu $p(x)$. $0 \in R[x]$ je nulový polynom a $a \in R \subseteq R[x]$ se nazývá konstantní polynom. Platí-li $\text{grad } p(x) = n$ a $a_n = 1$, pak se $p(x)$ nazývá normalizovaný polynom. Polynomy tvaru $ax + b$, kde $a \not= 0$ se nazývají lineární polynomy.
\end{mydef}

\begin{veta}
Je-li $R$ obor integrity, potom je také $R[x]$ obor integrity, a pro $p(x), q(x) \in R[x] \backslash \{0\}$ platí $\text{grad } (p(x)q(x)) = \text{grad } p(x) + \text{grad } q(x)$.
\end{veta}

Není-li R obor integrity, pak ani $R[x]$ není obor integrity, neboť $r$ je podokruh  okruhu $R[x]$.

\subsection{Polynomy $n$ neurčitých $x_1, \dots x_n$}

Indukcí definujeme

$$R[x_1] := R[x], R[x_1, \dots, x_n] := (r[x_1, \dots, x_{n-1}])[x_n], n > 1$$

Potom platí

$$ R[x_1, \dots, x_n] = \{ \sum\limits_{0 \leq i_i, \dots, i_n \leq m} a_{i_i \dots i_n} x_1^{i_1} \dots x_n^{i_n} | m \in \mathbb{N}_0, a_{i_1} \dots a_{i_n} \in R \} $$

\subsection{Polynomy a funkce}
\paragraph{Princip dosazování} Buď $(R, +, 0, -, \cdot, 1)$ komutativní okruh s jednotkovým prvkem a $p(x) = a_n x^n + \dots + a_1 x + a_0 \in R[x]$. Pro $a \in R$ je potom $p(a) := a_n a^n + \dots + a_1 a + a_0$ opět prvkem z $R$ který se nazývá \emph{hodnota polynomu v $a$}. Funkce

$$ \begin{dcases}
R \to R \\
a \mapsto p(a)
\end{dcases} $$

se nazývá polynomiální funkce indukovaná polynomem $p(x)$ a často se taká označuje $p$.

\begin{veta}
Zobrazení
$$ \begin{dcases}
R[x] \to R \\
p(x) \mapsto p(a)
\end{dcases} $$
je pro pevně dané $a \in R$ surjektivní homomorfizmus $R[x]$ na $R$.
\end{veta}

\begin{mydef}
Buď $p(x) \in R[x]$ (komutativní okruh s jednotkovým prvkem). Potom se $a \in R$ nazývá kořen polynomu $p(x): \Leftrightarrow p(a) = 0$. Polynom $p(x)$ se nazývá dělitelný polynomem $q(x) \in R[x]$ (formálně $q(x) | p(x)$): $\Leftrightarrow p(x) = q(x)r(x)$ kde $r(x) \in R[x]$.
\end{mydef}

\begin{veta}
Je-li $a$ kořen polynomu $p(x)$, pak je $p(x)$ dělitelný lineárním polynomem $x - a$ (a opačně).
\end{veta}

Nechť je $R$ obor integrity (např. $R = \mathbb{Z}$ nebo $R$ pole). Je-li $\text{grad } p(x) = n$ a platí $(x - a)^k | p(x)$ , tj. $p(x) = (x -a)^k q(x)$, potom je $k + \text{ grad } q(x) = \text{ grad } p(x) = n$, z čehož plyne $k \leq n$.

\begin{mydef}
Buď $p(x) \in R[x] \backslash \{0\}$ a nechť $a \in R$ je kořenem $p(x)$. Potom největší číslo $k \in \mathbb{N}$ takové, že $(x - a)^k | p(x)$, se nazývá násobnost kořene $a$.

$l \leq \text{ grad } p(x)$
\end{mydef}

\begin{veta}
Nechť $a_1, \dots a_r$ jsou po dvou různé kořeny polynomu $p(x) \in R[x] \backslash \{0\}$ s násobností $k_1 ,\dots, k_r$. Potom platí

$$ (x - a)^{k_1} \dots (x - a)^{k_r} | p(x)$$
\end{veta}

Potom platí: $k_1 + \dots + k_r \leq \text{ grad } p(x)$

\begin{veta}
Buďte $p(x), q(x) \in R[x] \backslash \{0\}, \text{ grad } p(x) \leq n$ a $p(b_i) = q(b_i)$ pro $n+1$ po dvou různých prvků $b_0, \dots, b_n$ množiny $R$. Potom platí $p(x) = q(x)$.
\end{veta}

\begin{mydef}
Pole $K$ se nazývá algebraicky uzavření, jestliže každý polynom $p(x) \in K[x] \ K$ má alespoň jeden kořen.
\end{mydef}
Pokud má nad oborem integrity každý lineární polynom kořen, pak je tento obor integrity pole ($ax - 1 (a \not=0)$ má kořen $c \Rightarrow ac = 1 \Rightarrow c = a^{-1}$).

\begin{veta}
(Gaussova základní věta algebry) Pole $\mathbb{C}$ je algebraicky uzavřené.
\end{veta}

\begin{veta}
Je-li $K$ pole, potom jsou následující tvrzení ekvivalentní:
\begin{enumerate}[a)]
	\item $K$ je algebraicky uzavřené
	\item Pro všechna $p(x) \in K[x]$, kde $\text{ grad } p(x) = n > 0$, platí $p(x) = c(x - b_1)^{k_1} \dots (x - b_r)^{k_r}$, kde $b_1, \dots, b_r, c \in K$ a $k_1 + \dots + k_r = n$.
\end{enumerate}
\end{veta}

Výpočet kořenů nad poli:
\begin{enumerate}[1)]
	\item $\text{grad } p(x) = 1$: $ax + b$:  triviální
	\item $\text{grad } p(x) = 2$: $p(x) = ax^2 + bx + c (a \not= 0)$ má kořeny $\frac{-b \pm \sqrt{b^2 - 4ac}}{2a}$ (2 resp. 4 zde označuje 1 + 1 resp. 1+1+1+1; vyjádření kořenů musí existovat a musí být $1+1 \not= 0$.
	\item $\text{grad } p(x) = 3, 4$ Cardanovy vzorce
	\item $\text{grad } p(x) > 4$  nejsou obecné vzorce
\end{enumerate}

\subsection{Interpolace pomocí polynomů}
Buď $K$ pole a $f: K \to K$ funkce. \textbf{Zadáno}: $b_i = f(a_i)$  pro po dvou různá $a_i \in K, 1 \leq i \leq n$ (např. výsledek řady měření). \textbf{Hledá se}: $p(x) \in K[x]$, kde $p(a_i) = b_i = f(a_i), 1 \leq i \leq n$ a $\text{ grad } p(x) < n$. (Existuje nejvýše jeden takový polynom $p(x)$ z $p(a_i), 1 \leq i \leq n$, kde $\text{grad } p(x), \text{ grad } q(x) < n$ totiž plyne $p = q$.)

\subsubsection{Lagrangerovy interpolační vzorce}
Buď
$$q_i(x) := \prod\limits_{1 \leq j \leq n, j \not=i} (x - a_j) = \dots (x - a_{i-1}) (x - a_{i+1}) \dots (x - a_n)$$
Potom platí:
$$q_i(a_k) =  \begin{dcases}
0 \text{pro} i \not= k \\
\prod\limits_{1 \leq j \leq n, j \not=i} (a_k - a_j) \not= 0 \text{pro } i = k
\end{dcases}$$
Pro 
$$ p(x) := \sum\limits_{i = 1}^n b_i \frac{q_i(x)}{q i (a_i)} $$
Potom platí $$ p(a_j) = b_j, 1 \leq j \leq n $$

Je-li $K$ konečné pole (např. $K = \mathbb{Z}_p, p$ prvočíslo), $f: K \to K$, potom existuje polynom $p(x) \in K[x]$ takový, že $f(a) = p(a)$ pro všechna $a \in K$.

% \subsubsection{Newtonovy interpolačnı́ vzorce}
% XXX ???

\section{Obory integrity a dělitelnost}
\subsection{Jednoduchá pravidla dělitelnosti}


\begin{mydef}
Buď $(I, +, 0, -, \cdot, 1)$ obor integrity. Jsou-li $a, b \in I$, potom říkáme, že prvek $a$ je dělitelný prvkem $b$ a $b$ se nazývá dělitel prvku $a$ ($b$ \uv{dělí} $a$, formálně $b|a$): $\Leftrightarrow \exists c \in I: a = bc$.
\end{mydef}

Elementární pravidla dělitelnosti:
\begin{enumerate}[1)]
	\item $\forall a \in I: a | 0$
	\item $\forall a \in I: 1 | a$
	\item $\forall a \in I: a | a$
	\item $\forall a,b,c \in I: a|b \land b|c \Rightarrow a|c$
	\item $\forall a,b,c \in I: a|b \Rightarrow a|bc$
	\item $\forall a,b,c \in I: a|b \land a|c \Rightarrow a|b+c$
	\item $\forall a,b,c \in I, c \not= 0: a|b \Leftrightarrow ac|bc$
	\item $\forall a,b,c,d \in I: a|b \land c|d \Rightarrow ac|bd$
	\item $\forall a,b \in I, n \in \mathbb{N}: a|b \Rightarrow a^n|b^n$
\end{enumerate}

\begin{mydef}
Buď $(I, +, 0, -, \cdot, 1)$ obor integrity. Dělitel prvku 1 se nazývá jednotka oboru integrity $I$. Buď $E(I)$ množina všech jednotek $I$. Prvky $a, b \in I$ se nazývají asociované (formálně $a \sim b$): $\Leftrightarrow \exists e \in E(I): a = be$.
\end{mydef}

\begin{veta}
\begin{enumerate}[a)]
	\item $e \in I$ je jednotka oboru integrity $I \Leftrightarrow \exists f \in I: ef = 1$
	\item $(E(I(, \cdot))$ je abelovská grupa, která se nazývá grupa jednotek oboru integrity  $I$.
	\item $\sim$ je relace kongruence na $(I, \cdot)$
	\item $\forall a, b \in I: a \sim b \Leftrightarrow a|b \lor b|a$
\end{enumerate}
\end{veta}

\begin{mydef}
Buď $(I, +, 0, -, \cdot, 1)$ obor integrity, $a \in I$.
\begin{description}
	\item[Triviální dělitelé prvku $a$] jsou všechna $e \in E(I)$ a všechna $b$ taková, že $b \sim a$
	\item[Vlastní dělitelé prvku $a$] jsou všechna $b$ taková, že $b | a, b \not\in E(I)$ a $b \not\sim a$
\end{description}
\end{mydef}

\begin{mydef}
Prvek $a \in I \backslash E(I), a \not= 0$ se nazývá ireducibilní prvek: $\Leftrightarrow a$ má pouze triviální dělitele. (např. prvočísla v $\mathbb{Z}$.)
\end{mydef}

\begin{mydef}
$p \in I \backslash E(I), p \not= 0$ se nazývá prvočinitel: $\Leftrightarrow p|ab \Rightarrow p|a \lor p|b$
\end{mydef}

Prvočinitel $\Leftrightarrow$ je ireducibilní prvek.

\section{Gaussovy okruhy}

\begin{mydef}
Obor integrity $I$ se nazývá Gaussův okruh: $\Leftrightarrow$ Ke každému prvku $a \in I \backslash E(I), a \not= 0$, existují prvočinitelé $p_1, \dots, p_r$ (nikoliv nutně po dvou různí) tak, že platí $a = p_1 \dots p_r$
\end{mydef}

\begin{veta}
(Jednoznačnost rozkladu na prvočinitele) Buď $I$ Gaussův okruh, $a \in I \backslash E(I), a \not= 0, a = p_1^{(1)} \dots p_{r_1}^{(1)} = p_1^{(2)} \dots p_{r_2}^{(2)}$, kde $p_i^{(1)}, p_j^{(2)}$ jsou prvočinitelé. Potom je $r_1 = r_2 := r$ a existuje permutace množiny $\{1, \dots r\}$ taková, že $p_i^{(1)} \sim p_{\pi(i)}^{(2)}, i = 1, \dots, r$.
\end{veta}

$\mathbb{Z}$ a $K[x]$ (K pole) jsou Gaussovy okruhy.

\begin{mydef}
Buď $I$ obor integrity, $a_1, \dots, a_n \in I$.
\begin{enumerate}[1)]
	\item $d \in I$ se nazývá \emph{největší společný dělitel} (NSD) prvků $a_1, \dots a_n \in I: \Leftrightarrow$ (i) $d | a_i, i = 1, \dots, n$ a (ii) $\forall t \in I: t|a_i, i = 1, \dots, n \Rightarrow t | d$
	\item $v \in I$ se nazývá \emph{nejmenší společný násobek} (NSN) prvků $a_1, \dots a_n \in I: \Leftrightarrow$ (i) $a_i | v, i = 1, \dots, n$ a (ii) $\forall w \in I: a_i | w, i = 1, \dots, n \Rightarrow v | w$
\end{enumerate}
\end{mydef}

\begin{veta}
V Gaussově okruhu $I$ je každý ireducibilní prvek prvočinitelem.
\end{veta}

\begin{veta}
Je-li $I$ Gaussův okruh, $a \in I \backslash E(I), a \not=0$, potom platí $a = e p_1^{e_1} \dots p_r^{e_r}$, kde $e \in E(I), p_1, \dots p_r$ jsou normovaní navzájem různí prvočinitelé $e_i \in \mathbb{N}$. 
\end{veta}

\begin{lemma}
Buď $I$ Gaussův okruh, $a, b \in I \backslash \{0\}, a = fp_1^{f_1} \dots p_r^{f_r}, b = g p_1^{g_1} \dots p_r^{g_r}$ ($p_j$ normovaní navzájem různí prvočinitelé, $f_j, g_j \in \mathbb{N}_0, f,g \in E(I)$). Potom platí $a|b \Leftrightarrow f_j \leq g_j$ pro $j = 1, \dots, r$.
\end{lemma}

\begin{veta}
Buď $I$ Guassův okruh, $a_1, \dots, a_n \in I, a_i \not= 0, a_i = e_i p_1^{e_{1i}} \dots p_r^{e_{ri}}, e_i \in E(I), p_j$ navzájem různí normovaní prvočinitelé, $e_{ji} \in \mathbb{N_0}$. Potom platí:

$$ \text{NSD}(a_1, \dots, a_n) = p_1^{\text{min}_{1 \leq n \leq n}(e_{1i})} \dots p_r^{\text{min}_{1 \leq n \leq n}(e_{ri})}$$

a

$$ \text{NSN}(a_1, \dots, a_n) = p_1^{\text{max}_{1 \leq n \leq n}(e_{1i})} \dots p_r^{\text{max}_{1 \leq n \leq n}(e_{ri})}$$

Jsou-li některá $a_i = 0$, potom je NDS$(a_1, \dots, a_n) = $NSD$(a_i|a_i \not= 0)$; jsou-li všechna $a_i = 0$, potom je NDS$(a_1, \dots a_n) = 0$. Jsou-li některá $a_i = 0$, pak je $NSN(a_1, \dots, a_n) = 0$.
\end{veta}

\begin{veta}
Buď $I$ Gaussův okruh a $\cap, \cup$ binární operace na $I / \sim= \{[a]_\sim | a \in I\}$ definovaní vztahy:

$$[a]_\sim \cap [b]_\sim := [NSD(a, b)]_\sim $$
$$[a]_\sim \cup [b]_\sim := [NSN(a, b)]_\sim $$

Potom jsou $\cap, \cup$ korektně definovány (tj. nezávisle na volbě reprezentantů) a $(I / \sim, \cap, \cup)$ je svaz s nulovým prvkem $[1]_\sim = E(I)$ a jednotkovým prvkem $[0]_\sim = \{0\}$ (svaz dělitelů). Příslušné uspořádání $\leq$ je dáno vztahem: $[a]_\sim \leq [b]_\sim : \Leftrightarrow a|b$.
\end{veta}

$(\mathbb{Z}/ \sim, \cap, \cup) \equiv (\mathbb{N}_0, NSD, NSN)$

\section{Eukleidovy okruhy}
\begin{mydef}
Obor integrity $I$ se nazývá Eukleidův okruh: $\Leftrightarrow$ existuje zobrazení $H: I \backslash \{0\} \to \mathbb{N}_0$ (Eukleidovské ohodnocení) s následující vlastností: pro všechna $a \in I \backslash \{0\}, b \in I$ existují $q, r \in I$ tak, že $b=aq + r$, kde $r = 0 \lor H(r) < H(a)$ (dělení se zbytkem).
\end{mydef}

$\mathbb{Z}$ je Ekleidův okruh, kde $H(a) := |a|$. Každé pole je Eukleidův okruh ($q = a^{-1}b, r = 0$).

\begin{veta}
$K[x]$ (K pole) je Eukleidův okruh, kde $H(p(x)) := \text{ grad }p(x)$, tj. pro $p(x) \not= 0$, $p_1(x)$ libovolné, je $p_1(x) = p(x)q(x) + r(x)$, kde $r(x) = 0$ nebo grad $r(x) < \text{grad }p(x)$.
\end{veta}

Libovolný polynom $p(x) \in K[x]$ a libovolný prvek $a \in K$ existuje $r \in K$ tak, že $p(x) = (x - a)q(x) + p(a)$.

\begin{veta}
Každý Eukleidův okruh je Gaussův okruh.
\end{veta}













%%%%%%%%%%%%%%%%%%%%%%%%%%%%%%%%%%%%%%%%%%%%%%%%%%%%%%%%%%%%%%%%%%%%%%%%%%%%%%%%
%%%%%%%%%%%%%%%%%%%%%%%%%%%%%%%%%%%%%%%%%%%%%%%%%%%%%%%%%%%%%%%%%%%%%%%%%%%%%%%%
\chapter{Teorie polí} \label{cha:11}

1. semestr, MAT, \texttt{Zaklady\_obecne\_algebry.pdf}, 6. kapitola

(minimální pole, rozšíření pole, konečná pole a jejich konstrukce)

\section{Minimální pole}
\begin{mydef}
Pole $(K, +, 0, -, \cdot, 1)$ se nazývá minimální, pokud nemá žádná jiná podpole než sama sebe.
\end{mydef}

\begin{veta}
Každé pole má vždy jediné podpole, které je minimální.
\end{veta}

\begin{lemma}
Buď $(R, +, 0, -, \cdot, 1)$ okruh s jednotkovým prvkem. Pak $\{n \cdot 1 | n \in \mathbb{Z}\}$ je komutativní podokruh okruhu R s tímtéž jednotkovým prvkem 1, totiž podokruh generovaný prvkem 1.
\end{lemma}

\begin{mydef}
Buď $(R, +, 0, -, \cdot)$ okruh. Pak symbolem char $R$ označíme charakteristiku okruhu $R$, tj. nejmenší číslo $n \in \mathbb{N}$ takové, že pro každé $a \in R$ platí $n \cdot a = 0$ (kde $n \cdot a := \underbrace{a + a + \dots + a}_{n\text{-krát}}$. Pokud takové číslo neexistuje, pak klademe char $R = 0$.

Je-li $(R, +, 0, -, \cdot, 1)$ okruh s jednotkovým prvkem a $n \in \mathbb{N}$, pak pro každé $a \in R$ platí $n \cdot a = 0$, právě když platí $n \cdot 1 = 0$ (platí-li $n \cdot 1 = 0$ a je-li $a \in R$ libovolný prvek, pak máme $n \cdot a = \underbrace{a + a + \dots + a}_{n\text{-krát}} = (\underbrace{1 + 1 + \dots + 1}_{n\text{-krát}})\cdot a = (n \cdot 1) \cdot a = 0 \cdot a = 0$; opačná implikace je zřejmá). Je-li tedy R okruh s jednotkovým prvkem $1$, pak char R je nejmenší číslo $n \in N$ pro něž platí $n \cdot 1 = 0$, případně char $R = 0$ pokud takové číslo neexistuje. Odtud ihned plyne, že platí

$\text{char }R = \begin{dcases}
o(1), \text{ pokud } o(1) \in \mathbb{N}, \\
0, \text{ pokud } o(1) = \infty
\end{dcases}$

($o(1)$ značí řád prvku $1$ v abelovské grupě $(R, +)$, tedy $o(1) = |\{n \cdot 1 | n \in \mathbb{Z}\}|$ pokud je tato kardinalita konečná, jinak $\infty$.
\end{mydef}

$\text{char }R = \begin{dcases}
|\{n \cdot 1 | n \in \mathbb{Z}\}|, \text{ pokud se jedná o končnou kardinality }, \\
0, \text{ jinak }
\end{dcases}$

Okruh zbytkových tříd: $\mathbb{Z}_n$: char $\mathbb{Z}_n = n (n \in \mathbb{N}_0)$, char $\mathbb{Z} = 0$.

\begin{lemma}
Buď $(R, +, 0, -, \cdot, 1)$ okruh s jednotkovým prvkem a nechť $m = \text{ char }R$. Potom $\{n \cdot 1 | n \in \mathbb{Z}\} \cong \mathbb{Z}_m$
\end{lemma}

\begin{lemma}
\begin{enumerate}[1)]
	\item Je-li $R$ obor integrity a $m = \text{ char }R$, potom také $\{n \cdot 1 | n \in \mathbb{Z}\}$, a tedy i $\mathbb{Z}_m$ je obor integrity, takže platí $m=0$ nebo $m \in \mathbb{P}$ ($\mathbb{P}$ značí množinu všech prvočísel).
	\item Je-li $R$ obor integrity a char $R \in \mathbb{P}$, potom $\{n \cdot 1 | n \in \mathbb{Z}\}$ je pole.
\end{enumerate}
\end{lemma}

\begin{veta}
Buď $(K, +, 0, -, \cdot, 1)$ pole takové, že char $K \in \mathbb{P}$. Potom $\{n \cdot 1 | n \in \mathbb{Z}$ je minimální podpole pole K. V tomto případě tedy platí: minimální podpole pole K je izomorfní se $\mathbb{Z}_m$, kde $m = \text{ char } K$.
\end{veta}

\begin{veta}
Buď $(K, +, 0, -, \cdot, 1)$ pole, kde char $K = 0$. Potom je $\{ \frac{n \cdot 1}{m \cdot 1} | n \in \mathbb{Z}, m \in \mathbb{Z} \backslash \{0\}\}$ nejmenším podpolem  a tudíž minimálním podpolem pole $K$. Toto minimální podpole je izomorfní s $\mathbb{Q}$. Přitom jsme položili $\frac{n \cdot 1}{m \cdot 1} := (n \cdot 1)(m \cdot 1)^{-1}$.
\end{veta}

Každé minimální pole je izomorfní se $\mathbb{Z}_p (p \in \mathbb{P})$ nebo $\mathbb{Q}$.

\section{Rozšíření pole}
\begin{mydef}
Buďte $K, L$ pole a $K$ podpole pole $L$. Potom $L$ se nazývá nadpole nebo rozšíření pole $K$.
\end{mydef}

Pole $L$ se nazývá \emph{kořenové pole} polynomu $f(x)$ vzhledem ke $K$, pokud je rozšířením pole s právě $n$ kořeny.

Buď $L$ takové kořenové pole polynomu $f(x)$ vzhledem ke $K$. Potom je

$K(\alpha_1, \dots, \alpha_n) := \bigcap \{M \subseteq L | M \text{ podpole pole } L, K \subseteq M, \alpha_1, dots, \alpha_n \in M \}$

nejmenší podpole pole L, které obsahuje pole $K$ a prvky $\alpha_1, \dots, \alpha_n$. $K(\alpha_1, \dots, \alpha_n)$ se nazývá \emph{rozkladové pole} polynomu $f(x)$ vzhledem ke K.

\begin{veta}
(Kroneckerova věta) Ke každému $f(x) \in K[x], f(x) \not= 0$, existuje kořenové pole a tedy rozkladové pole polynomu $f(x)$ vzhledem ke K.
\end{veta}

Je-li $L$ nadpole pole $K$, potom je $L$ také vektorovým prostorem nad $K$. Vztahem $\text{dim}_K K =: [L : K]$ definujeme tzv. stupeň rozšíření $L$ pole $K$. Je-li $[L : K] < \infty$, pak se $L$ nazývá \emph{konečné rozšíření} pole $K$.

\begin{enumerate}
	\item $[ L : K] = 1 \Leftrightarrow L = K$
	\item Je-li $p(x) \in K[x]$ ireducibilní polynom stupně $k$, pak existuje rozšíření $L$ pole $K$ a prvek $\alpha \in L$ tak, že $p(x) = 0$ a $\{1, \alpha, \dots, \alpha^{k-1}\}$ je báze $L$ nad $K$. Tedy platí $ [L : K] = k$.
\end{enumerate}

\begin{mydef}
Buď $L$ nadpole pole $K$ a $\alpha \in L$.
$\alpha$ se nazývá \emph{algebraický} prvek nad $K: \Leftrightarrow \exists f(x) \in K[x] \backslash \{0\}: f(\alpha) = 0$. \\
$\alpha$ se nazývá \emph{transcendentní} prvek nad $K: \Leftrightarrow \not\exists f(x) \in K[x] \backslash \{0\}: f(\alpha) = 0$.
\end{mydef}

\begin{mydef}
Je-li $L$ nadpole pole $K$ a $S \subseteq L$, pak definujeme rozšíření $K(S)$ pole $K$ takto:

$$K(S) := \bigcap \{E \subseteq L | E \text{ je podpole pole L, které obsahuje } K \cup S\}$$

Je-li $S = \{u_1, \dots, u_r\}$ konečné, pak píšeme $K(S) =: K(u_1, \dots, u_r)$. Je-li speciálně $S = \{\alpha\}$ jednoprvkové, pak píšeme $K(S) =: K(\alpha)$ (\uv{jednoduché rozšíření} pole K).
\end{mydef}

\section{Konečná pole (Galoisova pole)}

\begin{veta}
Řád každého konečného pole je mocnina prvočísla $p^n (p \in \mathbb{P}, n \in \mathbb{N})$. Obráceně ke každé mocnině čísla $p^n$ existuje až na izomorfizmus jediné pole $K$ takové, že $|K| = p^n$.
\end{veta}
Způsob zápisu pro $K$, kde $|K| = p^n: K = GF(p^n)$ (Galoisovo pole).

\begin{veta}
Je-li $K$ konečné pole, pak je grupa $(K \backslash \{0\}, \cdot)$ cyklická.
\end{veta}

\subsection{Konstrukce konečného pole $K$}

Každý generátor grupy $(K \ \{0\}, \cdot)$ se nazývá primitivní prvek $K$. Je-li $\alpha$ primitivní prvek $K$, pak $K = \{0, 1, \alpha, \alpha^2, \dots, \alpha^{|K|-2}\}$. Buď $\mathbb{Z}_q, q \in \mathbb{P}$, minimální podpole pole $K$. Pak pro libovolná primitivní prvek $\alpha$ z $K$ platí $K \cong \mathbb{Z}_q(\alpha)$ a $\alpha$ je algebraický prvek nad $\mathbb{Z}_q$ (neboť je kořenem polynomu $x^{|K|-1} - 1 \in \mathbb{Z}_q[x]$). Buď $f(x)$ minimální polynom kořene $\alpha$ vzhledem k $\mathbb{Z}_q$. Potom je $f(x)$ ireducibilní a platí

$$ \mathbb{Z}_q(\alpha) = \{a_0 + a_1 \alpha + \dots + a_{n-1}\alpha^{m-1} | a_i \in \mathbb{Z}_q\}$$

kde $m = \text{ grad } f(x)$. Odtud dostáváme $|\mathbb{Z_q}(\alpha)| = q^m$ a z podmínky $|\mathbb{Z}_q(\alpha)| = |K| = p^n$ nyní vyplývá $q = p$ a $m = n$.

Při určování konečného pole $K = GF(p^n)$, tj. při sestavování tabulek jeho operací, lze proto postupovat následujícím způsobem:
\begin{enumerate}
	\item Za minimální podpole pole $K$ se vezme $\mathbb{Z}_p$.
	\item Zvolíme normovaný ireducibilní polynom $q(x) \in \mathbb{Z}_p[x]$ stupně $n$. Nechť např. $q(x) = x^n - a_{n-1}x^{n-1} - \dots - a_1 x - x_0$ kde $a_i \in \mathbb{Z}_p$.
	\item Položíme $q(\alpha) = 0$ a uvažujeme bází $\{1, \alpha, \dots, \alpha^{n-1}\}$ vektorového prostoru $GF(p^n)$ nad $\mathbb{Z}_p$ (víme, že $[GF(p^n): \mathbb{Z}_p] = n$). Spočítáme použitím $q(\alpha = 0)$ (tj. $a^n = a_0 + x_1 \alpha + \dots + a_{n-1}x^{n-1}$ mocniny $\alpha$. Platí-li $\alpha^{p^n-1} = 1$ pro $1 \leq j < p^n - 1$, je $\alpha$ primitivní prvek $GF(p^n)$. Jinak učiníme další pokus s novým polynomem $q(x)$.
\end{enumerate}













%%%%%%%%%%%%%%%%%%
%%%%%%%%%%%%%%%%%%%%%%%%%%%%%%%%%%%%%%%%%%%%%%%%%%%%%%%%%%%%%%%%%%%%%%%%%%%%%%%%
\chapter{Metrické prostory (příklady, konvergence posloupností, spojitá a izometrická zobrazení, úplnost, Banachova věta o pevném bodu).} \label{cha:12}
\chapter{Normované a unitární prostory (základní vlastnosti a příklady, normované prostory konečné dimenze, uzavřené ortonormální systémy a Fourierovy řady).} \label{cha:13}
\chapter{Obyčejné grafy (stupně uzlů, sledy, souvislost, izomorfismy, stromy, kostry, Kruskalův a Primův algoritmus pro hledání minimální kostry ohodnoceného grafu, eulerovské a hamiltonovské grafy, planarita a obarvitelnost).} \label{cha:14}
\chapter{Orientované grafy (orientované sledy, souvislost a silná souvislost, turnaje, eulerovské a hamiltonovské grafy, Dijkstrův a Floyd-Warshallův algoritmus pro hledání cesty minimální délky).} \label{cha:15}
\chapter{Klasifikace gramatik, formálních jazyků a automatů přijímajících jazyky.} \label{cha:16}
\chapter{Vlastnosti formálních jazyků (typické vlastnosti a jejich rozhodnutelnost).} \label{cha:17}
\chapter{Konečné automaty (jazyky přijímané jazyky KA, varianty KA, minimalizace KA).} \label{cha:18}
\chapter{Regulární množiny, regulární výrazy a rovnice nad regulárními výrazy.} \label{cha:19}
\chapter{Transformace a normální formy bezkontextových gramatik.} \label{cha:20}
\chapter{Zásobníkové automaty (jazyky přijímané ZA, varianty ZA).} \label{cha:21}
\chapter{Turingovy stroje (jazyky přijímané TS, varianty TS, lineárně omezené automaty, univerzální TS).} \label{cha:22}
\chapter{Nerozhodnutelnost (problém zastavení TS, princip diagonalizace a redukce, Postův korespondenční problém).} \label{cha:23}
\chapter{Parciální rekurzivní funkce.} \label{cha:24}
\chapter{Časová a paměťová složitost (třídy složitosti, úplnost, SAT problém).} \label{cha:25}
\chapter{Ukazatele a zákony paralelního zpracování. Funkce konst. účinnosti a škálovatelnost.} \label{cha:26}
\chapter{Paralelizace programů: vzory programových a datových struktur, podpůrné struktury.} \label{cha:27}
\chapter{Paralelní zpracování v OpenMP, SPMD, smyčky, sekce a tasky. Synchronizační prostředky.} \label{cha:28}
\chapter{Architektury se sdílenou pamětí, UMA i NUMA, zajištění koherence pamětí cache.} \label{cha:29}
\chapter{Architektury distribuovaných systémů se zasílání zpráv.} \label{cha:30}
\chapter{Blokující a neblokující párové (point-to-point) komunikace v MPI.} \label{cha:31}
\chapter{Kolektivní komunikace v MPI, paralelní vstup a výstup.} \label{cha:32}
\chapter{Propojovací sítě: Topologie a směrovací algoritmu, přepínání a řízení toku.} \label{cha:33}
\chapter{Klasifikace metod komprese dat (ztrátové, bezeztrátové, intuitivní, algoritmické, četnost výskytu, pravděpodobnost výskytu), princip kódování délek sledů, kódování „přesuň na začátek.‘‘} \label{cha:34}
\chapter{Kódy s proměnnou délkou - princip, zdůvodnění, Huffmanovy kódy - různé typy, kanonický Huffmanův kód, adaptivní Huffmanův kód, aritmetický kód.} \label{cha:35}
\chapter{Slovníkové metody (LZ77, LZ78, práce se slovníkem, pohyblivé okno, prodlužování položek).} \label{cha:36}
\chapter{Informace a entropie, Shannova věta o kódování.} \label{cha:37}
\chapter{Bezpečnostní kódy: lineární, Hammingovy, cyklické, konvoluční. Detekce a oprava chyb.} \label{cha:38}
\chapter{Základní architektury přepínačů, algoritmy pro plánování, řešení blokování, vícestupňové přepínací sítě.} \label{cha:39}
\chapter{Základní funkce směrovače, zpracování paketů ve směrovači, typy architektur.} \label{cha:40}
\chapter{Sítě Peer-to-Peer (P2P), Milgramův problém malého světa, model sítě P2P, směrování v P2P sítích, strukturované a nestrukturované sítě.} \label{cha:41}
\chapter{Základní principy softwarově definovaných sítí SDN, architektura, technologie OpenFlow.} \label{cha:42}
\chapter{Formální metody v počítačových sítích.} \label{cha:43}
\chapter{Konflikty a závislosti při řetězovém zpracování instrukcí a jejich HW/SW ošetření.} \label{cha:44}
\chapter{Architektura superskalárních procesorů a algoritmy OOO zpracování instrukcí.} \label{cha:45}
\chapter{Procesory VLIW a používané optimalizační techniky s HW podporou.} \label{cha:46}
\chapter{Multivláknové procesory, hrubý, jemný a simultánní MT.} \label{cha:47}
\chapter{Datový paralelismus SIMD a SIMT, HW implementace a SW podpora.} \label{cha:48}
\chapter{Architektura grafických procesorů, odlišnosti od superskalárních procesorů.} \label{cha:49}
\chapter{Programovací jazyk CUDA, model vláken a paměťový model.} \label{cha:50}
\chapter{Základní rysy nízkopříkonových procesorů (požadavky, architektura, výkonnost).} \label{cha:51}
\chapter{Jazyky pro popis obvodů (VHDL, syntetizovatelné konstrukce).} \label{cha:52}
\chapter{Logická syntéza obvodů (návrh pro technologie FPGA a ASIC, fáze syntézy, optimalizace, mapování, techniky zřetězení a vyvážení).} \label{cha:53}
\chapter{Moderní přístupy k syntéze číslicových obvodů (reprezentace obvodu pomocí AIG, techniky odstraňování funkční redundance v AIG, tradiční mapování AIG do LUT).} \label{cha:54}
\chapter{Aplikace omezujících podmínek (časová a fyzická omezení).} \label{cha:55}
\chapter{Verifikace číslicových obvodů (metodologie OVM).} \label{cha:56}
\chapter{Technologie programovatelného hardware (architektura FPGA, struktura konfigurovatelných bloků a vestavěných bloků, propojovací sít, způsoby konfigurace, srovnání s technologií ASIC).} \label{cha:57}
\chapter{Vestavěný počítačový systém (shody a odlišnosti s běžným univerzálním počítačovým systémem).} \label{cha:58}
\chapter{Implementace funkcí vestavěného systému SW a HW prostředky (výhody a nevýhody - dopady SW a HW implementace konkrétní funkce na vlastnosti systému, příklad).} \label{cha:59}
\chapter{Číslicové vstupy a výstupy vestavěných systémů (problémy a jejich řešení, přizpůsobení napěťových úrovní, snímání stavu mechanického kontaktu, ovládání zátěže, posílení výstupu, H-můstek).} \label{cha:60}
\chapter{Architektura SW pro vestavěné systémy (hlavní smyčka, implementace stavového automatu, obsluha přerušení).} \label{cha:61}
\chapter{Konstrukce adaptéru systémové sběrnice: návrh adresového dekodérů, obsluha čtecí a zápisové transakce.} \label{cha:62}
\chapter{Architektura sběrnice PCI-Express a USB: typy transakcí, způsob komunikace a směrování transakcí, detekce chyb a způsob zotavení.} \label{cha:63}

% 2. semestr, IPR, zdroje % XXX


\end{document}

