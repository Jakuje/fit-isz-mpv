\documentclass[a4paper, 11pt]{report}
\usepackage[czech]{babel}
\usepackage[utf8]{inputenc}
\usepackage{multirow}
\usepackage{amsmath}
\usepackage{amsfonts}

\usepackage{geometry}
\usepackage{layout}

\geometry{
  includeheadfoot,
  hmargin=2.0cm,
  vmargin={0cm, 1.0cm}
}

\usepackage{color}
\usepackage[unicode,colorlinks,hyperindex,plainpages=false,pdftex]{hyperref}

\usepackage{listings}  
\definecolor{mygreen}{rgb}{0,0.6,0}
\lstset{language=VHDL,commentstyle=\color{mygreen},tabsize=4}

\usepackage{fancyhdr}
\pagestyle{fancyplain}
\fancyhf{}
\renewcommand{\headrulewidth}{0pt}

\cfoot{\hfill © Jakuje \hfill \thepage }


\begin{document}

\ref{cha:1}
\ref{cha:2}
\ref{cha:3}
\ref{cha:4}
\ref{cha:5}
\ref{cha:6}
\ref{cha:7}
\ref{cha:8}
\ref{cha:9}
\ref{cha:10}

\ref{cha:11}
\ref{cha:12}
\ref{cha:13}
\ref{cha:14}
\ref{cha:15}
\ref{cha:16}
\ref{cha:17}
\ref{cha:18}
\ref{cha:19}
\ref{cha:20}

\ref{cha:21}
\ref{cha:22}
\ref{cha:23}
\ref{cha:24}
\ref{cha:25}
\ref{cha:26}
\ref{cha:27}
\ref{cha:28}
\ref{cha:29}
\ref{cha:30}

\ref{cha:31}
\ref{cha:32}
\ref{cha:33}
\ref{cha:34}
\ref{cha:35}
\ref{cha:36}
\ref{cha:37}
\ref{cha:38}
\ref{cha:39}
\ref{cha:40}

\ref{cha:41}
\ref{cha:42}
\ref{cha:43}
\ref{cha:44}
\ref{cha:45}
\ref{cha:46}
\ref{cha:47}
\ref{cha:48}
\ref{cha:49}
\ref{cha:50}

\ref{cha:51}
\ref{cha:52}
\ref{cha:53}
\ref{cha:54}
\ref{cha:55}
\ref{cha:56}
\ref{cha:57}
\ref{cha:58}
\ref{cha:59}
\ref{cha:60}

\ref{cha:61}
\ref{cha:62}
\ref{cha:63}
\newpage

\tableofcontents

%%%%%%%%%%%%%%%%%%%%%%%%%%%%%%%%%%%%%%%%%%%%%%%%%%%%%%%%%%%%%%%%%%%%%%%%%%%%%%%%
%%%%%%%%%%%%%%%%%%%%%%%%%%%%%%%%%%%%%%%%%%%%%%%%%%%%%%%%%%%%%%%%%%%%%%%%%%%%%%%%
\chapter{Metodika návrhu HW/SW codesign, platformy, programovatelné obvody.} \label{cha:1}
\chapter{Výpočetní modely (StateCharts, Kahnova síť procesů, synchronní dataflow).} \label{cha:2}
\chapter{Specifikace (chování, struktura), syntéza (alokace, přidělení, plánování) a integrace systémů (rozhraní, synchronizace, komunikace).} \label{cha:3}
\chapter{Syntéza HW z vyšších programovacích jazyků (reprezentace, alokace, plánování, přiřazení) a jazyk Catapult C.} \label{cha:4}
\chapter{Odhady (přesnost, věrnost, metriky, metody) a optimalizace vlastností systému (příkon, energie).} \label{cha:5}
\chapter{Jazyk a sémantika predikátové logiky (termy, formule, realizace jazyka, pravdivost formulí).} \label{cha:6}
\chapter{Formální systém predikátové logiky (axiomy a odvozovací pravidla, dokazatelnost, model a důsledek teorie, věty o úplnosti a kompaktnosti, prenexní tvar formulí).} \label{cha:7}
\chapter{Algebraické struktury (grupy, okruhy, obory integrity a tělesa, svazy a Boolovy algebry, univerzální algebry).} \label{cha:8}
\chapter{Základní algebraické metody (podalgebry, homomorfismy, přímé součiny, kongruence a faktorové algebry, normální podgrupy a ideály okruhů).} \label{cha:9}
\chapter{Obory integrity a dělitelnost (okruhy polynomů, pravidla dělitelnosti, Gaussovy a Eukleidovy okruhy).} \label{cha:10}
\chapter{Teorie polí (minimální pole, rozšíření pole, konečná pole a jejich konstrukce).} \label{cha:11}
\chapter{Metrické prostory (příklady, konvergence posloupností, spojitá a izometrická zobrazení, úplnost, Banachova věta o pevném bodu).} \label{cha:12}
\chapter{Normované a unitární prostory (základní vlastnosti a příklady, normované prostory konečné dimenze, uzavřené ortonormální systémy a Fourierovy řady).} \label{cha:13}
\chapter{Obyčejné grafy (stupně uzlů, sledy, souvislost, izomorfismy, stromy, kostry, Kruskalův a Primův algoritmus pro hledání minimální kostry ohodnoceného grafu, eulerovské a hamiltonovské grafy, planarita a obarvitelnost).} \label{cha:14}
\chapter{Orientované grafy (orientované sledy, souvislost a silná souvislost, turnaje, eulerovské a hamiltonovské grafy, Dijkstrův a Floyd-Warshallův algoritmus pro hledání cesty minimální délky).} \label{cha:15}
\chapter{Klasifikace gramatik, formálních jazyků a automatů přijímajících jazyky.} \label{cha:16}
\chapter{Vlastnosti formálních jazyků (typické vlastnosti a jejich rozhodnutelnost).} \label{cha:17}
\chapter{Konečné automaty (jazyky přijímané jazyky KA, varianty KA, minimalizace KA).} \label{cha:18}
\chapter{Regulární množiny, regulární výrazy a rovnice nad regulárními výrazy.} \label{cha:19}
\chapter{Transformace a normální formy bezkontextových gramatik.} \label{cha:20}
\chapter{Zásobníkové automaty (jazyky přijímané ZA, varianty ZA).} \label{cha:21}
\chapter{Turingovy stroje (jazyky přijímané TS, varianty TS, lineárně omezené automaty, univerzální TS).} \label{cha:22}
\chapter{Nerozhodnutelnost (problém zastavení TS, princip diagonalizace a redukce, Postův korespondenční problém).} \label{cha:23}
\chapter{Parciální rekurzivní funkce.} \label{cha:24}
\chapter{Časová a paměťová složitost (třídy složitosti, úplnost, SAT problém).} \label{cha:25}
\chapter{Ukazatele a zákony paralelního zpracování. Funkce konst. účinnosti a škálovatelnost.} \label{cha:26}
\chapter{Paralelizace programů: vzory programových a datových struktur, podpůrné struktury.} \label{cha:27}
\chapter{Paralelní zpracování v OpenMP, SPMD, smyčky, sekce a tasky. Synchronizační prostředky.} \label{cha:28}
\chapter{Architektury se sdílenou pamětí, UMA i NUMA, zajištění koherence pamětí cache.} \label{cha:29}
\chapter{Architektury distribuovaných systémů se zasílání zpráv.} \label{cha:30}
\chapter{Blokující a neblokující párové (point-to-point) komunikace v MPI.} \label{cha:31}
\chapter{Kolektivní komunikace v MPI, paralelní vstup a výstup.} \label{cha:32}
\chapter{Propojovací sítě: Topologie a směrovací algoritmu, přepínání a řízení toku.} \label{cha:33}
\chapter{Klasifikace metod komprese dat (ztrátové, bezeztrátové, intuitivní, algoritmické, četnost výskytu, pravděpodobnost výskytu), princip kódování délek sledů, kódování „přesuň na začátek.‘‘} \label{cha:34}
\chapter{Kódy s proměnnou délkou - princip, zdůvodnění, Huffmanovy kódy - různé typy, kanonický Huffmanův kód, adaptivní Huffmanův kód, aritmetický kód.} \label{cha:35}
\chapter{Slovníkové metody (LZ77, LZ78, práce se slovníkem, pohyblivé okno, prodlužování položek).} \label{cha:36}
\chapter{Informace a entropie, Shannova věta o kódování.} \label{cha:37}
\chapter{Bezpečnostní kódy: lineární, Hammingovy, cyklické, konvoluční. Detekce a oprava chyb.} \label{cha:38}
\chapter{Základní architektury přepínačů, algoritmy pro plánování, řešení blokování, vícestupňové přepínací sítě.} \label{cha:39}
\chapter{Základní funkce směrovače, zpracování paketů ve směrovači, typy architektur.} \label{cha:40}
\chapter{Sítě Peer-to-Peer (P2P), Milgramův problém malého světa, model sítě P2P, směrování v P2P sítích, strukturované a nestrukturované sítě.} \label{cha:41}
\chapter{Základní principy softwarově definovaných sítí SDN, architektura, technologie OpenFlow.} \label{cha:42}
\chapter{Formální metody v počítačových sítích.} \label{cha:43}
\chapter{Konflikty a závislosti při řetězovém zpracování instrukcí a jejich HW/SW ošetření.} \label{cha:44}
\chapter{Architektura superskalárních procesorů a algoritmy OOO zpracování instrukcí.} \label{cha:45}
\chapter{Procesory VLIW a používané optimalizační techniky s HW podporou.} \label{cha:46}
\chapter{Multivláknové procesory, hrubý, jemný a simultánní MT.} \label{cha:47}
\chapter{Datový paralelismus SIMD a SIMT, HW implementace a SW podpora.} \label{cha:48}
\chapter{Architektura grafických procesorů, odlišnosti od superskalárních procesorů.} \label{cha:49}
\chapter{Programovací jazyk CUDA, model vláken a paměťový model.} \label{cha:50}
\chapter{Základní rysy nízkopříkonových procesorů (požadavky, architektura, výkonnost).} \label{cha:51}
\chapter{Jazyky pro popis obvodů (VHDL, syntetizovatelné konstrukce).} \label{cha:52}
\chapter{Logická syntéza obvodů (návrh pro technologie FPGA a ASIC, fáze syntézy, optimalizace, mapování, techniky zřetězení a vyvážení).} \label{cha:53}
\chapter{Moderní přístupy k syntéze číslicových obvodů (reprezentace obvodu pomocí AIG, techniky odstraňování funkční redundance v AIG, tradiční mapování AIG do LUT).} \label{cha:54}
\chapter{Aplikace omezujících podmínek (časová a fyzická omezení).} \label{cha:55}
\chapter{Verifikace číslicových obvodů (metodologie OVM).} \label{cha:56}
\chapter{Technologie programovatelného hardware (architektura FPGA, struktura konfigurovatelných bloků a vestavěných bloků, propojovací sít, způsoby konfigurace, srovnání s technologií ASIC).} \label{cha:57}
\chapter{Vestavěný počítačový systém (shody a odlišnosti s běžným univerzálním počítačovým systémem).} \label{cha:58}
\chapter{Implementace funkcí vestavěného systému SW a HW prostředky (výhody a nevýhody - dopady SW a HW implementace konkrétní funkce na vlastnosti systému, příklad).} \label{cha:59}
\chapter{Číslicové vstupy a výstupy vestavěných systémů (problémy a jejich řešení, přizpůsobení napěťových úrovní, snímání stavu mechanického kontaktu, ovládání zátěže, posílení výstupu, H-můstek).} \label{cha:60}
\chapter{Architektura SW pro vestavěné systémy (hlavní smyčka, implementace stavového automatu, obsluha přerušení).} \label{cha:61}
\chapter{Konstrukce adaptéru systémové sběrnice: návrh adresového dekodérů, obsluha čtecí a zápisové transakce.} \label{cha:62}
\chapter{Architektura sběrnice PCI-Express a USB: typy transakcí, způsob komunikace a směrování transakcí, detekce chyb a způsob zotavení.} \label{cha:63}

% 2. semestr, IPR, zdroje % XXX


\end{document}

